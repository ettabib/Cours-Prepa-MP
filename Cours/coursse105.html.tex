\textbf{Warning: 
requires JavaScript to process the mathematics on this page.\\ If your
browser supports JavaScript, be sure it is enabled.}

\begin{center}\rule{3in}{0.4pt}\end{center}

{[}
{[}
{[}{]}
{[}

\subsubsection{20.2 Intégrales multiples}

\paragraph{20.2.1 Pavés et subdivisions. Ensembles négligeables}

Définition~20.2.1 On appelle pavé de \mathbb{R}~^n tout ensemble P de
la forme P = {[}a1,b1{]}
\times⋯ \times {[}an,bn{]}. On
notera mesure du pavé P le nombre réel positif m(P)
= \∏ ~
i=1^n(bi - ai).

Remarque~20.2.1 Un pavé est clairement compact.

Définition~20.2.2 Soit P = {[}a1,b1{]}
\times⋯ \times {[}an,bn{]} un pavé
de \mathbb{R}~^n. On appelle subdivision de P toute famille \sigma =
(\sigma1,\\ldots,\sigman~)
où chaque \sigmai est une subdivision de
{[}ai,bi{]}. Si \sigmai =
(ai,\\\\jmathmathmathmath)1\leq\\\\jmathmathmathmath\leqni, les sous pavés
P\\\\jmathmathmathmath1,\\ldots,\\\\jmathmathmathmathn~
= {[}a1,\\\\jmathmathmathmath1-1,a1,\\\\jmathmathmathmath1{]}
\times⋯ \times
{[}an,\\\\jmathmathmathmathn-1,an,\\\\jmathmathmathmathn{]} sont appelés
les sous pavés de la subdivision. On appelle pas de la subdivision \sigma =
(\sigma1,\\ldots,\sigman~)
le plus grand des diamètres des sous pavés de \sigma.

Définition~20.2.3 Un sous-ensemble A de \mathbb{R}~^n est dit
négligeable (au sens de Riemann) si quelque soit \epsilon \textgreater{} 0, il
existe une famille finie de pavés (Pi)1\leqi\leqN
vérifiant

\begin{itemize}
\itemsep1pt\parskip0pt\parsep0pt
\item
  (i) A \subset~\⋃ ~
  i=1^NPi
\item
  (ii) \\sum ~
  i=1^Nm(Pi) \leq \epsilon.
\end{itemize}

Proposition~20.2.1

\begin{itemize}
\itemsep1pt\parskip0pt\parsep0pt
\item
  (i) tout ensemble négligeable est borné
\item
  (ii) si A est négligeable et B \subset~ A, alors B est aussi négligeable
\item
  (iii) une partie A est négligeable si et seulement
  si~\overlineA est négligeable
\item
  (iv) toute réunion finie de parties négligeables est négligeable
\end{itemize}

Démonstration Tout est à peu près évident. L'affirmation (iii) résulte
de ce que, si A \subset~\\⋃
 i=1^NPi, alors on a aussi
\overlineA
\subset~\⋃ ~
i=1^NPi puisque
\⋃ ~
i=1^NPi est fermé.

Théorème~20.2.2 Soit Q un pavé de \mathbb{R}~^n-1 et f une application
continue de Q dans \mathbb{R}~. Alors le graphe de f est une partie négligeable de
\mathbb{R}~^n.

Démonstration Puisque f est continue sur le compact Q, elle est
uniformément continue. Soit donc \epsilon \textgreater{} 0. Il existe \eta
\textgreater{} 0 tel que \forall~~x,x' \in Q,
\\textbar{}x - x'\\textbar{} \textless{} \eta
\rigtharrow~\textbar{}f(x) - f(x')\textbar{} \textless{} \epsilon \over
2m(Q) . Soit alors \sigma une subdivision de Q de pas inférieur strictement
à \eta et (Qi)1\leqi\leqN les sous pavés de la subdivision.
Choisissons un point xi dans chaque Qi et posons
Pi = Qi \times {[}f(xi) - \epsilon
\over 2m(Q) ,f(xi) + \epsilon \over
2m(Q) {]}. Chaque Pi est un pavé de \mathbb{R}~^n et
m(Pi) = \epsilon \over m(Q) m(Qi) si
bien que \\sum ~
m(Pi) = \epsilon \over m(Q)
 \\sum  m(Qi~)
= \epsilon \over m(Q) m(Q) = \epsilon. Mais soit (x,f(x)) un point
du graphe de f avec x \in Q~; soit i tel que x \in Qi~; on a alors
\\textbar{}x - xi\\textbar{} \leq
\delta(\sigma) \textless{} \eta et donc \textbar{}f(x) - f(xi)\textbar{}\leq
\epsilon \over 2m(Q) , soit encore f(x) \in {[}f(xi)
- \epsilon \over 2m(Q) ,f(xi) + \epsilon
\over 2m(Q) {]}, si bien que (x,f(x)) \in Pi.
On en déduit que le graphe de f est contenu dans
\⋃ ~
i=1^NPi avec
\\sum  m(Pi~) =
\epsilon. Donc le graphe de f est négligeable.

Corollaire~20.2.3 Toute partie de \mathbb{R}~^n qui est une réunion
finie de graphes d'applications continues sur des pavés

\begin{align*}
(x1,\\ldots,xi-1,xi+1,\\\ldots,xn~)&&
\%& \\ & \mapsto~&
xi =
f(x1,\\ldots,xi-1,xi+1,\\\ldots,xn~)\%&
\\ \end{align*}

est négligeable.

\paragraph{20.2.2 Intégrales multiples sur un pavé de \mathbb{R}~^n}

Remarque~20.2.2 On appelle point de discontinuité de f tout point où f
n'est pas continue. Si f est une application de l'espace métrique X dans
l'espace métrique E, on notera
\mathrmDisc~ (f) l'ensemble
des points de discontinuité de f.

Proposition~20.2.4 Soit E un espace vectoriel normé de dimension finie
et P un pavé de \mathbb{R}~^n. L'ensemble \mathcal{E} des fonctions f : P \rightarrow~ E
bornées et dont l'ensemble des points de discontinuité est négligeable
est un sous-espace vectoriel de l'espace des applications de P dans E.

Démonstration Cet ensemble est évidemment non vide (il contient par
exemple toutes les fonctions continues sur P)~; si f et g sont dans \mathcal{E},
si \alpha~ et \beta~ sont des scalaires, on a évidemment \alpha~f + \beta~g qui est bornée et
de plus \mathrmDisc~ (\alpha~f +
\beta~g) \subset~\mathrmDisc~ (f)
\cup\mathrmDisc~ (g) (puisque
là où f et g sont toutes deux continues, \alpha~f + \beta~g l'est également), donc
\mathrmDisc~ (\alpha~f + \beta~g) est
négligeable.

On admettra le théorème suivant

Théorème~20.2.5 Il existe une application qui à toute fonction f bornée
de P dans E dont l'ensemble des points de discontinuité est négligeable
associe un élément de E noté \int  P~f
vérifiant les propriétés suivantes

\begin{itemize}
\itemsep1pt\parskip0pt\parsep0pt
\item
  (i) l'application
  f\mapsto~\int  P~f
  est linéaire (\int  P~(\alpha~f + \beta~g) =
  \alpha~\int  P~f +
  \beta~\int  P~g)
\item
  (ii) \\textbar{}\int ~
  Pf\\textbar{} \leq\int ~
  P\\textbar{}f\\textbar{}
\item
  (iii) \int  P~1 = m(P)
\item
  (iv) si P est la réunion de deux pavés P1 et P2
  dont l'intersection est contenue dans l'intersection des frontières,
  alors \int  P~f
  =\int  P1~f
  +\int  P2~f
\item
  (v) si \x \in
  P∣f(x)\mathrel\neq~0\
  est négligeable, alors \int  P~f = 0.
\end{itemize}

Proposition~20.2.6

\begin{itemize}
\itemsep1pt\parskip0pt\parsep0pt
\item
  (i) Si f : P \rightarrow~ \mathbb{R}~ est une fonction bornée dont l'ensemble des points de
  discontinuité est négligeable et si f est positive, alors
  \int  p~f ≥ 0
\item
  (ii) Si f,g : P \rightarrow~ \mathbb{R}~ sont deux fonctions bornées dont l'ensemble des
  points de discontinuité est négligeable et si f \leq g alors
  \int  P~f \leq\\int
   Pg
\item
  (iii) Si f : P \rightarrow~ \mathbb{R}~ est une fonction bornée dont l'ensemble des points
  de discontinuité est négligeable, alors
  \\textbar{}\int ~
  Pf\\textbar{} \leq
  m(P)supx\inP~\\textbar{}f(x)\\textbar{}.
\end{itemize}

Démonstration (i) On a \int  P~f
=\int ~
P\textbar{}f\textbar{}≥\left
\textbar{}\int  P~f\right
\textbar{} d'où \int  P~f ≥ 0

(ii) On a \int  P~g
-\int  P~f =\\int
 P(g - f) ≥ 0 puisque g - f ≥ 0

(iii) Si M =\
supx\inP\\textbar{}f(x)\\textbar{},
on a \\textbar{}\int ~
Pf\\textbar{} \leq\int ~
P\\textbar{}f\\textbar{}
\leq\int  P~M =
M\int  P~1 = Mm(P)

Définition~20.2.4 Soit f : P \rightarrow~ E une application, soit \sigma une subdivision
de P, (Pi)1\leqi\leqN les sous pavés de la subdivision et
pour chaque i \in {[}1,N{]}, xi un point de Pi~; la
somme S(f,\sigma,x) =\ \\sum
 i=1^Nm(Pi)f(xi) sera appelée une
somme de Riemann associée à la subdivision \sigma et à la famille x =
(xi)1\leqi\leqN.

On admettra également le résultat suivant

Théorème~20.2.7 Soit f une fonction bornée de P dans E dont l'ensemble
des points de discontinuité est négligeable. Alors, pour tout \epsilon
\textgreater{} 0, il existe un réel \eta \textgreater{} 0 tel que pour
toute subdivision \sigma de P de pas plus petit que \eta et pour toute famille x
= (xi) associée, on a
\\textbar{}\int  P~f -
S(f,\sigma,x)\\textbar{} \textless{} \epsilon.

Remarque~20.2.3 Comme dans le cas des fonctions d'une variable, on peut
aussi définir, lorsque f est à valeurs réelles des sommes de Darboux
supérieure et inférieure D(f,\sigma) =\
\sum ~
i=1^Nm(Pi)Mi et
\\sum ~
i=1^Nm(Pi)mi où Mi
= supx\inPi~f(t) et
mi = inf~
x\inPif(t). La même démonstration que pour les fonctions
d'une variable montre alors que ces sommes de Darboux inférieure et
supérieure tendent vers l'intégrale de f sur P lorsque le pas de la
subdivision tend vers 0.

\paragraph{20.2.3 Intégrales multiples sur une partie de \mathbb{R}~^n}

Soit A une partie de \mathbb{R}~^n bornée de frontière négligeable et f
: A \rightarrow~ E continue et bornée. Soit P un pavé de \mathbb{R}~^n contenant A
et f^∗ l'application de P dans E définie par

 f^∗(x) = \left \
\cases f(x)&si x \in A \cr 0 &si x \in A
 \right .

L'ensemble des points de discontinuité de f^∗ est contenu
dans la frontière de A car si x est dans l'intérieur de A, la fonction
f^∗ coïncide avec la fonction continue f sur tout un
voisinage de x, donc est continue au point x et si x appartient à
l'intérieur de P \diagdown A, alors f^∗ coïncide avec la fonction
nulle sur tout un voisinage de x, donc est continue au point x. On peut
donc définir \int  Pf^∗~. De
plus, si P1 et P2 sont deux pavés contenant A, ''la
fonction f^∗'' est nulle sur P2 \diagdown P1 et
sur P1 \diagdown P2, si bien que \\int
 P1f^∗ =\int ~
P2f^∗, ce qui montre que
\int  Pf^∗~ ne dépend pas du
choix du pavé P contenant A.

Définition~20.2.5 Soit A une partie de \mathbb{R}~^n bornée de
frontière négligeable et f : A \rightarrow~ E continue et bornée~; on posera
\int  A~f =\\int
 Pf^∗.

Théorème~20.2.8

\begin{itemize}
\itemsep1pt\parskip0pt\parsep0pt
\item
  (i) L'application
  f\mapsto~\int  A~f
  est linéaire (\int  A~(\alpha~f + \beta~g) =
  \alpha~\int  A~f +
  \beta~\int  A~g)
\item
  (ii) \\textbar{}\int ~
  Af\\textbar{} \leq\int ~
  A\\textbar{}f\\textbar{}
\item
  (iii) si A1 \bigcap A2 est négligeable, alors
  \int  A1\cupA2~f
  =\int  A1~f
  +\int  A2~f
\item
  (iv) si A est négligeable, alors \int ~
  Af = 0.
\end{itemize}

Démonstration (i) et (ii) résultent de (\alpha~f + \beta~g)^∗ =
\alpha~f^∗ + \beta~g^∗ et de
\\textbar{}f^∗\\textbar{}
=\\textbar{} f\\textbar{}^∗ qui
sont évidents. Pour (iii) si on considère P un pavé contenant
A1 \cup A2, f^∗ l'extension de f de
A1 \cup A2 à P, f1^∗ et
f2^∗ les extensions de f depuis respectivement
A1 et A2 à P, on a f^∗ =
f1^∗ + f2^∗ sauf sur A1 \bigcap
A2. On a donc f^∗ = f1^∗ +
f2^∗ + g où g est une fonction nulle sauf sur
l'ensemble négligeable A1 \bigcap A2~; on en déduit que

\begin{align*} \int ~
A1\cupA2f& =& \int ~
Pf^∗ =\int ~
P(f1^∗ + f 2^∗ + g) \%&
\\ & =& \int ~
Pf1^∗ +\int ~
Pf2^∗ +\int ~
Pg =\int ~
Pf1^∗ +\int ~
Pf2^∗\%& \\ &
=& \int  A1~f
+\int  A2~f \%&
\\ \end{align*}

Quand à (iv), il est évident puisque \x \in
P∣f^∗(x)\mathrel\neq~0\
\subset~ A

\paragraph{20.2.4 Mesure d'un sous-ensemble borné de \mathbb{R}~^n}

Définition~20.2.6 On dit qu'une partie A de \mathbb{R}~^n est quarrable
si elle est bornée et de frontière négligeable.

Définition~20.2.7 Soit A une partie quarrable de \mathbb{R}~^n~; on
appelle mesure de A le nombre réel positif m(A)
=\int  A~1.

Proposition~20.2.9 Soit A une partie quarrable de \mathbb{R}~^n~; alors
l'intérieur et l'adhérence de A sont aussi quarrables et ont la même
mesure.

Démonstration En effet, la frontière de l'intérieur et de l'adhérence de
A sont contenues dans la frontière de A qui est négligeable. De plus
\int  \overlineA~1
=\int  \overlineA\diagdownA~1
+\int  A~1 =\\int
 A1 puisque \overlineA \diagdown A est contenu dans
la frontière de A et donc est négligeable~; la démonstration est
similaire pour l'intérieur.

Proposition~20.2.10 Si f : A \rightarrow~ E est une fonction continue bornée sur
l'ensemble quarrable A, alors
\\textbar{}\int ~
Af\\textbar{} \leq
m(A)supx\inA~\\textbar{}f(x)\\textbar{}.

Démonstration Si M =\
supx\inA\\textbar{}f(x)\\textbar{},
on a \\textbar{}\int ~
Af\\textbar{} \leq\int ~
A\\textbar{}f\\textbar{}
\leq\int  A~M =
M\int  A~1 = Mm(A)

{[}
{[}
{[}
{[}
