\textbf{Warning: 
requires JavaScript to process the mathematics on this page.\\ If your
browser supports JavaScript, be sure it is enabled.}

\begin{center}\rule{3in}{0.4pt}\end{center}

{[}
{[}
{[}{]}
{[}

\subsubsection{14.2 Séries trigonométriques}

\paragraph{14.2.1 Rappels d'intégration}

Lemme~14.2.1 Soit f : \mathbb{R}~ \rightarrow~ \mathbb{C} périodique de période T, continue par
morceaux. Alors, pour tout a \in \mathbb{R}~, \int ~
a^a+Tf(t) dt =\int ~
0^Tf(t) dt

Démonstration On écrit \int ~
a^a+Tf(t) dt =\int ~
a^0f(t) dt+\int ~
0^Tf(t) dt+\int ~
T^a+Tf(t) dt =\int ~
a^0f(t) dt+\int ~
0^Tf(t) dt+\int ~
0^af(u+T) du = \int ~
a^0f(t) dt +\int ~
0^Tf(t) dt +\int ~
0^af(u) du =\int ~
0^Tf(t) dt en faisant le changement de variable u = t -
T.

Lemme~14.2.2 Pour tout n \in ℤ, \int ~
0^2\pi~e^int dt = 2\pi~\deltan^0.

\paragraph{14.2.2 Généralités}

Définition~14.2.1 (forme réelle). Soit (an)n≥0 et
(bn)n≥1 deux suites de nombres complexes. On appelle
série trigonométrique associée la série de fonctions de \mathbb{R}~ dans \mathbb{C},

a0 + \sum n≥1(an~
\cos nx + bn \sin nx)

Remarque~14.2.1 Soit n \in \mathbb{N}~^∗ et an et bn
deux nombres complexes. On a alors an\
cos nx + bn sin~ nx =
cne^inx + c-ne^-inx avec
cn = an-ibn \over 2 et
c-n = an+ibn \over 2 .
Inversement, si on se donne deux nombres complexes cn et
c-n, on a cne^inx +
c-ne^-inx = an\
sin nx + bn cos~ nx avec
an = cn + c-n et bn =
i(cn - c-n). Ceci amène également à poser

Définition~14.2.2 (forme complexe). Soit (cn)n\inℤ une
suite de nombres complexes. On appelle série trigonométrique associée la
série de fonctions de \mathbb{R}~ dans \mathbb{C},

c0 + \\sum
n≥1(cne^inx + c
-ne^-inx)

On passe donc de la forme réelle à la forme complexe ou vice versa par
les formules

\begin{align*} a0& =& c0 \%&
\\ \forall~~n ≥
1,\quad cn& =& an - ibn
\over 2 ,\quad c-n =
an + ibn \over 2 \%&
\\ \forall~~n ≥
1,\quad an& =& cn +
c-n,\quad bn = i(cn -
c-n)\%& \\
\end{align*}

\paragraph{14.2.3 Un cas de convergence normale}

Théorème~14.2.3 On considère une série trigonométrique vérifiant les
conditions équivalentes

\begin{itemize}
\itemsep1pt\parskip0pt\parsep0pt
\item
  (i) les deux séries \\\sum
   \textbar{}an\textbar{} et
  \\sum ~
  \textbar{}bn\textbar{} sont convergentes.
\item
  (ii) les deux séries
  \\sum ~
  n≥0\textbar{}cn\textbar{} et
  \\sum ~
  n≥0\textbar{}c-n\textbar{} sont convergentes.
\end{itemize}

Alors la série trigonométrique converge normalement sur \mathbb{R}~, sa somme f
est une fonction continue périodique de période 2\pi~ et on a

\begin{align*} \forall~~n \in
ℤ,\quad cn& =& 1 \over 2\pi~
\int  0^2\pi~f(t)e^-int~
dt \%& \\ \forall~~n ≥
1,\quad an& =& 1 \over \pi~
\int ~
0^2\pi~f(t)cos~ nt
dt,\quad b n = 1 \over \pi~
\int ~
0^2\pi~f(t)sin~ nt dt\%&
\\ \end{align*}

Démonstration Les relations
\textbar{}an\textbar{}\leq\textbar{}cn\textbar{} +
\textbar{}c-n\textbar{},
\textbar{}bn\textbar{}\leq\textbar{}cn\textbar{} +
\textbar{}c-n\textbar{}, \textbar{}cn\textbar{}\leq 1
\over 2 (\textbar{}an\textbar{} +
\textbar{}bn\textbar{}) et
\textbar{}c-n\textbar{}\leq 1 \over 2
(\textbar{}an\textbar{} + \textbar{}bn\textbar{})
(que l'on déduit facilement des relations du paragraphe précédent)
montrent clairement l'équivalence. Alors on a

\forall~x \in \mathbb{R}~, \textbar{}cne^inx~
+ c -ne^-inx\textbar{}\leq\textbar{}c
n\textbar{} + \textbar{}c-n\textbar{}

qui est une série convergente indépendante de x. On a donc la
convergence normale de la série et en particulier la continuité de sa
somme. Cette somme est évidemment périodique de période 2\pi~ puisque
toutes les applications
x\mapsto~cne^inx +
c-ne^-inx le sont. Soit p \in ℤ. On a aussi
\forall~~x \in \mathbb{R}~,
\textbar{}(cne^inx +
c-ne^-inx)e^-ipx\textbar{}\leq\textbar{}cn\textbar{}
+ \textbar{}c-n\textbar{} ce qui montre que la série
c0e^-ipx +\
\sum  n≥1(cne^inx~
+ c-ne^-inx)e^-ipx converge normalement
sur \mathbb{R}~, donc sur {[}0,2\pi~{]}. Ceci \\\\jmathmathmathmathustifie donc dans le calcul suivant
l'interversion du signe d'intégrale et du signe somme

\begin{align*} \int ~
0^2\pi~f(t)e^-ipt dt&& \%&
\\ & =& \int ~
0^2\pi~\left (c 0e^-ipt
+ \sum n≥1(cne^int~
+ c -ne^-int)e^-ipt\right
) dt \%& \\ & =&
c0\int ~
0^2\pi~e^-ipt dt \%&
\\ & \text &
+\sum n=1^+\infty~~\left
(c n \\int  ~
0^2\pi~e^i(n-p)t dt + c -n
\\int  ~
0^2\pi~e^-i(n+p)t dt\right )\%&
\\ & =& 2\pi~\left
(c0\deltap^0 + \\sum
n=1^+\infty~(c n\deltap^n + c
-n\deltap^-n)\right ) = 2\pi~c p
\%& \\ \end{align*}

en distinguant les différents cas possibles p = 0, p ≥ 1 ou p \leq-1. Les
relations sur les an et bn s'en déduisent facilement
par les formules du premier paragraphe.

Remarque~14.2.2 La même technique permet d'aboutir aux mêmes formules
dès que la série trigonométrique converge uniformément sur un segment de
longueur 2\pi~.

{[}
{[}
{[}
{[}
