\begin{prop}
Soit E un K-espace vectoriel ~de dimension finie et $\Phi$
une forme quadratique sur E de forme polaire $\phi$. Soit $\mathcal{E}$ une base de E.
Les propositions suivantes sont �quivalentes :

\begin{itemize}

\item $\mathcal{E}$ est une base orthogonale (resp. orthonorm�e) 
\item $\mathrm{Mat}(\phi,\mathcal{E}) $ est
  diagonale (resp. la matrice identite )
\item $\forall x \in E, \Phi(x) =
  \sum  \alpha_i x_i^2 ~~
  (resp. \Phi(x) = \sum ~
  x_i^2) si~ x =
  \sum  x_ie_i~.$

\end{itemize}

\end{prop}
\begin{thm}
Soit E un K-espace vectoriel  de dimension finie et $\Phi$
une forme quadratique sur E de forme polaire $\phi$. Alors il existe des
bases de E orthogonales pour $\phi$.

\end{thm}
\begin{cor}
  Soit $A \in M_K(n)$ une matrice sym�trique. Alors
il existe une matrice inversible $P$ telle que $^tPAP$ soit
diagonale.

\end{cor}