\textbf{Warning: 
requires JavaScript to process the mathematics on this page.\\ If your
browser supports JavaScript, be sure it is enabled.}

\begin{center}\rule{3in}{0.4pt}\end{center}

{[}
{[}
{[}{]}
{[}

\subsubsection{20.3 Calcul des intégrales doubles et triples}

\paragraph{20.3.1 Théorème de Fubini sur une partie de \mathbb{R}~^2}

Théorème~20.3.1 (de Fubini pour un pavé de \mathbb{R}~^2). Soit P =
{[}a,b{]} \times {[}c,d{]} un pavé de \mathbb{R}~^2, E un espace vectoriel
normé de dimension finie, f : P \rightarrow~ E une fonction bornée dont l'ensemble
des points de discontinuité est négligeable vérifiant (i) pour chaque x
\in {[}a,b{]}, l'application y\mapsto~f(x,y) est
réglée de {[}c,d{]} dans E (ii) l'application \phi : {[}a,b{]} \rightarrow~ E,
x\mapsto~\int ~
c^df(x,y) dy est réglée. Alors

\int  P~f =\\int
 a^b\phi(x) dx =\int ~
a^b\left (\int ~
c^df(x,y) dy\right ) dx

Démonstration Quitte à prendre une base de E, il suffit de montrer le
résultat lorsque E = \mathbb{R}~. Soit n \in \mathbb{N}~, ai = a + i b-a
\over n et c\\\\jmathmathmathmath = c + \\\\jmathmathmathmath d-c
\over n . On définit ainsi une subdivision \sigman
de P dont le pas tend vers 0 quand n tend vers + \infty~. Soit Pi,\\\\jmathmathmathmath
= {[}ai-1,ai{]} \times {[}c\\\\jmathmathmathmath-1,c\\\\jmathmathmathmath{]}
les pavés de cette subdivision. Posons alors, pour (i,\\\\jmathmathmathmath) \in
{[}1,n{]}^2,

\mui,\\\\jmathmathmathmath = 1 \over c\\\\jmathmathmathmath -
c\\\\jmathmathmathmath-1 \int ~
c\\\\jmathmathmathmath-1^c\\\\jmathmathmathmath f(ai,y) dy = n
\over d - c \int ~
c\\\\jmathmathmathmath-1^c\\\\jmathmathmathmath f(ai,y) dy

et soit Sn =\
\sum  i,\\\\jmathmathmathmath\in{[}1,n{]}~ (b-a)(d-c)
\over n^2 \mui,\\\\jmathmathmathmath. Si l'on note
Mi,\\\\jmathmathmathmath =\
supx\inPi,\\\\jmathmathmathmathf(x) et mi,\\\\jmathmathmathmath
= inf x\inPi,\\\\jmathmathmathmath~f(x), on a
clairement

\begin{align*} d(f,\sigman)& =&
\sum i,\\\\jmathmathmathmath\in{[}1,n{]}~ (b - a)(d - c)
\over n^2 mi,\\\\jmathmathmathmath \%&
\\ & \leq& \\sum
i,\\\\jmathmathmathmath\in{[}1,n{]} (b - a)(d - c) \over
n^2 \mui,\\\\jmathmathmathmath \%& \\ &
\leq& \sum i,\\\\jmathmathmathmath\in{[}1,n{]}~ (b - a)(d - c)
\over n^2 Mi,\\\\jmathmathmathmath =
D(f,\sigman)\%& \\
\end{align*}

où d(f,\sigman) et D(f,\sigman) désignent respectivement les
sommes de Darboux inférieure et supérieure de f associées à la
subdivision \sigman. Quand n tend vers + \infty~, ces sommes de Darboux
admettent toutes deux pour limite \int ~
Pf, donc limn\rightarrow~+\infty~Sn~
=\int  P~f. Mais d'autre part,

\begin{align*} Sn& =&
\sum i=1^n~ b - a
\over n \\sum
\\\\jmathmathmathmath=1^n
\\int  ~
c\\\\jmathmathmathmath-1^c\\\\jmathmathmathmath f(ai,y) dy\%&
\\ & =& \\sum
i=1^n b - a \over n
\\int  ~
c^df(a i,y) dy \%&
\\ & =& \\sum
i=1^n b - a \over n \phi(ai)
\%& \\ \end{align*}

Il s'agit donc d'une somme de Riemann pour la fonction réglée \phi :
{[}a,b{]} \rightarrow~ \mathbb{R}~ associée à la subdivision régulière
(ai)0\leqi\leqn. On en déduit que
limn\rightarrow~+\infty~Sn~
=\int  a^b~\phi(x) dx, d'où
l'égalité recherchée.

Remarque~20.3.1 De même si on suppose que

\begin{itemize}
\itemsep1pt\parskip0pt\parsep0pt
\item
  (i) pour chaque y \in {[}c,d{]}, l'application
  x\mapsto~f(x,y) est réglée de {[}a,b{]} dans E
\item
  (ii) l'application \psi : {[}c,d{]} \rightarrow~ E,
  y\mapsto~\int ~
  a^bf(x,y) dx est réglée.
\end{itemize}

Alors

\int  P~f =\\int
 c^d\psi(y) dy =\int ~
c^d\left (\int ~
a^bf(x,y) dx\right ) dy

Tout ceci nous amène tout naturellement à introduire la notation
\int  \\int ~
Pf(x,y) dx dy pour l'intégrale d'une fonction f sur un pavé P de
\mathbb{R}~^2 puis à généraliser cette notation à toute partie
quarrable de \mathbb{R}~^2.

Corollaire~20.3.2 Soit P = {[}a,b{]} \times {[}c,d{]} un pavé de
\mathbb{R}~^2, f : {[}a,b{]} \rightarrow~ K, g : {[}c,d{]} \rightarrow~ K réglées. Alors

\int  \\int ~
{[}a,b{]}\times{[}c,d{]}f(x)g(y) dx dy = \left
(\int  a^b~f(x)
dx\right )\left
(\int  c^d~g(y)
dy\right )

Démonstration On a

\int  c^d~f(x)g(y) dy =
f(x)\int  c^d~g(y) dy = \lambda~f(x)

avec \lambda~ =\int  c^d~g(y) dy. On en
déduit que

\begin{align*} \int ~
\int  {[}a,b{]}\times{[}c,d{]}~f(x)g(y) dx
dy& =& \int  a^b~\lambda~f(x) dx =
\lambda~\int  a^b~f(x) dx\%&
\\ & =& \left
(\int  a^b~f(x)
dx\right )\left
(\int  c^d~g(y)
dy\right ) \%& \\
\end{align*}

Théorème~20.3.3 (théorème de Fubini pour une partie de \mathbb{R}~^2).
Soit \phi1 et \phi2 deux applications continues de
{[}a,b{]} dans \mathbb{R}~ vérifiant \forall~~t \in {[}a,b{]},
\phi1(t) \leq \phi2(t) et soit A = \(x,y) \in
\mathbb{R}~^2∣x \in
{[}a,b{]}\text et \phi1(x) \leq y \leq
\phi2(x)\. Soit f : A \rightarrow~ E continue. Alors A est
quarrable et

\int  \\int ~
Af(x,y) dx dy =\int ~
a^b\left (\int ~
\phi1(x)^\phi2(x)f(x,y)
dy\right ) dx

Démonstration A est quarrable car il est borné (\phi1 et
\phi2 continues sur des compacts sont bornées) et sa frontière
est formée de la réunion de quatre graphes de fonctions continues
x\mapsto~\phi1(x),
y\mapsto~b,
x\mapsto~\phi2(x) et
y\mapsto~a. Soit M =\
sup\phi2(x) et m = inf~
\phi1(x) si bien que A \subset~ {[}a,b{]} \times {[}m,M{]} et soit
f^∗ le prolongement par 0 de f de A à P = {[}a,b{]} \times
{[}m,M{]}. On sait dé\\\\jmathmathmathmathà que f^∗ est bornée et que
\mathrmDisc~ (f) est
négligeable (il est contenu dans la frontière de A). Pour x \in {[}a,b{]}
fixé, l'application y\mapsto~f^∗(x,y) est
continue par morceaux car f^∗(x,y) = \left
\ \cases f(x,y)&si \phi1(x) \leq
y \leq \phi2(x) \cr 0 &sinon 
\right .. De plus \int ~
m^Mf^∗(x,y) dy =\int ~
\phi1(x)^\phi2(x)f(x,y) dy est une fonction
continue de x comme on l'a vu dans le chapitre sur les intégrales de
fonctions d'une variable. On peut donc appliquer le théorème précédent
et on obtient

\begin{align*} \int ~
\int  A~f(x,y) dx dy& =&
\int  \\int ~
Pf^∗(x,y) dx dy \%& \\
& =& \int ~
a^b\left (\int ~
m^Mf^∗(x,y) dy\right ) dx \%&
\\ & =& \int ~
a^b\left (\int ~
\phi1(x)^\phi2(x)f(x,y)
dy\right ) dx\%& \\
\end{align*}

\includegraphics{cours16x.png}

Remarque~20.3.2 Ceci permet de ramener le calcul d'une intégrale double
sur une partie de \mathbb{R}~^2 délimitée par deux graphes au calcul de
deux intégrales de fonctions d'une variable. Pour un sous ensemble A
plus général, on cherchera à écrire A = A1
\cup\\ldots~ \cup
Ak où chaque partie Ai est limitée par deux graphes
de fonctions continues, soit de la forme \(x,y) \in
\mathbb{R}~^2∣x \in
{[}a,b{]}\text et \phi1(x) \leq y \leq
\phi2(x)\ ou de la forme \(x,y)
\in \mathbb{R}~^2∣y \in
{[}c,d{]}\text et \psi1(y) \leq x \leq
\psi2(y)\~; on écrira alors, à condition que les
intersections deux à deux des Ai soient négligeables,
\int  \\int ~
Af(x,y) dx dy =\
\sum ~
i=1^k\int ~
\int  Ai~f(x,y) dx dy puis on
ramènera le calcul de chaque intégrale double au calcul de deux
intégrales simples.

Exemple~20.3.1 Soit K = \(x,y) \in
\mathbb{R}~^2∣y ≥ 0, y \leq x,0 \leq x + 2y \leq
2\ et on cherche à calculer I
=\int  \\int ~
Kx^2 dx dy (moment d'inertie par rapport à l'axe Oy).

\includegraphics{cours17x.png}

Le théorème de Fubini nous permet de calculer cette intégrale de deux
manières différentes suivant que l'on commence à intégrer suivant x ou
suivant y. Dans une première méthode on peut écrire

\begin{align*} K& =& \(x,y) \in
\mathbb{R}~^2∣x \in {[}0, 2
\over 3 {]}\text et 0 \leq y \leq
x\ \%& \\ & \cup&
\(x,y) \in \mathbb{R}~^2∣x \in
{[} 2 \over 3 ,2{]}\text et 0 \leq y
\leq 1 - x \over 2 \\%&
\\ \end{align*}

d'où (en faisant sortir de l'intégrale par rapport à y le terme en
x^2 qui ne dépend pas de y)

\begin{align*} I& =& \\int
 0^2\diagup3x^2\left
(\int ~
0^xdy\right ) dx
+\int ~
2\diagup3^2x^2\left
(\int  0~^1- x
\over 2 dy\right ) dx\%&
\\ & =& \int ~
0^2\diagup3x^3 dx +\int ~
2\diagup3^2x^2\left (1 - x
\over 2 \right ) dx = 52
\over 81 \%& \\
\end{align*}

On peut aussi la calculer en considérant que

K = \(x,y) \in
\mathbb{R}~^2∣y \in {[}0, 2
\over 3 {]}\text et y \leq x \leq 2 -
2y\

d'où

I =\int ~
0^2\diagup3\left (\int ~
y^2-2yx^2 dx\right ) dy
=\int  0^2\diagup3~\left
( (2 - 2y)^3 \over 3 - y^3
\over 3 \right ) dy = 52
\over 81

\paragraph{20.3.2 Théorème de Fubini sur une partie de \mathbb{R}~^3}

On démontre d'une fa\ccon similaire le résultat
suivant pour les intégrales sur une partie de \mathbb{R}~^3

Théorème~20.3.4 (théorème de Fubini pour un pavé de \mathbb{R}~^3).
Soit P = {[}a1,b1{]} \times
{[}a2,b2{]} \times {[}a3,b3{]} un
pavé de \mathbb{R}~^3, E un espace vectoriel normé de dimension finie,
f : P \rightarrow~ E une fonction bornée dont l'ensemble des points de
discontinuité est négligeable vérifiant

\begin{itemize}
\item
  (i) pour chaque (x,y) \in {[}a1,b1{]} \times
  {[}a2,b2{]}, l'application
  z\mapsto~f(x,y,z) est réglée de {[}c,d{]} dans E
\item
  (ii) pour chaque x \in {[}a1,b1{]}, l'application
  y\mapsto~\int ~
  a3^b3f(x,y,z) dz est réglée de
  {[}a2,b2{]} dans E
\item
  (iii) l'application
  x\mapsto~\int ~
  a2^b2\left
  (\int ~
  a3^b3f(x,y,z)
  dz\right ) dy est réglée de
  {[}a1,b1{]} dans E
\item
  Alors

  \begin{align*} \int ~
  \int  \\int ~
  Pf(x,y,z) dx dy dz =& & \%&
  \\ & & \int ~
  a1^b1 \left
  (\int  a2^b2~
  \left (\int ~
  a3^b3 f(x,y,z)
  dz\right ) dy\right ) dx\%&
  \\ \end{align*}
\end{itemize}

Bien entendu, dans la limite de validité (c'est-à-dire si l'on est
certain que toutes les intégrales écrites ont bien un sens), on peut
regrouper deux des intégrales simples en une intégrale double en
utilisant le théorème de Fubini pour les intégrales doubles, et ainsi
obtenir les formules (à permutation près sur les noms des variables)

\begin{align*} \int ~
\int  \\int ~
Pf(x,y,z) dx dy dz&& \%& \\ &
=& \int  \\int ~
{[}a1,b1{]}\times{[}a2,b2{]}\left
(\int  a3^b3~
f(x,y,z) dz\right ) dx dy\%&
\\ & =& \int ~
a3^b3 \left
(\int  \\int ~
{[}a1,b1{]}\times{[}a2,b2{]}f(x,y,z)
dx dy\right ) dz\%& \\
\end{align*}

La première méthode d'intégration porte en général le nom d'intégration
par piles (on somme d'abord verticalement, puis ensuite
horizontalement), la deuxième méthode portant le nom d'intégration par
tranches (on somme d'abord horizontalement, puis ensuite verticalement).

De la même manière que pour les intégrales doubles et par prolongement
par 0 à un pavé, on montre alors les deux résultats suivants

Théorème~20.3.5 (théorème de Fubini pour une partie de \mathbb{R}~^3,
intégration par piles). Soit A un sous-ensemble quarrable de
\mathbb{R}~^2, \phi1 et \phi2 deux applications continues
de A dans \mathbb{R}~ vérifiant \forall~~(x,y) \in A,
\phi1(x,y) \leq \phi2(x,y) et soit K =
\(x,y,z) \in
\mathbb{R}~^2∣(x,y) \in
A\text et \phi1(x,y) \leq z \leq
\phi2(x,y)\. Soit f : K \rightarrow~ E continue. Alors

\int  \\int ~
\int  K~f(x,y,z) dx dy dz
=\int  \\int ~
A\left (\int ~
\phi1(x,y)^\phi2(x,y)f(x,y,z)
dz\right ) dx dy

Théorème~20.3.6 (théorème de Fubini pour une partie de \mathbb{R}~^3,
intégration par tranches). Soit K une partie de \mathbb{R}~^3 compacte
et quarrable. Soit f une application continue de K dans E. On suppose
vérifiées les conditions suivantes (en posant m
= inf (x,y,z)\inK~z et M
= sup(x,y,z)\inK~z) (i) pour tout z \in
{[}m,M{]}, Kz = \(x,y) \in
\mathbb{R}~^2∣(x,y,z) \in K\
(section de K par le plan horizontal de cote z) est un sous-ensemble
quarrable de \mathbb{R}~^2 (ii) l'application
z\mapsto~\int ~
\int  Kz~f(x,y,z) dx dy est
réglée sur {[}m,M{]}. Alors

\int  \\int ~
\int  K~f(x,y,z) dx dy dz
=\int  m^M~\left
(\int  \\int ~
Kzf(x,y,z) dx dy\right ) dz

Exemple~20.3.2 Supposons donnée une courbe dans un plan méridien donnée
en coordonnées cylindriques par r = \phi(z) où \phi : {[}a,b{]} \rightarrow~ \mathbb{R}~ est
continue positive. Considérons le volume K de révolution délimité par la
rotation de la courbe autour de l'axe Oz. Les sous-ensembles
Kz sont des disques de centre (0,0) de rayon \phi(z), donc de
mesure \pi~\phi(z)^2. On en déduit que la mesure de K est donnée
par

\begin{align*} m(K)& =& \\int
 \int  \\int  K~
dx dy dz =\int ~
a^b\left (\int ~
\int  Kz~ dx
dy\right ) dz =\int ~
a^bm(K z) dz\%&
\\ & =& \pi~\int ~
a^b\phi(z)^2 dz \%&
\\ \end{align*}

Par exemple, pour une boule de rayon R, on peut prendre a = -R, b = R et
\phi(z) = \sqrtR^2  - z^2, d'où la
mesure de la boule

m(K) = \pi~\int ~
-R^R(R^2 - z^2) dz =
\pi~\left {[}R^2z - z^3
\over 3 \right {]}-R^R
= 4 \over 3 \pi~R^3

\paragraph{20.3.3 Théorème de changement de variables dans les
intégrales multiples}

On admettra le théorème suivant de démonstration difficile

Théorème~20.3.7 (théorème de changement de variables). Soit K1
et K2 deux parties compactes de \mathbb{R}~^n de frontières
négligeables, \phi : K1 \rightarrow~ K2 continue, E un espace
vectoriel normé de dimension finie. On suppose que \phi réalise un
\mathcal{C}^1 difféomorphisme de l'intérieur de K1 sur
l'intérieur de K2. Soit f : K2 \rightarrow~ E continue. Alors
(si \\\\jmathmathmathmath\phi(x) désigne le \\\\jmathmathmathmathacobien de \phi au point x \in
K1^o)

\int  K2~f
=\int  K1~f \cdot \phi
\textbar{}\\\\jmathmathmathmath\phi\textbar{}

En particulier on a les formules suivantes pour les intégrales doubles
et triples

\int  \\int ~
K2f(x,y) dx dy =\int ~
\int  K1~f(\phi(u,v))
\textbar{}\\\\jmathmathmathmath\phi(u,v)\textbar{} du dv

\begin{align*} \int ~
\int  \\int ~
K2f(x,y,z) dx dy dz&& \%&
\\ & =& \int ~
\int  \\int ~
K1f(\phi(u,v,w)) \textbar{}\\\\jmathmathmathmath\phi(u,v,w)\textbar{}
du dv dw\%& \\
\end{align*}

Remarque~20.3.3 Le lecteur comparera ce théorème de changement de
variables avec le théorème de changement de variable pour les fonctions
d'une variable. Le \\\\jmathmathmathmathacobien \\\\jmathmathmathmathoue ici le rôle du terme \phi'(u). On prendra
garde qu'ici il est assorti d'une valeur absolue. Ceci est dû à
l'absence de convention de Chasles à partir de la dimension 2. En
dimension 1 et lorsque \phi est décroissante (donc \phi' \leq 0), les bornes se
retrouvent en sens contraire de l'ordre naturel et un rétablissement de
cet ordre transforme alors \phi' en - \phi' = \textbar{}\phi'\textbar{}.

Corollaire~20.3.8 En supposant vérifiées les hypothèses ci dessus pour
le changement de variable, on a les formules suivantes pour les passages
en coordonnées polaires, cylindriques ou sphériques

\begin{align*} \int ~
\int  K2~f(x,y) dx dy& =&
\int  \\int ~
K1f(rcos~
\theta,rsin~ \theta) \textbar{}r\textbar{} dr d\theta\%&
\\ & & \%&
\\ \end{align*}

\begin{align*} \int ~
\int  \\int ~
K2f(x,y,z) dx dy dz&& \%&
\\ & =& \int ~
\int  \\int ~
K1f(rcos~
\theta,rsin~ \theta,z) \textbar{}r\textbar{} dr d\theta dz
\%& \\ \int ~
\int  \\int ~
K2f(x,y,z) dx dy dz&& \%&
\\ & =& \int ~
\int  \\int ~
K1f(rcos~
\thetacos \phi,r\sin~
\thetacos \phi,r\sin~ \phi)
\textbar{}r^2 cos~ \phi\textbar{} dr d\theta
d\phi\%& \\ \end{align*}

Remarque~20.3.4 Le lecteur devra se persuader que le principal obstacle
au calcul explicite d'une intégrale multiple provient du domaine
d'intégration et non de la fonction à intégrer (penser par exemple
qu'une aire ou un volume peuvent être difficiles à calculer alors que la
fonction à intégrer est la constante 1). Ceci veut dire que lorsque l'on
recherche un changement de variable, on doit accorder une priorité
absolue à la simplification du domaine d'intégration, l'idéal étant de
transformer ce domaine en un pavé.

Exemple~20.3.3 Soit
(v1,\\ldots,vn~)
une famille libre de \mathbb{R}~^n et soit V le polytope construit sur
cette base, c'est-à-dire V = \t1v1
+ \\ldots~ +
tnvn∣\forall~~i
\in {[}1,n{]}, ti \in {[}0,1{]}\ (en dimension 2,
il s'agit d'un parallélogramme et en dimension 3 d'un parallélépipède).
Alors l'application \phi : {[}0,1{]}^n \rightarrow~ V ,
(t1,\\ldots,tn)\mapsto~t1v1~
+ \\ldots~ +
tnvn vérifie évidemment les conditions du théorème
de changement de variable. De plus son \\\\jmathmathmathmathacobien est égal au produit
mixte des n vecteurs, soit
{[}v1,\\ldots,vn~{]}.
On en déduit que la mesure de V est donnée par

\begin{align*} m(V )& =&
\int  V ~1 =\\int

{[}0,1{]}^n\textbar{}{[}v1,\\ldots,vn~{]}\textbar{}
\%& \\ & =&
\textbar{}{[}v1,\\ldots,vn{]}\textbar{}m({[}0,1{]}^n~)
= \textbar{}{[}v
1,\\ldots,vn~{]}\textbar{}\%&
\\ \end{align*}

\includegraphics{cours18x.png}

Exemple~20.3.4 On considère deux paraboles d'axe Ox tangentes en O à
l'axe Oy et deux paraboles d'axe Oy tangentes en O à l'axe Ox. On
cherche à calculer l'aire du domaine K compris entre les paraboles
(hachuré sur le dessin ci dessous)

\includegraphics{cours19x.png}

Les paraboles auront pour équations x^2 = 2p1y et
x^2 = 2p2y pour les paraboles verticales,
y^2 = 2q1x et y^2 = 2q2x pour
les paraboles horizontales. Il n'est pas raisonnable de tenter un calcul
par le théorème de Fubini. Nous allons donc faire un changement de
variable en paramétrant un point de la zone hachuré~; pour cela nous
considérerons que tout point de la zone hachurée est l'intersection
d'une parabole x^2 = 2py et d'une parabole y^2 =
2qx avec p \in {[}p1,p2{]} et q \in
{[}q1,q2{]}, autrement dit nous considérerons
l'application \phi : K \rightarrow~ {[}p1,p2{]} \times
{[}q1,q2{]} définie par \phi(x,y) = ( x^2
\over 2y , y^2 \over 2x
). Il est visible que \phi est bi\\\\jmathmathmathmathective, ce que confirmerait un calcul
simple. De plus

\\\\jmathmathmathmath\phi(x,y) = \left
\textbar{}\matrix\, x
\over y &- x^2 \over
2y^2 \cr - y^2
\over 2x^2 & y \over x
\right \textbar{} = 3 \over 4

On en déduit par le théorème d'inversion locale que \phi est un
\mathcal{C}^1 difféomorphisme de K^o sur
{]}p1,p2{[}\times{]}q1,q2{[} et que
\\\\jmathmathmathmath\phi^-1(p,q) = 1 \over
\\\\jmathmathmathmath\phi(\phi^-1(p,q)) = 4 \over 3 . On
a donc

\begin{align*} m(K)& =& \\int
 \int  K~1 dx dy \%&
\\ & =& \int ~
\int ~
{[}p1,p2{]}\times{[}q1,q2{]}1
\cdot \phi^-1(p,q)\left \textbar{}\\\\jmathmathmathmath
\phi^-1(p,q)\right \textbar{} dp dq\%&
\\ & =& 4 \over 3
m({[}p1,p2{]} \times {[}q1,q2{]}) =
4 \over 3 (p2 - p1)(q2 -
q1) \%& \\
\end{align*}

\paragraph{20.3.4 Théorème de Green-Riemann}

Soit \phi1 et \phi2 deux applications continues de
{[}a,b{]} dans \mathbb{R}~ vérifiant \forall~~x \in {[}a,b{]},
\phi1(x) \leq \phi2(x) et soit A = \(x,y) \in
\mathbb{R}~^2∣x \in
{[}a,b{]}\text et \phi1(x) \leq y \leq
\phi2(x)\. On considère la frontière \partial~A de A en
tant qu'arc paramétré orientée comme l'indique la figure ci dessous~: on
parcourt la frontière en laissant A à sa main gauche. Cette frontière
est la réunion de quatre arcs paramétrés, deux étant des graphes de
fonctions x\mapsto~y = \phii(x) (l'un
parcouru dans le sens direct, l'autre dans le sens indirect), deux étant
des graphes de fonctions
y\mapsto~\textconstante.

\includegraphics{cours20x.png}

Soit U un ouvert contenant A et P une fonction de classe \mathcal{C}^1
sur U. On cherche à calculer l'intégrale I =\int ~
\int  A \partial~P \over \partial~y~
(x,y) dx dy. D'après le théorème de Fubini, on a

\begin{align*} \int ~
\int  A \partial~P \over \partial~y~
(x,y) dx dy &=&\int ~
a^b\left (\int ~
\phi1(x)^\phi2(x) \partial~P \over
\partial~y (x,y) dy\right ) dx&&\%&
\\ & =& \int ~
a^b\Big {[}P(x,y)\Big
{]} y=\phi1(x)^y=\phi2(x) dx \%&
\\ & =& \int ~
a^bP(x,\phi 2(x)) dx -\\int
 a^bP(x,\phi 1(x)) dx\%&
\\ \end{align*}

La première intégrale \int ~
a^bP(x,\phi2(x)) dx n'est autre que l'intégrale
curviligne de la forme différentielle P(x,y) dx le long du graphe y =
\phi2(x), c'est-à-dire l'opposée de l'intégrale curviligne de la
forme différentielle P(x,y) dx le long du quart supérieur de la
frontière \partial~A (un changement d'orientation changeant l'intégrale
curviligne en son opposée). La deuxième intégrale
-\int  a^bP(x,\phi1~(x))
dx n'est autre que l'opposée de l'intégrale curviligne de la forme
différentielle P(x,y) dx le long du graphe y = \phi1(x),
c'est-à-dire l'opposée de l'intégrale curviligne de la forme
différentielle P(x,y) dx le long du quart inférieur de la frontière \partial~A .
Mais d'autre part les intégrales curvilignes de la forme différentielle
P(x,y) dx le long des quarts gauche et droite de la frontière sont
nulles, car sur ces arcs, x est une constante et donc dx = 0. On en
déduit donc que \int ~ \\int
 A \partial~P \over \partial~y (x,y) dx dy =
-\int  \partial~A~P(x,y) dx.

Soit \psi1 et \psi2 deux applications continues de
{[}c,d{]} dans \mathbb{R}~ vérifiant \forall~~y \in {[}c,d{]},
\psi1(y) \leq \psi2(y) et soit A = \(x,y) \in
\mathbb{R}~^2∣y \in
{[}c,d{]}\text et \psi1(y) \leq x \leq
\psi2(y)\. On considère la frontière \partial~A de A en
tant qu'arc paramétré orientée comme l'indique la figure ci dessous~: on
parcourt la frontière en laissant A à sa main gauche. Cette frontière
est la réunion de quatre arcs paramétrés, deux étant des graphes de
fonctions y\mapsto~x = \psii(y) (l'un
parcouru dans le sens direct, l'autre dans le sens indirect), deux étant
des graphes de fonctions
x\mapsto~\textconstante.

\includegraphics{cours21x.png}

Soit U un ouvert contenant A et Q une fonction de classe \mathcal{C}^1
sur U. La même méthode va nous fournir

\begin{align*} \int ~
\int  A \partial~Q \over \partial~y~
(x,y) dx dy &=&\int ~
c^d\left (\int ~
\psi1(y)^\psi2(y) \partial~Q \over
\partial~x (x,y) dx\right ) dy&&\%&
\\ & =& \int ~
c^d\Big{[} Q(x,y)\Big
{]} x=\psi1(y)^x=\psi2(y) dx \%&
\\ & =& \int ~
c^dQ(\psi 2(y),y) dy -\\int
 c^dQ(\psi 1(y),y) dy\%&
\\ \end{align*}

La première intégrale \int ~
c^dQ(\psi2(y),y) dy est l'intégrale de la forme
différentielle Q(x,y) dy le long du quart droit de la frontière \partial~A, la
seconde -\int ~
c^dQ(\psi1(y),y) dy est l'intégrale de la forme
différentielle Q(x,y) dy le long du quart gauche de la frontière \partial~A (à
cause du changement d'orientation). Mais d'autre part les intégrales
curvilignes de la forme différentielle Q(x,y) dy le long des quarts
supérieur et inférieur de la frontière sont nulles, car sur ces arcs, y
est une constante et donc dy = 0. On en déduit donc que
\int  \\int  A~
\partial~Q \over \partial~x (x,y) dx dy =\int ~
\partial~AQ(x,y) dy.

Si A est à la fois des deux formes en question, on pourra additionner
les deux résultats obtenus ce qui nous conduira à la formule

\begin{align*} \int ~
\partial~A(P(x,y) dx + Q(x,y) dy)&& \%&
\\ & =& \int ~
\int  A~\left ( \partial~Q
\over \partial~x (x,y) - \partial~P \over \partial~y
(x,y)\right ) dx dy\%& \\
\end{align*}

Définition~20.3.1 On dit qu'une partie A de \mathbb{R}~^2 est un
compact élémentaire s'il existe a,b,c,d \in \mathbb{R}~, deux fonctions
\phi1,\phi2 : {[}a,b{]} \rightarrow~ \mathbb{R}~ continues telles que
\forall~x \in {[}a,b{]}, \phi1~(x) \leq
\phi2(x) et deux fonctions \psi1 et \psi2 continues
de {[}c,d{]} dans \mathbb{R}~ vérifiant \forall~~y \in {[}c,d{]},
\psi1(y) \leq \psi2(y) telles que

\begin{align*} A& =& \(x,y) \in
\mathbb{R}~^2∣x \in
{[}a,b{]}\text et \phi 1(x) \leq y \leq
\phi2(x)\\%& \\
& =& \(x,y) \in
\mathbb{R}~^2∣y \in
{[}c,d{]}\text et \psi 1(y) \leq x \leq
\psi2(y)\\%& \\
\end{align*}

Définition~20.3.2 On dit qu'une partie A de \mathbb{R}~^2 est un
compact simple s'il existe un pavé P de \mathbb{R}~^2 contenant A et
une subdivision \sigma de P tels que pour tous les pavés Pi de la
subdivision, Pi \bigcap A soit un compact élémentaire.

On oriente la frontière d'un tel compact simple par la même règle que ci
dessus~: on parcourt la frontière en laissant le compact à sa main
gauche.

Exemple~20.3.5 Une couronne est un compact simple comme le montre le
dessin ci-dessous où on a exhibé une subdivision adaptée, ainsi que
l'orientation de la frontière de chaque compact élémentaire~:

\includegraphics{cours22x.png}

Posons alors Ai = Pi \bigcap A. On peut alors écrire

\begin{align*} \int ~
\int  A~\left ( \partial~Q
\over \partial~x (x,y) - \partial~P \over \partial~y
(x,y)\right ) dx dy&& \%&
\\ & =& \\sum
i \\int  ~ 
\\int  ~
Ai\left ( \partial~Q \over
\partial~x (x,y) - \partial~P \over \partial~y (x,y)\right )
dx dy\%& \\ & =&
\sum i~
\\int  ~
\partial~Ai(P(x,y) dx + Q(x,y) dy) \%&
\\ \end{align*}

Mais la réunion des frontières des Ai est constituée de deux
types d'arcs paramétrés~: des arcs faisant partie de la frontière de A
plus des segments horizontaux et verticaux provenant de la subdivision
du pavé. Or (sans vouloir formaliser complètement ce raisonnement) ces
segments sont parcourus deux fois pour un Ai et un
A\\\\jmathmathmathmath ad\\\\jmathmathmathmathacents, une fois dans un sens et une fois dans l'autre
(voir le dessin ci dessus), si bien que les intégrales curvilignes le
long de ces segments horizontaux ou verticaux n'appartenant pas à la
frontière de A s'annulent deux à deux. On obtient donc

\int  \\int ~
A\left ( \partial~Q \over \partial~x (x,y) -
\partial~P \over \partial~y (x,y)\right ) dx dy
=\int  \partial~A~(P(x,y) dx + Q(x,y) dy)

Théorème~20.3.9 (Green-Riemann). Soit A un compact simple de
\mathbb{R}~^2 de frontière orientée \partial~A, U un ouvert de \mathbb{R}~^2
contenant A, P,Q : U \rightarrow~ \mathbb{R}~ de classe \mathcal{C}^1. Alors

\int  \\int ~
A\left ( \partial~Q \over \partial~x (x,y) -
\partial~P \over \partial~y (x,y)\right ) dx dy
=\int  \partial~A~(P(x,y) dx + Q(x,y) dy)

Remarque~20.3.5 Le théorème de Green-Riemann permet (entre autres
choses) de ramener le calcul d'une intégrale du type
\int  \\int ~
Af(x,y) dx dy à celui d'une intégrale curviligne
\int  \partial~A~(P(x,y) dx + Q(x,y) dy)
(c'est-à-dire d'une intégrale simple) à condition de connaître deux
fonctions P et Q telles que f(x,y) = \partial~Q \over \partial~x
(x,y) - \partial~P \over \partial~y (x,y).

Corollaire~20.3.10 Soit A un compact simple de \mathbb{R}~^2 de
frontière orientée \partial~A. Alors l'aire de A est donnée par

m(A) =\int  \partial~A~x dy =
-\int  \partial~A~y dx = 1
\over 2 \int  \partial~A~(x dy
- y dx)

Démonstration Il suffit de prendre successivement P(x,y) = 0,Q(x,y) = x,
P(x,y) = -y,Q(x,y) = 0 et enfin P(x,y) = - y \over 2
,Q(x,y) = x \over 2 , couples pour lesquels  \partial~Q
\over \partial~x (x,y) - \partial~P \over \partial~y (x,y) =
1.

Corollaire~20.3.11 Soit A un compact simple de \mathbb{R}~^2 de
frontière orientée \partial~A. Alors l'aire de A est donnée en polaires par

m(A) = 1 \over 2 \int ~
\partial~A\rho^2 d\theta

Démonstration En effet x dy - y dx = \rho^2 d\theta.

{[}
{[}
{[}
{[}
