\textbf{Warning: 
requires JavaScript to process the mathematics on this page.\\ If your
browser supports JavaScript, be sure it is enabled.}

\begin{center}\rule{3in}{0.4pt}\end{center}

{[}
{[}
{[}{]}
{[}

\subsubsection{1.2 Cardinaux et entiers naturels}

\paragraph{1.2.1 Notion de cardinal}

Définition~1.2.1 On dit que deux ensembles E et F ont même cardinal s'il
existe une bi\\\\jmathmathmathmathection de E sur F. On notera
Card~E ou encore \textbar{}E\textbar{} le
cardinal d'un ensemble E.

Remarque~1.2.1 La relation il existe une bi\\\\jmathmathmathmathection de E sur F est bien
entendu une relation d'équivalence~; les cardinaux sont en quelque sorte
les classes d'équivalence pour cette relation (pas tout à fait puisque
l'ensemble de tous les ensembles n'existe pas).

Définition~1.2.2 On définit alors des opérations sur les cardinaux en
posant

\begin{align*} Card~A
+ CardB& =& \Card~A
\times\0\ \cup B
\times\1\\%&
\\
CardA.\Card~B& =&
Card~A \times B \%&
\\ \end{align*}

Remarque~1.2.2 Cette définition est \\\\jmathmathmathmathustifiée par le fait que si on a
CardA =\ Card~A' et
CardB =\ Card~B', on a
aussi

CardA \times\0\~ \cup
B \times\1\ =\
CardA' \times\0\ \cup B'
\times\1\

et Card~A \times B =\
CardA' \times B', comme on le vérifie facilement en construisant les
bi\\\\jmathmathmathmathections appropriées.

Définition~1.2.3 On pose Card~A
\leq Card~B s'il existe une in\\\\jmathmathmathmathection de A dans B.

On admettra que c'est une relation d'ordre total sur les cardinaux~; les
seuls points non évidents sont l'antisymétrie et la totalité~:
l'antisymétrie constitue le théorème de Cantor-Bernstein qui dit que
s'il existe une in\\\\jmathmathmathmathection de A dans B et une in\\\\jmathmathmathmathection de B dans A,
alors il existe une bi\\\\jmathmathmathmathection de A sur B~; la totalité résulte assez
facilement de l'axiome de Zorn.

\paragraph{1.2.2 Les entiers naturels}

On dira qu'un ensemble A est fini si Card~A
\textless{} Card~A + 1 (c'est équivalent à~: il
n'existe pas de bi\\\\jmathmathmathmathection de A sur une partie stricte de A). L'ensemble
des cardinaux finis forme alors un ensemble totalement ordonné appelé
ensemble des entiers naturels et noté \mathbb{N}~. Il vérifie les propriétés
suivantes qui le caractérisent à un isomorphime près d'ensembles
ordonnés

Axiome~1.2.1 (de Peano) \mathbb{N}~ est un ensemble infini où toute partie non
vide a un plus petit élément et où toute partie non vide ma\\\\jmathmathmathmathorée a un
plus grand élément

On en déduit immédiatement l'existence d'un successeur de tout élément a
de \mathbb{N}~ et on montre en théorie des cardinaux que ce n'est autre que a + 1.

L'existence d'un plus petit élément pour toute partie non vide conduit
immédiatement aux deux résultats suivants~:

Théorème~1.2.1 (Principe de récurrence forte) Soit P(n) une propriété
qui peut être vraie ou fausse pour tout entier naturel n. On suppose que
P(n0) est vraie et que si P(n) est vraie, alors P(n + 1) est
vraie. Alors P(n) est vraie pour tout n ≥ n0.

Démonstration Soit en effet X l'ensemble des n ≥ n0 tels que
P(n) soit fausse et supposons que X est non vide~; alors il admet un
plus petit élément n1 \inX. Comme
n0∉X, on a n1
\textgreater{} n0~; mais alors n1 -
1∉X et n1 - 1 ≥ n0~; donc
P(n1 - 1) est vraie, et il en est de même de P((n1 -
1) + 1) = P(n1), soit
n1∉X. C'est absurde. Donc X = \varnothing~, et
par conséquent, P(n) est vraie pour tout n ≥ n0.

Théorème~1.2.2 (Principe de récurrence faible) On suppose que
P(n0) est vraie et que si P(n0 + 1),P(n0 +
2),\\ldots~,P(n)
sont vraies, alors P(n + 1) est vraie. Alors P(n) est vraie pour tout n
≥ n0.

Démonstration Soit en effet X l'ensemble des n ≥ n0 tels que
P(n) soit fausse et supposons que X est non vide~; alors il admet un
plus petit élément n1 \inX. Comme
n0∉X, on a n1
\textgreater{} n0~; mais alors
n0,\\ldots,n1~
- 1∉X et donc
P(n0),\\ldots,P(n1~
- 1) sont vraies~; il en est donc de même de P(n1), soit
n1∉X. C'est absurde. Donc X = \varnothing~, et
par conséquent, P(n) est vraie pour tout n ≥ n0.

{[}
{[}
{[}
{[}
