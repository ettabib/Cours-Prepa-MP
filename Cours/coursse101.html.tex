\textbf{Warning: 
requires JavaScript to process the mathematics on this page.\\ If your
browser supports JavaScript, be sure it is enabled.}

\begin{center}\rule{3in}{0.4pt}\end{center}

{[}
{[}
{[}{]}
{[}

\subsubsection{19.2 Nappes réglées}

Remarque~19.2.1 Cette notion n'est pas au programme des classes
préparatoires.

\paragraph{19.2.1 Notion de nappe réglée}

Soit I un intervalle de \mathbb{R}~ et soit (Du)u\inI une
famille de droites indexée par u. Donnons nous pour chaque u \in I un
point f(u) de Du et un vecteur directeur
\vecg(u) \in\overrightarrow
Du \diagdown\0\ et supposons que
(I,f) et (I,\vecg) soient de classe \mathcal{C}^1.
La réunion des droites Du est alors paramétrée par F : I \times \mathbb{R}~ \rightarrow~
E, (u,v)\mapsto~f(u) + v\vecg(u).
Nous allons montrer qu'à équivalence près, cette nappe paramétrée ne
dépend pas du choix de (I,f) et de (I,\vecg).

Supposons tout d'abord que nous changeons l'arc paramétré (I,f) en
(I,f1)~; on a alors f1(u) = f(u) +
\phi(u)\vecg(u) et on vérifie facilement (par exemple à
l'aide d'une structure euclidienne) que \phi est de classe \mathcal{C}^1.
Mais alors l'application \theta : (u,v)\mapsto~(u,v +
\phi(u)) est de classe \mathcal{C}^1 et sa réciproque
(u,w)\mapsto~(u,w - \phi(u)) est aussi de classe
\mathcal{C}^1. Donc \theta est un difféomorphisme de I \times \mathbb{R}~ sur lui même et
on a F \cdot \theta(u,v) = F(u,v + \phi(u)) = f(u) + (v +
\alpha~(u))\vecg(u) = f1(u) +
v\vecg(u) = F1(u,v) ce qui montre bien que
les deux nappes sont effectivement équivalentes.

Supposons maintenant que nous changeons (I,\vecg) en
(I,\vecg1)~; on a alors
\vecg1(u) = \psi(u)\vecg(u)
avec une application \psi de classe \mathcal{C}^1 qui ne s'annule pas.
Mais alors l'application \theta : (u,v)\mapsto~(u,\psi(u)v)
est de classe \mathcal{C}^1 et sa réciproque
(u,w)\mapsto~(u, w \over \psi(u) )
est aussi de classe \mathcal{C}^1. Donc \theta est un difféomorphisme de I \times
\mathbb{R}~ sur lui même et on a F \cdot \theta(u,v) = F(u,\psi(u)v) = f(u) +
\beta~(u)v\vecg(u) = f(u) +
v\vecg1(u) = F1(u,v) ce qui
montre que les deux nappes sont bien équivalentes.

Définition~19.2.1 Une telle nappe sera appelée une nappe réglée de
classe \mathcal{C}^1. Les droites Du sont appelées les
génératrices de la nappe. Un arc (I,f) de classe \mathcal{C}^1 tel que
\forall~u \in I, f(u) \in Du~ sera appelé une
directrice de la nappe~; une directrice plane est appelée une base de la
nappe.

\paragraph{19.2.2 Plan tangent à une nappe réglée}

Donnons nous une nappe réglée de classe \mathcal{C}^1,
(Du)u\inI et soit F(u,v) = f(u) +
v\vecg(u) un paramétrage admissible de cette nappe.
Soit u0 \in I. On a alors  \partial~F \over \partial~u
(u0,v) = f'(u0) +
v\vecg'(u0) et  \partial~F \over
\partial~v (u0,v) =\vec g(u0). Un
vecteur normal à la nappe est alors le vecteur  \partial~F
\over \partial~u (u0,v) ∧ \partial~F \over
\partial~v (u0,v) = f'(u0) ∧\vec
g(u0) + v\vecg'(u0)
∧\vec g(u0).

Trois cas sont alors possibles~:

\begin{itemize}
\itemsep1pt\parskip0pt\parsep0pt
\item
  Premier cas La famille
  (f'(u0),\vecg(u0),\vecg'(u0))
  est libre. Alors, pout tout v \in \mathbb{R}~, la famille ( \partial~F
  \over \partial~u (u0,v), \partial~F \over
  \partial~v (u0,v)) est libre. Lorsque v varie, le vecteur normal
  tourne dans le plan
  \mathrmVect(f'(u0~)
  ∧\vec
  g(u0),\vecg'(u0)
  ∧\vec g(u0)) en occupant toutes les
  directions sauf une, \mathbb{R}~g'(u0) ∧\vec
  g(u0), qui est la direction limite lorsque v tend vers
  ±\infty~~; autrement dit, lorsque le point se déplace sur la génératrice, le
  plan tangent tourne autour de celle-ci (il doit forcément la contenir
  puisque c'est une courbe tracée sur la surface et qu'elle est sa
  propre tangente) en occupant toutes les positions sauf une, la
  position limite à l'infini.
\item
  Deuxième cas
  \mathrmrg(f'(u0),\vecg(u0),\vecg'(u0~))
  = 2. Alors le plan tangent, lorsqu'il existe, doit contenir la
  génératrice et être parallèle au plan
  \mathrmVect(f'(u0),\vecg(u0),\vecg'(u0~))~;
  il doit être constant le long de la génératrice
  Du0. D'autre part, les deux vecteurs
  f'(u0) ∧\vec g(u0) et
  \vecg'(u0) ∧\vec
  g(u0) ne peuvent pas être tous deux nuls, et donc il
  existe au plus un v tel que  \partial~F \over \partial~u
  (u0,v) ∧ \partial~F \over \partial~v (u0,v) =
  f'(u0) ∧\vec g(u0) +
  v\vecg'(u0) ∧\vec
  g(u0) = 0, autrement dit il y a au plus un point singulier
  sur la génératrice~; en ce point le plan tangent n'existe pas.
\item
  Troisième cas
  \mathrmrg(f'(u0),\vecg(u0),\vecg'(u0~))
  = 1. Alors, pour tout v \in \mathbb{R}~, on a  \partial~F \over \partial~u
  (u0,v) ∧ \partial~F \over \partial~v (u0,v) =
  f'(u0) ∧\vec g(u0) +
  v\vecg'(u0) ∧\vec
  g(u0) = 0. Tout point de la génératrice est singulier et
  il n'y a de plan tangent en aucun point de la génératrice. On dit que
  la génératrice Du0 est une génératrice singulière.
\end{itemize}

\paragraph{19.2.3 Nappes cylindriques. Nappes coniques}

Définition~19.2.2 Soit \vecD une direction de droite
dans E. On appelle nappe cylindrique de direction
\vecD toute nappe réglée de classe \mathcal{C}^1,
(Du)u\inI telle que toutes les droites Du
soient parallèles à \vecD.

Soit \veck un vecteur directeur de
\vecD. On peut donc choisir un paramétrage F(u,v) =
f(u) + v\vecg(u) avec \forall~~u \in
I, \vecg(u) =\vec k. On a alors
\vecg constante et donc \vecg'(u)
= 0. On voit donc que le premier cas de l'étude du plan tangent est
exclu et qu'il n'y a que deux possibilités pour un u0 \in I.

Premier cas
f'(u0)∉\vecD.
Alors  \partial~F \over \partial~u (u0,v) ∧ \partial~F
\over \partial~v (u0,v) = f'(u0)
∧\vec k. Tout point de la génératrice est régulier et
le plan tangent est constant le long de la génératrice.

Deuxième cas f'(u0) \in\vec D. Alors la
génératrice est singulière et le plan tangent n'existe en aucun point de
la génératrice.

Définition~19.2.3 Soit S un point de E. On appelle nappe conique de
sommet S toute nappe réglée de classe \mathcal{C}^1,
(Du)u\inI telle que toutes les droites Du
passent par le point S.

On peut alors choisir un paramétrage F(u,v) = f(u) +
v\vecg(u) avec \forall~~u \in I, f(u)
= S. Donc f est constante et f'(u) = 0. On voit donc que le premier cas
de l'étude du plan tangent est exclu et qu'il n'y a que deux
possibilités pour un u0 \in I.

Premier cas
(\vecg(u0),\vecg'(u0))
est libre. Alors  \partial~F \over \partial~u (u0,v) ∧ \partial~F
\over \partial~v (u0,v) =
v\vecg'(u0),\vecg(u0).
Tout point de la génératrice différent du sommet est régulier et le plan
tangent est constant le long de la génératrice.

Deuxième cas
(\vecg(u0),\vecg'(u0))
est liée. Alors la génératrice est singulière et le plan tangent
n'existe en aucun point de la génératrice.

Remarque~19.2.2 Les deux types de nappes réglées que nous venons
d'étudier vérifient la propriété remarquable que le plan tangent est
constant le long de chaque génératrice~; on appelle de telles nappes
réglées des nappes développables. Un autre type de nappes développables
peut être construit en prenant l'ensemble des tangentes à une courbe
gauche régulière. Soit (I,f) un tel arc paramétré régulier. On peut
paramétrer la tangente Du par
v\mapsto~f(u) + vf'(u)~; on peut donc prendre
\vecg(u) = f'(u). La famille
(f'(u0),\vecg(u0),\vecg'(u0))
est donc la famille (f'(u0),f''(u0)). Elle est donc
de rang au plus 2. On a deux possibilités~: soit u0 est un
point birégulier de (I,f), alors la famille est de rang 2 et le plan
tangent est donc constant le long de la génératrice (dont le seul point
singulier est d'ailleurs v = 0, c'est-à-dire le point de contact de la
tangente), soit u0 n'est pas birégulier et la génératrice est
singulière. En fait on peut montrer que ces trois types de nappes
(nappes cylindriques, nappes coniques et ensemble des tangentes à une
courbe) épuisent, au moins localement, les nappes développables.

{[}
{[}
{[}
{[}
