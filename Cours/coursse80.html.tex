\textbf{Warning: 
requires JavaScript to process the mathematics on this page.\\ If your
browser supports JavaScript, be sure it is enabled.}

\begin{center}\rule{3in}{0.4pt}\end{center}

{[}
{[}
{[}{]}
{[}

\subsubsection{14.4 Fonctions périodiques de période T}

Remarque~14.4.1 Remarquons que si f est périodique de période T, alors
\tildef définie par \tildef(t) =
f( T \over 2\pi~ t) est périodique de période 2\pi~ et l'on
a f(x) =\tilde f( 2\pi~ \over T x).
Ceci permet d'adapter tous les résultats précédents aux fonctions de
période T.

On pose

\begin{align*} (f∣g)&
=& 1 \over T \int ~
0^T\overlinef(t)g(t) dt = 1
\over T \int ~
a^a+T\overlinef(t)g(t) dt\%&
\\
\\textbar{}f\\textbar{}2^2&
=& (f∣f) = 1 \over T
\int ~
0^T\textbar{}f(t)\textbar{}^2 dt \%&
\\ en(t)& =&
e^2i\pi~nt\diagupT \%& \\
\end{align*}

Alors (en)n\inℤ est une famille orthonormée de C. On
définit les coefficients de Fourier de f \inC par

\begin{align*} \forall~~n \in
ℤ,\quad cn(f)& =&
(en∣f) = 1 \over
T \int ~
0^Tf(t)e^-2\pi~int\diagupT dt\%&
\\ \forall~~n ≥
0,\quad an(f)& =& 2 \over
T \int ~
0^Tf(t)cos~  2\pi~nt
\over T dt \%& \\
\forall~n ≥ 1,\quad bn~(f)&
=& 2 \over T \int ~
0^Tf(t)sin~  2\pi~nt
\over T dt \%& \\
\end{align*}

la série de Fourier de f par

\begin{align*} c0(f)& +&
\\sum
n≥1(cn(f)e^2\pi~inx\diagupT + c
-n(f)e^-2\pi~inx\diagupT) \%& \\ &
=& a0(f) \over 2 +
\\sum
n≥1(an(f)\cos  2\pi~nx
\over T + bn(f)\sin 2\pi~nx
\over T )\%& \\
\end{align*}

et on a les théorèmes

Théorème~14.4.1 (Bessel). Soit f : \mathbb{R}~ \rightarrow~ \mathbb{C} continue par morceaux et
périodique de période T. Alors la série
\textbar{}c0(f)\textbar{}^2
+ \\sum ~
n≥1(\textbar{}cn(f)\textbar{}^2 +
\textbar{}c-n(f)\textbar{}^2) est convergente et on
a

\textbar{}c0(f)\textbar{}^2 +
\sum n=1^+\infty~(\textbar{}c~
n(f)\textbar{}^2 + \textbar{}c
-n(f)\textbar{}^2) \leq\\textbar{}
f\\textbar{} 2^2

Théorème~14.4.2 (Dirichlet). Soit f : \mathbb{R}~ \rightarrow~ \mathbb{C} de classe \mathcal{C}^1 par
morceaux et périodique de période T. Alors la série de Fourier de f
converge sur \mathbb{R}~ et \forall~~x \in \mathbb{R}~,

\begin{align*} f(x^+) +
f(x^-) \over 2 && \%&
\\ & =& c0(f) +
\sum n=1^+\infty~(c~
n(f)e^2\pi~inx\diagupT + c -n(f)e^-2\pi~inx\diagupT)
\%& \\ & =& a0(f)
\over 2 + \\sum
n=1^+\infty~(a n(f)\cos  2\pi~nx
\over T + bn(f)\sin  2\pi~nx
\over T )\%& \\
\end{align*}

Théorème~14.4.3 (Dirichlet). Soit f : \mathbb{R}~ \rightarrow~ \mathbb{C} périodique de période Tde
classe \mathcal{C}^1 par morceaux et continue. Alors la série
\textbar{}c0(f)\textbar{} +\
\sum ~
n≥1(\textbar{}cn(f)\textbar{} +
\textbar{}c-n(f)\textbar{}) converge, la série de Fourier de f
converge normalement sur \mathbb{R}~ et on a \forall~~x \in \mathbb{R}~,

\begin{align*} f(x)& =& c0(f) +
\sum n=1^+\infty~(c~
n(f)e^2\pi~inx\diagupT + c -n(f)e^-2\pi~inx\diagupT)
\%& \\ & =& a0(f)
\over 2 + \\sum
n=1^+\infty~(a n(f)\cos  2\pi~nx
\over T + bn(f)\sin  2\pi~nx
\over T )\%& \\
\end{align*}

(autrement dit f est somme de sa série de Fourier).

Théorème~14.4.4 Soit f : \mathbb{R}~ \rightarrow~ \mathbb{C} périodique de période T de classe
C^k. Alors

\forall~n \in ℤ, cn~(f) =
\left ( 2\pi~in \over T
\right )^kc n(f^(k))

et, quand \textbar{}n\textbar{} tend vers + \infty~, cn(f) = o( 1
\over n^k ).

Théorème~14.4.5 (Parseval-Plancherel). Soit f : \mathbb{R}~ \rightarrow~ \mathbb{C} périodique de
période T et continue par morceaux. Alors

\begin{align*}
\\textbar{}f\\textbar{}2^2&
=& 1 \over T \int ~
0^T\textbar{}f(t)\textbar{}^2 dt \%&
\\ & =&
\textbar{}c0(f)\textbar{}^2 +
\sum n=1^+\infty~(\textbar{}c~
n(f)\textbar{}^2 + \textbar{}c
-n(f)\textbar{}^2) \%& \\ &
=& \textbar{}a0(f)\textbar{}^2
\over 4 + 1 \over 2
\sum n=1^+\infty~(\textbar{}a~
n(f)\textbar{}^2 + \textbar{}b
n(f)\textbar{}^2)\%& \\
\end{align*}

{[}
{[}
{[}
{[}
