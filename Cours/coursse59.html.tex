\textbf{Warning: 
requires JavaScript to process the mathematics on this page.\\ If your
browser supports JavaScript, be sure it is enabled.}

\begin{center}\rule{3in}{0.4pt}\end{center}

{[}
{[}
{[}{]}
{[}

\subsubsection{9.10 Convergence absolue, semi-convergence}

\paragraph{9.10.1 Critère de Cauchy pour les intégrales}

Théorème~9.10.1 (critère de Cauchy). Soit E un espace vectoriel normé
complet et -\infty~ \textless{} a \textless{} b \leq +\infty~. Soit f : {[}a,b{[}\rightarrow~ E
réglée. Alors l'intégrale \int ~
a^bf(t) dt converge si et seulement si~

\forall~~\epsilon \textgreater{} 0,
\exists~c \in {[}a,b{[}, c \textless{} u \textless{} v
\textless{} b \rigtharrow~\\textbar{}\int ~
u^vf(t) dt\\textbar{} \textless{} \epsilon

Démonstration Si F(x) =\int ~
a^xf(t) dt, la propriété ci dessus est équivalente à

\forall~~\epsilon \textgreater{} 0,
\exists~c \in {[}a,b{[}, c \textless{} u \textless{} v
\textless{} b \rigtharrow~\\textbar{} F(v) -
F(u)\\textbar{} \textless{} \epsilon

ce qui n'est autre que le critère de Cauchy pour l'existence de la
limite de F au point b.

\paragraph{9.10.2 Convergence absolue}

Définition~9.10.1 Soit f : {[}a,b{[}\rightarrow~ E réglée. On dit que l'intégrale
\int  a^b~f(t) dt converge
absolument si l'intégrale \int ~
a^b\\textbar{}f(t)\\textbar{}
dt converge.

Théorème~9.10.2 Soit E un espace vectoriel normé complet et -\infty~
\textless{} a \textless{} b \leq +\infty~. Soit f : {[}a,b{[}\rightarrow~ E réglée. Si
l'intégrale \int  a^b~f(t) dt
converge absolument, elle converge.

Démonstration Soit \epsilon \textgreater{} 0. Puisque l'intégrale
\int ~
a^b\\textbar{}f(t)\\textbar{}dt
converge, d'après le critère de Cauchy, il existe c \in {[}a,b{[} tel que
c \textless{} u \textless{} v \textless{} b \rigtharrow~\\int

u^v\\textbar{}f(t)\\textbar{}
dt \textless{} \epsilon. Alors c \textless{} u \textless{} v \textless{} b
\rigtharrow~\\textbar{}\int ~
u^vf(t) dt\\textbar{}
\leq\int ~
u^v\\textbar{}f(t)\\textbar{}
dt \textless{} \epsilon. Donc l'intégrale \int ~
a^bf(t) dt vérifie le critère de Cauchy, par conséquent
elle converge.

Remarque~9.10.1 L'avantage est évidemment que la convergence absolue
concerne la convergence d'une intégrale de fonction à valeurs réelles
positives pour laquelle nous disposons dé\\\\jmathmathmathmathà de critères simples.

\paragraph{9.10.3 Règles de convergence}

Proposition~9.10.3 Soit f : {[}a,b{[}\rightarrow~ E et g : {[}a,b{[}\rightarrow~ F réglées. On
suppose que f = 0(g) et que l'intégrale \int ~
a^bg(t) dt converge absolument. Alors l'intégrale
\int  a^b~f(t) dt converge
absolument.

Démonstration En effet f = 0(g) \Leftrightarrow
\\textbar{}f(t)\\textbar{} =
O(\\textbar{}g(t)\\textbar{}) au voisinage
de b.

Remarque~9.10.2 En général la fonction étalon g sera choisie à valeurs
réelles positives.

Théorème~9.10.4 Soit f : {[}a,b{[}\rightarrow~ E et g : {[}a,b{[}\rightarrow~ \mathbb{R}~ réglées. On
suppose que g est positive et qu'il existe \ell \in E
\diagdown\0\ tel que, au voisinage de b, f(t)
∼ \ellg(t). Alors (i) si \int ~
a^bg(t) dt converge, l'intégrale
\int  a^b~f(t) dt converge
absolument (ii) si \int  a^b~g(t)
dt diverge, l'intégrale \int ~
a^bf(t) dt diverge.

Démonstration On a bien entendu, f = O(g), et d'après la proposition
précédente, si \int  a^b~g(t) dt
converge, l'intégrale \int ~
a^bf(t) dt converge absolument. Inversement, supposons
que l'intégrale \int  a^b~f(t) dt
converge. On a f - \ellg =
o(\\textbar{}\ellg\\textbar{}) et donc il
existe c \in {[}a,b{[} tel que t \textgreater{} c
\rigtharrow~\\textbar{} f(t) - \ellg(t)\\textbar{} \leq 1
\over 2
\\textbar{}\ellg(t)\\textbar{} = 1
\over 2
\\textbar{}\ell\\textbar{}g(t). Soit alors c
\textless{} u \textless{} v \textless{} b~; on a
\\textbar{}\ell\\textbar{}\\int
 u^vg(t) dt =\\textbar{}
\ell\int  u^v~g(t)
dt\\textbar{} puisque g est réelle positive. On a donc

\begin{align*}
\\textbar{}\ell\\textbar{}\\int
 u^vg(t) dt& =&
\\textbar{}\int ~
u^v(\ellg(t) - f(t)) dt +\int ~
u^vf(t) dt\\textbar{}\%&
\\ & \leq& \int ~
u^v\\textbar{}\ellg(t) -
f(t)\\textbar{} dt
+\\textbar{}\int ~
u^vf(t) dt\\textbar{} \%&
\\ & \leq& 1 \over 2
\\textbar{}\ell\\textbar{}\\int
 u^vg(t) dt
+\\textbar{}\int ~
u^vf(t) dt\\textbar{} \%&
\\ \end{align*}

On en déduit que \int  u^v~g(t)
dt \leq 2 \over
\\textbar{}\ell\\textbar{}
\\textbar{}\int ~
u^vf(t) dt\\textbar{}. Comme
\int  a^b~f(t) dt converge,
l'intégrale vérifie le critère de Cauchy~; l'inégalité ci dessus montre
que l'intégrale \int  a^b~g(t) dt
vérifie également le critère de Cauchy, donc converge.

En utilisant alors nos fonctions ''étalons''  1 \over
t^\alpha~ en + \infty~ et  1 \over
(b-t)^\alpha~ en b \in \mathbb{R}~, on obtient les critères suivants

Théorème~9.10.5 Soit E un espace vectoriel normé. Soit f : {[}a,+\infty~{[}\rightarrow~ E
réglée. (i) S'il existe \alpha~ \textgreater{} 1 tel que f(t) = 0( 1
\over t^\alpha~ ), alors l'intégrale
\int  a^+\infty~~f(t) dt converge
absolument (ii) S'il existe \alpha~ \in \mathbb{R}~ et \ell \in E
\diagdown\0\ tels que f(t) ∼ \ell
\over t^\alpha~ alors l'intégrale
\int  a^+\infty~~f(t) dt converge
absolument si \alpha~ \textgreater{} 1 et diverge si \alpha~ \leq 1. (iii) Si E = \mathbb{R}~ et
f(t) ≥ 0, et s'il existe \alpha~ \leq 1 et \ell \textgreater{} 0 (y compris + \infty~) tel
que limt\rightarrow~+\infty~t^\alpha~~f(t) = \ell,
alors l'intégrale \int  a^+\infty~~f(t)
dt diverge.

Théorème~9.10.6 Soit E un espace vectoriel normé, b \in \mathbb{R}~. Soit f :
{[}a,b{[}\rightarrow~ E réglée. (i) S'il existe \alpha~ \textless{} 1 tel que f(t) = 0(
1 \over (b-t)^\alpha~ ), alors l'intégrale
\int  a^b~f(t) dt converge
absolument (ii) S'il existe \alpha~ \in \mathbb{R}~ et \ell \in E
\diagdown\0\ tels que f(t) ∼ \ell
\over (b-t)^\alpha~ alors l'intégrale
\int  a^b~f(t) dt converge
absolument si \alpha~ \textless{} 1 et diverge si \alpha~ ≥ 1. (iii) Si E = \mathbb{R}~ et
f(t) ≥ 0, et s'il existe \alpha~ ≥ 1 et \ell \textgreater{} 0 (y compris + \infty~) tel
que limt\rightarrow~b(b - t)^\alpha~~f(t) =
\ell, alors l'intégrale \int ~
a^bf(t) dt diverge.

Exemple~9.10.1 La fonction
t\mapsto~e^-t^2  est continue
sur {[}0,+\infty~{[} et en + \infty~ on a e^-t^2  = o( 1
\over t^2 ). Donc l'intégrale
\int ~
0^+\infty~e^-t^2  dt converge. De même,
considérons l'intégrale \int ~
0^+\infty~t^s-1e^-t dt. L'application
t\mapsto~t^s-1e^-t est
continue sur {]}0,+\infty~{[}, donc l'intégrale est a priori doublement
impropre en 0 et en + \infty~. En + \infty~, on a t^s-1e^-t =
o( 1 \over t^2 ) et donc l'intégrale
converge en + \infty~. En 0, on a t^s-1e^-t ∼
t^s-1 \textgreater{} 0, donc l'intégrale converge si et
seulement si~s - 1 \textgreater{} -1 soit s \textgreater{} 0. En
définitive, l'intégrale converge si et seulement si~s \textgreater{} 0.
On pose alors \Gamma(s) =\int ~
0^+\infty~t^s-1e^-t dt. Une intégration
par parties donne alors pour s \textgreater{} 0

\begin{align*} \int ~
x^yt^se^-t dt& =&
\left
{[}-t^se^-t\right {]}
x^y + s\int ~
x^yt^s-1e^-t dt \%&
\\ & =& x^se^-x -
y^se^-y + s\int ~
x^yt^s-1e^-t dt\%&
\\ \end{align*}

et en faisant tendre x vers 0 et y vers + \infty~, on obtient l'équation
fonctionnelle \Gamma(s) = s\Gamma(s - 1). Tenant compte de \Gamma(1)
=\int  0^+\infty~e^-t~ dt =
1, on obtient \forall~~n \in \mathbb{N}~, \Gamma(n) = (n - 1)!.
Remarquons également qu'en faisant le changement de variables t =
\sqrtu qui est de classe \mathcal{C}^1 sur {[}x,y{]}
pour 0 \textless{} x \textless{} y \textless{} +\infty~, on obtient

\int ~
x^ye^-t^2  dt = 1
\over 2 \int ~
\sqrtx^\sqrty
e^-u \over \sqrtu du

En faisant tendre x vers 0 et y vers + \infty~, on obtient
\int ~
0^+\infty~e^-t^2  dt = 1
\over 2 \Gamma( 1 \over 2 ).

\paragraph{9.10.4 Semi-convergence}

On dit qu'une intégrale \int ~
a^bf(t) dt est semi-convergente si elle converge, sans
être absolument convergente. L'outil essentiel pour montrer une
convergence non absolue est l'intégration par parties~; les autres
outils sont un théorème d'Abel ou le retour pur et simple au critère de
Cauchy.

Exemple~9.10.2 Etude de l'intégrale \int ~
1^+\infty~ sin~ t
\over t^\alpha~ dt. On a 
sin t \over t^\alpha~~ =
O( 1 \over t^\alpha~ ), donc si \alpha~ \textgreater{}
1 l'intégrale converge absolument.

Si 0 \textless{} \alpha~ \leq 1, on a après intégration par parties

\int  1^x~
sin t \over t^\alpha~~ dt
= cos 1 - \cos~ x
\over x^\alpha~ +\int ~
1^x cos~ t
\over t^\alpha~+1 dt

Mais limx\rightarrow~+\infty~~
cos x \over x^\alpha~~ =
0 et l'intégrale \int  1^+\infty~~
cos t \over t^\alpha~+1~
dt converge absolument puisque  cos~ t
\over t^\alpha~+1 = O( 1 \over
t^\alpha~+1 ). On en déduit que le terme de droite de l'égalité
ci dessus a une limite en + \infty~, et donc le terme de gauche aussi. En
conséquence, l'intégrale \int ~
1^+\infty~ sin~ t
\over t^\alpha~ dt converge. Montrons qu'elle ne
converge pas absolument~; on a

\begin{align*} \int ~
1^x \textbar{}sin~ t\textbar{}
\over t^\alpha~ & ≥& \\int
 1^x sin ^2~t
\over t^\alpha~ dt = 1 \over 2
\int  1^x~ 1
- cos~ (2t) \over
t^\alpha~ dt\%& \\ & =& 1
\over 2 \int ~
1^x 1 \over t^\alpha~ dt - 1
\over 2 \int ~
1^x cos~ (2t)
\over t^\alpha~ dt \%&
\\ \end{align*}

Mais l'intégrale \int  1^+\infty~~ 1
\over t^\alpha~ dt est divergente (car \alpha~ \leq 1),
alors que l'intégrale \int ~
1^+\infty~ cos~ (2t)
\over t^\alpha~ dt converge (même méthode
d'intégration par parties). On en déduit que
limx\rightarrow~+\infty~~\\int
 1^x sin ^2~t
\over t^\alpha~ dt = +\infty~ et donc aussi
limx\rightarrow~+\infty~~\\int
 1^x \textbar{} sin~
t\textbar{} \over t^\alpha~ dt = +\infty~.

Si \alpha~ \leq 0, posons \beta~ = -\alpha~. On a (en posant t = u + n\pi~),

\begin{align*} \left
\textbar{}\int ~
n\pi~^(n+1)\pi~t^\beta~ sin~ t
dt\right \textbar{}& =& \int ~
0^\pi~(u + n\pi~)^\beta~ sin~ u
du \%& \\ & ≥&
(n\pi~)^\beta~\int ~
0^\pi~ sin~ u du =
2(n\pi~)^\beta~\%& \\
\end{align*}

qui ne tend pas vers 0 quand n tend vers + \infty~~; le critère de Cauchy
n'est pas vérifié, et donc l'intégrale diverge.

Dans certains cas, le théorème d'Abel peut rendre des services (mais
très souvent, une simple intégration par parties peut s'y substituer)

Théorème~9.10.7 (Abel). Soit f : {[}a,b{[}\rightarrow~ \mathbb{R}~ de classe \mathcal{C}^1
et g : {[}a,b{[}\rightarrow~ \mathbb{R}~ continue. On suppose que

\begin{itemize}
\itemsep1pt\parskip0pt\parsep0pt
\item
  (i) f est positive, décroissante, de limite 0 en b
\item
  (ii) \existsM ≥ 0, \\forall~~x \in
  {[}a,b{[},\quad \left
  \textbar{}\int  a^x~g(t)
  dt\right \textbar{}\leq M
\end{itemize}

Alors l'intégrale \int ~
a^bf(t)g(t) dt converge.

Démonstration On montre que cette intégrale impropre vérifie le critère
de Cauchy à l'aide de la deuxième formule de la moyenne (dont les
hypothèses sur {[}u,v{]} sont bien vérifiées)

\begin{align*} \left
\textbar{}\int  u^v~f(t)g(t)
dt\right \textbar{}& =& f(u)\left
\textbar{}\int  u^w~g(t)
dt\right \textbar{} \%& \\
& =& f(u)\left \textbar{}\int ~
a^wg(t) dt -\int ~
a^ug(t) dt\right \textbar{}\leq 2Mf(u)\%&
\\ \end{align*}

Soit \epsilon \textgreater{} 0. Il existe c \in {[}a,b{[} tel que c \textless{} u
\textless{} b \rigtharrow~ 2Mf(u) \textless{} \epsilon. Alors c \textless{} u \textless{}
v \textless{} b \rigtharrow~\left
\textbar{}\int  u^v~f(t)g(t)
dt\right \textbar{} ce qui assure la convergence de
l'intégrale.

{[}
{[}
{[}
{[}
