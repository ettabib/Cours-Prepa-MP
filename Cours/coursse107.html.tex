\textbf{Warning: 
requires JavaScript to process the mathematics on this page.\\ If your
browser supports JavaScript, be sure it is enabled.}

\begin{center}\rule{3in}{0.4pt}\end{center}

{[}
{[}
{[}{]}
{[}

\subsubsection{20.4 Introduction aux intégrales de surface}

Définition~20.4.1 Soit \Sigma = (D,F) une nappe paramétrée de classe
\mathcal{C}^1de \mathbb{R}~^3, où D est un compact de \mathbb{R}~^2
de frontière négligeable. Soit f une fonction définie et continue sur
l'image de \Sigma et à valeurs dans l'espace vectoriel normé E. On appelle
intégrale de f le long de \Sigma et on note \int ~
\int  \Sigma~f(m) d\sigma l'élément de E

\int  \\int ~
\Sigmaf(m) d\sigma =\int ~
\int  D~f(F(u,v))
\\textbar{} \partial~F \over \partial~u (u,v) ∧ \partial~F
\over \partial~v (u,v)\\textbar{} du dv

En particulier, on appelle aire de \Sigma le nombre réel positif

m(\Sigma) =\int  \\int ~
\Sigma d\sigma =\int ~ \\int
 D\\textbar{} \partial~F \over \partial~u
(u,v) ∧ \partial~F \over \partial~v (u,v)\\textbar{}
du dv

Le principal résultat sur ces intégrales de surface est l'invariance par
changement de paramétrage admissible

Théorème~20.4.1 Soit \Sigma1 = (D1,F1) et
\Sigma2 = (D2,F2) deux nappes paramétrées de
classe \mathcal{C}^1 équivalentes, où D1 et D2 sont
des compacts de \mathbb{R}~^2 de frontières négligeables. Soit f une
fonction définie sur l'image de \Sigma1 et \Sigma2, à valeurs
dans l'espace vectoriel normé E. Alors

\int  \\int ~
\Sigma1f(m) d\sigma =\int ~
\int  \Sigma2~f(m) d\sigma

En particulier, l'aire de la nappe est invariante par changement de
paramétrage.

Démonstration Soit \theta : D1 \rightarrow~ D2 un difféomorphisme de
l'intérieur de D1 sur l'intérieur de D2 vérifiant
F1 = F2 \cdot \theta. Un calcul fait dans le chapitre sur les
nappes paramétrées montre que (si on note
(u,v)\mapsto~F1(u,v) et
(\lambda~,\mu)\mapsto~F2(\lambda~,\mu))

 \partial~F1 \over \partial~u (u,v) ∧ \partial~F1
\over \partial~v (u,v) = \\\\jmathmathmathmath\theta(u,v) \partial~F2
\over \partial~\lambda~ (\theta(u,v)) ∧ \partial~F2
\over \partial~\mu (\theta(u,v))

On en déduit que

\begin{align*} \int ~
\int  \Sigma1~f(m) d\sigma&& \%&
\\ & =& \int ~
\int  D1f(F1~(u,v))
\\textbar{} \partial~F \over \partial~u (u,v) ∧ \partial~F
\over \partial~v (u,v)\\textbar{} du dv \%&
\\ & =& \int ~
\int  D1f(F2~ \cdot
\theta(u,v)) \\textbar{} \partial~F2 \over
\partial~\lambda~ (\theta(u,v)) ∧ \partial~F2 \over \partial~\mu
(\theta(u,v))\\textbar{} \textbar{}\\\\jmathmathmathmath\theta(u,v)\textbar{}
du dv\%& \\ & =&
\int  \\int ~
D2f(F2(\lambda~,\mu)) \\textbar{}
\partial~F2 \over \partial~\lambda~ (\lambda~,\mu) ∧ \partial~F2
\over \partial~\mu (\lambda~,\mu)\\textbar{} d\lambda~ d\mu \%&
\\ \end{align*}

par le théorème de changement de variables dans les intégrales doubles.

Remarque~20.4.1 Le lecteur attentif aura remarqué que nous avons modifié
légèrement les définitions d'une nappe paramétrée et de l'équivalence de
deux nappes paramétrées, de fa\ccon à ce que cela
nous arrange. Nous réclamons toute son indulgence pour ces modifications
de détail.

Comme cas particulier, cherchons l'aire d'une nappe de révolution d'axe
Oz. Soit \Gamma une méridienne de cette nappe, paramétrée en coordonnées
cylindriques par r = \phi(t) et z = \psi(t), t \in {[}a,b{]}. Un paramétrage de
la nappe est alors F(t,\theta) = O + \phi(t)\vecu(\theta) +
\psi(t)\veck, (t,\theta) \in {[}a,b{]} \times {[}0,2\pi~{]} si bien que
 \partial~F \over \partial~t (t,\theta) = \phi'(t)\vecu(\theta)
+ \psi'(t)\veck et  \partial~F \over \partial~\theta (t,\theta)
= \phi(t)\vecu'(\theta). On a donc

 \partial~F \over \partial~t (t,\theta) ∧ \partial~F \over \partial~\theta
(t,\theta) = \phi(t)\left (\phi'(t)\veck -
\psi(t)\vecu(\theta)\right )

et donc

\\textbar{} \partial~F \over \partial~t (t,\theta) ∧ \partial~F
\over \partial~\theta (t,\theta)\\textbar{} =
\textbar{}\phi(t)\textbar{}\sqrt\phi'(t)^2  +
\psi'(t)^2

On en déduit que

\begin{align*} m(\Sigma)& =& \\int
 \int ~
{[}a,b{]}\times{[}0,2\pi~{]}\textbar{}\phi(t)\textbar{}\sqrt\phi'(t)^2
 + \psi'(t)^2 dt d\theta\%& \\ &
=& 2\pi~\int ~
a^b\textbar{}\phi(t)\textbar{}\sqrt\phi'(t)^2
 + \psi'(t)^2 dt \%& \\ &
=& 2\pi~\int  \Gamma~\textbar{}r\textbar{} ds
\%& \\ \end{align*}

en notant r = \phi(t) et ds = \sqrt\phi'(t)^2  +
\psi'(t)^2 dt la différentielle de l'abscisse curviligne sur
\Gamma. On obtient donc

Proposition~20.4.2 Soit \Sigma la nappe de révolution engendrée par la
rotation de la méridienne \Gamma autour de la droite D. Soit ds la
différentielle de l'abscisse curviligne de \Gamma et r la distance d'un point
de \Gamma à la droite D. Alors l'aire de la nappe est égale à
2\pi~\int  \Gamma~r ds.

{[}
{[}
{[}
{[}
