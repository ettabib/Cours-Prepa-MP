\textbf{Warning: 
requires JavaScript to process the mathematics on this page.\\ If your
browser supports JavaScript, be sure it is enabled.}

\begin{center}\rule{3in}{0.4pt}\end{center}

{[}
{[}
{[}{]}
{[}

\subsubsection{2.5 Dualité~: approche générale}

Cette section ne figure pas au programme des classes préparatoires. Elle
reprend les définitions et les résultats de la section précédente en les
généralisant.

\paragraph{2.5.1 Notion de dual. Orthogonalité}

Définition~2.5.1 Soit E un K-espace vectoriel . On appelle forme
linéaire sur E toute application linéaire de E dans K. On appelle dual
de E le K-espace vectoriel E^∗ = L(E,K).

Remarque~2.5.1 On dispose d'une application bilinéaire de
E^∗\times E dans K donnée par \langle
f∣x\rangle = f(x) appelée la
forme bilinéaire canonique. A cette forme bilinéaire est associée une
notion d'orthogonalité. On notera donc

\begin{itemize}
\itemsep1pt\parskip0pt\parsep0pt
\item
  (i) si A \subset~ E, A^\bot = \f \in
  E^∗∣\forall~~x
  \in A, f(x) = 0\
\item
  (ii) si B \subset~ E^∗, B^o = \x \in
  E∣\forall~~f \in B, f(x) =
  0\
\end{itemize}

Proposition~2.5.1 Les notations A,A1,A2 désignant
des parties de E et B,B1,B2 désignant des parties de
E^∗, on a

\begin{itemize}
\itemsep1pt\parskip0pt\parsep0pt
\item
  (i) A^\bot et B^o sont des sous-espaces vectoriels
  de E^∗ et E~; A^\bot =\
  \mathrmVect(A)^\bot et B^o
  =\
  \mathrmVect(B)^o
\item
  (ii) A1 \subset~ A2 \rigtharrow~ A1^\bot⊃
  A2^\bot et B1 \subset~ B2 \rigtharrow~
  B1^o ⊃ B2^o
\item
  (iii) A \subset~ (A^\bot)^o et B \subset~
  (B^o)^\bot
\item
  (iv) Soit A un sous-espace vectoriel de E, alors A^\bot =
  \0\ \Leftrightarrow A =
  E et A^\bot = E^∗\Leftrightarrow A =
  \0\.
\item
  (v) Soit B un sous-espace vectoriel de E^∗, alors
  B^o = E \Leftrightarrow B =
  \0\.
\end{itemize}

Démonstration Les propriétés (i),(ii) et (iii) sont évidentes ainsi que
les parties '' ⇐'' de (iv) et (v).

Montrons donc que A\neq~E \rigtharrow~
A^\bot\neq~\0\.
Soit (ei)i\inI une base de A que l'on complète en
(ei)i\inJ base de E. Soit i0 \in J \diagdown I et f
l'application qui à x associe sa i0-ième coordonnée dans la
base. On a f\neq~0 et f \in A^\bot.

Montrons maintenant que
A\neq~\0\ \rigtharrow~
A^\bot\neq~E^∗. Pour cela soit
x \in A \diagdown\0\. On complète x en une base
(ei)i\inI de E avec x = ei0. Soit
f l'application qui à x associe sa i0-ième coordonnée dans la
base. On a f(x)\neq~0, donc
f∉A^\bot.

Montrons maintenant que
B\neq~\0\ \rigtharrow~
B^o\neq~E. Soit f \in B
\diagdown\0\. On a
f\neq~0, donc \exists~x \in E,
f(x)\neq~0. Dans ce cas
x∉B^o, ce qui achève la
démonstration.

On prendra garde qu'on peut avoir B^o =
\0\ avec
B\neq~E^∗ (prendre par exemple E =
\mathbb{R}~{[}X{]} et B =\
\mathrmVect(\epsilonx,x \in ℤ) où \epsilonx(P)
= P(x)~; on a B^o = \0\
alors que \epsilon1\diagup2∉B).

\paragraph{2.5.2 Hyperplans}

Définition~2.5.2 On appelle hyperplan de E tout sous-espace vectoriel H
de E vérifiant les conditions équivalentes

\begin{itemize}
\itemsep1pt\parskip0pt\parsep0pt
\item
  (i) dim~ E\diagupH = 1
\item
  (ii) \exists~f \in
  E^∗\diagdown\0\, H
  = \mathrmKer~f
\item
  (iii) H admet une droite comme supplémentaire.
\end{itemize}

Démonstration (i) \rigtharrow~(ii)~: prendre \overlinee une base
de E\diagupH et écrire \pi~(x) = f(x)\overlinee.

(ii) \rigtharrow~ (iii)~: on prend a \in E tel que f(a)\neq~0.
Tout élément x s'écrit de manière unique sous la forme x = (x - f(x)
\over f(a) a) + f(x) \over f(a) a
avec x - f(x) \over f(a) a
\in\mathrmKer~f, soit E
= \mathrmKer~f \oplus~ Ka.

(iii) \rigtharrow~(i)~: tout supplémentaire de H est isomorphe à E\diagupH.

Théorème~2.5.2 Soit H un hyperplan de E. Alors H^\bot est de
dimension 1 (droite vectorielle)~: deux formes linéaires nulles sur H
sont proportionnelles.

Démonstration Si E = H \oplus~ Ka et H =\
\mathrmKerf, soit g \in H^\bot. Alors g et 
g(a) \over f(a) f coïncident sur H et sur Ka, donc sont
égales.

\paragraph{2.5.3 Bidual}

Définition~2.5.3 On désigne par E^∗∗ le dual de
E^∗.

Remarque~2.5.2 Si E est de dimension finie, E^∗ aussi et
dim E^∗~ =\
dim E. On en déduit que E^∗∗ est aussi de dimension finie
encore égale à dim~ E.

Théorème~2.5.3 L'application u : E \rightarrow~ E^∗∗,
x\mapsto~ux définie par ux(f) =
f(x) est une application linéaire in\\\\jmathmathmathmathective. Si E est un espace
vectoriel de dimension finie, c'est un isomorphisme d'espaces
vectoriels.

Démonstration En effet, cette application est visiblement linéaire et si
x \in\mathrmKer~u, on a

\forall~f \in E^∗, f(x) = u x~(f) =
0(f) = 0

et donc x \in (E^∗)^o =
\0\~; elle est donc in\\\\jmathmathmathmathective. Si E
est un espace vectoriel de dimension finie, on a une application
linéaire in\\\\jmathmathmathmathective entre deux espaces de même dimension finie, elle est
donc bi\\\\jmathmathmathmathective.

\paragraph{2.5.4 Transposée}

Définition~2.5.4 Soit u \in L(E,F). On note ^tu :
F^∗\rightarrow~ E^∗ définie par ^tu(g) = g \cdot u
(c'est une application linéaire).

Remarque~2.5.3 Cela revient à poser, pour x \in E et g \in F^∗,
\langle
^tu(g)∣x\rangle
E =\langle
g∣u(x)\rangle F.

Théorème~2.5.4 On a les propriétés suivantes

\begin{itemize}
\itemsep1pt\parskip0pt\parsep0pt
\item
  (i) u\mapsto~^tu est linéaire de L(E,F)
  dans L(F^∗,E^∗).
\item
  (ii) u \in L(E,F),v \in L(F,G)~; alors ^t(v \cdot u) =
  ^tu \cdot^tv
\item
  (iii) Si u est bi\\\\jmathmathmathmathective, ^tu aussi et
  (^tu)^-1 = ^t(u^-1)
\item
  (iv)
  \mathrmKer^t~u
  =
  (\mathrmImu)^\bot~
\item
  (v)
  \mathrmIm^t~u =
  (\mathrmKeru)^\bot~
\end{itemize}

Démonstration (i) et (ii) sont très faciles à partir de la définition.
(iii) découle immédiatement de (ii) en écrivant que v \cdot u =
\mathrmIdE et u \cdot v =
\mathrmIdF.

Pour (iv), on a g
\in\mathrmKer^t~u
\Leftrightarrow \forall~~x \in E,
^tu(g)(x) = 0 \Leftrightarrow
\forall~~x \in E, g(u(x)) = 0
\Leftrightarrow g \in
(\mathrmImu)^\bot~.

Pour (v), on remarque d'abord que f
\in\mathrmIm^t~u
\rigtharrow~\existsg, f = g \cdot u \rigtharrow~\\forall~~x
\in\mathrmKer~u, f(x) = 0 \rigtharrow~ f
\in
(\mathrmKeru)^\bot~,
soit
\mathrmIm^t~u \subset~
(\mathrmKeru)^\bot~.
Inversement, soit f \in
(\mathrmKeru)^\bot~.
On définit g1 forme linéaire sur
\mathrmIm~u par
g1(y) = f(x) si y = u(x)~; on vérifie en effet que f(x) est
indépendant du choix de x tel que y = u(x) car f est nulle sur
\mathrmKer~u. Soit alors V
un supplémentaire de
\mathrmIm~u dans F. On
définit g : F \rightarrow~ K par g(y1 + y2) =
g1(y1) si y1
\in\mathrmImu et y2~
\in V . On a bien f = g \cdot u = ^tu(g). Donc
(\mathrmKeru)^\bot\subset~\\mathrmIm^t~u
et donc l'égalité.

\paragraph{2.5.5 Dualité en dimension finie}

Proposition~2.5.5 Soit E un espace vectoriel de dimension finie, \mathcal{E} =
(e1,\\ldots,en~)
une base de E. La famille \mathcal{E}' =
(e1^∗,\\ldots,en^∗~)
de E^∗ définie par ei^∗(e\\\\jmathmathmathmath) =
\deltai^\\\\jmathmathmathmath est une base de E^∗ appelée la base
duale de la base \mathcal{E}

Démonstration On vérifie en effet immédiatement qu'elle est libre et
elle a le bon cardinal.

Théorème~2.5.6 Soit E un espace vectoriel de dimension finie.
L'application \mathcal{E}\rightarrow~\mathcal{E}' est une bi\\\\jmathmathmathmathection de l'ensemble des bases de E sur
l'ensemble des bases de E^∗.

Démonstration In\\\\jmathmathmathmathectivité~: si
((e1,\\ldots,en~)
et
(e1',\\ldots,en~')
sont deux bases qui ont même base duale, on a pour toute f \in
E^∗, f(ei) = f(ei') et donc ei
= ei'. Sur\\\\jmathmathmathmathectivité~: soit ℱ =
(f1,\\ldots,fn~)
une base de E^∗ et soit ℱ sa base duale (dans
E^∗∗). Soit u l'isomorphisme de E sur E^∗∗, et \mathcal{E} =
u^-1(ℱ'), base de E. On a alors fi(e\\\\jmathmathmathmath) =
ue\\\\jmathmathmathmath(fi) =
f\\\\jmathmathmathmath^∗(fi) = \deltai^\\\\jmathmathmathmath, donc ℱ est
la base duale de la base \mathcal{E}.

Corollaire~2.5.7 Soit E un espace vectoriel de dimension finie.

\begin{itemize}
\item
  (i) Soit A un sous-espace vectoriel de E. On a

  dim E =\ dim~ A
  + dim A^\bot~\text
  et (A^\bot)^o = A
\item
  (ii) Soit B un sous-espace vectoriel de E^∗. On a

  dim E =\ dim~ B
  + dim B^o~\text
  et (B^o)^\bot = B
\end{itemize}

Démonstration Soit
(e1,\\ldots,ep~)
une base de A que l'on complète en
(e1,\\ldots,en~)
base de E. On vérifie immédiatement que A^\bot
=\
\mathrmVect(ep+1^∗,\\ldots,en^∗~)
d'où le résultat sur la dimension. On montre de même le résultat sur la
dimension de B^o. Les égalités découlent alors des inclusions
et du fait que les espaces ont même dimension.

Corollaire~2.5.8 Soit E un espace vectoriel de dimension finie,
f1,\\ldots,fk~
\in E^∗, V = \x \in
E∣f1(x) =
\\ldots~ =
fk(x) = 0\. Alors
dim V =\ dim~ E
-\mathrmrg(f1,\\\ldots,fk~).

Théorème~2.5.9 Soit E et F deux espaces vectoriels de dimensions finies,
u \in L(E,F). Alors
\mathrmrg~u
= \mathrmrg^t~u.

Démonstration
\mathrmrg^t~u
= dim~
\mathrmIm^t~u
= dim~
(\mathrmKeru)^\bot~
= dim E -\ dim~
\mathrmKer~u
= \mathrmrg~u.

{[}
{[}
{[}
{[}
