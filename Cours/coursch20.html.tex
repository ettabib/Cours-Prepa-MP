\textbf{Warning: 
requires JavaScript to process the mathematics on this page.\\ If your
browser supports JavaScript, be sure it is enabled.}

\begin{center}\rule{3in}{0.4pt}\end{center}

{[}
{[}
{[}{]}
{[}

\subsection{Chapitre~19\\Surfaces}

Dans tout le chapitre, E désignera un espace vectoriel de dimension 3,
muni de sa structure d'espace affine naturelle (obtenue en le faisant
opérer sur lui même par addition). En cas de nécessité, cet espace sera
supposé muni d'une structure euclidienne orientée~; dans ce cas on
notera x ∧ y le produit vectoriel de deux vecteurs x et y,
(x∣y) leur produit scalaire et {[}x,y,z{]} le
produit mixte de trois vecteurs.

~19.1  \\
~~19.1.1 {Notion de nappe
paramétrée. Equivalence} \\ ~~19.1.2
 \\ ~~19.1.3
{Plan tangent à une nappe
paramétrée, vecteur normal} \\ ~~19.1.4
{Points réguliers et nappes
cartésiennes} \\ ~~19.1.5
{Intersection de nappes
paramétrées} \\ ~~19.1.6
{Intersection d'une nappe et de
son plan tangent} \\ ~19.2
 \\ ~~19.2.1
 \\
~~19.2.2 {Plan tangent à une
nappe réglée} \\ ~~19.2.3
{Nappes cylindriques. Nappes
coniques} \\ ~19.3 {Equations de
surfaces} \\ ~~19.3.1 {Surfaces
cartésiennes et nappes paramétrées} \\ ~~19.3.2
 \\ ~~19.3.3
 \\ ~~19.3.4
 \\
~19.4  \\ ~~19.4.1
 \\
~~19.4.2 {Réduction des
quadriques} \\ ~~19.4.3
{Classification des quadriques
en dimension 2 et 3} \\ ~~19.4.4
{Quadriques réglées, quadriques
de révolution}

{[}
{[}
{[}
{[}
