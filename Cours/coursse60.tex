\textbf{Warning: 
requires JavaScript to process the mathematics on this page.\\ If your
browser supports JavaScript, be sure it is enabled.}

\begin{center}\rule{3in}{0.4pt}\end{center}

{[}
{[}{]}
{[}

\subsubsection{10.1 Suites de fonctions}

\paragraph{10.1.1 Convergence simple, convergence uniforme}

Définition~10.1.1 Soit X un ensemble, E un espace métrique,
(fn)n\in\mathbb{N}~ une suite d'applications de X dans E. On dit
que la suite converge simplement si pour tout x \in X, la suite
(fn(x))n\in\mathbb{N}~ converge dans E. Dans ce cas, on pose
f(x) = limfn~(x) et on dit que f : X
\rightarrow~ E est limite simple de la suite (fn).

Remarque~10.1.1 La traduction en métrique de f est limite simple de la
suite (fn) est

\forall~x \in X, \\forall~~\epsilon
\textgreater{} 0, \exists~N(\epsilon,x),\quad
n ≥ N(\epsilon,x) \rigtharrow~ d(f(x),fn(x)) \textless{} \epsilon

où l'entier N dépend à la fois de \epsilon et de x \in X.

Exemple~10.1.1 Soit fn : {[}0,1{]} \rightarrow~ \mathbb{R}~,
x\mapsto~x^n. La suite fn
converge simplement vers f : {[}0,1{]} \rightarrow~ \mathbb{R}~ définie par f(x) =
\left \ \cases 1&si x
= 1 \cr 0&si x\neq~1 
\right .. Pour un \epsilon \textless{} 1 donné, le meilleur
N(\epsilon,x) que l'on puisse prendre est 0 si x = 1 ou x = 0, et E(
log~ \epsilon \over
log x~ ) si 0 \textless{} x \textless{} 1. On
constate que supx\in{[}0,1{]}~N(\epsilon,x) =
+\infty~. Il n'est donc pas question de prendre le même N pour tous les x.

Définition~10.1.2 Soit X un ensemble, E un espace métrique,
(fn)n\in\mathbb{N}~ une suite d'applications de X dans E. On dit
que la suite converge uniformément s'il existe f : X \rightarrow~ E vérifiant les
conditions équivalentes

\begin{itemize}
\itemsep1pt\parskip0pt\parsep0pt
\item
  (i) \forall~~\epsilon \textgreater{} 0,
  \exists~N(\epsilon),\quad n ≥ N(\epsilon)
  \rigtharrow~\forall~x \in X, d(f(x),fn~(x)) \textless{}
  \epsilon
\item
  (ii) limn\rightarrow~+\infty~\mun~ = 0 où
  l'on a posé \mun =\
  supx\inXd(fn(x),f(x)) \in \mathbb{R}~ \cup\ +
  \infty~\.
\end{itemize}

Démonstration L'équivalence est claire puisque \mun \textless{}
\epsilon \rigtharrow~ (\forall~x \in X, d(fn~(x),f(x))
\textless{} \epsilon) et qu'inversement (\forall~~x \in X,
d(fn(x),f(x)) \textless{} \epsilon) \rigtharrow~ \mun \leq \epsilon.

Remarque~10.1.2 Il est clair que si la suite (fn) converge
uniformément vers f, elle converge simplement vers f. On en déduit que
la fonction f est unique.

\paragraph{10.1.2 Plan d'étude d'une suite de fonctions}

Soit X un ensemble, E un espace métrique, (fn)n\in\mathbb{N}~
une suite d'applications de X dans E.

On commence par étudier la convergence simple de la suite de fonctions.
Pour chaque x \in X on étudie la suite (fn(x)) d'éléments de E.
Dans le cas où cette suite est convergente pour chaque x \in X, on définit
f : X \rightarrow~ E par f(x) = limfn~(x)~;
l'application f est limite simple de la suite (fn).

On étudie ensuite la convergence uniforme de la suite (fn)
vers f. Pour montrer une convergence uniforme, on peut soit chercher une
suite (\alpha~n) de limite 0 indépendante de x telle que
\forall~x \in X, d(fn~(x),f(x)) \leq
\alpha~n, soit étudier directement la suite (\mun) où
\mun =\
supx\inXd(fn(x),f(x)) \in \mathbb{R}~ \cup\ +
\infty~\. Pour montrer une non convergence uniforme, on peut
soit utiliser un des théorèmes suivants qui garantissent qu'un certain
nombre de propriétés des fonctions fn sont conservées par
limite uniforme, soit utiliser la proposition suivante

Proposition~10.1.1 Soit X un ensemble, E un espace métrique,
(fn)n\in\mathbb{N}~ une suite d'applications de X dans E. Alors
la suite (fn) converge uniformément vers f si et seulement
si~pour toute suite (xn) de X, on a
limd(f(xn),fn(xn~))
= 0.

Démonstration La condition est évidemment nécessaire puisque 0 \leq
d(f(xn),fn(xn)) \leq \mun.
Inversement, si la suite ne converge pas uniformément vers f, on a, en
niant la propriété (i)

\exists~\epsilon \textgreater{} 0,
\forall~N \in \mathbb{N}~, \\exists~n ≥ N,
\existsxn~ \in X,\quad
d(f(xn),fn(xn)) ≥ \epsilon

Ceci définit xn pour une infinité de n. Pour les autres, on
choisit un xn arbitraire. On a pour une infinité de n,
d(f(xn),fn(xn)) ≥ \epsilon et donc la suite
d(f(xn),fn(xn)) ne tend pas vers 0.

Exemple~10.1.2 Soit fn : {[}0,1{]} \rightarrow~ \mathbb{R}~,
x\mapsto~x^n. La suite fn
converge simplement vers f : {[}0,1{]} \rightarrow~ \mathbb{R}~ définie par f(x) =
\left \ \cases 1&si x
= 1 \cr 0&si x\neq~1 
\right .. Prenons xn = 1 - 1
\over n . On a fn(xn) -
f(xn) = (1 - 1 \over n )^n de
limite  1 \over e et non 0. Donc la suite ne converge
pas uniformément.

Remarque~10.1.3 Lorsque la convergence n'est pas uniforme sur X tout
entier, on peut rechercher des parties de X sur lesquelles cette
convergence est uniforme.

Exemple~10.1.3 Soit fn : {[}0, \pi~\over 2 {]}
\rightarrow~ \mathbb{R}~ définie par fn(t) =
n^\alpha~ sin~
^ntcos~ t. Il est clair que
\forall~t \in {[}0, \pi~\over 2~ {]},
limfn~(t) = 0~: si t \in {[}0,
\pi~\over 2 {[} on a 0 \leq sin~ t
\textless{} 1 et si t = \pi~\diagup2, on a cos~ t = 0.
La suite converge simplement vers la fonction nulle. On a \mun
= sup~t\in{[}0,\pi~\over
2 {]}\textbar{}f(t) - fn(t)\textbar{}
= sup~t\in{[}0,\pi~\over
2 {]}fn(t). Mais fn'(t) =
n^\alpha~ sin ^n-1~t(n - (n +
1)sin ^2~t) et on a donc le tableau
de variation, en posant tn = arcsin~
\sqrt n\over n+1

\begin{center}\rule{3in}{0.4pt}\end{center}

\begin{center}\rule{3in}{0.4pt}\end{center}

\begin{center}\rule{3in}{0.4pt}\end{center}

\begin{center}\rule{3in}{0.4pt}\end{center}

\begin{center}\rule{3in}{0.4pt}\end{center}

\begin{center}\rule{3in}{0.4pt}\end{center}

t

0

tn

\pi~\over 2

\begin{center}\rule{3in}{0.4pt}\end{center}

\begin{center}\rule{3in}{0.4pt}\end{center}

\begin{center}\rule{3in}{0.4pt}\end{center}

\begin{center}\rule{3in}{0.4pt}\end{center}

\begin{center}\rule{3in}{0.4pt}\end{center}

\begin{center}\rule{3in}{0.4pt}\end{center}

fn(t)

0

\nearrow

\mun

\searrow

0

\begin{center}\rule{3in}{0.4pt}\end{center}

\begin{center}\rule{3in}{0.4pt}\end{center}

\begin{center}\rule{3in}{0.4pt}\end{center}

\begin{center}\rule{3in}{0.4pt}\end{center}

\begin{center}\rule{3in}{0.4pt}\end{center}

\begin{center}\rule{3in}{0.4pt}\end{center}

On a donc

\mun = fn(tn) =
n^\alpha~\left ( n\over n +
1\right )^n\diagup2 1\over
\sqrtn + 1 ∼ n^\alpha~\over
\sqrte\sqrtn

La suite converge uniformément si et seulement si~\alpha~ \textless{}
1\over 2. Par contre, soit a \textless{}
\pi~\over 2 et soit N tel que
arcsin~ \sqrt
N\over N+1 \textgreater{} a. Alors dès que n ≥ N,
la fonction fn est croissante sur {[}0,a{]} et donc
supt\leqafn~(t) =
fn(a) qui tend vers 0 quand n tend vers + \infty~. On en déduit que
la suite fn converge uniformément vers la fonction nulle sur
tout intervalle {[}0,a{]} (mais pas sur leur réunion {[}0,
\pi~\over 2 {[}).

A titre d'introduction à ce qui suit, calculons
\int  0~^\pi~\over
2 fn(t) dt~; on a par un simple changement de variables u
= sin t, \\int ~
0^\pi~\over 2 fn(t) dt =
n^\alpha~\int ~
0^1u^n du =
n^\alpha~\over n+1. On voit donc que bien que
\forall~t \in {[}0, \pi~\over 2~ {]},
limn\rightarrow~+\infty~fn~(t) = 0, la suite
\int  0~^\pi~\over
2 fn(t) dt ne converge vers 0 que si \alpha~ \textless{} 1. Si \alpha~
= 1, elle converge vers 1, et si \alpha~ \textgreater{} 1, elle converge vers
+ \infty~. Autrement dit, si \alpha~ ≥ 1, on a
limn\rightarrow~+\infty~~\\int
 0^1fn(t)
dt\neq~\int ~
0^1\
limn\rightarrow~+\infty~fn(t) dt. On voit qu'une convergence simple
ne permet pas d'intervertir le symbole limite et le symbole d'intégrale.

\paragraph{10.1.3 Critère de Cauchy uniforme}

Définition~10.1.3 Soit X un ensemble, E un espace métrique. On dit
qu'une suite (fn)n\in\mathbb{N}~ d'applications de X dans E
vérifie le critère de Cauchy uniforme si on a

\forall~~\epsilon \textgreater{} 0,
\existsN \in \mathbb{N}~, p,q ≥ N \rigtharrow~\\forall~~x
\in X, d(fp(x),fq(x)) \textless{} \epsilon

Remarque~10.1.4 Il est clair que si la suite (fn) vérifie le
critère de Cauchy uniforme, pour chaque x \in X, la suite
(fn(x)) d'éléments de E est une suite de Cauchy.

Théorème~10.1.2 Soit X un ensemble, E un espace métrique complet. Alors
une suite (fn)n\in\mathbb{N}~ d'applications de X dans E est
uniformément convergente si et seulement si~elle vérifie le critère de
Cauchy uniforme.

Démonstration Le sens direct se démontre de la manière habituelle et
n'utilise pas la complétude de E~: si (fn) converge
uniformément vers f, soit \epsilon \textgreater{} 0 et N \in \mathbb{N}~ tel que n ≥ N
\rigtharrow~\forall~x \in X, d(f(x),fn~(x)) \textless{}
\epsilon \over 2 . Alors, si p,q ≥ N, on a
\forall~x \in X, d(fp(x),fq~(x)) \leq
d(fp(x),f(x)) + d(f(x),fq(x)) \textless{} \epsilon
\over 2 + \epsilon \over 2 = \epsilon.

Pour la réciproque, supposons que E est complet et que la suite
(fn) vérifie le critère de Cauchy uniforme. D'après la
remarque précédente, pour chaque x \in X, la suite (fn(x))
d'éléments de E est une suite de Cauchy, donc elle converge. On pose
f(x) = limfn~(x). Montrons que la
suite converge uniformément vers f. Soit \epsilon \textgreater{} 0, et soit N \in
\mathbb{N}~ tel que p,q ≥ N \rigtharrow~\forall~~x \in X,
d(fp(x),fq(x)) \textless{} \epsilon \over
2 . Fixons p ≥ N et faisons tendre q vers + \infty~. On obtient, en tenant
compte de limfq~(x) = f(x) et de la
continuité de la fonction distance, \forall~~x \in X,
d(fp(x),f(x)) \leq \epsilon \over 2 \textless{} \epsilon, ce
qui montre la convergence uniforme vers f.

\paragraph{10.1.4 Fonctions bornées, norme de la convergence uniforme}

Soit X un ensemble, E un espace vectoriel normé. On notera ℬ(X,E)
l'ensemble des applications bornées de X dans E. Pour f \inℬ(X,E), on
posera \\textbar{}f\\textbar{}\infty~
=\
supt\inX\\textbar{}f(t)\\textbar{}
\in \mathbb{R}~.

Proposition~10.1.3 L'application
f\mapsto~\\textbar{}f\\textbar{}\infty~
est une norme sur l'espace vectoriel ℬ(X,E). Soit (fn) une
suite de ℬ(X,E). Alors la suite (fn) converge uniformément si
et seulement si~elle converge dans (ℬ(X,E),\\textbar{}
\\textbar{}\infty~), avec la même limite.

Démonstration La vérification des propriétés des normes est élémentaire.
Si la suite (fn) converge dans
(ℬ(X,E),\\textbar{} \\textbar{}\infty~), soit f
sa limite. On a alors \mun =\
supx\inX\\textbar{}f(x) -
fn(x)\\textbar{} =\\textbar{} f
- fn\\textbar{}\infty~. On en déduit que la suite
converge uniformément vers f. Inversement, si la suite converge
uniformément vers f : X \rightarrow~ E, il existe N \in \mathbb{N}~ tel que n ≥ N \rigtharrow~
\mun =\
supx\inX\\textbar{}f(x) -
fn(x)\\textbar{} \textless{} 1. La fonction f -
fN est donc bornée~; comme fN est bornée, la
fonction f est également bornée. On a alors \\textbar{}f
- fn\\textbar{}\infty~ = \mun, ce qui montre
que la suite (fn) converge vers f dans
(ℬ(X,E),\\textbar{} \\textbar{}\infty~).

Remarque~10.1.5 On voit en particulier qu'une suite de fonctions bornées
qui converge uniformément a une limite qui est également une fonction
bornée.

Remarque~10.1.6 Soit (fn) une suite de ℬ(X,E). Alors la suite
(fn) vérifie le critère de Cauchy uniforme si et seulement
si~c'est une suite de Cauchy de (ℬ(X,E),\\textbar{}
\\textbar{}\infty~) (immédiat). On en déduit, d'après un
théorème précédent, que si E est complet,
(ℬ(X,E),\\textbar{} \\textbar{}\infty~) est lui
aussi complet.

\paragraph{10.1.5 Opérations sur les fonctions}

Bien entendu, les théorèmes de continuité des opérations algébriques
s'appliquent immédiatement aux suites simplement convergentes puisque si
f(x) = limfn~(x) et g(x)
= limgn~(x), on a (\alpha~f + \beta~g)(x)
= lim(\alpha~fn + \beta~gn~)(x) et
f(x)g(x) = limfn(x)gn~(x).

La convergence uniforme est stable par combinaisons linéaires comme le
montre le théorème suivant.

Théorème~10.1.4 Soit X un ensemble, E un espace vectoriel normé. Soit
(fn) et (gn) deux suites d'applications de X dans E
qui convergent uniformément. Soit \alpha~ et \beta~ des scalaires. Alors la suite
(\alpha~fn + \beta~gn)n\in\mathbb{N}~ converge uniformément.

Démonstration Soit f = limfn~ et g
= limgn~. Soit \epsilon \textgreater{} 0, et
N \in \mathbb{N}~ tel que

n ≥ N \rigtharrow~\forall~~x \in X, \\textbar{}f(x)
- fn(x)\\textbar{} \leq \epsilon \over
2(1 + \textbar{}\alpha~\textbar{})

et

\\textbar{}g(x) -
gn(x)\\textbar{} \leq \epsilon \over 2(1
+ \textbar{}\beta~\textbar{})

Alors pour n ≥ N, on a \forall~~x \in X,
\\textbar{}(\alpha~f + \beta~g)(x) - (\alpha~fn +
\beta~gn)(x)\\textbar{} \leq\textbar{}\alpha~\textbar{} \epsilon
\over 2(1+\textbar{}\alpha~\textbar{}) +
\textbar{}\beta~\textbar{} \epsilon \over
2(1+\textbar{}\beta~\textbar{}) \textless{} \epsilon.

Par contre, la convergence uniforme n'est pas stable par produit comme
le montre l'exemple suivant~:

Exemple~10.1.4 Soit fn : \mathbb{R}~ \rightarrow~ \mathbb{R}~ définie par fn(x) =
1 \over n . La suite (fn) converge
uniformément vers la fonction nulle. Soit g : \mathbb{R}~ \rightarrow~ \mathbb{R}~ définie par g(x) =
x. Alors la suite (fng) converge simplement vers 0, mais pas
uniformément puisque
supx\in\mathbb{R}~\textbar{}fn~(x)g(x)\textbar{}
= supx\in\mathbb{R}~~\left
\textbar{} x \over n \right \textbar{}
= +\infty~. A fortiori, la convergence uniforme d'une suite (fn) et
d'une suite (gn) n'implique pas la convergence uniforme de la
suite (fngn) (prendre gn = g). Cependant,
on a le résultat suivant

Théorème~10.1.5 Soit X un ensemble. Soit (fn) et
(gn) deux suites d'applications bornées de X dans K qui
convergent uniformément. Alors la suite (fngn)
converge uniformément.

Démonstration Soit f = limfn~ et g
= limgn~. On sait dé\\\\jmathmathmathmathà que f et g
sont bornées. On écrit alors

\begin{align*} f(x)g(x) -
fn(x)gn(x)& =& (fn(x) -
f(x))(gn(x) - g(x)) \%& \\ &
\text & +f(x)(gn(x) - g(x)) +
g(x)(fn(x) - f(x))\%& \\
\end{align*}

ce qui nous donne

\begin{align*} \textbar{}f(x)g(x) -
fn(x)gn(x)\textbar{}& \leq& \textbar{}fn(x)
- f(x)\textbar{}\textbar{}gn(x) - g(x)\textbar{} \%&
\\ & &
+\textbar{}f(x)\textbar{}\textbar{}gn(x) - g(x)\textbar{} +
\textbar{}g(x)\textbar{}\textbar{}fn(x) - f(x)\textbar{}\%&
\\ \end{align*}

puis \\textbar{}fg -
fngn\\textbar{}\infty~
\leq\\textbar{} fn -
f\\textbar{}\infty~\\textbar{}gn -
g\\textbar{}\infty~ +\\textbar{}
f\\textbar{}\infty~\\textbar{}f -
fn\\textbar{}\infty~ +\\textbar{}
g\\textbar{}\infty~\\textbar{}g -
gn\\textbar{}\infty~. On obtient donc
lim~\\textbar{}fg -
fngn\\textbar{}\infty~ = 0 et donc
(fngn) converge uniformément vers fg.

\paragraph{10.1.6 Propriétés de la convergence uniforme}

Exemple~10.1.5 Soit fn : {[}0,1{]} \rightarrow~ \mathbb{R}~,
x\mapsto~x^n. La suite fn
converge simplement vers f : {[}0,1{]} \rightarrow~ \mathbb{R}~ définie par f(x) =
\left \ \cases 1&si x
= 1 \cr 0&si x\neq~1 
\right .. Chacune des fonctions fn est continue
au point 1, alors que f ne l'est pas. De nouveau, on a 1
= limn\rightarrow~+\infty~~\left
(limx\rightarrow~1^-x^n~\right
)\neq~limx\rightarrow~1^-~\left
(limn\rightarrow~+\infty~x^n~\right
) = 0.

Théorème~10.1.6~(conservation de la continuité) Soit E et F deux espaces
métriques, (fn)n\in\mathbb{N}~ une suite d'applications de E
dans F qui converge simplement vers f : E \rightarrow~ F. Soit a \in E. On suppose
que (i) chacune des fn est continue au point a (ii) il existe
U voisinage de a telle que la suite (fn) converge uniformément
sur U Alors f est continue au point a.

Démonstration Soit \epsilon \textgreater{} 0 et soit N \in \mathbb{N}~ tel que n ≥ N
\rigtharrow~\forall~x \in U, d(f(x),fn~(x)) \textless{}
\epsilon \over 3 . Comme fN est continue au point a,
il existe V voisinage de a tel que x \in V \rigtharrow~
d(fN(x),fN(a)) \textless{} \epsilon \over
3 . Alors, pour x \in U \bigcap V , on a d(f(x),f(a)) \leq
d(f(x),fN(x)) + d(fN(x),fN(a)) +
d(fN(a),f(a)) \leq \epsilon \over 3 + \epsilon
\over 3 + \epsilon \over 3 = \epsilon. Donc f est
continue au point a.

Corollaire~10.1.7 Soit E et F deux espaces métriques,
(fn)n\in\mathbb{N}~ une suite d'applications continues de E dans
F qui converge uniformément vers f : E \rightarrow~ F. Alors f est continue.

Remarque~10.1.7 Il suffit évidemment que tout point ait un voisinage sur
lequel la suite converge uniformément, ce que l'on appelle la
convergence uniforme locale.

Théorème~10.1.8~(interversion des limites) Soit E un espace métrique, F
un espace métrique complet, (fn)n\in\mathbb{N}~ une suite de
fonctions de E dans F. Soit a \in E, A \subset~ E tel que a
\in\overlineA et \forall~~n \in \mathbb{N}~, A
\subset~ Def (fn~). On suppose que

\begin{itemize}
\itemsep1pt\parskip0pt\parsep0pt
\item
  (i) la suite fn converge uniformément vers f sur A
\item
  (ii) chacune des fn a une limite \elln en a suivant A
\end{itemize}

Alors la suite (\elln) admet une limite \ell et f admet \ell pour
limite en a suivant A, autrement dit

limn\rightarrow~+\infty~~\left
(limx\rightarrow~a,x\inAfn~(x)\right
) = limx\rightarrow~a,x\inA~\left
(limn\rightarrow~+\infty~fn~(x)\right
)

Démonstration Pour montrer que la suite (\elln) admet une limite
\ell, il suffit de montrer que c'est une suite de Cauchy. Mais, la suite
(fn) vérifie le critère de Cauchy uniforme sur A. Soit \epsilon
\textgreater{} 0~; il existe N \in \mathbb{N}~ tel que p,q ≥ N
\rigtharrow~\forall~x \in A, d(fp(x),fq~(x))
\textless{} \epsilon. Soit p,q ≥ N~; en faisant tendre x vers a en restant dans
A, on obtient d(\ellp,\ellq) \leq \epsilon ce qui montre
effectivement que la suite (\elln) est une suite de Cauchy de F,
donc qu'elle converge.

Soit \epsilon \textgreater{} 0 et soit N \in \mathbb{N}~ tel que n ≥ N
\rigtharrow~\forall~x \in A, d(f(x),fn~(x)) \textless{}
\epsilon \over 3 et soit N' \in \mathbb{N}~ tel que n ≥ N' \rigtharrow~
d(\elln,\ell) \textless{} \epsilon \over 3 . Soit n
= max(N,N'). Comme fn~ admet
\elln pour limite en a suivant A, il existe U voisinage de a dans
E tel que x \in U \bigcap A \rigtharrow~ d(fn(x),\elln) \leq \epsilon
\over 3 . Alors, pour x \in U \bigcap A, on a

\begin{align*} d(f(x),\ell)& \leq&
d(f(x),fn(x)) + d(fn(x),\elln) +
d(\elln,\ell)\%& \\ &
\textless{}& \epsilon \over 3 + \epsilon \over
3 + \epsilon \over 3 = \epsilon \%&
\\ \end{align*}

ce qui montre que limx\rightarrow~a,x\inA~f(x) =
\ell.

Remarque~10.1.8 Le résultat suivant s'applique en particulier dans le
cas où a = +\infty~ et A = \mathbb{N}~, c'est-à-dire au cas d'une suite double
(xn,p) d'éléments de F~: avec les hypothèses

\begin{itemize}
\itemsep1pt\parskip0pt\parsep0pt
\item
  (i) limn\rightarrow~+\infty~xn,p~ =
  yp uniformément par rapport à p
\item
  (ii) limp\rightarrow~+\infty~xn,p~ =
  \elln
\end{itemize}

Alors la suite (\elln) admet une limite \ell et on a
limp\rightarrow~+\infty~yp~ = \ell, autrement
dit

limn\rightarrow~+\infty~~\left
(limp\rightarrow~+\infty~xn,p~\right
) = limp\rightarrow~+\infty~~\left
(limn\rightarrow~+\infty~xn,p~\right
)

Exemple~10.1.6 Le résultat précédent utilise de manière essentielle la
convergence uniforme par rapport à p comme le montre l'exemple
xn,p = n \over n+p pour lequel on a

0 = limn\rightarrow~+\infty~~\left
(limp\rightarrow~+\infty~xn,p~\right
)\neq~limp\rightarrow~+\infty~~\left
(limn\rightarrow~+\infty~xn,p~\right
) = 1

Théorème~10.1.9~(intégration) Soit (fn) une suite de fonctions
réglées de {[}a,b{]} dans E (espace vectoriel normé complet) qui
converge uniformément vers f : {[}a,b{]} \rightarrow~ E. Alors f est réglée et la
suite (\int  a^bfn~(t)
dt) admet la limite \int ~
a^bf(t) dt.

Démonstration Soit \epsilon \textgreater{} 0~; il existe N \in \mathbb{N}~ tel que n ≥ N
\rigtharrow~\\textbar{} f - fn\\textbar{}\infty~
\textless{} \epsilon \over 2 . Mais puisque fN est
réglée, il existe \phi : {[}a,b{]} \rightarrow~ E en escalier telle que
\\textbar{}fN - \phi\\textbar{}\infty~
\textless{} \epsilon \over 2 . On a donc
\\textbar{}f - \phi\\textbar{}\infty~
\leq\\textbar{} f - fN\\textbar{}\infty~
+\\textbar{} fN - \phi\\textbar{}\infty~
\textless{} \epsilon ce qui montre que f est encore réglée. On a alors

\\textbar{}\int ~
a^bf -\int ~
a^bf n\\textbar{}
\leq\int ~
a^b\\textbar{}f - f
n\\textbar{} \leq (b - a)\\textbar{}f -
fn\\textbar{}\infty~

ce qui montre que la suite (\int ~
a^bfn(t) dt) admet la limite
\int  a^b~f(t) dt.

Remarque~10.1.9 Comme le montre la démonstration précédente, le fait que
l'intervalle soit borné est essentiel. Le résultat précédent ne s'étend
donc pas aux intégrales sur des intervalles non bornés (voir pour cela
le paragraphe sur les fonctions intégrables). Par contre on a

Corollaire~10.1.10 Soit I un intervalle de \mathbb{R}~, (fn) une suite
de fonctions réglées de I dans E (espace vectoriel normé complet) qui
converge uniformément vers f : I \rightarrow~ E. Alors f est réglée. Soit a \in I,
Fn(x) =\int ~
a^xfn(t) dt et F(x)
=\int  a^x~f(t) dt. Alors la
suite (Fn) converge uniformément vers F sur tout segment
inclus dans I.

Démonstration Le théorème précédent montre que f est réglée sur tout
segment inclus dans I, donc réglée. De plus, si J est un segment inclus
dans I, on peut, quitte à l'agrandir, supposer qu'il contient a. On a
alors, pour x \in J,

\begin{align*} \\textbar{}F(x) -
Fn(x)\\textbar{}& \leq& \left
\textbar{}\int ~
a^x\\textbar{}f(t) - f
n(t)\\textbar{} dt\right \textbar{} \%&
\\ & \leq& \textbar{}x -
a\textbar{}\\textbar{}f -
fn\\textbar{}\infty~ \leq
\ell(J)\\textbar{}f -
fn\\textbar{}\infty~\%&
\\ \end{align*}

(en travaillant séparément dans les cas a \leq x et a \textgreater{} x), ce
qui montre la convergence uniforme sur J de Fn vers F.

Par contre, la convergence uniforme d'une suite de fonctions dérivables
n'implique pas que la limite soit elle-même dérivable. C'est même de
cette manière, par limite uniforme, qu'ont été construits les premiers
exemples de fonctions continues n'admettant de dérivée en aucun point
(voir le paragraphe sur les séries de fonctions). Par contre on a

Théorème~10.1.11 Soit I un intervalle de \mathbb{R}~, (fn) une suite de
fonctions de I dans E (espace vectoriel normé complet) qui converge
simplement vers f : I \rightarrow~ E. On suppose que (i) chacune des fn
est de classe \mathcal{C}^1 (ii) la suite (fn') converge
uniformément sur I vers une fonction g. Alors f est de classe
\mathcal{C}^1 et f' = g.

Démonstration Soit a \in I. Puisque chaque fn est de classe
\mathcal{C}^1, on a \forall~x \in I, fn~(x) =
fn(a) +\int ~
a^xfn'(t) dt. D'après le théorème précédent la
suite x\mapsto~\int ~
a^xfn'(t) dt converge uniformément sur tout
segment inclus dans I vers \int ~
a^xg(t) dt. On obtient donc, en faisant tendre n vers +
\infty~, \forall~~x \in I, f(x) = f(a)
+\int  a^x~g(t) dt. Comme g est
continue (limite uniforme de fonctions continues), f est de classe
\mathcal{C}^1 et f' = g.

Remarque~10.1.10 Comme le montre la démonstration précédente, il suffit,
avec les mêmes hypothèses, que la suite (fn) converge en un
point a pour qu'elle converge simplement sur I, cette convergence étant
d'ailleurs uniforme sur tout segment inclus dans I. On retiendra que,
pour montrer la dérivabilité d'une limite de suites de fonctions, il
faut s'attacher à la convergence uniforme de la suite des dérivées, et
non à celle de la suite elle-même.

\paragraph{10.1.7 Suites de fonctions intégrables sur un intervalle}

Remarque~10.1.11 Les théorèmes du type
lim\\int  fn~
=\int  \limfn~
démontrés précédemment ont des hypothèses trop restrictives~: ils
nécessitent d'une part que l'intervalle soit borné et d'autre part que
la suite de fonctions converge uniformément sur tout l'intervalle. La
théorie de Lebesgue étend ces théorèmes à des situations plus générales
que nous n'étudierons pas en détail, mais d'où nous extrairons un
certain nombre de résultats utiles, qui ne seront pas démontrés en toute
généralité, mais seulement avec quelques hypothèses supplémentaires.

Nous admettrons le résultat fondamental suivant suivant dont la
démonstration est difficile

Lemme~10.1.12 Soit J un segment de \mathbb{R}~, (fn)n\in\mathbb{N}~ une
suite de fonctions continues par morceaux de J dans \mathbb{R}~^+
vérifiant

\begin{itemize}
\itemsep1pt\parskip0pt\parsep0pt
\item
  il existe M ≥ 0 tel que \forall~~n \in \mathbb{N}~,
  \forall~t \in J, fn~(t) \leq M
\item
  la suite (fn)n\in\mathbb{N}~ converge simplement vers 0, soit
  \forall~~t \in J,
  limn\rightarrow~+\infty~fn~(t) = 0
\end{itemize}

Alors la suite (\int ~
Jfn)n\in\mathbb{N}~ converge vers 0.

On en déduit le lemme suivant

Lemme~10.1.13~(Convergence bornée sur un segment) Soit J un segment de
\mathbb{R}~, (fn)n\in\mathbb{N}~ une suite de fonctions continues par
morceaux de J dans \mathbb{C} qui converge simplement vers f continue par
morceaux. On suppose qu'il existe M ≥ 0 tel que
\forall~n \in \mathbb{N}~, \\forall~~t \in J,
\textbar{}fn(t)\textbar{}\leq M. Alors la suite
(\int  Jfn)n\in\mathbb{N}~~
admet la limite \int  J~f.

Démonstration On pose gn(t) = \textbar{}f(t) -
fn(t)\textbar{}. Comme \forall~~n \in \mathbb{N}~,
\forall~t \in J, \textbar{}fn~(t)\textbar{}\leq
M, en passant à la limite on a \textbar{}f(t)\textbar{}\leq M, soit encore
0 \leq g(t) \leq 2M. D'autre part, \forall~~t \in J,
limn\rightarrow~+\infty~gn~(t) = 0. D'après
le lemme précédent, on a
lim\\int ~
Jgn = 0. Or

\left \textbar{}\int  J~f
-\int ~
Jfn\right \textbar{} =
\left \textbar{}\int ~
J(f - fn)\right
\textbar{}\leq\int  J~\textbar{}f -
fn\textbar{} =\int ~
Jgn

qui tend vers 0. Autrement dit la suite (\\int
 Jfn)n\in\mathbb{N}~ admet la limite
\int  J~f.

Théorème~10.1.14~(convergence dominée) Soit I un intervalle de \mathbb{R}~,
(fn) une suite de fonctions de I dans \mathbb{C} continues par morceaux
qui converge simplement vers f : I \rightarrow~ \mathbb{C} continue par morceaux. On suppose
qu'il existe \phi : I \rightarrow~ \mathbb{R}~^+ continue par morceaux et intégrable
sur I telle que \forall~~n \in \mathbb{N}~,
\textbar{}fn\textbar{}\leq \phi (hypothèse de domination). Alors les
fonctions fn et f sont intégrables sur I et la suite
(\int  Ifn~) est convergente
de limite \int  I~f~:

\int  I~f =\
limn\rightarrow~+\infty~\int  Ifn~

Démonstration Comme \textbar{}fn(t)\textbar{}\leq \phi(t) et que \phi
est intégrable, les fonctions fn sont intégrables sur I. De
plus, en faisant tendre n vers + \infty~, on a aussi \textbar{}f(t)\textbar{}\leq
\phi(t), donc f est également intégrable sur I. Soit J un segment inclus
dans I. On a

\begin{align*} \left
\textbar{}\int  I~f
-\int ~
Ifn\right \textbar{}& \leq&
\left \textbar{}\int  I~f
-\int  J~f\right
\textbar{} + \left \textbar{}\\int
 Jf -\int ~
Jfn\right \textbar{} +
\left \textbar{}\int ~
Jfn -\int ~
Ifn\right \textbar{}\%&
\\ & =& \left
\textbar{}\int ~
I\diagdownJf\right \textbar{} + \left
\textbar{}\int  J~f
-\int ~
Jfn\right \textbar{} +
\left \textbar{}\int ~
I\diagdownJfn\right \textbar{} \%&
\\ & \leq& \int ~
I\diagdownJ\phi + \left
\textbar{}\int  J~f
-\int ~
Jfn\right \textbar{}
+\int  I\diagdownJ~\phi \%&
\\ \end{align*}

Comme \phi est intégrable positive, on a \int ~
I\phi =\
sup\\int ~
J\phi∣J \subset~ I\. Soit donc
\epsilon \textgreater{} 0~; il existe J segment inclus dans I tel que
\int  I\phi -\epsilon\over 3~
\leq\int  J~\phi \leq\\int
 I\phi, soit encore 0 \leq\int ~
I\diagdownJ\phi \leq \epsilon\over 3. Fixons un tel segment J~;
sur ce segment, la suite fn converge simplement vers f et
\textbar{}fn(t)\textbar{}\leq \phi(t)\textbar{}\leq M avec M
= supt\inJ~\phi(t) (qui existe puisque \phi
est continue par morceaux, donc bornée sur tout segment). Le lemme de
convergence bornée sur un segment assure que \\int
 Jf =\
limn\rightarrow~+\infty~\int  Jfn~~;
donc il existe N \in \mathbb{N}~ tel que n ≥ N \rigtharrow~\left
\textbar{}\int  J~f
-\int ~
Jfn\right \textbar{} \textless{}
\epsilon\over 3. Alors, pour n ≥ N, on a

\left \textbar{}\int  I~f
-\int ~
Ifn\right \textbar{}\leq
2\epsilon\over 3 + \epsilon\over 3 = \epsilon

ce qui montre bien que la suite (\int ~
Ifn) est convergente de limite
\int  I~f

Remarque~10.1.12 Il est important de constater que l'hypothèse de
domination par une fonction intégrable \phi indépendante de n sert non
seulement à garantir l'intégrabilité des fn et de f, mais est
également un argument essentiel de la démonstration de
\int  I~f =\
limn\rightarrow~+\infty~\int  Ifn~,
et donc de la validité du résultat. Comme on l'a dé\\\\jmathmathmathmathà vu avec la suite
de fonctions continues sur {[}0, \pi~\over 2 {]},
t\mapsto~n^\alpha~\
sin ^n-1tcos~ t, une suite de
fonctions intégrables peut très bien converger simplement vers une
fonction intégrable sans que l'on ait \int ~
If =\
limn\rightarrow~+\infty~\int  Ifn~.

Théorème~10.1.15~(convergence monotone) Soit I un intervalle de \mathbb{R}~,
(fn) une suite croissante de fonctions de I dans \mathbb{R}~ continues
par morceaux et intégrables sur I, qui converge simplement vers f : I \rightarrow~
\mathbb{R}~ continue par morceaux. Alors la suite (\int ~
Ifn) est ma\\\\jmathmathmathmathorée si et seulement si la fonction f est
intégrable. Dans ces conditions on a

\int  I~f =\
supn\in\mathbb{N}~\int  Ifn~
= limn\rightarrow~+\infty~~\\int
 Ifn

Démonstration En rempla\ccant éventuellement
fn par fn - f0 et f par f - f0,
on peut supposer que les fonctions fn sont positives, et donc
f également.

Supposons tout d'abord que la fonction f est intégrable. On a alors
\forall~t \in I, \textbar{}fn~(t)\textbar{} =
fn(t) \leq f(t), et le théorème de convergence dominée assure que
la suite (croissante) (\int ~
Ifn)n\in\mathbb{N}~ converge vers
\int  I~f~; en particulier elle est
ma\\\\jmathmathmathmathorée.

Supposons en sens inverse que que la suite
(\int  Ifn)n\in\mathbb{N}~~ est
ma\\\\jmathmathmathmathorée par M. Soit J un segment inclus dans I~; on a donc 0
\leq\int  Jfn~
\leq\int  Ifn~ \leq M, mais d'autre
part, on a \forall~~t \in J,
\textbar{}fn(t)\textbar{} = fn(t) \leq f(t) et f est
intégrable sur le segment J puisqu'elle est continue par morceaux sur ce
segment. On a donc \int  J~f
= lim\\int ~
Jfn \leq M par le théorème de convergence dominée. Pour
tout segment J \subset~ I, on a \int  J~f \leq M
et f est positive, par définition même, elle est intégrable sur I, ce
qui achève la démonstration de l'équivalence

{[}
{[}
