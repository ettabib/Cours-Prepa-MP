\textbf{Warning: 
requires JavaScript to process the mathematics on this page.\\ If your
browser supports JavaScript, be sure it is enabled.}

\begin{center}\rule{3in}{0.4pt}\end{center}

{[}
{[}
{[}{]}
{[}

\subsubsection{4.2 Espaces métriques}

\paragraph{4.2.1 Distances}

Définition~4.2.1 Soit E un ensemble. On appelle distance sur E toute
application d : E \times E \rightarrow~ \mathbb{R}~^+ vérifiant pour tout x,y,z \in E

\begin{itemize}
\itemsep1pt\parskip0pt\parsep0pt
\item
  (i) d(x,y) = 0 \Leftrightarrow x = y (propriété de
  séparation)
\item
  (ii) d(x,y) = d(y,x) (propriété de symétrie)
\item
  (iii) d(x,z) \leq d(x,y) + d(y,z) (inégalité triangulaire)
\end{itemize}

On appelle espace métrique un couple (E,d) d'un ensemble E et d'une
distance d sur E.

Proposition~4.2.1 Soit d une distance sur E Alors

\forall~~x,y,z \in E, \textbar{}d(x,z) -
d(y,z)\textbar{}\leq d(x,y)

Démonstration On a d(x,z) - d(y,z) \leq d(x,y) d'après l'inégalité
triangulaire. En échangeant x et y, on a aussi d(y,z) - d(x,z) \leq d(x,y),
d'où le résultat.

Exemple~4.2.1 Sur tout ensemble, d(x,y) = \left
\ \cases 1&si
x\neq~y \cr 0&si x = y
\cr  \right . est une distance sur E
appelée la distance discrète. Sur K = \mathbb{R}~ ou K = \mathbb{C}, d(x,y) = \textbar{}x -
y\textbar{} est une distance appelée la distance usuelle. Sur
K^n on trouve classiquement trois distances utiles
d1(x,y) =\ \\sum
 i\textbar{}xi - yi\textbar{},
d2(x,y) =
\sqrt\\\sum
 i\textbar{}xi -
yi\textbar{}^2 et d\infty~(x,y)
= maxi\textbar{}xi~ -
yi\textbar{} si x =
(x1,\\ldots,xn~)
et y =
(y1,\\ldots,yn~).

Définition~4.2.2 On appelle boule ouverte de centre a de rayon r
\textgreater{} 0~: B(a,r) = \x \in
E∣d(a,x) \textless{} r\.

On appelle boule fermée de centre a de rayon r \textgreater{} 0~:
B'(a,r) = \x \in E∣d(a,x) \leq
r\.

On appelle sphère de centre a de rayon r \textgreater{} 0~: S(a,r) =
\x \in E∣d(a,x) =
r\

Définition~4.2.3 Soit (E,d) un espace métrique et dF la
restriction de d à F \times F. Alors dF est encore une distance sur
F appelée la distance induite par d.

Remarque~4.2.1 On a clairement BdF(a,r) =
Bd(a,r) \bigcap F et le résultat similaire pour les boules fermées,
si a \in F.

Définition~4.2.4 Soit
(E1,d1),\\ldots,(Ek,dk~)
des espaces métriques. Soit E = E1
\times⋯ \times Ek. On définit alors sur E une
distance produit par d(x,y) =\
maxidi(xi,yi) si x =
(x1,\\ldots,xk~)
et y =
(y1,\\ldots,yk~).

Définition~4.2.5

\begin{itemize}
\itemsep1pt\parskip0pt\parsep0pt
\item
  (i) Soit x \in E et A \subset~ E, A\neq~\varnothing~. On appelle
  distance de x à A le réel d(x,A) = inf~
  \d(x,a)∣a \in
  A\
\item
  (ii) A,B \subset~ E non vides. On appelle distance de A et B le réel d(A,B)
  = inf~
  \d(a,b)∣a \in A,b \in
  B\
\item
  (iii) On appelle diamètre de A \subset~ E, A\neq~\varnothing~, le
  nombre \delta(A) =\
  sup\d(a,a')∣a,a' \in
  A\ \in \mathbb{R}~ \cup\ + \infty~\~; on
  dit que A est bornée si \delta(A) \textless{} +\infty~.
\end{itemize}

Définition~4.2.6 Soit (E,d) et (F,\delta) deux espaces métriques. On appelle
isométrie de E sur F toute application f : E \rightarrow~ F bi\\\\jmathmathmathmathective qui conserve
la distance~:

\forall~~x,y \in E, \delta(f(x),f(y)) = d(x,y)

\paragraph{4.2.2 Topologie définie par une distance}

Définition~4.2.7 Soit (E,d) un espace métrique. On appelle topologie
définie sur E par la distance d l'ensemble des parties U de E (les
ouverts de la topologie) vérifiant

\forall~x \in U, \\exists~r
\textgreater{} 0,\quad B(x,r) \subset~ U

Démonstration C'est bien une topologie~: clairement E et \varnothing~ sont des
ouverts~; si U et U' sont des ouverts et x \in U \bigcap U', il existe r
\textgreater{} 0 et r' \textgreater{} 0 tels que B(x,r) \subset~ U et B(x,r') \subset~
U' et alors r0 = min~(r,r')
\textgreater{} 0 est tel que B(x,r0) \subset~ U \bigcap U'. Si les
Ui, i \in I sont des ouverts, soit x
\in\⋃ ~
i\inIUi. Il existe i0 tel que x \in
Ui0 puis r \textgreater{} 0 tel que B(x,r) \subset~
Ui0. On a alors B(x,r)
\subset~\⋃ ~
i\inIUi.

Proposition~4.2.2 Dans un espace métrique, toute boule ouverte est un
ouvert, toute boule fermée est un fermé.

Démonstration Soit x \in B(a,r) et \rho = r - d(a,x) \textgreater{} 0. Si y \in
B(x,\rho), on a d(a,y) \leq d(a,x) + d(x,y) \textless{} d(a,x) + \rho = r soit
B(x,\rho) \subset~ B(a,r). De même on montre que si
x∉B'(a,r) et si \rho = d(a,x) - r \textgreater{}
0, alors B(x,\rho) \subset~ E \diagdown B'(a,r). Donc E \diagdown B'(a,r) est ouvert et B'(a,r)
est fermé.

Remarque~4.2.2

\begin{itemize}
\itemsep1pt\parskip0pt\parsep0pt
\item
  (i) V \in V (a) \Leftrightarrow
  \exists~r \textgreater{} 0, B(a,r) \subset~ V
\item
  (ii) a \in A^o \Leftrightarrow
  \exists~r \textgreater{} 0, B(a,r) \subset~ A
\item
  (iii) \overlineA = \x \in
  E∣\forall~~r
  \textgreater{} 0, B(x,r) \bigcap
  A\neq~\varnothing~\
\item
  (iv) \mathrmFr~(A) =
  \x \in
  E∣\forall~~r
  \textgreater{} 0, B(x,r) \bigcap
  A\neq~\varnothing~\text et B(x,r)
  \bigcapcA\neq~\varnothing~\
\end{itemize}

Proposition~4.2.3 Soit (E,d) un espace métrique et F \subset~ E. Alors la
topologie induite sur F est la topologie définie par la distance
dF.

Démonstration On remarque que si a \in F, BdF(a,r) =
Bd(a,r) \bigcap F. Soit V un ouvert pour la topologie induite, soit
U ouvert de E tel que V = U \bigcap F. On a a \in U, donc il existe r
\textgreater{} 0 tel que Bd(a,r) \subset~ U. On a alors
BdF(a,r) = Bd(a,r) \bigcap F \subset~ U \bigcap F \subset~ U.
Inversement, soit V un ouvert pour la distance dF. Pour tout x
\in V , il existe rx \textgreater{} 0 tel que
BdF(x,rx) \subset~ V . On a alors V
= \⋃  ~x\inV
BdF(x,rx) (cette réunion contient V de
manière évidente et est contenue dans V car réunion de parties de V ).
On pose alors U =\ \⋃
 x\inV Bd(x,rx). C'est un ouvert de E et
on a V = U \bigcap F.

Remarque~4.2.3 Ceci montre que la topologie définie par la distance
dF ne dépend que de la topologie sur E et pas vraiment de la
distance d. Montrons de même que la topologie définie par la distance
produit ne dépend que des topologies sur les espaces et pas des
distances elles-mêmes

Proposition~4.2.4 Soit (E1,d1) et
(E2,d2) deux espaces métriques et (E1 \times
E2,\delta) l'espace métrique produit. Soit U \subset~ E1 \times
E2. Alors U est ouvert si et seulement si~

\forall~(a1,a2~) \in U,
\existsV 1 \in V (a1~),
\existsV 2~ \in V
(a2),\quad V 1 \times V 2 \subset~ U

Démonstration Supposons que U est ouvert pour la distance produit. Si
(a1,a2) \in U, il existe r \textgreater{} 0 tel que
B\delta((a1,a2),r) \subset~ U. Mais on a

\begin{align*}
B\delta((a1,a2),r)& =&
\(x1,x2)∣max(d1(x1,a1),d2(a2,r2~))
\textless{} r\\%& \\ &
=& Bd1(a1,r) \times
Bd2(a2,r) \%&
\\ \end{align*}

et donc V 1 = Bd1(a1,r) et V
2 = Bd2(a2,r) sont des voisinages
de a1 et a2 tels que V 1 \times V 2 \subset~
U. Inversement, si U vérifie cette propriété, soit
(a1,a2) \in U et soit V 1 \in V
(a1), V 2 \in V (a2) tels que V 1
\times V 2 \subset~ U. Il existe r1 \textgreater{} 0 et
r2 \textgreater{} 0 tels que
Bdi(ai,ri) \subset~ V i. Soit
r = min(r1,r2~)
\textgreater{} 0. On a

\begin{align*}
B\delta((a1,a2),r)& =&
Bd1(a1,r) \times
Bd2(a2,r) \%&
\\ & \subset~&
Bd1(a1,r1) \times
Bd2(a2,r2) \subset~ V 1 \times V
2 \subset~ U\%& \\
\end{align*}

donc U est un ouvert pour \delta, ce qui achève la démonstration.

Remarque~4.2.4 En particulier, si U1 et U2 sont des
ouverts de E1 et E2, alors U1 \times
U2 est un ouvert de E1 \times E2~; un tel
ouvert sera dit ouvert élémentaire.

\paragraph{4.2.3 Points isolés, points d'accumulation}

Soit tou\\\\jmathmathmathmathours F une partie de E et x \in\overlineF. On
sait que \forall~~V \in V (x) V \bigcap
F\neq~\varnothing~. On a alors deux possibilités suivant que
V \bigcap F peut être réduit à \x\ ou non.

Définition~4.2.8

\begin{itemize}
\itemsep1pt\parskip0pt\parsep0pt
\item
  (i) On dit que x \in F est point isolé de F, s'il existe V voisinage de
  x dans E tel que V \bigcap F = \x\
\item
  (ii) On dit que x \in E est point d'accumulation de F si pour tout
  voisinage V de x dans E, V \bigcap F
  \diagdown\x\\neq~\varnothing~.
\end{itemize}

Théorème~4.2.5 Soit E un espace métrique.

\begin{itemize}
\itemsep1pt\parskip0pt\parsep0pt
\item
  (i) x \in F est point isolé de F si et seulement
  si~\x\ est ouvert dans F
\item
  (ii) x \in E est point d'accumulation de F si et seulement si~pour tout
  voisinage V de x dans E, V \bigcap F est infini.
\end{itemize}

Démonstration (i) est tout à fait élémentaire et résulte de la
définition de la topologie induite. En ce qui concerne (ii), la partie (
⇐) est évidente. Montrons donc la partie ( \rigtharrow~). Soit x un point
d'accumulation de F, V un voisinage de x et r \textgreater{} 0 tel que
B(x,r) \subset~ V . Alors (B(x,r) \diagdown\x\) \bigcap
F\neq~\varnothing~. Soit x1 \in (B(x,r)
\diagdown\x\) \bigcap F. Si xn est supposé
construit, on pose rn = d(x,xn) \textgreater{} 0 et
on choisit xn+1 \in (B(x,rn)
\diagdown\x\) \bigcap
F\neq~\varnothing~. Alors la suite (d(x,xn)) est
strictement décroissante, ce qui montre que les xn sont deux à
deux distincts. Ils sont tous dans F et dans B(x,r) donc dans V .

\paragraph{4.2.4 Propriété de séparation}

Théorème~4.2.6 Soit E un espace métrique, a et b deux points distincts
de E. Alors il existe U ouvert contenant a et V ouvert contenant b tels
que U \bigcap V = \varnothing~.

Démonstration Soit r = 1 \over 3 d(a,b), U = B(a,r)
et V = B(b,r) conviennent.

Corollaire~4.2.7 Soit E un espace métrique et \Delta =
\(x,x) \in E \times E\. Alors \Delta est fermée
dans E \times E.

Démonstration Soit (a,b) \in E \times E \diagdown \Delta. On a donc
a\neq~b. Il existe U ouvert contenant a et V
ouvert contenant b tels que U \bigcap V = \varnothing~. Alors U \times V est un ouvert de E \times
E (élémentaire) et (U \times V ) \bigcap \Delta = \varnothing~, soit U \times V \subset~ E \times E \diagdown \Delta. Donc E \times E
\diagdown \Delta est voisinage de tous ses points et il est ouvert. Donc \Delta est
fermée.

\paragraph{4.2.5 Changement de distances}

Définition~4.2.9 Soit E un ensemble. On dit que deux distances
d1 et d2 sur E sont topologiquement équivalentes si
elles définissent la même topologie (il s'agit clairement d'une relation
d'équivalence).

Théorème~4.2.8 Soit E un ensemble, d et d' deux distances sur E. Ces
distances sont topologiquement équivalentes si et seulement si~elles
vérifient

\begin{itemize}
\itemsep1pt\parskip0pt\parsep0pt
\item
  (i) \forall~a \in E, \\forall~~r
  \textgreater{} 0, \exists~r' \textgreater{}
  0,\quad Bd'(a,r') \subset~ Bd(a,r)
\item
  (ii) \forall~a \in E, \\forall~~r'
  \textgreater{} 0, \exists~r \textgreater{}
  0,\quad Bd(a,r) \subset~ Bd'(a,r')
\end{itemize}

Démonstration Ces conditions sont évidemment nécessaires puisque les
boules ouvertes pour d doivent être des ouverts pour d' et
réciproquement. Supposons maintenant que (i) est vérifiée et soit U un
ouvert pour d. Soit a \in U. Il existe r \textgreater{} 0 tel que
Bd(a,r) \subset~ U. Alors \exists~r'
\textgreater{} 0,\quad Bd'(a,r') \subset~
Bd(a,r) \subset~ U. On en déduit que U est ouvert pour d', donc
Td \subset~Td'. De même (ii) traduit l'inclusion
Td' \subset~Td.

Définition~4.2.10 Soit E un ensemble. On dit que deux distances
d1 et d2 sur E sont équivalentes s'il existe \alpha~ et \beta~
strictement positifs tels que

\forall~~x,y \in E,\quad
\alpha~d1(x,y) \leq d2(x,y) \leq \beta~d1(x,y)

Proposition~4.2.9 Deux distances équivalentes sont topologiquement
équivalentes.

Démonstration On a d1(a,x) \textless{} r
\over \beta~ \rigtharrow~ d2(x,y) \textless{} r soit
Bd1(a, r \over \beta~ ) \subset~
Bd2(a,r). De même Bd2(a,\alpha~r) \subset~
Bd1(a,r).

Remarque~4.2.5 Soit d une distance sur E et posons d'(x,y)
= min~(1,d(x,y)). On vérifie facilement que d
est une distance sur E, que d et d' sont topologiquement équivalentes
(Bd(a,r) \subset~ Bd'(a,r) et
Bd'(a,min~( 1 \over
2 ,r)) \subset~ Bd(a,r)). Mais en général, d et d' ne sont pas
équivalentes (d' est tou\\\\jmathmathmathmathours bornée alors que d ne l'est pas en
général).

\paragraph{4.2.6 La droite numérique achevée}

On pose \overline\mathbb{R}~ = \mathbb{R}~
\cup\-\infty~,+\infty~\ muni de la relation d'ordre
évidente. Les intervalles ouverts sont donc les intervalles de la forme

\begin{itemize}
\itemsep1pt\parskip0pt\parsep0pt
\item
  (i) I ={]}a,b{[}= \x \in \mathbb{R}~∣a
  \textless{} x \textless{} b\ pour a,b
  \in\overline\mathbb{R}~
\item
  (ii) I ={]}a,+\infty~{]} = \x
  \in\overline\mathbb{R}~∣a
  \textless{} x\ ou I = {[}-\infty~,a{[}= \x
  \in\overline\mathbb{R}~∣x
  \textless{} a\
\item
  (iii) I = {[}-\infty~,+\infty~{]} = \overline\mathbb{R}~
\end{itemize}

Comme sur \mathbb{R}~, ces intervalles ouverts engendrent une topologie appelée la
topologie usuelle de \overline\mathbb{R}~. On a alors

\begin{itemize}
\item
  (i) si a \in \mathbb{R}~, V \in V (a) \Leftrightarrow
  \exists~\epsilon \textgreater{} 0,\quad
  {]}a - \epsilon,a + \epsilon{[}\subset~ V
\item
  (ii) si a = +\infty~,

  V \in V (+\infty~) \Leftrightarrow \exists~A
  \textgreater{} 0,\quad {]}A,+\infty~{]} \subset~ V
\item
  (iii) si a = -\infty~,

  V \in V (-\infty~) \Leftrightarrow \exists~A
  \textless{} 0,\quad {[}-\infty~,A{[}\subset~ V
\end{itemize}

Théorème~4.2.10 La topologie usuelle sur \overline\mathbb{R}~
est définie par une distance.

Démonstration Soit \phi : \overline\mathbb{R}~ \rightarrow~ {[}-1,1{]}
définie par \phi(x) = \left \
\cases  x \over
1+\textbar{}x\textbar{} &si x \in \mathbb{R}~ \cr 1 &si x = +\infty~
\cr -1 &si x = -\infty~ \cr 
\right .. L'application \phi est une bi\\\\jmathmathmathmathection strictement
croissante donc respecte les intervalles ouverts, donc les topologies
usuelles~: si U \subset~\overline\mathbb{R}~, U est ouvert dans
\overline\mathbb{R}~ si et seulement si~\phi(U) est ouvert dans
{[}-1,1{]}~; comme la topologie sur {[}-1,1{]} est définie par la
distance \textbar{}x - y\textbar{}, la topologie sur
\overline\mathbb{R}~ est définie par la distance d(x,y) =
\textbar{}\phi(x) - \phi(y)\textbar{} (pour cette distance, \phi devient une
isométrie).

Remarque~4.2.6 On vérifie immédiatement que la topologie usuelle de
\overline\mathbb{R}~ induit sur \mathbb{R}~ la topologie usuelle de \mathbb{R}~.

{[}
{[}
{[}
{[}
