\textbf{Warning: 
requires JavaScript to process the mathematics on this page.\\ If your
browser supports JavaScript, be sure it is enabled.}

\begin{center}\rule{3in}{0.4pt}\end{center}

{[}
{[}{]}
{[}

\subsubsection{16.1 Notions générales}

\paragraph{16.1.1 Solutions d'une équation différentielle}

Définition~16.1.1 Soit E un espace vectoriel normé de dimension finie, n
≥ 1, W un ouvert de \mathbb{R}~ \times E^n+1 et G : W \rightarrow~ E' une application.
On appelle solution de l'équation différentielle
G(t,y,y',\\ldots,y^(n)~)
= 0 tout couple (I,\phi) d'un intervalle I de \mathbb{R}~ et d'une application \phi : I
\rightarrow~ E de classe C^n telle que

\begin{align*} \forall~~t \in I,&
&
(t,\phi(t),\phi'(t),\\ldots,\phi^(n)~(t))
\in W \%& \\ & & \text
et
G(t,\phi(t),\phi'(t),\\ldots,\phi^(n)~(t))
= 0\%& \\
\end{align*}

On dira alors que n est l'ordre de l'équation différentielle.

Remarque~16.1.1 Dans le cas particulier où E' = E et où
G(t,y0,\\ldots,yn~)
= yn -
F(t,y0,\\ldots,yn-1~)
avec F application d'un ouvert U de \mathbb{R}~ \times E^n dans E, on
obtient une équation différentielle de la forme y^(n) =
F(t,y,y',\\ldots,y^(n-1)~).
On dira qu'une telle équation est sous forme normale. Une solution d'une
telle équation est donc un couple (I,\phi) où \phi : I \rightarrow~ E est de classe
C^n et vérifie

\begin{align*} \forall~~t \in I,&
&
(t,\phi(t),\\ldots,\phi^(n-1)~(t))
\in U \%& \\ & & \text
et \phi^(n)(t) =
F(t,\phi(t),\phi'(t),\\ldots,\phi^(n-1)~(t))\%&
\\ \end{align*}

Par la suite on s'intéressera plus particulièrement aux équations
différentielles sous forme normale. Il est parfois possible de passer
d'une équation sous forme générale à une équation sous forme normale en
résolvant l'équation
G(t,y0,\\ldots,yn~)
= 0 sous la forme yn =
F(t,y0,\\ldots,yn-1~).
A cet égard, le théorème des fonctions implicites peut rendre de grands
services.

\paragraph{16.1.2 Type de problèmes}

Nous distinguerons par la suite deux types de problèmes concernant les
équations différentielles. Le premier type de problème est appelé le
problème à condition initiale (ou problème de Cauchy-Lipschitz), le
second problème à conditions aux limites (ou conditions au bord).

Problème 1 (à conditions initiales). On considère une équation
différentielle sous forme normale y^(n) =
F(t,y,y',\\ldots,y^(n-1)~)
où F est une application d'un ouvert U de \mathbb{R}~ \times E^n dans E et
on se donne
(t0,y0,\\ldots,yn-1~)
\in U. Peut-on trouver une solution (I,\phi) de cette équation différentielle
telle que t0 \in I, \phi(t0) =
y0,\\ldots,\phi^(n-1)(t0~)
= yn-1~? Si oui, a-t-on en un certain sens unicité d'une
solution~?

Problème 2 (à conditions aux limites). On considère une équation
différentielle sous forme normale y^(n) =
F(t,y,y',\\ldots,y^(n-1)~)
où F est une application d'un ouvert U de \mathbb{R}~ \times E^n dans E et
on se donne a et b \in \mathbb{R}~. Peut-on trouver une solution (I,\phi) de cette
équation différentielle vérifiant des équations
G1(a,\phi(a),\\ldots,\phi^(n-1)~(a))
= 0 et
G2(b,\phi(b),\\ldots,\phi^(n-1)~(b))
= 0, où G1 et G2 sont deux applications de \mathbb{R}~ \times
E^n respectivement dans deux espaces vectoriels normés
E1 et E2.

Le problème avec conditions aux limites est beaucoup plus difficile et a
des réponses beaucoup plus complexes que le problème avec conditions
initiales. Nous ne le traiterons donc pas en dehors d'exercices
particuliers et nous intéresserons presque exclusivement au problème à
conditions initiales.

\paragraph{16.1.3 Réduction à l'ordre 1}

Considérons une équation différentielle sous forme normale
y^(n) =
f(t,y,y',\\ldots,y^(n-1)~)
où f est une application d'un ouvert U de \mathbb{R}~ \times E^n dans E et
soit (I,\phi) une solution de l'équation différentielle. Posons
\phi1(t) =
\phi(t),\\ldots,\phin~(t)
= \phi^(n-1)(t) et \Phi(t) =
(\phi1(t),\\ldots,\phin~(t))
=
(\phi(t),\phi'(t),\\ldots,\phi^(n-1)~(t)).
Alors (I,\Phi) est de classe \mathcal{C}^1 et on a

\begin{align*} \Phi'(t)& =& \left
(\phi'(t),\\ldots,\phi^(n)~(t)\right
) \%& \\ & =& \left
(\phi'(t),\\ldots,\phi^(n-1)(t),f(t,\phi(t),\\\ldots,\phi^(n-1)~(t))\right
) \%& \\ & =& \left
(\phi2(t),\\ldots,\phin(t),f(t,\phi1(t),\\\ldots,\phin~(t))\right
) = F(t,\Phi(t))\%& \\
\end{align*}

si l'on définit F : U \rightarrow~ E^n par
F(t,(y1,\\ldots,yn~))
=
(y2,\\ldots,yn,F(t,y1,\\\ldots,yn~)).
Donc (I,\Phi) est solution de l'équation différentielle Y ' = F(t,Y ).

Inversement, donnons nous une solution (I,\Phi) de l'équation
différentielle Y ' = F(t,Y ) où F : U \rightarrow~ E^n est définie par
F(t,(y1,\\ldots,yn~))
=
(y2,\\ldots,yn,f(t,y1,\\\ldots,yn~)).
Posons \Phi(t) =
(\phi1(t),\\ldots,\phin~(t)).
On a donc

\begin{align*} \Phi'(t)& =&
(\phi1'(t),\\ldots,\phin-1'(t),\phin~'(t))
\%& \\ & =& F(t,\Phi(t)) =
(\phi2(t),\\ldots,\phin(t),f(t,\phi1(t),\\\ldots,\phin~(t)))\%&
\\ \end{align*}

On en déduit que pour i \in {[}1,n - 1{]} on a \phii'(t) =
\phii+1(t) et une récurrence évidente montre que pour i \in
{[}2,n{]}, \phii(t) = \phi1^(i-1)(t). Mais alors la
dernière équation se traduit par \phi1^(n)(t) =
(\phi^(n-1))'(t) = \phin'(t) =
f(t,\phi1(t),\\ldots,\phin~(t))
=
f(t,\phi1(t),\phi1'(t),\\ldots,\phi1^(n-1)~(t)),
si bien que (I,\phi1) est de classe C^n et solution de
l'équation différentielle y^(n) =
f(t,y,y',\\ldots,y^(n-1)~).
On en déduit donc le théorème suivant

Théorème~16.1.1 Soit U un ouvert de U \times E^n et f : U \rightarrow~ E.
Soit F : U \rightarrow~ E^n définie par
F(t,(y1,\\ldots,yn~))
=
(y2,\\ldots,yn,F(t,y1,\\\ldots,yn~)).
Alors l'application (I,\phi)\mapsto~(I,\Phi) définie par
\Phi(t) =
(\phi(t),\\ldots,\phi^(n-1)~(t))
est une bi\\\\jmathmathmathmathection de l'ensemble des solutions de l'équation
différentielle d'ordre n, y^(n) =
f(t,y,y',\\ldots,y^(n-1)~),
sur l'ensemble des solutions de l'équation différentielle d'ordre un Y '
= F(t,Y ).

Remarque~16.1.2 En ce qui concerne le type de problème étudié, il est
clair que cette bi\\\\jmathmathmathmathection préserve les problèmes à conditions initiales.
Autrement dit (I,\phi) est une solution de y^(n) =
f(t,y,y',\\ldots,y^(n-1)~)
vérifiant les conditions initiales \phi(t0) =
y0,\\ldots,\phi^(n-1)(t0~)
= yn-1 si et seulement si~(I,\Phi) est une solution de Y ' =
F(t,Y ) vérifiant la condition initiale \Phi(t0) = Y 0
avec Y 0 =
(y0,\\ldots,yn-1~).
Nous pourrons donc par la suite, pour ce qui concerne les problèmes
d'existence et d'unicité du problème à conditions initiales, nous borner
à l'étude des équations différentielles d'ordre 1.

\paragraph{16.1.4 Equivalence avec une équation intégrale}

Théorème~16.1.2 Soit E un espace vectoriel normé de dimension finie, U
un ouvert de \mathbb{R}~ \times E, F : U \rightarrow~ E continue, (t0,y0) \in U,
I un intervalle de \mathbb{R}~ contenant t0 et \phi une application
continue de I dans E. Alors on a équivalence de

\begin{itemize}
\itemsep1pt\parskip0pt\parsep0pt
\item
  (i) (I,\phi) est une solution de l'équation différentielle y' = F(t,y)
  vérifiant \phi(t0) = y0
\item
  (ii) \forall~t \in I, \phi(t) = y0~
  +\int  t0^t~F(u,\phi(u))
  du.
\end{itemize}

Démonstration Supposons (i) vérifié. Alors, comme \phi est de classe
\mathcal{C}^1, on a

\phi(t) = \phi(t0) +\int ~
t0^t\phi'(u) du = y 0
+\int  t0^t~F(u,\phi(u))
du

ce qui montre que (i) \rigtharrow~(ii). Inversement supposons (ii) vérifié. Il est
clair que \phi(t0) = y0. De plus comme
u\mapsto~F(u,\phi(u)) est continue,
t\mapsto~\int ~
t0^tF(u,\phi(u)) du est de classe \mathcal{C}^1
et sa dérivée est F(t,\phi(t))~; on en déduit que \phi est de classe
\mathcal{C}^1 et que \phi'(t) = F(t,\phi(t)), ce qui achève la démonstration.

\paragraph{16.1.5 Le lemme de Gronwall}

On utilisera par la suite le lemme suivant~:

Lemme~16.1.3 (Gronwall). Soit c un réel positif, g : {[}a,b{[}\rightarrow~ \mathbb{R}~
continue positive. Soit u : I \rightarrow~ \mathbb{R}~ continue telle que
\forall~~x \in {[}a,b{[}, \textbar{}u(x)\textbar{}\leq c
+\int ~
a^x\textbar{}u(t)\textbar{}g(t) dt. Alors
\forall~~x \in {[}a,b{[}, \textbar{}u(x)\textbar{}\leq
cexp (\\int ~
a^xg(t) dt).

Démonstration Posons v(x) = c +\int ~
a^x\textbar{}u(t)\textbar{}g(t) dt. Comme u et g sont
continues, v est de classe \mathcal{C}^1 et on a v'(x) =
\textbar{}u(x)\textbar{}g(x) \leq v(x)g(x) puisque
\textbar{}u(x)\textbar{}\leq v(x) et g(x) ≥ 0 par hypothèse. Soit w(x) =
v(x)exp (-\\int ~
a^xg(t) dt). On a alors

\begin{align*} w'(x)& =&
v'(x)exp (-\\int ~
a^xg(t) dt) - v(x)g(x)exp~
(-\int  a^x~g(t) dt)\%&
\\ & =& (v'(x) -
v(x)g(x))exp (-\\int ~
a^xg(t) dt) \leq 0 \%& \\
\end{align*}

On en déduit que w est décroissante, et donc
\forall~~x \in {[}a,b{[}, w(x) \leq w(a). Mais w(a) = v(a)
= c. On a donc, pour x \in {[}a,b{[}, \textbar{}u(x)\textbar{}\leq v(x) \leq
cexp (\\int ~
a^xg(t) dt), ce qu'on voulait démontrer.

Remarque~16.1.3 De la même fa\ccon on montre que si
\forall~~x \in{]}b,a{]}, \textbar{}u(x)\textbar{}\leq c
+\int ~
x^a\textbar{}u(t)\textbar{}g(t) dt, alors
\forall~~x \in{]}b,a{]}, \textbar{}u(x)\textbar{}\leq
cexp (\\int ~
x^ag(t) dt).

Remarque~16.1.4 Le cas c = 0 \\\\jmathmathmathmathouera un rôle crucial dans les
démonstrations d'unicité. On obtient alors la nullité de u sur
l'intervalle en question.

{[}
{[}
