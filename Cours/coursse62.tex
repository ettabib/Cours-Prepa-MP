\textbf{Warning: 
requires JavaScript to process the mathematics on this page.\\ If your
browser supports JavaScript, be sure it is enabled.}

\begin{center}\rule{3in}{0.4pt}\end{center}

{[}
{[}
{[}{]}
{[}

\subsubsection{10.3 Intégrales dépendant d'un paramètre}

\paragraph{10.3.1 Position du problème}

Soit E un espace métrique, a,b \in \mathbb{R}~, E' un espace vectoriel normé complet
et f : E \times {[}a,b{]} \rightarrow~ E', (x,t)\mapsto~f(x,t). On
suppose que \forall~~x \in E, l'application
t\mapsto~f(x,t) est réglée de {[}a,b{]} dans E'. On
peut donc définir une application F : E \rightarrow~ E' par F(x)
=\int  a^b~f(x,t) dt. Nous allons
nous intéresser ici aux propriétés de la fonction F (continuité,
dérivabilité, intégration) en fonction de celles de f.

\paragraph{10.3.2 Continuité}

Théorème~10.3.1~(Continuité par convergence dominée) Soit E un espace
métrique, I un intervalle de \mathbb{R}~, f : E \times I \rightarrow~ \mathbb{C},
(x,t)\mapsto~f(x,t). On suppose (i) pour chaque x \in
E, l'application t\mapsto~f(x,t) est continue par
morceaux sur I (ii) pour chaque t \in I, l'application
x\mapsto~f(x,t) est continue sur E (iii) il existe
une fonction \phi : I \rightarrow~ \mathbb{R}~^+, intégrable, telle que
\forall~~(x,t) \in E \times I, \textbar{}f(x,t)\textbar{}\leq
\phi(t) (hypothèse de domination). Alors, pour tout x \in E, la fonction
t\mapsto~f(x,t) est intégrable sur I et
l'application F : E \rightarrow~ \mathbb{C},
x\mapsto~\int ~
If(x,t) dt est continue sur E.

Démonstration L'intégrabilité de t\mapsto~f(x,t) est
claire avec la ma\\\\jmathmathmathmathoration \textbar{}f(x,t)\textbar{}\leq \phi(t). Soit alors x
\in E et (xn) une suite de E de limite x. Posons gn(t)
= f(xn,t) et g(t) = f(x,t). La suite (gn) est une
suite de fonctions continues par morceaux sur I qui converge vers g
continue par morceaux et on a \textbar{}gn\textbar{}\leq \phi avec \phi
intégrable~; le théorème de convergence dominée assure que
lim\\int ~
Ign =\int  I~g soit
encore limn\rightarrow~+\infty~F(xn~) =
F(x). Donc F est bien continue.

Remarque~10.3.1 Pour montrer que F est continue, il suffit de montrer
que sa restriction à tout compact K contenu dans E est continue, donc
qu'à tout compact K contenu dans E, on peut associer une fonction
\phiK intégrable telle que \forall~~(x,t) \in K \times
I, \textbar{}f(x,t)\textbar{}\leq \phiK(t).

Corollaire~10.3.2 Soit U un ouvert de \mathbb{R}~^n, a,b \in \mathbb{R}~ et f : U \times
{[}a,b{]} \rightarrow~ \mathbb{C} continue, (x,t)\mapsto~f(x,t). Alors
l'application F : U \rightarrow~ \mathbb{C} définie par F(x) =\int ~
a^bf(x,t) dt est continue.

Démonstration Soit x0 \in U et r \textgreater{} 0 tel que la
boule fermée B'(x0,r) soit contenue dans U. La fonction f est
continue sur le compact B'(x0,r) \times {[}a,b{]}, donc elle y est
bornée~: soit M ≥ 0 tel que \forall~~(x,t) \in
B'(x0,r) \times {[}a,b{]}, \textbar{}f(x,t)\textbar{}\leq M. Comme la
fonction constante t\mapsto~M est intégrable sur
{[}a,b{]} et bien entendue indépendante de x, le théorème de continuité
par convergence dominée montre que F est continue sur B'(x0,r)
et en particulier qu'elle est continue au point x0.

Exemple~10.3.1 Soit 0 \textless{} a \textless{} b \textless{} +\infty~ et soit
\Gammaa,b(x) =\int ~
a^bt^x-1e^-t dt. On a ici, f(x,t) =
t^x-1e^-t = exp~ (-t - (x
- 1)log~ t) qui est une fonction continue de \mathbb{R}~
\times {[}a,b{]} dans \mathbb{R}~ (composée de fonctions continues). On en déduit que
\Gammaa,b est continue sur \mathbb{R}~.

\paragraph{10.3.3 Dérivabilité}

Nous supposerons ici que E = I intervalle de \mathbb{R}~. Soit t0 \in
{[}a,b{]}~; lorsque l'application
x\mapsto~f(x,t0) est dérivable en un point
x0 \in I, sa dérivée au point x0 sera notée  \partial~f
\over \partial~x (x0,t0).

Théorème~10.3.3~(Dérivabilité par convergence dominée) Soit J un
intervalle de \mathbb{R}~, I un intervalle de \mathbb{R}~, f : J \times I \rightarrow~ \mathbb{C}, (x,t) \rightarrow~ f(x,t)
continue, admettant une dérivée partielle par rapport à x,
(x,t)\mapsto~ \partial~f \over \partial~x (x,t),
continue sur J \times I. On suppose que pour tout x \in E, la fonction
t\mapsto~f(x,t) est intégrable sur I et qu'il existe
une fonction \phi : I \rightarrow~ \mathbb{R}~^+, intégrable, telle que
\forall~~(x,t) \in J \times I, \textbar{} \partial~f
\over \partial~x (x,t)\textbar{}\leq \phi(t) (hypothèse de
domination). Alors, l'application F : J \rightarrow~ \mathbb{C},
x\mapsto~\int ~
If(x,t) dt est de classe \mathcal{C}^1 sur J et

\forall~x \in J, F'(x) =\\int ~
I \partial~f \over \partial~x (x,t) dt

Démonstration L'intégrabilité de t\mapsto~ \partial~f
\over \partial~x (x,t) est claire avec la ma\\\\jmathmathmathmathoration
\textbar{} \partial~f \over \partial~x (x,t)\textbar{}\leq \phi(t). Soit
alors x \in J et xn une suite de J
\diagdown\x\ de limite x. Posons
gn(t) = f(x,t)-f(xn,t) \over
x-xn et g(t) = \partial~f \over \partial~x (x,t). La
suite (gn) est une suite de fonctions continues sur I qui
converge vers g continue. L'inégalité des accroissements finis assure
que \textbar{}f(x,t) - f(xn,t)\textbar{}\leq\textbar{}x -
xn\textbar{}supy\in{]}x,xn{[}~\left
\textbar{} \partial~f \over \partial~x (y,t)\right
\textbar{}\leq\textbar{}x - xn\textbar{}\phi(t), d'où

\textbar{}gn(t)\textbar{} = \left \textbar{}
f(x,t) - f(xn,t) \over x - xn
\right \textbar{}\leq \phi(t)

avec \phi intégrable. Le théorème de convergence dominée assure alors que
lim\\int ~
Ign =\int  I~g, soit
encore que limn\rightarrow~+\infty~~
F(xn)-F(x) \over xn-x
=\int  I \partial~f \over \partial~x~
(x,t) dt. Comme la suite (xn) est quelconque, on a

limt\rightarrow~x~ F(t) - F(x)
\over t - x =\int  I~
\partial~f \over \partial~x (x,t) dt

donc F est dérivable et F'(x) =\int ~
I \partial~f \over \partial~x (x,t) dt. La continuité de F'
relève du théorème précédent relatif à la continuité d'une intégrale
dépendant d'un paramètre, la fonction étant dominée indépendamment du
paramètre.

Corollaire~10.3.4 Soit J un intervalle de \mathbb{R}~, I un intervalle de \mathbb{R}~, f : J
\times I \rightarrow~ \mathbb{C}, (x,t)\mapsto~f(x,t) continue, admettant des
dérivées partielles par rapport à x, (x,t)\mapsto~
\partial~^if \over \partial~x^i (x,t), continues
sur J \times I, i =
1,\\ldots~,k. On
suppose que pour tout x \in E, la fonction
t\mapsto~f(x,t) est intégrable sur I et qu'il existe
des fonctions
\phi1,\\ldots,\phik~
: I \rightarrow~ \mathbb{R}~^+, intégrables, telles que
\forall~(x,t) \in J \times I, \textbar{} \partial~^i~f
\over \partial~x^i (x,t)\textbar{}\leq \phii(t)
(hypothèses de domination). Alors, l'application F : J \rightarrow~ \mathbb{C},
x\mapsto~\int ~
If(x,t) dt est de classe C^k sur J et

\forall~i \in {[}1,k{]}, \\forall~~x \in
J, F^(i)(x) =\int  I~
\partial~^if \over \partial~x^i (x,t) dt

Démonstration Récurrence évidente à partir du théorème précédent

Théorème~10.3.5 Soit I un intervalle de \mathbb{R}~, a,b \in \mathbb{R}~ et f : I \times {[}a,b{]}
\rightarrow~ \mathbb{C}, (x,t)\mapsto~f(x,t). On suppose que (i) Pour
chaque x \in I, l'application t\mapsto~f(x,t) est
continue par morceaux sur {[}a,b{]} (ii) Pour chaque (x,t) \in I \times
{[}a,b{]}, f admet une dérivée partielle par rapport à x,  \partial~f
\over \partial~x (x,t) et que l'application
(x,t)\mapsto~ \partial~f \over \partial~x (x,t)
est continue. Alors F : I \rigtharrow~ \mathbb{C},
x\mapsto~\int ~
a^bf(x,t) dt est de classe \mathcal{C}^1 et
\forall~x0 \in I, F'(x0~)
=\int  a^b~ \partial~f
\over \partial~x (x0,t) dt.

Démonstration Il suffit, comme dans le théorème correspondant de
continuité pour une intégrale dépendant d'un paramètre sur un segment,
de prendre x0 \in I, un segment K, voisinage de x0
dans I, et d'utiliser le fait que la fonction
(x,t)\mapsto~ \partial~f \over \partial~x (x,t)
est continue, donc bornée par un certain M sur le compact K \times {[}a,b{]}.
La fonction constante M ainsi introduite est intégrable sur le segment
{[}a,b{]} et fournit ainsi une fonction dominante de la fonction
(x,t)\mapsto~ \partial~f \over \partial~x (x,t)
sur K \times {[}a,b{]}. Par le théorème de dérivation par convergence
dominée, F est dérivable sur K et en particulier elle est dérivable au
point x0 avec F'(x0) =\int ~
a^b \partial~f \over \partial~x (x0,t) dt.

Exemple~10.3.2 Soit 0 \textless{} a \textless{} b \textless{} +\infty~ et soit
\Gammaa,b(x) =\int ~
a^bt^x-1e^-t dt. On a ici, f(x,t) =
t^x-1e^-t = exp~ (-t + (x
- 1)log~ t) qui admet une dérivée par rapport à
x,  \partial~f \over \partial~x (x,t) = log~
t t^x-1e^-t qui est une fonction continue de \mathbb{R}~ \times
{[}a,b{]} dans \mathbb{R}~ (composée de fonctions continues). On en déduit que
\Gammaa,b est de classe \mathcal{C}^1 sur \mathbb{R}~ et que
\Gammaa,b'(x) =\int ~
a^bt^x-1 log~ t
e^-t dt. Une récurrence évidente montrera alors que
\Gammaa,b est de classe C^\infty~ et que
\Gammaa,b^(n)(x) =\int ~
a^bt^x-1(log~
t)^ne^-t dt.

\paragraph{10.3.4 Théorème de Fubini sur un produit de segments}

Théorème~10.3.6~(Fubini sur un produit de segments) Soit f : {[}a,b{]} \times
{[}c,d{]} \rightarrow~ \mathbb{C} continue. Alors

\int  a^b~\left
(\int  c^d~f(x,y)
dy\right ) dx =\int ~
c^d\left (\int ~
a^bf(x,y) dx\right ) dy

Démonstration On va démontrer que \forall~~t \in
{[}a,b{]}, F(t) = G(t) où F(t) =\int ~
a^t\left (\int ~
c^df(x,y) dy\right ) dx et G(t)
=\int  c^d~\left
(\int  a^t~f(x,y)
dx\right ) dy. Pour t = b, on aura alors le résultat
voulu. Comme F(a) = G(a) = 0, il suffit de démontrer que F et G sont
dérivables et que F' = G'. Mais on a F(t) =\\int
 a^t\phi(x) dx avec \phi(x) =\int ~
c^df(x,y) dy. Le théorème de continuité des intégrales
dépendant d'un paramètre sur un segment (conséquence du théorème de
continuité par convergence dominée) assure que \phi est continue, donc que
F est dérivable et que F'(t) = \phi(t) =\int ~
c^df(t,y) dy.

On a aussi G(t) =\int  c^d~\psi(t,y)
dy avec \psi(t,y) =\int  a^t~f(x,y)
dx. Comme x\mapsto~f(x,y) est continue,
t\mapsto~\psi(t,y) est de classe C^1 et
\partial~\psi\over \partial~t (t,y) = f(t,y). Mais f étant continue sur
le compact {[}a,b{]} \times {[}c,d{]}, elle y est bornée et on a donc

\forall~~(t,y) \in {[}a,b{]} \times {[}c,d{]},
\left \textbar{}\partial~\psi\over \partial~t
(t,y)\right \textbar{}\leq\\textbar{}
f\\textbar{}\infty~

qui est une fonction (constante) de la variable y, intégrable sur
{[}c,d{]} et indépendante de t. Le théorème de dérivation par
convergence dominée assure que l'application G :
t\mapsto~\int ~
c^d\psi(t,y) dy est dérivable et que G'(t)
=\int ~
c^d\partial~\psi\over \partial~t (t,y) dy, soit encore
G'(t) =\int  c^d~f(t,y) dy = \phi(t)
= F'(t), ce qui achève la démonstration.

\paragraph{10.3.5 Intégrales sur un pavé ou un rectangle}

Définition~10.3.1 On appelle pavé de \mathbb{R}~^2 (resp. rectangle de
\mathbb{R}~^2) toute partie de \mathbb{R}~^2 de la forme {[}a,b{]} \times
{[}c,d{]} (resp I \times I' où I et I' sont des intervalles de \mathbb{R}~).

En appliquant le théorème de Fubini pour une fonction continue sur un
produit de segments, on est amené à donner la définition suivante~:

Définition~10.3.2 Soit P = {[}a,b{]} \times {[}c,d{]} un pavé de
\mathbb{R}~^2 et f : P \rightarrow~ \mathbb{C} une fonction continue. On appelle intégrale
de la fonction f sur le pavé P le nombre complexe noté indifféremment
\int  \\int  P~f
ou \int  \\int ~
Pf(x,y) dx dy défini par

\int  \\int  P~f
=\int  \\int ~
Pf(x,y) dx dy =\int ~
a^b\left (\int ~
c^df(x,y) dy\right ) dx
=\int  c^d~\left
(\int  a^b~f(x,y)
dx\right ) dy

On peut alors répéter les définitions et résultats qui nous ont permis
de définir les fonctions intégrables sur un intervalle à partir de
l'intégrale sur un segment, pour définir des fonctions intégrables sur
un rectangle à partir de la notion de intégrale sur un pavé. On donnera
donc les définitions et propriétés suivantes sans commentaire ou
démonstration.

\begin{itemize}
\itemsep1pt\parskip0pt\parsep0pt
\item
  Soit R un rectangle et f : R \rightarrow~ \mathbb{R}~ continue positive. On dit que f est
  intégrable sur R s'il existe M ≥ 0 tel que pour tout pavé P \subset~ R on ait
  \int  \\int  P~f
  \leq M. On pose alors \int ~
  \int  R~f =\
  supP\subset~Rf~; on montre que si (Pn)n\in\mathbb{N}~ est
  une suite croissante de pavés contenus dans R dont la réunion est R,
  alors \int  \\int ~
  Rf =\
  limn\rightarrow~+\infty~\int ~
  \int  Pn~f.
\item
  Soit R un rectangle et f : R \rightarrow~ \mathbb{C} continue. On dit que f est intégrable
  sur R si la fonction continue positive \textbar{}f\textbar{} est
  intégrable sur R~; on montre alors que si (Pn)n\in\mathbb{N}~
  est une suite croissante de pavés contenus dans R dont la réunion est
  R, la suite \left (\int ~
  \int ~
  Pnf\right )n\in\mathbb{N}~ converge et
  que sa limite est indépendante du choix de la suite
  (Pn)n\in\mathbb{N}~~; on pose donc \\int
   \int  R~f
  = limn\rightarrow~+\infty~~\\int
   \int  Pn~f.
\item
  L'ensemble des fonctions continues de R dans \mathbb{C} intégrables sur R est
  un sous-espace vectoriel de l'espace des fonctions continues et
  l'application f\mapsto~\\int
   \int  R~f est linéaire.
\item
  Si un rectangle R est la réunion de deux rectangles R1 et
  R2 ne se rencontrant que suivant un de leurs côtés, alors f
  est intégrable sur R si et seulement si elle est intégrable sur
  R1 et R2 et dans ce cas, \\int
   \int  R~f
  =\int  \\int ~
  R1f +\int ~
  \int  R2~f.
\end{itemize}

On utilisera plusieurs fois le lemme suivant

Lemme~10.3.7 Soit R et R' deux rectangles tels que R \subset~ R' et soit f : R'
\rightarrow~ \mathbb{C} continue et intégrable sur R'. Alors f est intégrable sur R et

\left \textbar{}\int ~
\int  R'f -\\int ~
\int  R~f\right
\textbar{}\leq\int  \\int ~
R'\textbar{}f\textbar{}-\int ~
\int  R~\textbar{}f\textbar{}

Démonstration Tout pavé P inclus dans R est inclus dans R' et donc
supP\subset~R\\int ~
\int ~
P\textbar{}f\textbar{}\leq\
supP\subset~R'\int ~
\int  P~\textbar{}f\textbar{}
=\int  \\int ~
R'\textbar{}f\textbar{} \textless{} +\infty~ ce qui garantit
l'intégrabilité de f sur R. De plus (avec quelques conventions
d'écritures évidentes pour l'intégrale sur la différence de deux
rectangles)

\left \textbar{}\int ~
\int  R'f -\\int ~
\int  R~f\right
\textbar{} = \left \textbar{}\int ~
\int  R'\diagdownR~f\right
\textbar{}\leq\int  \\int ~
R'\diagdownR\textbar{}f\textbar{}\leq\int ~
\int ~
R'\textbar{}f\textbar{}-\int ~
\int  R~\textbar{}f\textbar{}

Lemme~10.3.8 Soit R = I \times I' un rectangle, f : R \rightarrow~ \mathbb{C} une fonction
continue, intégrable sur R. Soit (Jn)n\in\mathbb{N}~ et
(Kn)n\in\mathbb{N}~ des suites croissantes de segments dont les
réunions sont respectivement I et I'. Alors

\int  \\int  R~f
= limn\rightarrow~+\infty~~\\int
 \int  I\timesKn~f
= limn\rightarrow~+\infty~~\\int
 \int  Jn\timesI'~f

Démonstration Premier cas~: f est à valeurs réelles positives. On a
alors

\int  \\int ~
Jn\timesKnf \leq\int ~
\int  I\timesKn~f
\leq\int  \\int ~
I\timesI'f

et comme \int  \\int ~
I\timesI'f =\
limn\rightarrow~+\infty~\int ~
\int  Jn\timesKn~f (les
Jn \times Kn forment une suite croissante de pavés dont
la réunion est I \times I'), on a \int ~
\int  I\timesI'~f =\
limn\rightarrow~+\infty~\int ~
\int  I\timesKn~f. On démontre
l'autre formule de manière similaire.

Deuxième cas~: f est à valeurs complexes. On remarque que, d'après le
lemme précédent

\left \textbar{}\int ~
\int  I\timesI'~f -\\int
 \int ~
I\timesKnf\right
\textbar{}\leq\int  \\int ~
I\timesI'\textbar{}f\textbar{}-\int ~
\int ~
I\timesKn\textbar{}f\textbar{}

qui tend vers 0 d'après le premier cas. Donc \\int
 \int  I\timesI'~f
= limn\rightarrow~+\infty~~\\int
 \int  I\timesKn~f. On démontre
l'autre formule de manière similaire.

\paragraph{10.3.6 Théorème de Fubini sur un produit d'intervalles}

Lemme~10.3.9 Soit I' un intervalle de \mathbb{R}~, f : {[}a,b{]} \times I' \rightarrow~ \mathbb{C}
continue. On fait les hypothèses suivantes~:

\begin{itemize}
\itemsep1pt\parskip0pt\parsep0pt
\item
  pour tout x \in {[}a,b{]}, y\mapsto~f(x,y) est
  intégrable sur I'
\item
  l'application g :
  x\mapsto~\int ~
  I'f(x,y) dy est continue par morceaux sur {[}a,b{]}
\item
  f est intégrable sur le rectangle {[}a,b{]} \times I'
\end{itemize}

Alors \int  \\int ~
{[}a,b{]}\timesI'f =\int ~
a^bg =\int ~
a^b\left (\int ~
I'f(x,y) dy\right )dx.

Démonstration Soit Kn une suite croissante de segments dont la
réunion est I'. On sait que

\int  \\int ~
{[}a,b{]}\timesI'f =\
limn\rightarrow~+\infty~\int ~
\int  {[}a,b{]}\timesKn~f
= limn\rightarrow~+\infty~~\\int
 a^b\left (\\int
 Knf(x,y) dy\right )dx
= limn\rightarrow~+\infty~~\\int
 a^bg n(x) dx

en posant gn(x) =\int ~
Knf(x,y) dy~; le théorème de continuité des intégrales
dépendant d'un paramètre sur un segment nous garantit que gn
est continue~; de plus, comme la fonction
y\mapsto~f(x,y) est intégrable sur I', la suite
(gn)n\in\mathbb{N}~ converge simplement vers g et on peut écrire

\begin{align*} \textbar{}gn+1(x) -
gn(x)\textbar{}& =& \left
\textbar{}\int  Kn+1~f(x,y) dy
-\int  Kn~f(x,y)
dy\right \textbar{} \%& \\
& =& \left \textbar{}\int ~
Kn+1\diagdownKnf(x,y) dy\right
\textbar{} \leq\int ~
Kn+1\diagdownKn\textbar{}f(x,y)\textbar{} dy\%&
\\ & =& \int ~
Kn+1\textbar{}f(x,y)\textbar{} dy
-\int ~
Kn\textbar{}f(x,y)\textbar{} dy \%&
\\ \end{align*}

Posons un =\int ~
a^b\left (\int ~
Kn\textbar{}f(x,y)\textbar{} dy\right
)dx. En intégrant l'inégalité ci-dessus de a à b, on obtient

\int  a^b\textbar{}g~
n+1(x)-gn(x)\textbar{} dx \leq\int ~
a^b\left (\int ~
Kn+1\textbar{}f(x,y)\textbar{} dy\right
)-\int  a^b~\left
(\int ~
Kn\textbar{}f(x,y)\textbar{} dy\right )
= un+1-un

Mais \\sum ~
n=0^N(un+1 - un) = uN+1 -
u0 qui admet la limite \int ~
\int ~
{[}a,b{]}\timesI'\textbar{}f\textbar{}. Donc la série
\\sum  (un+1~ -
un) converge, et par conséquent, il en est de même de la série
\\sum ~
\int ~
a^b\textbar{}gn+1(x) -
gn(x)\textbar{} dx. Le théorème d'intégration termes à termes
pour les séries de fonctions assure que

\\sum
n=0^+\infty~\\\int
  a^b(g n+1 - gn) =
\\int  ~
a^b \\sum
n=0^+\infty~(g n+1 - gn) =
\\int  ~
a^b(g - g 0)

Mais

\sum n=0^N-1~
\\int  ~
a^b(g n+1 - gn) =
\\int  ~
a^bg N
-\\int  ~
a^bg 0

Autrement dit on a
limN\rightarrow~+\infty~~\left
(\int  a^bgN~
-\int ~
a^bg0\right )
=\int  a^b(g - g0~),
soit encore
limN\rightarrow~+\infty~~\\int
 a^bgN =\int ~
a^bg, c'est à dire \int ~
\int  {[}a,b{]}\timesI'~f
=\int  a^b~f, ce que nous
cherchions à démontrer.

Nous sommes maintenant en mesure de démontrer un premier théorème
reliant l'intégrale sur un rectangle et les intégrales partielles sur
les intervalles pro\\\\jmathmathmathmathections de ce rectangle sur les deux axes.

Théorème~10.3.10 Soit R = I \times I' un rectangle, f : R \rightarrow~ \mathbb{C} continue. On
suppose que

\begin{itemize}
\itemsep1pt\parskip0pt\parsep0pt
\item
  f est intégrable sur R
\item
  pour tout x \in I, y\mapsto~f(x,y) est intégrable
  sur I'
\item
  les applications
  x\mapsto~\int ~
  I'\textbar{}f(x,y)\textbar{} dy et g :
  x\mapsto~\int ~
  I'f(x,y) dy sont continues par morceaux sur I
\end{itemize}

Alors g est intégrable sur I et \int ~
\int  I\timesI'~f
=\int  I~g =\\int
 I\left (\int ~
I'f(x,y) dy\right )dx.

Démonstration Soit J un segment inclus dans I. D'après le lemme
précédent dont les hypothèses sont évidemment vérifiées,

\int  J~\textbar{}g\textbar{}
=\int  I~\left
\textbar{}\int  I'~f(x,y)
dy\right \textbar{} dx \leq\int ~
J\left (\int ~
I'\textbar{}f(x,y)\textbar{} dy\right ) dx
=\int  \\int ~
J\timesI'\textbar{}f\textbar{}\leq\int ~
\int  I\timesI'~\textbar{}f\textbar{}

ce qui garantit que g est intégrable sur I.

Soit maintenant (Jn)n\in\mathbb{N}~ une suite croissante de
segments dont la réunion est I. Alors, en combinant les deux lemmes
précédents, on a

\int  \\int ~
I\timesI'f =\
limn\rightarrow~+\infty~\int ~
\int  Jn\timesI'~f
= limn\rightarrow~+\infty~~\\int
 Jng =\int  I~g

ce que nous voulions démontrer.

Nous allons en déduire un théorème nous permettant d'intervertir les
signes d'intégration sur des intervalles.

Théorème~10.3.11 Soit I et I' deux intervalles de \mathbb{R}~, f : I \times I' \rightarrow~ \mathbb{C}
continue. On suppose que

\begin{itemize}
\itemsep1pt\parskip0pt\parsep0pt
\item
  pour tout x \in I, y\mapsto~f(x,y) est intégrable
  sur I'
\item
  pour tout y \in I', x\mapsto~f(x,y) est intégrable
  sur I et l'application
  y\mapsto~\int ~
  If(x,y) dx est continue par morceaux
\item
  l'application x\mapsto~\\int
   I'f(x,y) dy est continue par morceaux sur I et
  x\mapsto~\int ~
  I'\textbar{}f(x,y)\textbar{} dy est continue par morceaux et
  intégrable sur I
\end{itemize}

Alors l'application
y\mapsto~\int ~
If(x,y) dx est intégrable sur I' et on a

\int  I~\left
(\int  I'~f(x,y) dy\right
)dx =\int  I'~\left
(\int  I~f(x,y) dx\right
)dy

Démonstration Soit P = J \times K un pavé contenu dans I \times I'. On a alors

\int  \\int ~
J\timesK\textbar{}f\textbar{} =\int ~
J\left (\int ~
K\textbar{}f(x,y)\textbar{} dy\right )dx
\leq\int  J~\left
(\int  I'~\textbar{}f(x,y)\textbar{}
dy\right )dx \leq\int ~
I\left (\int ~
I'\textbar{}f(x,y)\textbar{} dy\right )dx

ce qui montre que \textbar{}f\textbar{} est intégrable sur le rectangle
I \times I'. Il en est donc de même pour f.

Le théorème précédent assure que l'application
x\mapsto~\int ~
I'f(x,y) dy est intégrable sur I et que

\int  \\int ~
I\timesI'f =\int ~
I\left (\int ~
I'f(x,y) dy\right )dx

On applique à nouveau le théorème précédent en intervertissant le rôle
de I et I', donc des variables x et y. On obtient que l'application
y\mapsto~\int ~
If(x,y) dx est intégrable sur I et que
\int  I'~\left
(\int  I~f(x,y) dx\right
)dy =\int  \\int ~
I\timesI'f ce qui nous donne l'égalité recherchée.

Remarque~10.3.2 Moyennant la vérification que toutes les fonctions
intégrées sont continues par morceaux, on pourra retenir ce théorème
sous la forme

\left .\array
\int  I~\left
(\int  I'~\textbar{}f(x,y)\textbar{}
dy\right )dx \textless{} +\infty~ \cr
\forall~y \in E, \\int ~
I\textbar{}f(x,y)\textbar{} dx \textless{} +\infty~ 
\right \\rigtharrow~\int ~
I\left (\int ~
I'f(x,y) dy\right )dx
=\int  I'~\left
(\int  I~f(x,y) dx\right
)dy

\paragraph{10.3.7 La fonction \Gamma}

Définition~10.3.3 Pour x \in{]}0,+\infty~{[}, on pose \Gamma(x)
=\int ~
0^+\infty~t^x-1e^-t dt.

Démonstration En + \infty~, t^x-1e^-t = o( 1
\over t^2 ) donc la fonction est intégrable
sur {[}1,+\infty~{[}. En 0, on a t^x-1e^-t ∼
t^x-1 \textgreater{} 0 donc la fonction est intégrable sur
{]}0,1{]} si et seulement si~x \textgreater{} 0. La fonction \Gamma est donc
définie pour x \textgreater{} 0.

Proposition~10.3.12 La fonction \Gamma est de classe C^\infty~ sur
{]}0,+\infty~{[} et

\forall~k \in \mathbb{N}~, \\forall~~x
\in{]}0,+\infty~{[}, \Gamma^(k)(x) =\int ~
0^+\infty~(log~
t)^ke^-tt^x-1 dt

Démonstration Soit 0 \textless{} a \textless{} 1 \textless{} b
\textless{} +\infty~ et posons f(x,t) = t^x-1e^-t pour
(x,t) \in {[}a,b{]}\times{]}0,+\infty~{[}. Soit J = {[}a,b{]} et I ={]}0,+\infty~{[}~; la
fonction f : J \times I \rightarrow~ \mathbb{C}, (x,t)\mapsto~f(x,t) est
continue et admet des dérivées partielles par rapport à x,
(x,t)\mapsto~\partial~^if\over
\partial~x^i(x,t) = (log~
t)^ie^-tt^x-1, continues sur J \times I, i =
1,\\ldots~,k. Soit
\phii :{]}0,+\infty~{[}\rightarrow~ \mathbb{R}~^+ définie par

 \phii(t) = \left \
\cases \textbar{}log~
t\textbar{}^ie^-tt^a-1&si t \in{]}0,1{]}
\cr (log~
t)^ie^-tt^b-1&si t ≥ 1 
\right .

Alors \phii est continue par morceaux, intégrable sur {]}0,+\infty~{[}
et on a

\forall~~(x,t) \in {[}a,b{]}\times{]}0,+\infty~{[}, \textbar{}
\partial~^if \over \partial~x^i (x,t)\textbar{}\leq
\phii(t)

D'après le théorème de dérivation des intégrales dépendant d'un
paramètre, \Gamma(x) =\int ~
0^+\infty~f(x,t) dt est de classe C^k sur {[}a,b{]}
et \Gamma^(i)(x) =\int ~
{]}0,+\infty~{[} \partial~^if \over
\partial~x^i (x,t) dt. Comme a et b sont quelconques, le résultat
reste valide sur la réunion des intervalles {[}a,b{]}, donc \Gamma est de
classe C^k sur {]}0,+\infty~{[} et

\forall~k \in \mathbb{N}~, \\forall~~x
\in{]}0,+\infty~{[}, \Gamma^(k)(x) =\int ~
0^+\infty~(log~
t)^ke^-tt^x-1 dt

Proposition~10.3.13 Pour tout x \in{]}0,+\infty~{[}, \Gamma(x + 1) = x\Gamma(x). En
particulier \forall~~n \in \mathbb{N}~, \Gamma(n + 1) = n!.

Démonstration Soit 0 \textless{} a \textless{} b \textless{} +\infty~. On a
par une intégration par parties

\int ~
a^be^-tt^x dt =
\left
{[}-e^-tt^x\right {]}
a^b + x\int ~
a^be^-tt^x-1 dt

Il suffit alors de faire tendre a vers 0 et b vers + \infty~~; le crochet
admet la limite 0 et on obtient \Gamma(x + 1) = x\Gamma(x). Comme \Gamma(1) = 1, une
récurrence immédiate donne \Gamma(n + 1) = n!.

Proposition~10.3.14 \Gamma( 1 \over 2 ) =
2\int ~
0^+\infty~e^-t^2  dt =
\sqrt\pi~.

Démonstration Soit 0 \textless{} a \textless{} b \textless{} +\infty~. Le
changement de variable t = u^2 donne

\int ~
a^be^-tt^-1\diagup2 dt =
2\int ~
\sqrta^\sqrtbe^-u^2
 du

Il suffit alors de faire tendre a vers 0 et b vers + \infty~, pour avoir \Gamma(
1 \over 2 ) = 2\int ~
0^+\infty~e^-t^2  dt. La valeur 
\sqrt\pi~ \over 2 de cette dernière
intégrale sera admise (démontrée en exercice).

\paragraph{10.3.8 Méthodes directes}

En ce qui concerne la continuité et la dérivabilité de F, on peut aussi
tenter de ma\\\\jmathmathmathmathorer directement les expressions F(x) - F(x0)
=\int  a^b~(f(x,t) -
f(x0,t)) dt et F(x) - F(x0) - (x -
x0)\int  a^b~ \partial~f
\over \partial~x (x0,t) dt
=\int  a^b~\left
(f(x,t) - f(x0,t) - (x - x0) \partial~f
\over \partial~x (x0,t)\right ) dten
utilisant en particulier l'inégalité des accroissements finis et
l'inégalité de Taylor Lagrange à l'ordre 2 qui nous donneront, moyennant
des hypothèses raisonnables,

\\textbar{}f(x,t) -
f(x0,t)\\textbar{} \leq\textbar{}x -
x0\textbar{}supy\in{[}x0,x{]}~\\textbar{}
\partial~f \over \partial~x (y,t)\\textbar{}

et

\begin{align*} \\textbar{}f(x,t) -
f(x0,t) - (x - x0) \partial~f \over \partial~x
(x0,t)\\textbar{}&& \%&
\\ & & \quad
\quad \leq \textbar{}x -
x0\textbar{}^2 \over 2
supy\in{[}x0,x{]}~\\textbar{}
\partial~^2f \over \partial~x^2
(y,t)\\textbar{}\%& \\
\end{align*}

Des méthodes directes similaires peuvent être utilisées pour
l'intégration.

Exemple~10.3.3 Soit F(x) =\int ~
0^+\infty~ sin~ t
\over t e^-tx dt. Il est clair que la
fonction est intégrable pour x \textgreater{} 0. Montrons que F est
dérivable sur {]}0,+\infty~{[}. On a, pour x0 \textgreater{} 0 et x
\textgreater{} x0 \over 2

e^-tx - e^-tx0  + (x -
x0)te^-tx0  = (x -
x0)^2 \over 2
t^2e^-t\xi~, \xi~ \in {[}x 0,x{]}

en appliquant la formule de Taylor Lagrange à l'application
y\mapsto~e^-ty sur {[}x0,x{]}
(ou {[}x,x0{]}), soit encore \textbar{}e^-tx -
e^-tx0 + (x -
x0)te^-tx0\textbar{}\leq
(x-x0)^2 \over 2
t^2e^- tx0 \over 2 .
On en déduit que \textbar{}F(x) - F(x0) + (x -
x0)\int ~
0^+\infty~sin~
te^-tx0 dt\textbar{}\leq
(x-x0)^2 \over 2
\int ~
0^+\infty~\textbar{}sin~
t\textbar{}te^- tx0 \over 2  dt,
toutes les intégrales ayant manifestement un sens. En divisant par
\textbar{}x - x0\textbar{}, on en déduit que F est dérivable
en x0 et que F'(x0) = -\\int
 0^+\infty~sin~
te^-tx0 dt = - 1 \over
1+x0^2 (facile). On obtient donc F(x) = K
-\mathrmarctg~ x, pour x
\textgreater{} 0. Nous laissons en exercice au lecteur le soin de
montrer que limx\rightarrow~\infty~~F(x) = 0, ce qui
montrera que K = \pi~ \over 2 . Bien entendu, on aurait
pu aussi utiliser le théorème de convergence dominée pour démontrer la
dérivabilité de F sur {]}0,+\infty~{[} (ou plutôt sur {[}a,+\infty~{[} pour tout a
\textgreater{} 0).

{[}
{[}
{[}
{[}
