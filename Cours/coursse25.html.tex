\textbf{Warning: 
requires JavaScript to process the mathematics on this page.\\ If your
browser supports JavaScript, be sure it is enabled.}

\begin{center}\rule{3in}{0.4pt}\end{center}

{[}
{[}
{[}{]}
{[}

\subsubsection{4.8 Espaces et parties compactes}

\paragraph{4.8.1 Propriété de Bolzano-Weierstrass}

Définition~4.8.1 Soit E un espace métrique. On dit que E est compact
s'il vérifie la propriété de Bolzano Weierstrass~: toute suite de E a
une valeur d'adhérence dans E. On dit qu'une partie A de E est compacte
si toute suite de A a une valeur d'adhérence dans A.

Remarque~4.8.1 La compacité est une notion purement topologique et non
métrique. Le fait pour une partie A d'être compacte ne dépend que de la
topologie de la partie et pas de l'espace ambiant E (comparer avec le
fait pour A d'être ouverte ou fermée qui dépend de E).

Théorème~4.8.1 Soit E un espace métrique et F une partie de E. (i) Si F
est compacte, alors F est fermée et bornée dans E (ii) Inversement, si E
est compact et F fermée dans E alors F est compacte

Démonstration (i) Soit x \in E qui est limite d'une suite (xn)
de F. La suite (xn) est une suite dans F, donc admet une
valeur d'adhérence \ell dans F. Mais la suite étant convergente, a une
seule valeur d'adhérence dans E, on a x = \ell \in F. Donc F est fermé dans
E. Le fait d'être borné résultera du lemme suivant

Lemme~4.8.2 Soit F une partie compacte~; alors pour tout \epsilon
\textgreater{} 0, F peut être recouvert par un nombre fini de boules de
rayon \epsilon (propriété de précompacité)

Démonstration Supposons que F ne peut pas être recouvert par un nombre
fini de boules de rayon \epsilon et soit x0 \in F~; on a
F⊄B'(x0,1)~; soit x1 \in F \diagdown B'(x0,\epsilon)~;
supposons
x0,\\ldots,xn~
construits~; alors F⊄B(x0,\epsilon)
\cup\\ldots~ \cup
B(xn,\epsilon) et on prend xn+1 \in F \diagdown\left
(B(x0,\epsilon)
\cup\\ldots~ \cup
B(xn,\epsilon)\right ). On construit ainsi une suite
(xn) telle que \forall~~p,q,
p\neq~q \rigtharrow~ d(xp,xq) ≥ \epsilon. Cette
suite n'admet aucune sous suite de Cauchy, donc aucune sous suite
convergente, donc pas de valeur d'adhérence. C'est absurde.

(ii) Soit (xn) une suite dans F~; c'est aussi une suite dans E
donc elle admet une valeur d'adhérence x \in E (car E est compact), x
= limx\phi(n)~~; mais comme F est fermé
et x est limite d'une suite d'éléments de F, x appartient à F et il est
évidemment limite dans F de la suite (x\phi(n)). Donc la suite
admet une valeur d'adhérence dans F et F est compacte.

Théorème~4.8.3 Soit f : E \rightarrow~ F continue. Pour toute partie compacte A de
E, f(A) est une partie compacte de F (et en particulier elle est fermée
et bornée).

Démonstration Soit (bn) une suite de f(A). On pose
bn = f(an), an \in A. Alors an
admet une valeur d'adhérence dans A, a =\
lima\phi(n). Par continuité de f au point a, on a f(a)
= limf(a\phi(n)~) et donc la suite
(bn) a une valeur d'adhérence dans f(A).

Corollaire~4.8.4 Soit E un espace métrique compact et f : E \rightarrow~ F
bi\\\\jmathmathmathmathective et continue. Alors f est un homéomorphisme.

Démonstration Il faut montrer que f^-1 est continue autrement
dit que pour tout fermé A de E, (f^-1)^-1(A) =
f(A) est fermée dans F~; mais une telle partie A est fermée dans un
compact, donc compacte et donc f(A) est compacte dans F donc fermée.
Ceci montre la continuité de f^-1.

Proposition~4.8.5 Si E1 et E2 sont deux espaces
métriques compacts, alors l'espace métrique produit est compact.

Démonstration Soit (zn) une suite dans E = E1 \times
E2, zn = (xn,yn). La suite
(xn) est une suite dans E compact, donc admet une sous suite
convergente (x\phi(n)). La suite (y\phi(n)) est une suite
dans E2 compact, donc admet une sous suite convergente
(y\phi(\psi(n))). La suite (x\phi(\psi(n))) est une sous suite
d'une suite convergente, donc encore convergente et donc la suite
(z\phi(\psi(n))) est convergente. Toute suite de E admet bien une
valeur d'adhérence.

Théorème~4.8.6 (Heine). Soit E un espace métrique compact et f : E \rightarrow~ F
continue. Alors f est uniformément continue.

Démonstration Supposons f non uniformément continue. Alors

\exists~\epsilon \textgreater{} 0,
\forall~~\eta \textgreater{} 0,\quad
\exists~a,b \in E, d(a,b) \textless{}
\eta\text et d(f(a),f(b)) ≥ \epsilon

en prenant \eta = 1 \over n+1 , on trouve an
et bn tels que d(an,bn) \textless{} 1
\over n+1 alors que d(f(an),f(bn))
≥ \epsilon. La suite (an) admet une sous suite convergente
(a\phi(n)) de limite a~; comme d(a\phi(n),b\phi(n))
\textless{} 1 \over \phi(n)+1 on a aussi
limb\phi(n)~ = a. Cependant
d(f(a\phi(n)),f(b\phi(n))) ≥ \epsilon, ce qui montre que la suite
(d(f(a\phi(n)),f(b\phi(n)))) ne tend pas vers 0, alors que
les deux suites f(a\phi(n)),f(b\phi(n)) admettent la même
limite f(a) (continuité de f au point a). C'est absurde.

\paragraph{4.8.2 Propriété de Borel Lebesgue}

Définition~4.8.2 On dit qu'un espace topologique E vérifie la propriété
de Borel Lebesgue si on a les conditions équivalentes (i) Pour toute
famille d'ouverts (Ui)i\inI telle que E
= \⋃ ~
i\inIUi, il existe
i1,\\ldots,ik~
\in I tels que E =\ \⋃
 p=1^kUip (ii) Pour toute famille
de fermés (Fi)i\inI telle que
\⋂ ~
i\inIFi = \varnothing~, il existe
i1,\\ldots,ik~
\in I tels que \⋂ ~
p=1^kFip = \varnothing~

Démonstration Ces deux propriétés sont équivalentes par passage au
complémentaire.

Remarque~4.8.2 On peut formuler (i) sous la forme~: de tout recouvrement
de E par des ouverts, on peut extraire un sous recouvrement fini.

On a le lemme suivant, qui nous servira pour la démonstration du
théorème~:

Lemme~4.8.7 Soit (E,d) un espace métrique compact et
(Ui)i\inI une famille d'ouverts telle que E
= \⋃ ~
i\inIUi. Alors, il existe \epsilon \textgreater{} 0 tel que

\forall~~x \in E,
\existsix~ \in I, B(x,\epsilon) \subset~
Uix

Démonstration Par l'absurde~; supposons que

\forall~~\epsilon \textgreater{} 0,
\existsx \in E, \\forall~~i \in I,
B(x,\epsilon)⊄Ui

Prenons \epsilon = 1 \over n+1 et xn
correspondant. La suite (xn) a donc une valeur d'adhérence x.
Il existe i0 \in I tel que x \in Ui0 et donc
un \eta \textgreater{} 0 tel que B(x,\eta) \subset~ Ui0. Mais x
est valeur d'adhérence de la suite xn et donc il existe n
\textgreater{} 2\diagup\eta tel que xn \in B(x,\eta\diagup2). Alors, si y \in
B(xn, 1 \over n+1 ), on a d(y,x) \leq
d(y,xn) + d(xn,x) \textless{} 1
\over n+1 + \eta \over 2 \textless{} \eta
soit B(xn, 1 \over n+1 ) \subset~ B(x,\eta) \subset~
Ui0. Mais ceci contredit la définition de
xn~: \forall~i \in I, B(xn~, 1
\over n+1 )⊄Ui. C'est absurde.

Théorème~4.8.8 Un espace métrique E est compact si et seulement si~il
vérifie la propriété de Borel-Lebesgue.

Démonstration ⇐ Supposons que E vérifie la propriété de Borel-Lebesgue,
et soit (xn) une suite de E. Pour N \in \mathbb{N}~, posons XN =
\xn∣n ≥
N\. On a

\begin{align*} x\text valeur
d'adhérence de (xn)&& \%&
\\ & \Leftrightarrow &
\forall~V \in V (x), \\forall~~N \in \mathbb{N}~,
\existsn ≥ N, xn~ \in V \%&
\\ & \Leftrightarrow &
\forall~V \in V (x), \\forall~~N \in \mathbb{N}~,
V \bigcap XN\neq~\varnothing~ \%&
\\ & \Leftrightarrow &
\forall~~N \in \mathbb{N}~, x
\in\overlineXN \%&
\\ & \Leftrightarrow & x
\in\⋂
N\in\mathbb{N}~\overlineXN \%&
\\ \end{align*}

Supposons donc que la suite n'a pas de valeur d'adhérence~; on a alors
\⋂ ~
N\in\mathbb{N}~\overlineXN = \varnothing~ et comme ce sont
des fermés de E qui vérifie la propriété de Borel-Lebesgue, il existe
N1,\\ldots,Nk~
tels que \⋂ ~
p=1^k\overlineXNp
= \varnothing~. Mais la suite (XN) est décroissante, et donc la suite
(\overlineXN) aussi. On a donc
\⋂ ~
p=1^k\overlineXNp
=
\overlineXmax(Np)\mathrel\neq~~\varnothing~.
C'est absurde. Donc E est compact.

\rigtharrow~ Soit (Ui)i\inI une famille d'ouverts telle que E
= \⋃ ~
i\inIUi. Alors, il existe \epsilon \textgreater{} 0 tel que

\forall~~x \in E,
\existsix~ \in I, B(x,\epsilon) \subset~
Uix

Par le lemme de précompacité, on peut recouvrir E par un nombre fini de
boules de rayon \epsilon~: E = B(x1,\epsilon)
\cup\\ldots~ \cup
B(xk,\epsilon). Mais alors E \subset~ Uix 1
\cup\\ldots~ \cup
Uix k \subset~ E, ce qui démontre que l'on peut
recouvrir E par un nombre fini de Ui.

\paragraph{4.8.3 Compacts de \mathbb{R}~ et \mathbb{R}~^n}

Lemme~4.8.9 Tout segment {[}a,b{]} de \mathbb{R}~ est compact.

Démonstration Soit (xn) une suite de {[}a,b{]}. On définit
deux suites (ap) et (bp) de la manière suivante~:
a0 = a et b0 = b~; si ap et bp
sont construits, on pose ap+1 = ap et bp+1
= ap+bp \over 2 si
\n \in \mathbb{N}~∣xn \in
{[}ap, ap+bp \over 2
{]}\ est infini~; sinon on pose ap+1 =
ap+bp \over 2 et bp+1 =
bp. On a évidemment~: (ap) croissante,
(bp) décroissante, bp - ap = b-a
\over 2^p et \n \in
\mathbb{N}~∣xn \in
{[}ap,bp{]}\ est infini. Les deux
suites étant ad\\\\jmathmathmathmathacentes, soit \ell leur limite commune et \epsilon \textgreater{}
0. Il existe n \in \mathbb{N}~ tel que \ell - \epsilon \textless{} an \leq \ell \leq
bn \textless{} \ell + \epsilon et donc \n \in
\mathbb{N}~∣xn \in{]}\ell - \epsilon,\ell +
\epsilon{[}\ est infini. Donc \ell est valeur d'adhérence de la
suite (xn).

Théorème~4.8.10 Les parties compactes de \mathbb{R}~ et \mathbb{R}~^n sont les
parties à la fois fermées et bornées pour une des distances usuelles.

Démonstration On sait dé\\\\jmathmathmathmathà qu'une partie compacte doit être fermée et
bornée. Inversement soit A une partie fermée et bornée de \mathbb{R}~. Il existe
a,b \in \mathbb{R}~ tels que A \subset~ {[}a,b{]}. Alors A = A \bigcap {[}a,b{]} est fermé dans
{[}a,b{]} donc compacte. Même chose dans \mathbb{R}~^n en
rempla\ccant {[}a,b{]} par
{[}a1,b1{]} \times⋯ \times
{[}an,bn{]} qui est compact comme produit de
compacts.

Corollaire~4.8.11 Soit E un espace métrique compact. Toute application
continue de E dans \mathbb{R}~ est bornée et atteint ses bornes inférieure et
supérieure.

Démonstration f(E) est compacte donc bornée et fermée (donc contient ses
bornes).

Corollaire~4.8.12 \mathbb{R}~ est complet.

Démonstration Une suite de Cauchy est bornée, donc peut être incluse
dans un segment qui est compact~; elle y admet donc une valeur
d'adhérence et donc elle converge.

Remarque~4.8.3 Bien entendu la validité de cette démonstration dépend de
la construction de \mathbb{R}~ qui est employée.

{[}
{[}
{[}
{[}
