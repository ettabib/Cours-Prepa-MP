\textbf{Warning: 
requires JavaScript to process the mathematics on this page.\\ If your
browser supports JavaScript, be sure it is enabled.}

\begin{center}\rule{3in}{0.4pt}\end{center}

{[}
{[}
{[}{]}
{[}

\subsubsection{14.5 Produit de convolution}

\paragraph{14.5.1 Convolution de fonctions périodiques}

Définition~14.5.1 Soit f,g : \mathbb{R}~ \rightarrow~ \mathbb{C} continues par morceaux et périodiques
de période 2\pi~. On définit le produit de convolution de f et g comme la
fonction f ∗ g : \mathbb{R}~ \rightarrow~ \mathbb{C} définie par

\forall~~x \in \mathbb{R}~, f ∗ g(x) = 1 \over
2\pi~ \int  0^2\pi~~f(t)g(x - t) dt

Théorème~14.5.1

\begin{itemize}
\itemsep1pt\parskip0pt\parsep0pt
\item
  (i) la fonction f ∗ g est continue et périodique de période 2\pi~
\item
  (ii) l'application (f,g)\mapsto~f ∗ g est
  bilinéaire
\item
  (iii) le produit de convolution est commutatif~: g ∗ f = f ∗ g
\item
  (iv) le produit de convolution est associatif~: (f ∗ g) ∗ h = f ∗ (g ∗
  h)
\end{itemize}

Démonstration (i) On a f ∗ g(x + 2\pi~) = 1 \over 2\pi~
\int  0^2\pi~~f(t)g(x + 2\pi~ - t) dt
= 1 \over 2\pi~ \int ~
0^2\pi~f(t)g(x - t) dt = f ∗ g(x) puisque g est périodique
de période 2\pi~. Montrons la continuité de f ∗ g. Supposons tout d'abord
que f est en escalier et soit a0 = 0 \leq a1
\leq\\ldots~ \leq
ap = 2\pi~ une subdivision de {[}0,2\pi~{]} adaptée à f, si bien que
\forall~t \in{]}ai-1,ai~{[}, f(t) =
\lambda~i~; on a alors

\begin{align*} \int ~
0^2\pi~f(t)g(x - t) dt& =& \\sum
i=1^p\lambda~ i
\\int  ~
ai-1^ai g(x - t) dt \%&
\\ & =& -\\sum
i=1^p\lambda~ i
\\int  ~
x-ai-1^x-ai g(u) du\%&
\\ \end{align*}

en faisant le changement de variable u = x - t. Comme une intégrale de
fonction réglée dépend de fa\ccon continue des bornes
d'intégration, l'application
x\mapsto~\int ~
x-ai-1^x-aig(u) du est continue et
donc f ∗ g est continue. Si maintenant f est continue par morceaux, soit
(fn) une suite d'applications en escalier qui converge
uniformément vers f. On a alors

\begin{align*} \textbar{}f ∗ g(x) - fn ∗
g(x)& =& \left \textbar{} 1 \over 2\pi~
\int  0^2\pi~(f(t) - f~
n(t))g(x - t) dt\right \textbar{}\%&
\\ & \leq& 1 \over 2\pi~
\int  0^2\pi~~\textbar{}f(t) -
f n(t)\textbar{}\,\textbar{}g(x - t)\textbar{}
dt \%& \\ & \leq&
\\textbar{}f -
fn\\textbar{}\infty~\\textbar{}g\\textbar{}\infty~
\%& \\ \end{align*}

ce qui montre que la suite (fn ∗ g) converge uniformément vers
f ∗ g. Comme ces applications sont continues, il en est de même de f ∗
g.

(ii) est évident

(iii) On a, en faisant le changement de variable u = x - t et en
remarquant que la fonction intégrée étant périodique de période 2\pi~, son
intégrale sur le segment de longueur 2\pi~, {[}x - 2\pi~,x{]} est égale à
l'intégrale sur {[}0,2\pi~{]}

\begin{align*} f ∗ g(x)& =& 1
\over 2\pi~ \int ~
0^2\pi~f(t)g(x - t) dt \%& \\
& =& - 1 \over 2\pi~ \int ~
x^x-2\pi~f(x - u)g(u) du \%&
\\ & =& 1 \over 2\pi~
\int  x-2\pi~^x~f(x - u)g(u) du \%&
\\ & =& 1 \over 2\pi~
\int  0^2\pi~~f(x - u)g(u) du = g ∗
f(x)\%& \\
\end{align*}

Ceci démontre la commutativité.

(iv) On a

\begin{align*} (f ∗ g) ∗ h(x)& =& 1
\over 2\pi~ \int ~
0^2\pi~f ∗ g(t)h(x - t) dt \%&
\\ & =& 1 \over
4\pi~^2 \int  0^2\pi~~h(x
- t)\left (\int ~
0^2\pi~f(u)g(t - u) du\right ) dt\%&
\\ & =& 1 \over
4\pi~^2 \int ~
0^2\pi~\left (\int ~
0^2\pi~f(u)g(t - u)h(x - t) du\right ) dt\%&
\\ \end{align*}

Si f, g et h sont continues, le théorème de Fubini permet d'intervertir
les deux signes d'intégration et on obtient

\begin{align*} (f ∗ g) ∗ h(x)&& \%&
\\ & =& 1 \over
4\pi~^2 \int ~
0^2\pi~\left (\int ~
0^2\pi~f(u)g(t - u)h(x - t) dt\right ) du
\%& \\ & =& 1 \over
4\pi~^2 \int ~
0^2\pi~f(u)\left (\\int
 0^2\pi~g(t - u)h(x - t) dt\right ) du \%&
\\ & =& 1 \over
4\pi~^2 \int ~
0^2\pi~f(u)\left (\\int
 -u^2\pi~-ug(v)h(x - u - v) dv\right )
du\%& \\ & =& 1 \over
4\pi~^2 \int ~
0^2\pi~f(u)\left (\\int
 0^2\pi~g(v)h(x - u - v) dv\right ) du \%&
\\ & =& 1 \over 2\pi~
\int  0^2\pi~~f(u) g ∗ h(x - u) du =
f ∗ (g ∗ h)(x) \%& \\
\end{align*}

en faisant le changement de variable v = t - u, soit t = v + u, dans
l'intégrale interne et en utilisant le fait que la fonction est
périodique de période 2\pi~. Si f et g sont seulement continues par
morceaux, il suffit d'utiliser des subdivisions adaptées et de découper
les intégrales suivant ces subdivisions.

Théorème~14.5.2 Soit f et g des applications de \mathbb{R}~ dans \mathbb{C} périodiques de
période 2\pi~. On suppose que f est continue par morceaux et que g est de
classe C^k. Alors f ∗ g est de classe C^k et (f
∗ g)^(k) = f ∗ (g^(k)).

Démonstration Une récurrence évidente permet d'obtenir le résultat pour
k quelconque à partir de k = 1. Quitte à utiliser une subdivision de
{[}0,2\pi~{]} et à découper l'intégrale, il suffit de montrer que
x\mapsto~\int ~
a^bf(t)g(x - t) dt est de classe \mathcal{C}^1 lorsque f
est continue sur {[}a,b{]} et g de classe \mathcal{C}^1 sur \mathbb{R}~. Mais
l'application (x,t)\mapsto~f(t)g(x - t) admet une
dérivée partielle par rapport à x égale à  \partial~ \over \partial~x
(f(t)g(x - t)) = f(t)g'(x - t) qui est une fonction continue du couple
(x,t). Le théorème de dérivation des intégrales dépendant d'un paramètre
montre que x\mapsto~\int ~
a^bf(t)g(x - t) dt est de classe \mathcal{C}^1 et que

(f ∗ g)'(x) =\int  a^b~ \partial~
\over \partial~x (f(t)g(x - t)) dt =\\int
 a^bf(t)g'(x - t) dt

Ceci montre que f ∗ g est de classe \mathcal{C}^1 et que (f ∗ g)' = f ∗
(g').

\paragraph{14.5.2 Produit de convolution et séries de Fourier}

Théorème~14.5.3 Soit f et g des applications de \mathbb{R}~ dans \mathbb{C} périodiques de
période 2\pi~, continues par morceaux. Alors \forall~~n \in
ℤ, cn(f ∗ g) = cn(f)cn(g).

Démonstration On a pour f continue par morceaux,

\begin{align*} f ∗ en(x)& =& 1
\over 2\pi~ \int ~
0^2\pi~f(t)e^in(x-t) dt = 1
\over 2\pi~ e^inx\int ~
0^2\pi~f(t)e^-int dt\%&
\\ & =& cn(f)e^inx
\%& \\ \end{align*}

On en déduit que

\begin{align*} cn(f ∗ g)& =& (f ∗ g) ∗
en(0) = f ∗ (g ∗ en)(0) \%&
\\ & =& f ∗
(cn(g)en)(0) = cn(g)(f ∗ en)(0)
= cn(g)cn(f)\%& \\
\end{align*}

Remarque~14.5.1 Pour une fonction g donnée, l'application
f\mapsto~f ∗ g se traduit donc comme un filtre sur
le signal f~: l'amplitude cn(f) de l'harmonique de f
correspondant à la fréquence n est multipliée par le coefficient
cn(g). Comme les cn(g) tendent vers 0 quand
\textbar{}n\textbar{} tend vers + \infty~, on voit qu'il ne peut exister
d'élément neutre pour le produit de convolution, c'est-à-dire de
fonction \epsilon telle que \forall~~f \inC, f ∗ \epsilon = f.

{[}
{[}
{[}
{[}
