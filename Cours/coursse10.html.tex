\textbf{Warning: 
requires JavaScript to process the mathematics on this page.\\ If your
browser supports JavaScript, be sure it is enabled.}

\begin{center}\rule{3in}{0.4pt}\end{center}

{[}
{[}
{[}{]}
{[}

\subsubsection{2.4 Dualité~: approche restreinte}

\paragraph{2.4.1 Formes linéaires, dual, formes coordonnées}

Définition~2.4.1 Soit E un K-espace vectoriel . On appelle forme
linéaire sur E toute application linéaire de E dans K. On appelle dual
de E le K-espace vectoriel E^∗ = L(E,K).

Remarque~2.4.1 Soit (ei)i\inI une base de E et
i0 \in I. Tout vecteur x de E s'écrit de manière unique sous la
forme x = \\sum ~
i\inIxiei. L'application
\phii0 :
x\mapsto~xi0 est clairement une
forme linéaire sur E, appelée forme linéaire coordonnée d'indice
i0 dans la base (ei)i\inI. Elle est définie
par \phii0(ei0) = 1 et
\phii0(ei) = 0 si
i\neq~i0, soit encore par
\phii0(ei) =
\deltai0^i.

Proposition~2.4.1 Soit E un K-espace vectoriel et x \in E,
x\neq~0. Alors il existe une forme linéaire \phi sur
E telle que \phi(x) = 1.

Démonstration Le vecteur x forme à lui tout seul une famille libre que
l'on peut compléter en une base (ei)i\inI de E avec x
= ei0. Soit \phi la forme linéaire qui associe à tout
vecteur de E sa i0-ième coordonnée dans cette base. On a bien
entendu \phi(x) = 1.

Remarque~2.4.2 Le résultat précédent peut encore s'interpréter sous la
forme~: si x \in E,

\forall~\phi \in E^∗~, \phi(x) =
0\quad \Leftrightarrow x = 0

\paragraph{2.4.2 Base duale d'un espace vectoriel de dimension finie}

Définition~2.4.2 Soit E un espace vectoriel de dimension finie, \mathcal{E} =
(e1,\\ldots,en~)
une base de E. Pour i \in {[}1,n{]}, soit \phii la forme linéaire
coordonnée d'indice i dans la base \mathcal{E}. Alors \mathcal{E}^∗ =
(\phi1,\\ldots,\phin~)
est une base du dual E^∗, appelée la base duale de la base \mathcal{E}.
Elle est caractérisée par les relations \forall~~i,\\\\jmathmathmathmath \in
{[}1,n{]}, \phii(e\\\\jmathmathmathmath) = \deltai^\\\\jmathmathmathmath.

Démonstration Tout d'abord, montrons que \mathcal{E}^∗ est une famille
libre en utilisant les relations de définition des formes coordonnées

\forall~i,\\\\jmathmathmathmath \in {[}1,n{]}, \phii(e\\\\jmathmathmathmath~) =
\deltai^\\\\jmathmathmathmath

Soit
\lambda~1,\\ldots,\lambda~n~
\in K tels que \lambda~1\phi1 +
\\ldots~ +
\lambda~n\phin = 0~; on a alors, pout tout i \in {[}1,n{]}

0 = 0(ei) = \lambda~1\phi1(ei) +
\\ldots~ +
\lambda~n\phin(ei) = \lambda~i

ce qui montre bien que la famille est libre. Pour montrer qu'elle est
génératrice, soit \phi \in E et considérons \psi =\
\sum ~
i=1^n\phi(ei)\phii~; on a alors pour tout \\\\jmathmathmathmath
\in {[}1,n{]},

\psi(e\\\\jmathmathmathmath) = \\sum
i=1^n\phi(e i)\phii(e\\\\jmathmathmathmath) =
\sum i=1^n\phi(e~
i)\deltai^\\\\jmathmathmathmath = \phi(e \\\\jmathmathmathmath)

Les deux applications linéaires \phi et \psi coïncidant sur une base, sont
égales, ce qui montre que la famille est génératrice.

Remarque~2.4.3 Attention~: la dimension finie est essentielle~; elle
garantit qu'il n'y a qu'un nombre fini de \phi(ei) non nuls et
permet de considérer la somme
\\sum ~
i\inI\phi(ei)\phii~; en dimension infinie,
\mathcal{E}^∗ n'est pas une base de E^∗, car elle n'est pas
génératrice (considérer la forme linéaire \phi qui à tout ei
associe 1).

Corollaire~2.4.2 La dimension de l'espace dual d'un espace vectoriel de
dimension finie est égale à la dimension de l'espace.

\paragraph{2.4.3 Orthogonalité 1}

Soit E un K-espace vectoriel de dimension finie,
(e1,\\ldots,ep~)
une famille d'éléments de E. Nous pouvons associer à cette famille
l'application u : E^∗\rightarrow~ K^p, définie par u(\phi) =
(\phi(e1),\\ldots,\phi(ep~)).
On vérifie immédiatement que u est linéaire. Son noyau est constitué des
\phi \in E^∗ vérifiant \forall~~i \in {[}1,p{]},
\phi(ei) = 0.

Proposition~2.4.3 Si
(e1,\\ldots,ep~)
est une base de E, alors u est un isomorphisme d'espace vectoriel de
E^∗ sur K^p.

Démonstration En effet dans ce cas, u envoie la base duale
\mathcal{E}^∗ sur la base canonique de K^p~; c'est donc un
isomorphisme.

Proposition~2.4.4 La famille
(e1,\\ldots,ep~)
est libre si et seulement si u est sur\\\\jmathmathmathmathective. Sous ces conditions,
\mathrmKer~u est de
codimension p et

\forall~~x \in E,\quad (x
\in\mathrmVect(e1,\\\ldots,ep~)
\Leftrightarrow \forall~~\phi
\in\mathrmKer~u, \phi(x) = 0)

Démonstration Si u est sur\\\\jmathmathmathmathective, notons
(\epsilon1,\\ldots,\epsilonp~)
la base canonique de K^p et soit \phii \in
E^∗ tel que u(\phii) = \epsiloni~; on a donc
\forall~i,\\\\jmathmathmathmath \in {[}1,p{]}, \phii(e\\\\jmathmathmathmath~) =
\deltai^\\\\jmathmathmathmath ce qui implique évidemment que la famille est
libre~: si \\sum ~
\\\\jmathmathmathmath=1^p\lambda~\\\\jmathmathmathmathe\\\\jmathmathmathmath = 0, on a pour tout i \in
{[}1,p{]}

0 = \phii(0) = \phii(\\sum
\\\\jmathmathmathmath=1^p\lambda~ \\\\jmathmathmathmathe\\\\jmathmathmathmath) =
\sum \\\\jmathmathmathmath=1^p\lambda~~
\\\\jmathmathmathmath\phii(e\\\\jmathmathmathmath) = \lambda~i

Inversement, si la famille est libre, on peut compléter la famille
(e1,\\ldots,ep~)
en une base
(e1,\\ldots,en~)
de E et soit
(\phi1,\\ldots,\phin~)
la base duale. On a alors \forall~~i,\\\\jmathmathmathmath \in {[}1,p{]},
\phii(e\\\\jmathmathmathmath) = \deltai^\\\\jmathmathmathmath, soit
\forall~i \in {[}1,p{]}, u(\phii~) =
\epsiloni. L'image de u contient une base de K^p, c'est
donc K^p et u est sur\\\\jmathmathmathmathective. Dans ces conditions, on peut
appliquer le théorème du rang, et donc dim~
\mathrmKer~u
= dim E^∗~- p
= dim~ E - p.

Soit \phi \in\mathrmKer~u~; alors
\forall~i \in {[}1,p{]}, \phi(ei~) = 0 et donc
\forall~~x
\in\mathrmVect(e1,\\\ldots,ep~),
\phi(x) = 0. Inversement, supposons que
x∉\mathrmVect(e1,\\\ldots,ep~)~;
alors la famille
(e1,\\ldots,ep~,x)
est libre, on peut la compléter en une base de E et la forme coordonnée
suivant x dans cette base, soit \phi, appartient à
\mathrmKer~u alors que \phi(x)
= 1. On a donc bien l'équivalence

\forall~~x \in E,\quad (x
\in\mathrmVect(e1,\\\ldots,ep~)
\Leftrightarrow \forall~~\phi
\in\mathrmKer~u, \phi(x) = 0)

Remarque~2.4.4 Application~: Soit F un sous-espace vectoriel de E,
(e1,\\ldots,ep~)
une base de F, u : E^∗\rightarrow~ K^p l'application linéaire
associée,
(\phi1,\\ldots,\phin-p~)
une base de \mathrmKer~u~;
alors x \in F \Leftrightarrow \forall~~i \in
{[}1,n - p{]}, \phii(x) = 0. On dit encore que F est défini par
le système d'équations linéaires \phi1(x) =
0,\\ldots,\phin-p~(x)
= 0.

\paragraph{2.4.4 Hyperplans}

Définition~2.4.3 On appelle hyperplan de E tout sous-espace vectoriel H
de E vérifiant les conditions équivalentes

\begin{itemize}
\itemsep1pt\parskip0pt\parsep0pt
\item
  (i) dim~ E\diagupH = 1
\item
  (ii) \exists~f \in E
  \diagdown\0\, H =\
  \mathrmKerf
\item
  (iii) H admet une droite comme supplémentaire.
\end{itemize}

Démonstration

\begin{itemize}
\itemsep1pt\parskip0pt\parsep0pt
\item
  (i) \rigtharrow~(ii)~: prendre \overlinee une base de E\diagupH et
  écrire \pi~(x) = f(x)\overlinee.
\item
  (ii) \rigtharrow~ (iii)~: on prend a \in E tel que
  f(a)\neq~0. Tout élément x s'écrit de manière
  unique sous la forme x = (x - f(x) \over f(a) a)
  + f(x) \over f(a) a avec x - f(x)
  \over f(a) a
  \in\mathrmKer~f, soit E
  = \mathrmKer~f \oplus~ Ka.
\item
  (iii) \rigtharrow~(i)~: tout supplémentaire de H est isomorphe à E\diagupH.
\end{itemize}

Théorème~2.4.5 Soit H un hyperplan de E. Alors deux formes linéaires
nulles sur H sont proportionnelles.

Démonstration Si E = H \oplus~ Ka et H =\
\mathrmKerf, soit g \in E^∗ nulle sur H.
Alors g et  g(a) \over f(a) f coïncident sur H et sur
Ka, donc sont égales.

\paragraph{2.4.5 Orthogonalité 2}

Remarque~2.4.5 Soit E un K-espace vectoriel de dimension finie,
(\phi1,\\ldots,\phip~)
une famille d'éléments de E^∗. Nous pouvons associer à cette
famille l'application v : E \rightarrow~ K^p, définie par v(x) =
(\phi1(x),\\ldots,\phip~(x)).
On vérifie immédiatement que v est linéaire. Son noyau est constitué de
l'intersection des
\mathrmKer\phii~ (en
général des hyperplans, sauf si la forme linéaire est nulle).

Proposition~2.4.6 Si
(\phi1,\\ldots,\phip~)
est une base de E^∗, alors v est un isomorphisme d'espace
vectoriel de E^∗ sur K^p.

Démonstration En effet dans ce cas, v est in\\\\jmathmathmathmathective car

\begin{align*} v(x) = 0&
\Leftrightarrow & \forall~~i \in
{[}1,p{]}, \phii(x) = 0\%& \\ &
\Leftrightarrow & \forall~~\phi \in
E^∗, \phi(x) = 0 \%& \\ &
\Leftrightarrow & x = 0 \%&
\\ \end{align*}

Comme dim E =\ dim~
E^∗ = p = dim K^p~, il
s'agit d'un isomorphisme.

Théorème~2.4.7 Soit
(\phi1,\\ldots,\phip~)
une base de E^∗~; alors il existe une unique base
(e1,\\ldots,ep~)
de E dont
(\phi1,\\ldots,\phip~)
soit la base duale.

Démonstration On a en effet \forall~~i \in
{[}1,p{]},\phii(e\\\\jmathmathmathmath) = \deltai^\\\\jmathmathmathmath
\Leftrightarrow v(e\\\\jmathmathmathmath) = \epsilon\\\\jmathmathmathmath (\\\\jmathmathmathmath-ième
vecteur de la base canonique). La famille
(e1,\\ldots,ep~)
est donc l'image de la base canonique de K^p par
l'isomorphisme v^-1.

Proposition~2.4.8 La famille
(\phi1,\\ldots,\phip~)
est libre si et seulement si v est sur\\\\jmathmathmathmathective. Sous ces conditions,
\mathrmKer~v est de
codimension p et \forall~\phi \in E^∗~,

(\phi
\in\mathrmVect(\phi1,\\\ldots,\phip~)
\Leftrightarrow \forall~~x
\in\mathrmKer~v, \phi(x) = 0)

Démonstration Si v est sur\\\\jmathmathmathmathective, notons
(\epsilon1,\\ldots,\epsilonp~)
la base canonique de K^p et soit ei \in E tel que
v(ei) = \epsiloni~; on a donc
\forall~i,\\\\jmathmathmathmath \in {[}1,p{]}, \phii(e\\\\jmathmathmathmath~) =
\deltai^\\\\jmathmathmathmath ce qui implique évidemment que la famille est
libre~: si \\sum ~
\\\\jmathmathmathmath=1^p\lambda~\\\\jmathmathmathmath\phi\\\\jmathmathmathmath = 0, on a pour tout i \in
{[}1,p{]}

0 = 0(ei) = \\sum
\\\\jmathmathmathmath=1^p\lambda~ \\\\jmathmathmathmath\phi\\\\jmathmathmathmath(ei) =
\sum \\\\jmathmathmathmath=1^p\lambda~~
\\\\jmathmathmathmath\phi\\\\jmathmathmathmath(ei) = \lambda~i

Inversement, si la famille est libre, on peut compléter la famille
(\phi1,\\ldots,\phip~)
en une base
(\phi1,\\ldots,\phin~)
de E^∗ qui est la base duale de la base
(e1,\\ldots,en~)
de E. On a alors \forall~~i,\\\\jmathmathmathmath \in {[}1,p{]},
\phii(e\\\\jmathmathmathmath) = \deltai^\\\\jmathmathmathmath, soit
\forall~i \in {[}1,p{]}, v(ei~) =
\epsiloni. L'image de v contient une base de K^p, c'est
donc K^p et v est sur\\\\jmathmathmathmathective. Dans ces conditions, on peut
appliquer le théorème du rang, et donc dim~
\mathrmKer~v
= dim~ E - p.

Soit x \in\mathrmKer~v~; alors
\forall~i \in {[}1,p{]}, \phii~(x) = 0 et donc
\forall~~\phi
\in\mathrmVect(\phi1,\\\ldots,\phip~),
\phi(x) = 0. Inversement, supposons que
\phi∉\mathrmVect(\phi1,\\\ldots,\phip~)~;
alors la famille
(\phi1,\\ldots,\phip~,\phi)
est libre, on peut la compléter en une base de E^∗ qui est la
base duale d'une base
(e1,\\ldots,en~)
de E~; on a alors ep+1
\in\mathrmKer~v et
\phi(ep+1) = 1. On a donc bien l'équivalence

\begin{align*} \forall~~\phi \in
E^∗,\quad (\phi
\in\mathrmVect(\phi~
1,\\ldots,\phip~)&&\%&
\\ & \Leftrightarrow &
\forall~~x
\in\mathrmKer~v, \phi(x) = 0)\%&
\\ \end{align*}

Remarque~2.4.6 Application~: soit
H1,\\ldots,Hp~
des hyperplans de E d'équations respectives \phi1(x) =
0,\\ldots,\phip~(x)
= 0~; soit r =\
\mathrmrg(\phi1,\\ldots,\phip~).
Quitte à renuméroter les Hi, on peut supposer que
(\phi1,\\ldots,\phir~)
est une base de
\mathrmVect(\phi1,\\\ldots,\phip~).
On a alors, si v : E \rightarrow~ K^r est l'application linéaire
associée à cette famille,

\begin{align*} x \in\⋂
i=1^pH i& \Leftrightarrow &
\forall~i \in {[}1,p{]}, \phii~(x) = 0\%&
\\ & \Leftrightarrow &
\forall~i \in {[}1,r{]}, \phii~(x) = 0\%&
\\ & \Leftrightarrow & x
\in\mathrmKer~v \%&
\\ \end{align*}

ce qui montre que \\⋂
 i=1^pHi est un sous-espace vectoriel de
dimension dim~ E - r. Soit alors H un hyperplan
de E d'équation \phi(x) = 0. On a alors

\begin{align*} \⋂
i=1^pH i \subset~ H& \Leftrightarrow
& \mathrmKer~v
\subset~\mathrmKer~\phi \%&
\\ & \Leftrightarrow & \phi
\in\mathrmVect(\phi1,\\\ldots,\phir~)
=\
\mathrmVect(\phi1,\\ldots,\phip~)\%&
\\ \end{align*}

\paragraph{2.4.6 Application~: polynômes d'interpolation de Lagrange}

Théorème~2.4.9 Soit K un corps commutatif,
x1,\\ldots,xn~
\in K distincts. Soit
a1,\\ldots,an~
\in K. Alors il existe un unique polynôme P \in K{[}X{]} tel que
deg P \leq n - 1 et \\forall~~i
\in {[}1,n{]}, P(xi) = ai.

Démonstration Soit \phii : Kn-1{[}X{]} \rightarrow~ K,
P\mapsto~P(xi) (où Kn-1{[}X{]} =
\P \in
K{[}X{]}∣deg~ P \leq n
- 1\). Les \phii sont des formes linéaires sur
l'espace vectoriel Kn-1{[}X{]} de dimension n~; soit v :
Kn-1{[}X{]} \rightarrow~ K^n,
P\mapsto~(\phi1(P),\\ldots,\phin~(P))
=
(P(x1),\\ldots,P(xn~)).
Alors v est une application linéaire in\\\\jmathmathmathmathective car

\begin{align*} v(P) = 0&
\Leftrightarrow & \forall~~i \in
{[}1,n{]}, P(xi) = 0 \%& \\ &
\Leftrightarrow & \∏
i=1^n(X - x
i)∣P(X) \mathrel\Leftrightarrow P =
0\%& \\ \end{align*}

pour des raisons de degré évidentes. On en déduit que v est un
isomorphisme d'espaces vectoriels, ce qui démontre le résultat.

Remarque~2.4.7 Comme v est sur\\\\jmathmathmathmathective, la famille
(\phi1,\\ldots,\phin~)
est libre~; comme son cardinal est n, c'est une base du dual
Kn-1{[}X{]}^∗. Cherchons la base dont c'est la
duale, c'est-à-dire des polynômes Pi vérifiant
Pi(x\\\\jmathmathmathmath) = \deltai^\\\\jmathmathmathmath~; un tel polynôme
doit être divisible par
\∏ ~
\\\\jmathmathmathmath\neq~i(X - x\\\\jmathmathmathmath). Pour des
raisons de degrés, il doit lui être proportionnel et le fait que
Pi(xi) = 1 nécessite

Pi(X) = \∏
\\\\jmathmathmathmath\neq~i(X - x\\\\jmathmathmathmath)
\over \∏
\\\\jmathmathmathmath\neq~i(xi - x\\\\jmathmathmathmath)

On a alors

\forall~P \in Kn-1~{[}X{]}, P =
\sum i=1^n\phi~
i(P)Pi = \\sum
i=1^nP(x i) \∏
\\\\jmathmathmathmath\neq~i(X - x\\\\jmathmathmathmath)
\over \∏
\\\\jmathmathmathmath\neq~i(xi - x\\\\jmathmathmathmath)

{[}
{[}
{[}
{[}
