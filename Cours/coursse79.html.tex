\textbf{Warning: 
requires JavaScript to process the mathematics on this page.\\ If your
browser supports JavaScript, be sure it is enabled.}

\begin{center}\rule{3in}{0.4pt}\end{center}

{[}
{[}
{[}{]}
{[}

\subsubsection{14.3 Série de Fourier d'une fonction}

\paragraph{14.3.1 Les espaces C et D}

Définition~14.3.1 On considère l'espace vectoriel C des fonctions de \mathbb{R}~
dans \mathbb{C}, continues par morceaux et périodiques de période 2\pi~. On
désignera par D le sous-espace vectoriel des applications f : \mathbb{R}~ \rightarrow~ \mathbb{C},
continues par morceaux, périodiques de période 2\pi~ et vérifiant
\forall~~x \in \mathbb{R}~, f(x) =
f(x^+)+f(x^-) \over 2 (où
f(x^+) et f(x^-) désignent respectivement les
limites à gauche et à droite de f au point x). Pour f,g \inC, on posera
(f∣g) = 1 \over 2\pi~
\int ~
0^2\pi~\overlinef(t)g(t) dt,
\\textbar{}f\\textbar{}2 =
\sqrt(f∣ f) et
en : t\mapsto~e^int.

Théorème~14.3.1 L'application
(f,g)\mapsto~(f\mathrel∣g) est
une forme hermitienne positive sur C dont la restriction à D est définie
positive. La famille (en)n\inℤ est une famille
orthonormée de C. Pour toute f \inC, on a
\\textbar{}f\\textbar{}2
\leq\\textbar{} f\\textbar{}\infty~
(norme de la convergence uniforme).

Démonstration Le caractère sesquilinéaire et la symétrie hermitienne
sont évidents. Si f \inC, on a (f∣f) = 1
\over 2\pi~ \int ~
0^2\pi~\textbar{}f(t)\textbar{}^2 dt ≥ 0. La
nullité de (f∣f) nécessite que f soit nulle
en tout point de {[}0,2\pi~{]} où elle est continue, soit sur {[}0,2\pi~{]}
privé d'un nombre fini de points. Si f est dans D, alors en chacun de
ces points on a f(x^+) = f(x^-) = 0 (car il existe
tout un intervalle ouvert à gauche de x sur lequel f est nul, et de même
à droite) et donc f(x) = 0, par conséquent f est la fonction nulle sur
{[}0,2\pi~{]}, donc sur \mathbb{R}~.

Remarque~14.3.1 On prendra garde que si f est seulement continue par
morceaux,
\\textbar{}f\\textbar{}2 = 0
n'implique pas f = 0.

\paragraph{14.3.2 Coefficients de Fourier d'une fonction continue par
morceaux}

Définition~14.3.2 Soit f : \mathbb{R}~ \rightarrow~ \mathbb{C} continue par morceaux et périodique de
période 2\pi~. On définit les coefficients de Fourier de la fonction f par

\begin{align*} \forall~~n \in
ℤ,\quad cn(f)& =&
(en∣f) = 1 \over
2\pi~ \int ~
0^2\pi~f(t)e^-int dt\%&
\\ \forall~~n ≥
0,\quad an(f)& =& 1 \over
\pi~ \int ~
0^2\pi~f(t)cos~ nt dt \%&
\\ \forall~~n ≥
1,\quad bn(f)& =& 1 \over
\pi~ \int ~
0^2\pi~f(t)sin~ nt dt \%&
\\ \end{align*}

Remarque~14.3.2 Les fonctions intégrées étant périodiques de période 2\pi~,
on a aussi pour tout a \in \mathbb{R}~, cn(f) = 1 \over
2\pi~ \int ~
a^a+2\pi~f(t)e^-int dt, an(f) = 1
\over \pi~ \int ~
a^a+2\pi~f(t)cos~ nt dt,
bn(f) = 1 \over \pi~
\int ~
a^a+2\pi~f(t)sin~ nt dt et en
particulier cn(f) = 1 \over 2\pi~
\int  -\pi~^\pi~f(t)e^-int~
dt, an(f) = 1 \over \pi~
\int ~
-\pi~^\pi~f(t)cos~ nt dt,
bn(f) = 1 \over \pi~
\int ~
-\pi~^\pi~f(t)sin~ nt dt

Proposition~14.3.2 On a les relations suivantes

\begin{align*} c0(f)& =&
a0(f) \over 2 \%&
\\ \forall~~n ≥
1,\quad cn(f)& =& an(f) -
ibn(f) \over 2 ,\quad
c-n(f) = an(f) + ibn(f)
\over 2 \%& \\
\forall~n ≥ 1,\quad an~(f)&
=& cn(f) + c-n(f),\quad bn
= i(cn(f) - c-n(f)) \%&
\\ \end{align*}

Démonstration Elémentaire

Proposition~14.3.3 Soit f : \mathbb{R}~ \rightarrow~ \mathbb{C} continue par morceaux et périodique de
période 2\pi~. Si f est à valeurs réelles, on a an(f) \in \mathbb{R}~,
bn(f) \in \mathbb{R}~ et c-n(f) =
\overlinecn(f). Si f est paire (resp.
impaire) on a bn(f) = 0 (resp. an(f) = 0).

Démonstration Si f est à valeurs réelles, il en est de même de
x\mapsto~f(x)cos~ nx et de
x\mapsto~f(x)sin~ nx ce qui
montre que an(f) et bn(f) sont réels~; de plus
f(x)e^inx = \overlinef(x)e^-inx
ce qui montre que c-n(f) =
\overlinecn(f). Si f est paire, on a
bn(f) = 1 \over 2\pi~
\int ~
-\pi~^\pi~f(x)sin~ nx dx = 0 puisque
la fonction f(x)sin~ nx est impaire. Le
raisonnement est similaire si f est impaire avec les an(f).

Définition~14.3.3 Soit f : \mathbb{R}~ \rightarrow~ \mathbb{C} continue par morceaux et périodique de
période 2\pi~. On appelle série de Fourier de la fonction f la série
trigonométrique

\begin{align*} c0(f) +
\sum n≥1(cn(f)e^inx~
+ c -n(f)e^-inx)&& \%&
\\ & & = a0(f)
\over 2 + \\sum
n≥1(an(f)\cos nx +
bn(f)\sin nx)\%&
\\ \end{align*}

Définition~14.3.4 Pour n ≥ 1, on posera (sommes partielles de la série
de Fourier)

\begin{align*} Sn(f)(x)& =&
c0(f) + \\sum
p=1^n(c p(f)e^ipx + c
-p(f)e^-ipx) \%& \\ & =&
a0(f) \over 2 + \\sum
p=1^n(a p(f)\cos px +
bp(f)\sin px)\%&
\\ \end{align*}

\paragraph{14.3.3 Inégalité de Bessel et théorème de Riemann-Lebesgue}

Définition~14.3.5 Pour N ≥ 1, on posera TN
=\
\mathrmVect(e-N,e-N+1,\\ldots,e-1,e0,e1,\\\ldots,eN-1,eN~)
(espace vectoriel des polynômes trigonométriques de degré inférieur ou
égal à N.

Remarque~14.3.3 On a également

TN = \x\mapsto~
a0 \over 2 + \\sum
p=1^N(a p \cos px +
bp \sin px)\

Par définition même
(e-N,e-N+1,\\ldots,e-1,e0,e1,\\\ldots,eN-1,eN~)
est une base orthonormée de TN.

Lemme~14.3.4 Soit f : \mathbb{R}~ \rightarrow~ \mathbb{C} continue par morceaux et périodique de
période 2\pi~. Alors
\\textbar{}SN(f)\\textbar{}2^2
= \\sum ~
k=-N^N\textbar{}ck(f)\textbar{}^2.

Démonstration
c-N(f),\\ldots,c0(f),\\\ldots,cN~(f)
sont les coordonnées de SN(f) dans la base orthonormée
(e-N,e-N+1,\\ldots,e-1,e0,e1,\\\ldots,eN-1,eN~)~;
la norme au carré de SN(f) est donc la somme des carrés des
modules de ces coordonnées~; d'où le résultat.

Lemme~14.3.5 Soit f : \mathbb{R}~ \rightarrow~ \mathbb{C} continue par morceaux et périodique de
période 2\pi~. Alors SN(f) est la pro\\\\jmathmathmathmathection orthogonale de f sur
le sous-espace vectoriel TN.

Démonstration Puisque SN(f) appartient à TN, il
suffit de montrer que f - SN(f) \bot TN ou encore que
\forall~~n \in {[}-N,N{]},
(en∣f - SN(f)) = 0, ou
encore que \forall~~n \in {[}-N,N{]},
(en∣f) =
(en∣SN(f)). Mais
(en∣SN(f)) est la
coordonnée suivant en de SN(f) (puisque la base est
orthonormée), c'est donc cn(f) =
(en∣f) par définition, ce qui
montre le résultat.

Théorème~14.3.6 (Bessel). Soit f : \mathbb{R}~ \rightarrow~ \mathbb{C} continue par morceaux et
périodique de période 2\pi~. Alors la série
\textbar{}c0(f)\textbar{}^2
+ \\sum ~
n≥1(\textbar{}cn(f)\textbar{}^2 +
\textbar{}c-n(f)\textbar{}^2) est convergente et on
a

\textbar{}c0(f)\textbar{}^2 +
\sum n=1^+\infty~(\textbar{}c~
n(f)\textbar{}^2 + \textbar{}c
-n(f)\textbar{}^2) \leq\\textbar{}
f\\textbar{} 2^2

Démonstration Puisque SN(f) est la pro\\\\jmathmathmathmathection orthogonale de f
sur TN, on a f = SN(f) + (f - SN(f)) avec
SN(f) \bot f - SN(f). Le théorème de Pythagore assure
que
\\textbar{}f\\textbar{}2^2
=\\textbar{}
SN(f)\\textbar{}2^2
+\\textbar{} f -
SN(f)\\textbar{}2^2, d'où
encore d'après le lemme 1

\textbar{}c0(f)\textbar{}^2 +
\sum n=1^N(\textbar{}c~
n(f)\textbar{}^2 + \textbar{}c
-n(f)\textbar{}^2) =\\textbar{} S
N(f)\\textbar{}2^2
\leq\\textbar{} f\\textbar{}
2^2

La série à termes positifs
\textbar{}c0(f)\textbar{}^2
+ \\sum ~
n≥1(\textbar{}cn(f)\textbar{}^2 +
\textbar{}c-n(f)\textbar{}^2) a ses sommes
partielles ma\\\\jmathmathmathmathorées par
\\textbar{}f\\textbar{}2^2,
donc elle converge et sa somme est ma\\\\jmathmathmathmathorée par
\\textbar{}f\\textbar{}2^2,
ce qui achève la démonstration.

Remarque~14.3.4 Un calcul élémentaire montre que pour n ≥ 1,

\textbar{}cn(f)\textbar{}^2 + \textbar{}c
-n(f)\textbar{}^2 = 1 \over 2
(\textbar{}an(f)\textbar{}^2 + \textbar{}b
n(f)\textbar{}^2)

ce qui montre que les séries
\\sum ~
\textbar{}an(f)\textbar{}^2 et
\\sum ~
\textbar{}bn(f)\textbar{}^2 convergent et que (en
tenant compte de a0(f) = c0(f)
\over 2 )

 \textbar{}a0(f)\textbar{}^2 \over
4 + 1 \over 2 \\sum
n=1^+\infty~(\textbar{}a
n(f)\textbar{}^2 + \textbar{}b
n(f)\textbar{}^2) \leq\\textbar{}
f\\textbar{}^2

Théorème~14.3.7 (Riemann-Lebesgue). Soit f : \mathbb{R}~ \rightarrow~ \mathbb{C} continue par morceaux
et périodique de période 2\pi~. Alors

limn\rightarrow~±\infty~cn~(f)
= limn\rightarrow~+\infty~an~(f)
= limn\rightarrow~+\infty~bn~(f) = 0

Démonstration Puisque les séries
\\sum ~
n≥1(\textbar{}cn(f)\textbar{}^2 +
\textbar{}c-n(f)\textbar{}^2),
\\sum ~
\textbar{}an(f)\textbar{}^2 et
\\sum ~
\textbar{}bn(f)\textbar{}^2 sont convergentes,
leurs termes généraux admettent la limite 0, ce qui montre le résultat.

\paragraph{14.3.4 Les théorèmes de Dirichlet}

Nous aurons besoin par la suite du lemme suivant

Lemme~14.3.8 Pour tout entier n ≥ 1 et pour
t∉2\pi~ℤ,
\\sum ~
k=-n^ne^ikt = sin~
(2n+1) t \over 2 \over
sin  t \over 2 ~ .

Démonstration On a en effet

\begin{align*} \\sum
k=-n^ne^ikt& =& e^-int
\sum k=0^2ne^ikt~ =
e^-int e^(2n+1)it - 1 \over
e^it - 1 \%& \\ & =&
e^(n+1)it - e^-int \over
e^it - 1 = e^(n+ 1 \over 2
)it - e^-(n+ 1 \over 2 )it
\over e^i t \over 2  -
e^-i t \over 2  \%&
\\ \end{align*}

en multipliant numérateur et dénominateur par e^-it\diagup2. On en
déduit immédiatement la formule souhaitée.

Théorème~14.3.9 (Dirichlet). Soit f : \mathbb{R}~ \rightarrow~ \mathbb{C} de classe \mathcal{C}^1 par
morceaux et périodique de période 2\pi~. Alors la série de Fourier de f
converge sur \mathbb{R}~ et

\forall~x \in \mathbb{R}~, f(x^+~) +
f(x^-) \over 2 = c0(f) +
\sum n=1^+\infty~(c~
n(f)e^inx + c -n(f)e^-inx)

Démonstration On a

\begin{align*} Sn(f)(x)& =& 1
\over 2\pi~ \\sum
k=-n^ne^inx
\\int  ~
0^2\pi~f(t)e^-int dt\%&
\\ & =& 1 \over 2\pi~
\int ~
0^2\pi~f(t)\left (\\sum
k=-n^ne^in(x-t)\right )
dt\%& \\ & =& 1 \over
2\pi~ \int  0^2\pi~~f(t)
sin (2n + 1) x-t \over 2~
\over sin~  x-t
\over 2  dt \%& \\
\end{align*}

Faisons le changement de variable t = x + u, on obtient

\begin{align*} Sn(f)(x)& =& 1
\over 2\pi~ \int ~
-x^2\pi~-xf(x + u) sin~ (2n +
1) u \over 2 \over
sin  u \over 2 ~ du\%&
\\ & =& 1 \over 2\pi~
\int  -\pi~^\pi~~f(x + u)
sin (2n + 1) u \over 2~
\over sin~  u
\over 2  du \%& \\
\end{align*}

puisque la fonction intégrée est périodique de période 2\pi~ et que donc
son intégrale sur tout intervalle de longueur 2\pi~ est la même. Coupons
l'intégrale en deux, l'une de - \pi~ à 0, l'autre de 0 à \pi~. Dans la
première faisons le changement de variable u = -2v et dans la seconde le
changement de variable u = 2v. On obtient

\begin{align*} Sn(f)(x)& =& 1
\over \pi~ \int ~
0^\pi~\diagup2f(x - 2v) sin~ (2n + 1)v
\over sin v~ dv \%&
\\ & \text & + 1
\over \pi~ \int ~
0^\pi~\diagup2f(x + 2v) sin~ (2n + 1)v
\over sin v~ dv \%&
\\ & =& 1 \over \pi~
\int  0^\pi~\diagup2~(f(x + 2v) + f(x -
2v)) sin~ (2n + 1)v \over
sin v~ dv\%&
\\ \end{align*}

Appliquons le résultat précédent à la fonction constante f0 :
x\mapsto~1. On a bien entendu
Sn(f0)(x) = 1 puisque c0(f0) = 1
et cn(f0) = 0 pour n\neq~0~;
on obtient

1 = 2 \over \pi~ \int ~
0^\pi~\diagup2 sin~ (2n + 1)v
\over sin v~ dv

On en déduit que

\begin{align*} Sn(f)(x) -
f(x^+) + f(x^-) \over 2 =&&
\%& \\ & & 1 \over \pi~
\int  0^\pi~\diagup2~ f(x + 2v) -
f(x^+) + f(x - 2v) - f(x -) \over
sin v \sin~ (2n +
1)v dv\%& \\
\end{align*}

Considérons la fonction g périodique de période 2\pi~ définie par

\begin{align*} g(v)& =& f(x + 2v) -
f(x^+) + f(x - 2v) - f(x-) \over
sin v \text pour ~v
\in{]}0, \pi~ \over 2 {]}\%&
\\ g(0)& =& 2(f'(x^+) -
f'(x^-)) \%& \\ g(v)& =&
0\text pour v \in{]} \pi~ \over 2
,2\pi~{[} \%& \\
\end{align*}

Comme la fonction \tildef définie par
\tildef(x) = f(x^+) et
\tildef(t) = f(t) pour t \textgreater{} x est
dérivable à droite au point x (puisque f est de classe \mathcal{C}^1
par morceaux), on a, quand v tend vers 0 par valeurs supérieures,

\begin{align*} f(x + 2v) - f(x^+))
\over sin v~ &
∼v\rightarrow~0,v\textgreater{}0& f(x + 2v) - f(x^+)
\over v \%& \\ & = &
2 \tildef(x + 2v) -\tilde f(x)
\over 2v \%& \\
\end{align*}

de limite 2f'(x^+). De même on a

limv\rightarrow~0,v\textgreater{}0~ f(x - 2v)
- f(x-) \over sin v~
= -2f'(x^-)

ce qui montre que g est continue à droite au point 0. On en déduit
immédiatement que g est continue par morceaux. Mais alors

\begin{align*} Sn(f)(x) -
f(x^+) + f(x^-) \over 2 && \%&
\\ & =& 1 \over \pi~
\int ~
0^2\pi~g(v)sin~ (2n + 1)v dv =
b 2n+1(g)\%& \\
\end{align*}

D'après le théorème de Riemann-Lebesgue, cette expression tend vers 0
quand n tend vers + \infty~, ce qui montre à la fois la convergence de la
série et donne la valeur de sa somme.

Lemme~14.3.10 Soit f : \mathbb{R}~ \rightarrow~ \mathbb{C} périodique de période 2\pi~de classe
\mathcal{C}^1 par morceaux et continue. Alors
\forall~n \in ℤ, cn(f') = incn~(f)
(où f' désigne la fonction de D égale à la dérivée de f sauf en un
nombre fini de points modulo 2\pi~).

Démonstration Soit \sigma = (ai)0\leqi\leqp une subdivision de
{[}0,2\pi~{]} adaptée à f. En tout point de {[}0,2\pi~{]}
\diagdown\a0,\\ldots,ap\~,
f'(t) est la dérivée de f et on pose f'(ai) = 1
\over 2 (f'(ai^+) + f'(a
i^-)), si bien que f' \inD. Une intégration par parties donne, si
{[}a,b{]} \subset~{]}ai-1,ai{[},

\begin{align*} \int ~
a^bf'(t)e^-int dt& =& \left
{[}f(t)e^-int\right {]} a^b +
in\int  a^bf(t)e^-int~
dt \%& \\ & =& f(b)e^-inb -
f(a)e^-ina + in\int ~
a^bf(t)e^-int dt\%&
\\ \end{align*}

En faisant tendre a vers ai-1 et b vers ai, en
tenant compte de la continuité de f aux points ai-1 et
ai on obtient

\begin{align*} \int ~
ai-1^ai f'(t)e^-int dt&
=& f(a i)e^-inai  -
f(ai-1)e^-inai-1 \%&
\\ & \text &
+in\int ~
ai-1^ai f(t)e^-int dt \%&
\\ \end{align*}

et en sommant

\begin{align*} \int ~
0^2\pi~f'(t)e^-int dt&& \%&
\\ & =& \\sum
i=1^p
\\int  ~
ai-1^ai f'(t)e^-int dt
\%& \\ & =& \\sum
i=1^p\left (f(a
i)e^-inai  -
f(ai-1)e^-inai-1 \right )
+ in\\int  ~
ai-1^ai f(t)e^-int dt\%&
\\ & =&
f(ap)e^-inap  -
f(a0)e^-ina0  +
in\int  a0^ap~
f(t)e^-int dt \%& \\ & =&
in\int ~
0^2\pi~f(t)e^-int dt \%&
\\ \end{align*}

puisque a0 = 0, ap = 2\pi~,
f(ap)e^-inap = f(2\pi~)e^-in2\pi~ =
f(2\pi~) = f(0) = f(a0)e^-ina0. En divisant
par 2\pi~, on obtient cn(f') = incn(f).

Théorème~14.3.11 (Dirichlet). Soit f : \mathbb{R}~ \rightarrow~ \mathbb{C} périodique de période 2\pi~ de
classe \mathcal{C}^1 par morceaux et continue. Alors la série
\textbar{}c0(f)\textbar{} +\
\sum ~
n≥1(\textbar{}cn(f)\textbar{} +
\textbar{}c-n(f)\textbar{}) converge, la série de Fourier de f
converge normalement sur \mathbb{R}~ et on a

\forall~x \in \mathbb{R}~, f(x) = c0~(f) +
\sum n=1^+\infty~(c~
n(f)e^inx + c -n(f)e^-inx)

(autrement dit f est somme de sa série de Fourier).

Démonstration Pour a et b réels on a ab \leq 1 \over 2
(a^2 + b^2)~; on en déduit que si
n\neq~0, on a 0
\leq\textbar{}cn(f)\textbar{} = \left \textbar{}
cn(f') \over in \right
\textbar{}\leq 1 \over 2
(\textbar{}cn(f)\textbar{}^2 + 1
\over n^2 ). D'après le théorème de Bessel,
la série \\sum ~
n≥0\textbar{}cn(f')\textbar{}^2 converge et
d'après la théorie des séries de Riemann la série
\\sum ~  1
\over n^2 converge. On en déduit que la
série \\sum ~
n≥1\textbar{}cn(f)\textbar{} converge. On montre de la
même fa\ccon que la série
\\sum ~
n≥1\textbar{}c-n(f)\textbar{} converge, d'où la
convergence de la série
\\sum ~
n≥1(\textbar{}cn(f)\textbar{} +
\textbar{}c-n(f)\textbar{}). La convergence normale de la
série de Fourier en résulte immédiatement puisque

\forall~~x \in \mathbb{R}~,
\textbar{}cn(f)e^inx + c
-n(f)e^-inx\textbar{}\leq\textbar{}c n(f)\textbar{}
+ \textbar{}c-n(f)\textbar{}

qui est une série convergente indépendante de x. La formule résulte du
premier théorème de Dirichlet en remarquant que si f est continue, f(x)
= f(x^+)+f(x^-) \over 2 .

\paragraph{14.3.5 Coefficients de Fourier des fonctions de classe
C^k}

Théorème~14.3.12 Soit f : \mathbb{R}~ \rightarrow~ \mathbb{C} périodique de période 2\pi~ de classe
C^k. Alors

\forall~n \in ℤ, cn~(f) =
(in)^kc n(f^(k))

et, quand \textbar{}n\textbar{} tend vers + \infty~, cn(f) = o( 1
\over n^k ).

Démonstration On a vu que cn(f') = incn(f) et il
suffit de faire une récurrence évidente sur k pour obtenir
cn(f) = (in)^kcn(f^(k)). Comme
le théorème de Riemann-Lebesgue assure que
lim\textbar{}n\textbar{}\rightarrow~+\infty~cn(f^(k)~)
= 0, on a cn(f) = o( 1 \over n^k
).

Remarque~14.3.5 Autrement dit, plus la fonction est régulière, plus vite
les coefficients de Fourier tendent vers 0 à l'infini. Si f est de
classe C^\infty~, on a pour tout k \in \mathbb{N}~,
lim\textbar{}n\textbar{}\rightarrow~+\infty~n^kcn~(f)
= 0 (typiquement les coefficients de Fourier seront à décroissance
exponentielle).

\paragraph{14.3.6 Le théorème de Parseval}

Lemme~14.3.13 Soit f : \mathbb{R}~ \rightarrow~ \mathbb{C} périodique de période 2\pi~ et continue par
morceaux. Alors, pour tout \epsilon \textgreater{} 0, il existe g : \mathbb{R}~ \rightarrow~ \mathbb{C}
périodique de période 2\pi~, de classe \mathcal{C}^1 par morceaux et
continue telle que \\textbar{}f -
g\\textbar{}2 \textless{} \epsilon.

Démonstration Supposons tout d'abord que f est en escalier et soit 0 =
a0 \textless{} a1 \textless{}
\\ldots~ \textless{}
ap = 2\pi~ une subdivision de {[}0,2\pi~{]} adaptée à f avec f(t) =
\lambda~i pour t \in{]}ai-1,ai{[}. Soit \delta le pas de
la subdivision. Pour  2 \over n \textless{} \eta
définissons une fonction gn par

\begin{itemize}
\itemsep1pt\parskip0pt\parsep0pt
\item
  (i) \forall~~i \in {[}0,p{]},
  gn(ai) = 0
\item
  (ii) \forall~~i \in {[}1,p{]},
  \forall~t \in {[}ai-1~ + 1
  \over n ,ai - 1 \over n
  {]}, gn(t) = \lambda~i
\item
  (iii) gn est affine sur chacun des intervalles
  {[}ai-1,ai-1 + 1 \over n {]} et
  {[}ai - 1 \over n ,ai{]}.
\end{itemize}

Il est clair que gn est continue, affine par morceaux. Comme
de plus gn(0) = gn(2\pi~) = 0 elle se prolonge en une
application continue et périodique de période 2\pi~ sur \mathbb{R}~. Puisque
gn est affine par morceaux, elle est a fortiori de classe
\mathcal{C}^1 par morceaux. On a

\begin{align*} \int ~
ai-1^ai \textbar{}f(t) -
gn(t)\textbar{}^2 dt& =&
\int ~
ai-1^ai-1+ 1 \over n
\textbar{}f(t) - gn(t)\textbar{}^2 dt + \%&
\\ & \text &
\int  ai~- 1
\over n ^ai \textbar{}f(t) -
gn(t)\textbar{}^2 dt \%&
\\ \end{align*}

Mais on a g(t) = n\lambda~i(t - ai) pour t \in
{[}ai-1,ai-1 + 1 \over n {]} et
g(t) = -n\lambda~i(t - ai) pour t \in {[}ai - 1
\over n ,ai{]}. On a donc

\begin{align*} \int ~
ai-1^ai \textbar{}f(t) -
gn(t)\textbar{}^2 dt&& \%&
\\ & =&
\textbar{}\lambda~i\textbar{}^2\left
(\int ~
ai-1^ai-1+ 1 \over n
(1 - n(t - ai-1))^2 dt\right .
\%& \\ & \text &
\quad \quad + \left
.\int  ai~- 1
\over n ^ai (1 + n(t -
ai))^2 dt\right ) \%&
\\ & =&
\textbar{}\lambda~i\textbar{}^2\left
(\int  0~^ 1 \over
n (1 - nu)^2 dt +\int  ~-
1 \over n ^0(1 + nu)^2
dt\right )\%& \\ & =&
\textbar{}\lambda~i\textbar{}^2 \over 3n
\left (\left {[}-(1 -
nu)^3\right {]} 0^ 1
\over n  + \left {[}(1 +
nu)^3\right {]}- 1 \over
n ^0\right ) \%&
\\ & =&
2\textbar{}\lambda~i\textbar{}^2 \over 3n
\%& \\ \end{align*}

soit encore

2\pi~\\textbar{}f -
gn\\textbar{}2^2
=\int  0^2\pi~~\textbar{}f(t) -
g n(t)\textbar{}^2 dt = 2 \over
3n \\sum
i=1^p\textbar{}\lambda~ i\textbar{}^2

On en déduit que
limn\rightarrow~+\infty~~\\textbar{}f -
gn\\textbar{}2 = 0 et que donc on
peut trouver un n tel que  2 \over n \textless{} \eta
avec \\textbar{}f -
gn\\textbar{}2 \textless{} \epsilon.

Supposons maintenant que f est continue par morceaux. Sa restriction à
{[}0,2\pi~{]} est réglée et donc on peut trouver \phi en escalier sur
{[}0,2\pi~{[} (et que l'on prolonge par périodicité) telle que
\\textbar{}f - \phi\\textbar{}\infty~
\textless{} \epsilon \over 2 . On a alors

\begin{align*} \\textbar{}f -
\phi\\textbar{}2^2& =& 1
\over 2\pi~ \int ~
0^2\pi~\textbar{}f(t) - \phi(t)\textbar{}^2 dt \leq 1
\over 2\pi~ \int ~
0^2\pi~\\textbar{}f -
\phi\\textbar{} \infty~^2 dt\%&
\\ & =& \\textbar{}f -
\phi\\textbar{}\infty~^2 \%&
\\ \end{align*}

soit encore \\textbar{}f -
\phi\\textbar{}2 \leq\\textbar{} f -
\phi\\textbar{}\infty~ \textless{} \epsilon
\over 2 . Mais d'autre part, comme \phi est en escalier,
on sait qu'on peut trouver g continue et affine par morceaux telle que
\\textbar{}\phi - g\\textbar{}2
\textless{} \epsilon \over 2 . On a alors
\\textbar{}f - g\\textbar{}2
\leq\\textbar{} f - \phi\\textbar{}2
+\\textbar{} \phi - g\\textbar{}2
\textless{} \epsilon ce qui démontre le lemme.

Théorème~14.3.14 (Parseval-Plancherel). Soit f : \mathbb{R}~ \rightarrow~ \mathbb{C} périodique de
période 2\pi~ et continue par morceaux. Alors

\begin{align*}
\\textbar{}f\\textbar{}2^2&
=& 1 \over 2\pi~ \int ~
0^2\pi~\textbar{}f(t)\textbar{}^2 dt \%&
\\ & =&
\textbar{}c0(f)\textbar{}^2 +
\sum n=1^+\infty~(\textbar{}c~
n(f)\textbar{}^2 + \textbar{}c
-n(f)\textbar{}^2) \%& \\ &
=& \textbar{}a0(f)\textbar{}^2
\over 4 + 1 \over 2
\sum n=1^+\infty~(\textbar{}a~
n(f)\textbar{}^2 + \textbar{}b
n(f)\textbar{}^2)\%& \\
\end{align*}

Démonstration On sait que
\textbar{}c0(f)\textbar{}^2
+ \\sum ~
n=1^N(\textbar{}cn(f)\textbar{}^2 +
\textbar{}c-n(f)\textbar{}^2)
=\\textbar{}
SN(f)\\textbar{}2^2. D'autre
part, à l'aide du théorème de Pythagore et puisque SN(f) est
la pro\\\\jmathmathmathmathection orthogonale de f sur le sous-espace TN des
polynômes trigonométriques de degré au plus N, on a
\\textbar{}f\\textbar{}2^2
=\\textbar{}
SN(f)\\textbar{}2^2
+\\textbar{} f -
SN(f)\\textbar{}2^2. Le
résultat à démontrer est donc équivalent à
limN\rightarrow~+\infty~\\textbar{}SN(f)\\textbar{}2^2~
=\\textbar{}
f\\textbar{}2^2, soit encore à
limN\rightarrow~+\infty~~\\textbar{}f -
SN(f)\\textbar{}2 = 0.

Supposons tout d'abord que f est \mathcal{C}^1 par morceaux et
continue. On sait que la série de Fourier de f converge normalement,
donc uniformément vers f. On a donc
limN\rightarrow~+\infty~~\\textbar{}f -
SN(f)\\textbar{}\infty~ = 0, mais comme ci
dessus, on a \\textbar{}f -
SN(f)\\textbar{}2
\leq\\textbar{} f -
SN(f)\\textbar{}\infty~ ce qui montre que
limN\rightarrow~+\infty~~\\textbar{}f -
SN(f)\\textbar{}2 = 0.

Si maintenant f est seulement continue par morceaux, soit \epsilon
\textgreater{} 0 et g : \mathbb{R}~ \rightarrow~ \mathbb{C} périodique de période 2\pi~, de classe
\mathcal{C}^1 par morceaux et continue telle que
\\textbar{}f - g\\textbar{}2
\textless{} \epsilon \over 2 . D'après le premier cas, on a
limN\rightarrow~+\infty~~\\textbar{}g -
SN(g)\\textbar{}2 = 0 et donc il
existe N0 \in \mathbb{N}~ tel que N ≥ N0
\rigtharrow~\\textbar{} g -
SN(g)\\textbar{}2 \textless{} \epsilon
\over 2 . Mais comme SN(g) \in TN et
que SN(f) est la pro\\\\jmathmathmathmathection orthogonale de f sur TN,
on a \\textbar{}f -
SN(f)\\textbar{}2
\leq\\textbar{} f -
SN(g)\\textbar{}2 soit encore, pour N
≥ N0,

\begin{align*} \\textbar{}f -
SN(f)\\textbar{}2& \leq&
\\textbar{}f -
SN(g)\\textbar{}2
\leq\\textbar{} f - g\\textbar{}2
+\\textbar{} g -
SN(g)\\textbar{}2\%&
\\ & \textless{}& \epsilon
\over 2 + \epsilon \over 2 = \epsilon \%&
\\ \end{align*}

ce qui démontre le résultat. La deuxième formule résulte d'un calcul
précédent qui montre que

\begin{align*}
\textbar{}c0(f)\textbar{}^2& +&
\sum n=1^+\infty~(\textbar{}c~
n(f)\textbar{}^2 + \textbar{}c
-n(f)\textbar{}^2) \%& \\ &
=& \textbar{}a0(f)\textbar{}^2
\over 4 + 1 \over 2
\sum n=1^+\infty~(\textbar{}a~
n(f)\textbar{}^2 + \textbar{}b
n(f)\textbar{}^2)\%& \\
\end{align*}

Corollaire~14.3.15 (in\\\\jmathmathmathmathectivité de la transformation de Fourier). Soit f
et g deux fonctions de D telles que \forall~~n \in \mathbb{N}~,
cn(f) = cn(g). Alors f = g.

Démonstration On a \forall~n \in \mathbb{N}~, cn~(f - g)
= 0, soit encore d'après le théorème de Parseval,
\\textbar{}f - g\\textbar{}2 =
0. Comme f - g appartient à D sur laquelle le produit scalaire est
défini positif, on a f - g = 0.

Remarque~14.3.6 Si on suppose seulement que f et g sont continues par
morceaux, on obtient seulement que f et g coïncident sauf en un nombre
fini de points (sur un intervalle de longueur 2\pi~).

{[}
{[}
{[}
{[}
