\textbf{Warning: 
requires JavaScript to process the mathematics on this page.\\ If your
browser supports JavaScript, be sure it is enabled.}

\begin{center}\rule{3in}{0.4pt}\end{center}

{[}
{[}
{[}{]}
{[}

\subsubsection{13.4 Endomorphismes d'un espace hermitien}

\paragraph{13.4.1 Notion d'ad\\\\jmathmathmathmathoint}

Soit E un espace préhilbertien complexe

Définition~13.4.1 Soit E un espace préhilbertien complexe. Soit u,v \in
L(E). On dit que u et v sont des endomorphismes ad\\\\jmathmathmathmathoints si

\forall~x,y \in E, (u(x)\mathrel∣~y)
= (x∣v(y))

Remarque~13.4.1 La symétrie hermitienne du produit scalaire montre
clairement que u et v \\\\jmathmathmathmathouent des rôles symétriques, donc que u est
ad\\\\jmathmathmathmathoint de v si et seulement si~v est ad\\\\jmathmathmathmathoint de u.

Théorème~13.4.1 Soit E un espace hermitien. Tout endomorphisme de E
admet un unique ad\\\\jmathmathmathmathoint u^∗. Si u \in L(E), \mathcal{E} une base de E, \Omega
= \mathrmMat~ (\phi,\mathcal{E}) et A
= \mathrmMat~ (u,\mathcal{E}), alors

\mathrmMat~
(u^∗,\mathcal{E}) = \Omega^-1A^∗\Omega

Démonstration Soit \mathcal{E} une base de E et \Omega =\
\mathrmMat (\phi,\mathcal{E}). Comme \phi est non dégénérée, la
matrice \Omega est inversible. Soit u,v \in L(E), A =\
\mathrmMat (u,\mathcal{E}) et B =\
\mathrmMat (v,\mathcal{E}). Si x,y \in E, on a
(u(x)∣y) = (AX)^∗\OmegaY =
X^∗A^∗\OmegaY et (x∣v(y)) =
X^∗\OmegaBY . L'unicité de la matrice de la forme sesquilinéaire
(x,y)\mapsto~(u(x)\mathrel∣y)
montre que

\begin{align*} \forall~~x,y \in E,
(u(x)∣y) =
(x∣v(y))&& \%&
\\ & \Leftrightarrow &
A^∗\Omega = \OmegaB \Leftrightarrow B =
\Omega^-1A^∗\Omega\%& \\
\end{align*}

ce qui montre à la fois l'existence et l'unicité de l'ad\\\\jmathmathmathmathoint et la
formule voulue.

Proposition~13.4.2 Soit E un espace hermitien. L'application
u\mapsto~u^∗ est un endomorphisme
semi-linéaire involutif de L(E). Si u,v \in L(E), alors u \cdot v aussi et
(u \cdot v)^∗ = v^∗\cdot u^∗. Si u \in L(E) est
inversible, alors u^∗ est inversible et
(u^-1)^∗ = (u^∗)^-1.

Démonstration On a dé\\\\jmathmathmathmathà vu que la relation u et v sont ad\\\\jmathmathmathmathoints était
symétrique, donc si u \in L(E), u^∗ aussi et u^∗∗ =
u. Si u,v \in L(E), \alpha~,\beta~ \in \mathbb{C}, on a

\begin{align*} ((\alpha~u +
\beta~v)(x)∣y)& =& (\alpha~u(x) +
\beta~v(x)∣y) \%&
\\ & =&
\overline\alpha~(u(x)∣y) +
\overline\beta~(v(x)∣y) \%&
\\ & =&
\overline\alpha~(x∣u^∗(y))
+
\overline\beta~(x∣v^∗(y))\%&
\\ & =&
(x∣(\overline\alpha~u^∗
+ \overline\beta~v^∗)(y)) \%&
\\ \end{align*}

ce qui montre que (\alpha~u + \beta~v)^∗ =
\overline\alpha~u^∗ +
\overline\beta~v^∗ et donc la semilinéarité de
u\mapsto~u^∗. Si u,v \in L(E), on a

(u \cdot v(x)∣y) =
(v(x)∣u^∗(y)) =
(x∣v^∗\cdot u^∗(y))

ce qui montre que u \cdot v admet v^∗\cdot u^∗ comme
ad\\\\jmathmathmathmathoint.

Si u est inversible, on a u^-1 \cdot u =
\mathrmIdE d'où (u^-1 \cdot
u)^∗ = \mathrmIdE^∗, soit
u^∗\cdot (u^-1)^∗ =
\mathrmIdE. De même u \cdot u^-1 =
\mathrmIdE donne par passage à l'ad\\\\jmathmathmathmathoint
(u^-1)^∗\cdot u^∗ =
\mathrmIdE. Ceci montre que u^∗
est inversible et que (u^-1)^∗ =
(u^∗)^-1

Proposition~13.4.3 Soit E un espace hermitien, u \in L(E). Alors

\begin{itemize}
\itemsep1pt\parskip0pt\parsep0pt
\item
  (i) \mathrm{det}~
  u^∗ =
  \overline\mathrm{det}~
  u,
  \mathrm{tr}u^∗~ =
  \overline\mathrm{tr}u~,
  \chiu^∗ = \overline\chiu
\item
  (ii)
  \mathrmKeru^∗~
  =
  (\mathrmImu)^\bot~,
  \mathrmImu^∗~ =
  (\mathrmKeru)^\bot~
\item
  (iii)
  \mathrmKeru^∗~u
  = \mathrmKer~u et
  \mathrmImu^∗~u
  = \mathrmImu^∗~
\end{itemize}

Démonstration (i) Soit \mathcal{E} une base de E, \Omega =\
\mathrmMat (\phi,\mathcal{E}) et A =\
\mathrmMat (u,\mathcal{E}), alors
\mathrmMat~
(u^∗,\mathcal{E}) = \Omega^-1A^∗\Omega. On a donc
\mathrm{det} u^∗~
= \mathrm{det}~
\Omega^-1A^∗\Omega =\
\mathrm{det} A^∗ =
\overline\mathrm{det}~
A =
\overline\mathrm{det}~
u. La démonstration est la même pour la trace et pour le polynôme
caractéristique.

(ii) On a

\begin{align*} x
\in\mathrmKeru^∗~&
\Leftrightarrow & u^∗(x) = 0
\Leftrightarrow \forall~~y \in E,
(u^∗(x)∣y) = 0 \%&
\\ & \Leftrightarrow &
\forall~y \in E, (x\mathrel∣~u(y)) =
0 \Leftrightarrow x \in
(\mathrmImu)^\bot~\%&
\\ \end{align*}

En appliquant ce résultat à u^∗ on obtient,
\mathrmKer~u =
(\mathrmImu^∗)^\bot~
et en prenant l'orthogonal,
\mathrmImu^∗~ =
(\mathrmKeru)^\bot~

(iii) On a visiblement u(x) = 0 \rigtharrow~ u^∗u(x) = 0, donc
\mathrmKer~u
\subset~\mathrmKeru^∗~u~;
mais d'autre part, si x
\in\mathrmKeru^∗~u,
on a

\\textbar{}u(x)\\textbar{}^2 =
(u(x)∣u(x)) =
(u^∗u(x)∣x) =
(0∣x) = 0

et donc u(x) = 0, soit
\mathrmKeru^∗~u
\subset~\mathrmKer~u et l'égalité.
On en déduit alors que

\mathrmImu^∗~u =
(\mathrmKer(u^∗u)^∗)^\bot~
=
(\mathrmKeru^∗u)^\bot~
=
(\mathrmKeru)^\bot~
= \mathrmImu^∗~

Une des propriétés essentielles de l'ad\\\\jmathmathmathmathoint que nous utiliserons de
fa\ccon systématique pour la réduction des
endomorphismes est la suivante

Théorème~13.4.4 Soit u \in L(E). Soit F un sous-espace de E stable par u~;
alors F^\bot est stable par u^∗.

Démonstration Soit x \in F^\bot. Si y \in F, on a
\phi(u^∗(x),y) = \phi(x,u(y)) = 0 puisque u(y) \in F et x \in
F^\bot. Donc u^∗(x) \in F^\bot et
F^\bot est stable par u^∗.

\paragraph{13.4.2 Endomorphismes hermitiens}

Définition~13.4.2 Soit E un espace hermitien, u \in L(E). On dit que u est
hermitien (ou autoad\\\\jmathmathmathmathoint) s'il vérifie les conditions équivalentes

\begin{itemize}
\itemsep1pt\parskip0pt\parsep0pt
\item
  (i) u^∗ = u
\item
  (ii) \forall~~x,y \in E,
  (u(x)∣y) =
  (x∣u(y))
\end{itemize}

Remarque~13.4.2 Si la base \mathcal{E} est orthonormée, alors
\mathrmMat~ ((
∣ ),\mathcal{E}) = In et
\mathrmMat~
(u^∗,\mathcal{E}) =\
\mathrmMat (u,\mathcal{E})^∗~; en particulier

Théorème~13.4.5 Soit \mathcal{E} une base orthonormée de E~; alors u est hermitien
si et seulement
si~\mathrmMat~ (u,\mathcal{E}) est une
matrice hermitienne.

Proposition~13.4.6 L'ensemble H(E) des endomorphismes hermitiens est un
\mathbb{R}~-sous-espace vectoriel de L(E) (mais pas un \mathbb{C} sous-espace vectoriel).
On a L(E) = H(E) \oplus~ iH(E)

Démonstration L'endomorphisme de L^∗(E),
u\mapsto~u^∗ étant \mathbb{R}~ linéaire et
involutif, l'espace L(E) est somme directe du sous-espace propre associé
à la valeur propre 1 (les endomorphismes hermitiens) et du sous-espace
propre associé à la valeur propre -1 (les endomorphismes antihermitiens,
qui ne sont autre que les endomorphismes hermitiens multipliés par i).

\paragraph{13.4.3 Groupe unitaire}

Soit E un espace hermitien

Définition~13.4.3 On dit que u \in L(E) est un endomorphisme unitaire si
on a les propriétés équivalentes

\begin{itemize}
\itemsep1pt\parskip0pt\parsep0pt
\item
  (i) \forall~~x \in E,
  \\textbar{}u(x)\\textbar{}
  =\\textbar{} x\\textbar{}
\item
  (ii) \forall~~x,y \in E,
  (u(x)∣u(y)) =
  (x∣y)
\item
  (iii) u est inversible et u^-1 = u^∗
\item
  (iv) u \cdot u^∗ = \mathrmIdE
\item
  (v) u^∗\cdot u = \mathrmIdE
\end{itemize}

Démonstration (ii) \rigtharrow~(i) est évident (faire y = x). (i) \rigtharrow~(ii) provient de
l'identité de polarisation et de la linéarité de u. Pour un
endomorphisme d'un espace vectoriel de dimension finie, on sait que
l'inversibilité est équivalente à l'inversibilité à gauche ou à droite.
On a donc (iii) \Leftrightarrow (iv)
\Leftrightarrow (v). Supposons (ii) vérifié. Alors \phi(x,y) =
\phi(u(x),u(y)) = \phi(x,u^∗\cdot u(y)), ce qui montre (puisque \phi est
non dégénérée) que u^∗\cdot u =
\mathrmIdE~; donc (ii) \rigtharrow~(v). De même (v)
\rigtharrow~(ii) puisque \phi(u(x),u(y)) = \phi(x,u^∗\cdot u(y)).

Théorème~13.4.7 L'ensemble U(E) des endomorphismes unitaires de E est un
sous-groupe de (GL(E),\cdot). Pour tout endomorphisme unitaire u de E, on a
\textbar{}\mathrm{det}~
u\textbar{} = 1. L'ensemble SU(E) des endomorphismes unitaires de
déterminant 1 est un sous-groupe distingué de U(E).

Démonstration On a clairement \mathrmIdE \in
U(E). La définition (i) montre évidemment que si u et v sont unitaires,
il en est de même de u \cdot v. De plus, soit u \in U(E)~; on a
\\textbar{}u^-1(x)\\textbar{}
=\\textbar{}
u(u^-1(x))\\textbar{}
=\\textbar{} x\\textbar{} ce qui montre
que u^-1 \in U(E). Donc U(E) est un sous-groupe de (GL(E),\cdot).
On a alors 1 = \mathrm{det}~
\mathrmIdE =\
\mathrm{det} (u^∗\cdot u)
= \mathrm{det}~
u^∗\mathrm{det}~ u
= \textbar{}\mathrm{det}~
u\textbar{}^2, soit
\textbar{}\mathrm{det}~
u\textbar{} = 1. L'application de U(E) dans le groupe multiplicatif des
nombres complexes de module 1,
u\mapsto~\mathrm{det}~
u est un morphisme de groupes~; son noyau SU(E) est donc un sous groupe
distingué.

Théorème~13.4.8 Soit u \in L(E).

\begin{itemize}
\itemsep1pt\parskip0pt\parsep0pt
\item
  (i) Si u est unitaire, il envoie toute base orthonormée sur une base
  orthonormée.
\item
  (ii) Inversement, s'il existe une base orthonormée \mathcal{E} de E telle que
  u(\mathcal{E}) soit encore orthonormée, alors u est un endomorphisme unitaire.
\end{itemize}

Démonstration (i) On a
(u(ei)∣u(e\\\\jmathmathmathmath)) =
(ei∣e\\\\jmathmathmathmath) =
\deltai^\\\\jmathmathmathmath.

(ii) Soit x = \\sum ~
xiei \in E. On a
\\textbar{}x\\textbar{}^2
= \\sum ~
\textbar{}xi\textbar{}^2. Mais on a aussi u(x)
= \\sum ~
xiu(ei) et comme u(\mathcal{E}) est orthonormée,
\\textbar{}u(x)\\textbar{}^2
= \\sum ~
\textbar{}xi\textbar{}^2~; on a donc
\forall~~x \in E,
\\textbar{}u(x)\\textbar{}
=\\textbar{} x\\textbar{}.

Théorème~13.4.9 Soit u un endomorphisme unitaire et F un sous-espace de
E stable par u. Alors F^\bot est stable par u.

Démonstration On a u(F) \subset~ F et comme u est inversible, on a
dim u(F) =\ dim~ F. On
a donc u(F) = F. Soit donc x \in F^\bot et y \in F~; il existe z \in F
tel que u(z) = y, d'où (u(x)∣y) =
(u(x)∣u(z)) =
(x∣z) = 0, et donc u(x) \in F^\bot.

\paragraph{13.4.4 Matrices unitaires}

Proposition~13.4.10 Soit E un espace hermitien. Soit u \in L(E), \mathcal{E} une
base de E, \Omega = \mathrmMat~
(( ∣ ),\mathcal{E}) et A =\
\mathrmMat (u,\mathcal{E}). Alors u est un endomorphisme
unitaire si et seulement si~A^∗\OmegaA = \Omega.

Démonstration On a \phi(u(x),u(y)) = (AX)^∗\Omega(AY ) =
X^∗A^∗\OmegaAY . L'unicité de la matrice d'une forme
bilinéaire montre que

\forall~~x,y \in E,
(u(x)∣u(y)) =
(x∣y) \mathrel\Leftrightarrow
A^∗\OmegaA = \Omega

En particulier, si \mathcal{E} est une base orthonormée de E, u est un
endomorphisme unitaire si et seulement si~A^∗A =
In. Ceci conduit à la définition suivante

Définition~13.4.4 Soit A \in M\mathbb{C}(n). On dit que A est une matrice
unitaire si elle vérifie les conditions équivalentes

\begin{itemize}
\itemsep1pt\parskip0pt\parsep0pt
\item
  (i) A est inversible et A^-1 = A^∗
\item
  (ii) A^∗A = In
\item
  (iii) AA^∗ = In
\end{itemize}

Théorème~13.4.11 L'ensemble U(n) des matrices carrées unitaires d'ordre
n est un sous-groupe de (GL\mathbb{C}(n),.). Pour toute matrice
unitaire A, on a
\textbar{}\mathrm{det}~
A\textbar{} = 1. L'ensemble SU(n) des matrices unitaires de déterminant
1 est un sous-groupe distingué de U(n) .

Démonstration On a clairement In \in U(n). La définition (i)
montre évidemment que si A et B sont unitaires, il en est de même de AB.
De plus, soit A \in U(n)~; on a
A^-1(A^-1)^∗ =
A^-1(A^∗)^∗ = A^-1A =
In ce qui montre que A^-1 \in U(n). Donc U(n) est un
sous-groupe de (GL\mathbb{C}(n),.). On a alors 1
= \mathrm{det} In~
= \mathrm{det}~
(A^∗A) =
\textbar{}\mathrm{det}~
A\textbar{}^2, soit
\textbar{}\mathrm{det}~
A\textbar{} = 1. L'application de U(n) dans le groupe multiplicatif des
nombres complexes de module 1,
A\mapsto~\mathrm{det}~
A est un morphisme de groupes multiplicatifs~; son noyau SU(n) est donc
un sous-groupe distingué.

Dans ce paragraphe, on munira \mathbb{C}^n de la forme sesquilinéaire
hermitienne naturelle (qui rend la base canonique orthonormée),
c'est-à-dire que l'on posera (x∣y)
= \\sum ~
i=1^n\overlinexiyi

Théorème~13.4.12 Une matrice A \in M\mathbb{C}(n) est unitaire si et
seulement si~ses vecteurs colonnes (resp. lignes) forment une base
orthonormée de \mathbb{C}^n.

Démonstration Soit
(c1,\\ldots,cn~)
les vecteurs colonnes de A,
(l1,\\ldots,ln~)
ses vecteurs lignes. On a

\begin{align*} A \in U(n)&
\Leftrightarrow & A^∗A = I n
\Leftrightarrow \forall~~i,\\\\jmathmathmathmath,
(A^∗A) i,\\\\jmathmathmathmath = \deltai^\\\\jmathmathmathmath \%&
\\ & \Leftrightarrow &
\forall~~i,\\\\jmathmathmathmath, \\sum
k=1^n\overlinea
k,iak,\\\\jmathmathmathmath = \deltai^\\\\jmathmathmathmath
\Leftrightarrow \forall~i,\\\\jmathmathmathmath, (c~
i∣c\\\\jmathmathmathmath) = \deltai^\\\\jmathmathmathmath\%&
\\ \end{align*}

De la même fa\ccon, en traduisant la relation
AA^∗ = In, on obtiendrait
(li∣l\\\\jmathmathmathmath) =
\deltai^\\\\jmathmathmathmath.

Théorème~13.4.13 Soit E un espace hermitien. Soit \mathcal{E} une base orthonormée
de E, \mathcal{E}' une base de E. Alors on a équivalence de

\begin{itemize}
\itemsep1pt\parskip0pt\parsep0pt
\item
  (i) \mathcal{E}' est orthonormée
\item
  (ii) la matrice P\mathcal{E}^\mathcal{E}' de passage de la base \mathcal{E} à la
  base \mathcal{E}' est unitaire.
\end{itemize}

Démonstration On sait que P\mathcal{E}^\mathcal{E}'
= \mathrmMat~ (u,\mathcal{E}) où u est
l'endomorphisme de E défini par \forall~~i,
u(ei) = ei'. Or d'après les résultats du paragraphe
précédent, u est un endomorphisme unitaire si et seulement si~\mathcal{E}' est
orthonormée~; mais d'autre part, comme \mathcal{E} est orthonormée, u est unitaire
si et seulement
si~\mathrmMat~ (u,\mathcal{E}) est une
matrice unitaire, d'où l'équivalence entre (i) et (ii).

\paragraph{13.4.5 Réduction des endomorphismes normaux}

Définition~13.4.5 Soit E un espace hermitien et u \in L(E). On dit que u
est un endomorphisme normal si

u^∗u = uu^∗

Lemme~13.4.14 Soit u un endomorphisme normal. Alors
\mathrmKeru^∗~
= \mathrmKer~u.

Démonstration On a

\begin{align*} x
\in\mathrmKeru^∗~&
\Leftrightarrow &
(u^∗(x)∣u^∗(x)) = 0
\Leftrightarrow
(uu^∗(x)∣x) = 0\%&
\\ & \Leftrightarrow &
(u^∗u(x)∣x) = 0
\Leftrightarrow (u(x)\mathrel∣u(x)) = 0
\%& \\ & \Leftrightarrow &
x \in\mathrmKer~u \%&
\\ \end{align*}

Lemme~13.4.15 2. Soit u un endomorphisme normal. Alors, pour tout \lambda~ \in \mathbb{C},
\mathrmKer(u^∗-\overline\lambda~\mathrmIdE~)
= \mathrmKer~(u -
\lambda~\mathrmIdE).

Démonstration Il suffit de remarquer que u -
\lambda~\mathrmId est encore normal (élémentaire) et de lui
appliquer le lemme précédent en remarquant que
u^∗-\overline\lambda~\mathrmIdE
= (u - \lambda~\mathrmIdE)^∗

Théorème~13.4.16 Soit u un endomorphisme d'un espace hermitien. On a
équivalence de

\begin{itemize}
\itemsep1pt\parskip0pt\parsep0pt
\item
  (i) u est normal
\item
  (ii) u est diagonalisable dans une base orthonormée.
\end{itemize}

Démonstration (ii) \rigtharrow~(i) Soit \mathcal{E} une base orthonormée de diagonalisation
de u. Alors \mathrmMat~
(u,\mathcal{E}) =\
diag(\lambda~1,\\ldots,\lambda~n~).
Comme \mathcal{E} est orthonormée, on a
\mathrmMat~
(u^∗,\mathcal{E}) =\
\mathrmMat (u,\mathcal{E})^∗
=\
diag(\overline\lambda~1,\\ldots,\overline\lambda~n~).
Les deux matrices diagonales commutant, on a uu^∗ =
u^∗u, donc u est normal.

(i) \rigtharrow~(ii) Montrons le résultat par récurrence sur
dim~ E, le résultat étant évident si
dim~ E = 1. Supposons que u est normal. Comme \mathbb{C}
est algébriquement clos, u admet une valeur propre \lambda~. Comme
\mathrmKer(u^∗-\overline\lambda~\mathrmIdE~)
= \mathrmKer~(u -
\lambda~\mathrmIdE), Eu(\lambda~)
= \mathrmKer~(u -
\lambda~\mathrmIdE) est stable par u^∗
et donc Eu(\lambda~)^\bot est stable par u^∗∗ = u.
Mais comme Eu(\lambda~) est stable par u, le sous-espace
Eu(\lambda~)^\bot est stable par u^∗. Soit v =
u\textbar{} Eu(\lambda~)^\bot. La
relation (v(x)∣y) =
(u(x)∣y) =
(x∣u^∗(y)) pour x,y \in
Eu(\lambda~)^\bot montre que v^∗ =
u\textbar{} Eu(\lambda~)^\bot^∗,
donc v^∗v = vv^∗ et donc v est un endomorphisme
normal de Eu(\lambda~)^\bot. Par hypothèse de récurrence, il
existe une base orthonormée de Eu(\lambda~)^\bot formée de
vecteurs propres de v donc de u. Comme E = Eu(\lambda~) \bot \oplus~
Eu(\lambda~)^\bot, si on réunit cette base avec une base
orthonormée de Eu(\lambda~), on obtient une base orthonormée de E
formée évidemment de vecteurs propres de u, ce qui achève la
démonstration.

Remarque~13.4.3 Soit \mathcal{E} une telle base. Alors
\mathrmMat~ (u,\mathcal{E})
=\
diag(\lambda~1,\\ldots,\lambda~n~).
L'endomorphisme u est hermitien si et seulement si~sa matrice dans la
base orthonormée \mathcal{E} est hermitienne, c'est-à-dire si et seulement
si~\forall~i, \lambda~i~ \in \mathbb{R}~~; de même u est
unitaire si et seulement si~sa matrice dans la base orthonormée \mathcal{E} est
unitaire, c'est-à-dire si et seulement si~\forall~~i,
\textbar{}\lambda~i\textbar{} = 1. Comme il est clair que tout
endomorphisme hermitien ou unitaire est normal on obtient les deux
corollaires

Corollaire~13.4.17 Soit u un endomorphisme d'un espace hermitien. On a
équivalence de

\begin{itemize}
\itemsep1pt\parskip0pt\parsep0pt
\item
  (i) u est hermitien
\item
  (ii) u est diagonalisable dans une base orthonormée et
  \mathrm{Sp}~(u) \subset~ \mathbb{R}~
\end{itemize}

Corollaire~13.4.18 Soit u un endomorphisme d'un espace hermitien. On a
équivalence de

\begin{itemize}
\itemsep1pt\parskip0pt\parsep0pt
\item
  (i) u est unitaire
\item
  (ii) u est diagonalisable dans une base orthonormée et
  \mathrm{Sp}~(u) \subset~ U
  (ensemble des nombres complexes de module 1)
\end{itemize}

\paragraph{13.4.6 Réduction des matrices normales}

En traduisant le paragraphe précédent en terme de matrices (en utilisant
le produit hermitien canonique sur \mathbb{C}^2 défini par
(x∣y) =\
\sum ~
i\overlinexiyi) on
obtient la définition et les résultats suivants.

Définition~13.4.6 Soit A \in M\mathbb{C}(n). On dit que A est une matrice
normale si

A^∗A = AA^∗

Théorème~13.4.19 Soit A \in M\mathbb{C}(n). On a équivalence de

\begin{itemize}
\itemsep1pt\parskip0pt\parsep0pt
\item
  (i) A est normal
\item
  (ii) Il existe P unitaire telle que P^-1AP =
  P^∗AP soit diagonale.
\end{itemize}

Corollaire~13.4.20 Soit A \in M\mathbb{C}(n). On a équivalence de

\begin{itemize}
\itemsep1pt\parskip0pt\parsep0pt
\item
  (i) A est hermitienne
\item
  (ii) Il existe P unitaire telle que P^-1AP =
  P^∗AP soit diagonale réelle
\end{itemize}

Corollaire~13.4.21 Soit A \in M\mathbb{C}(n). On a équivalence de

\begin{itemize}
\itemsep1pt\parskip0pt\parsep0pt
\item
  (i) A est unitaire
\item
  (ii) Il existe P unitaire telle que P^-1AP =
  P^∗AP soit diagonale à éléments diagonaux dans U (ensemble
  des nombres complexes de module 1)
\end{itemize}

{[}
{[}
{[}
{[}
