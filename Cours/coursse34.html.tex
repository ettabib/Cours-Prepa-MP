\textbf{Warning: 
requires JavaScript to process the mathematics on this page.\\ If your
browser supports JavaScript, be sure it is enabled.}

\begin{center}\rule{3in}{0.4pt}\end{center}

{[}
{[}
{[}{]}
{[}

\subsubsection{6.3 Développements asymptotiques}

\paragraph{6.3.1 Echelles de comparaison, parties principales}

Définition~6.3.1 On appelle échelle de comparaison en a suivant A toute
famille (\phii)i\inI de fonctions de ℱa,A(\mathbb{R}~)
vérifiant

\begin{itemize}
\itemsep1pt\parskip0pt\parsep0pt
\item
  (i) \forall~i \in I, \\forall~~V \in V
  (a), \exists~t \in V \bigcap A,
  \phii(t)\neq~0~; autrement dit, aucune
  des \phii n'est identiquement nulle au voisinage de a suivant A
\item
  (ii) si i\neq~\\\\jmathmathmathmath, l'une des deux fonctions
  \phii ou \phi\\\\jmathmathmathmath est négligeable devant l'autre
\end{itemize}

Remarque~6.3.1 On obtient une relation d'ordre strict sur I en posant i
\textless{} \\\\jmathmathmathmath \Leftrightarrow \phi\\\\jmathmathmathmath =
o(\phii).

Exemple~6.3.1 Au voisinage d'un point a \in \mathbb{R}~, on a plusieurs échelles de
comparaison classiques

\begin{itemize}
\item
  a) la famille des t\mapsto~(t - a)^n, n
  \in \mathbb{N}~ (l'échelle qui conduit aux développements limités)
\item
  b) la famille des t\mapsto~(t - a)^n, n
  \in ℤ
\item
  c) la famille des t\mapsto~\textbar{}t -
  a\textbar{}^\alpha~, \alpha~ \in \mathbb{R}~
\item
  d) la famille des t\mapsto~\textbar{}t -
  a\textbar{}^\alpha~\textbar{}log~
  t\textbar{}^\beta~, \alpha~,\beta~ \in \mathbb{R}~~: on a alors

  (\alpha~,\beta~) \textless{} (\alpha~',\beta~') \Leftrightarrow
  \bigl (\alpha~ \textless{} \alpha~'\text ou (\alpha~
  = \alpha~'\text et \beta~ \textgreater{}
  \beta~')\bigr )
\end{itemize}

Exemple~6.3.2 Au voisinage de a = +\infty~, on a plusieurs échelles de
comparaison classiques

\begin{itemize}
\item
  a) la famille des t\mapsto~t^n, n \in ℤ
  (l'ordre obtenu est l'inverse de l'ordre naturel)
\item
  b) la famille des t\mapsto~t^\alpha~, \alpha~ \in \mathbb{R}~
  (l'ordre obtenu est l'inverse de l'ordre naturel)
\item
  c) la famille des
  t\mapsto~t^\alpha~(log~
  t)^\beta~, \alpha~,\beta~ \in \mathbb{R}~~: on a alors

  (\alpha~,\beta~) \textless{} (\alpha~',\beta~') \Leftrightarrow
  \bigl (\alpha~ \textgreater{} \alpha~'\text ou
  (\alpha~ = \alpha~'\text et \beta~ \textgreater{}
  \beta~')\bigr )
\item
  c) la famille des
  t\mapsto~e^P(t)t^\alpha~(log~
  t)^\beta~, P \in \mathbb{R}~{[}X{]},\alpha~,\beta~ \in \mathbb{R}~~: on a alors

  (P,\alpha~,\beta~) \textless{} (Q,\alpha~',\beta~') \Leftrightarrow
  \left \\cases
  limt\rightarrow~+\infty~~(Q(t) - P(t)) = +\infty~
  \cr \cr \textou &
  \cr P = Q\text et \alpha~ \textgreater{}
  \alpha~' \cr \textou & \cr
  P = Q\text et \alpha~ = \alpha~'\text et \beta~
  \textgreater{} \beta~'  \right .
\end{itemize}

Définition~6.3.2 Soit (\phii)i\inI une échelle de
comparaison en a suivant A et f \inℱa,A(E). On dit que f admet
une partie principale suivant l'échelle de comparaison s'il existe i \in I
et ai \in E \diagdown\0\ tels que f(t)
∼ ai\phii(t). Une telle partie principale si elle
existe est unique.

Démonstration Si on a f(t) ∼ ai\phii(t) ∼
b\\\\jmathmathmathmath\phi\\\\jmathmathmathmath(t), on a nécessairement i = \\\\jmathmathmathmath car sinon une des
deux fonctions serait négligeable devant l'autre. On a alors
(ai - bi)\phii = o(\phii) ce qui n'est
possible que si ai = bi.

\paragraph{6.3.2 Développements asymptotiques}

Définition~6.3.3 Soit (\phii)i\inI une échelle de
comparaison en a suivant A et f \inℱa,A(E). On dit que f admet
un développement asymptotique à la précision \phi\\\\jmathmathmathmath suivant
l'échelle de comparaison s'il existe i0 \textless{}
i1 \textless{}
\\ldots~ \textless{}
ip \leq \\\\jmathmathmathmath et
a0,a1,\\ldots,ap~
\in E \diagdown\0\ tels que

f(t) = a0\phii0(t) +
\\ldots~ +
ap\phiip(t) + o(\phi\\\\jmathmathmathmath(t))

Remarque~6.3.2 Un tel développement est nécessairement unique puisque
a0\phii0(t) est nécessairement la partie
principale de f(t), a1\phii1(t) celle de f(t)
- a0\phii0(t) et ainsi de suite \\\\jmathmathmathmathusqu'à
ap\phiip(t) qui doit être la partie
principale de f(t) - a0\phii0(t)
-\\ldots~ -
ap-1\phiip-1(t).

\paragraph{6.3.3 Opérations sur les développements asymptotiques}

Il est clair que si f et g admettent des développements asymptotiques à
la précision \phi\\\\jmathmathmathmath et \phi\\\\jmathmathmathmath', alors \alpha~f + \beta~g admet un
développement asymptotique à la précision
\phimin(\\\\jmathmathmathmath,\\\\jmathmathmathmath')~ obtenu de la manière
évidente (additionner les deux et supprimer les termes non
significatifs).

Si l'échelle de comparaison est stable par produit, en faisant le
produit de deux développements asymptotiques on obtient un développement
asymptotique du produit des deux fonctions, à une précision à évaluer
suivant les cas. Ceci peut permettre également de composer
développements asymptotiques et développements limités.

De plus, les théorèmes de comparaison des intégrales impropres peuvent
permettre d'intégrer des développements asymptotiques~:

à condition que la fonction g soit positive au voisinage de a et que son
intégrale converge au point a,

f = o(g) \rigtharrow~\int  a^x~f =
o(\int  a^x~g)

Exemple~6.3.3 ~: on a pour x \textgreater{} 0 au voisinage du point 0,

\begin{align*} d \over dx
(arcsin~ (1 - x))& =& - 1
\over \sqrt2x - x^2 = -
1 \over \sqrt2x  1
\over \sqrt1 - x \over
2  \%& \\ & =& - 1
\over \sqrt2x (1 + x
\over 4 + x^2 \over 16 +
o(x^2)) \%& \\ & =& -
\sqrt2 \over
2\sqrtx - \sqrt2
\over 8 \sqrtx -
\sqrt2 \over 32 x^3\diagup2 +
o(x^3\diagup2)\%& \\
\end{align*}

En intégrant de 0 à x on va obtenir, en tenant compte de \phi(t) =
o(t^3\diagup2) \rigtharrow~\int ~
0^x\phi(t) dt = o(\int ~
0^xt^3\diagup2 dt)

arcsin (1 - x) = \pi~ \over 2~
-\sqrt2x - \sqrt2
\over 12 x^3\diagup2 - \sqrt2
\over 80 x^5\diagup2 + o(x^5\diagup2)

{[}
{[}
{[}
{[}
