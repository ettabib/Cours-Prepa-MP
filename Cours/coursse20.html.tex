\textbf{Warning: 
requires JavaScript to process the mathematics on this page.\\ If your
browser supports JavaScript, be sure it is enabled.}

\begin{center}\rule{3in}{0.4pt}\end{center}

{[}
{[}
{[}{]}
{[}

\subsubsection{4.3 Suites}

\paragraph{4.3.1 Suites convergentes, limites}

Définition~4.3.1 Soit E un espace métrique et
(xn)n\in\mathbb{N}~ une suite de E. On dit qu'elle est
convergente s'il existe \ell \in E vérifiant les conditions équivalentes

\begin{itemize}
\itemsep1pt\parskip0pt\parsep0pt
\item
  (i) \forall~V \in V (\ell), \\exists~N
  \in \mathbb{N}~, n ≥ N \rigtharrow~ xn \in V
\item
  (ii) \forall~~\epsilon \textgreater{} 0,
  \existsN \in \mathbb{N}~, n ≥ N \rigtharrow~ d(xn~,\ell)
  \textless{} \epsilon
\end{itemize}

Une suite qui n'est pas convergente est dite divergente.

Démonstration La propriété (ii) ne fait que traduire (i) pour V =
B(\ell,\epsilon). Or toute boule est un voisinage et tout voisinage contient une
boule. Donc (i) et (ii) sont équivalents.

Proposition~4.3.1 Si la suite (xn)n\in\mathbb{N}~ est
convergente, l'élément \ell de E est unique~; on l'appelle la limite de la
suite. On note \ell = limxn~.

Démonstration Si \ell et \ell' conviennent avec \ell\neq~\ell', il existe d'après la
propriété de séparation U ouvert contenant \ell et V ouvert contenant \ell'
tels que U \bigcap V = \varnothing~. Mais
\existsN,\quad n ≥ N \rigtharrow~ xn~ \in
U et \existsN', n ≥ N' \rigtharrow~ xn~ \in V . Pour n
≥ max(N,N'), on a xn~ \in U \bigcap V . C'est
absurde.

Remarque~4.3.1 On prendra garde à ne pas introduire le symbole
limxn~ de manière opératoire avant
d'avoir démontré son existence. On remarquera d'autre part que les
notions de convergence et de limites sont purement topologiques
puisqu'on peut les exprimer en terme de voisinages~; elles sont donc
inchangées par changement de distance topologiquement équivalente.

\paragraph{4.3.2 Sous suites, valeurs d'adhérences}

Définition~4.3.2 Soit (xn)n\in\mathbb{N}~ une suite d'éléments
de E et soit \phi : \mathbb{N}~ \rightarrow~ \mathbb{N}~ strictement croissante. On dit alors que la suite
(x\phi(n))n\in\mathbb{N}~ est une sous suite de la suite
(xn)n\in\mathbb{N}~.

Théorème~4.3.2 Si la suite (xn)n\in\mathbb{N}~ est convergente
de limite \ell, alors toute sous suite converge et a la même limite.

C'est une conséquence du lemme suivant qui se démontre à l'aide d'une
récurrence facile.

Lemme~4.3.3 Soit \phi : \mathbb{N}~ \rightarrow~ \mathbb{N}~ strictement croissante. Alors
\forall~~n \in \mathbb{N}~, \phi(n) ≥ n.

Démonstration (du théorème) Soit V voisinage de \ell et N \in \mathbb{N}~ tel que n ≥ N
\rigtharrow~ xn \in V . Alors pour n ≥ N on a \phi(n) ≥ n ≥ N donc
x\phi(n) \in V . Donc \ell est encore limite de la suite
(x\phi(n))n\in\mathbb{N}~.

Définition~4.3.3 Soit (xn)n\in\mathbb{N}~ une suite d'éléments
de E et a \in E. On dit que a est valeur d'adhérence de la suite si on a
les conditions équivalentes

\begin{itemize}
\itemsep1pt\parskip0pt\parsep0pt
\item
  (i) \forall~V \in V (a), \\forall~~N
  \in \mathbb{N}~, \exists~n ≥ N,\quad
  xn \in V
\item
  (i)' \forall~~\epsilon \textgreater{} 0,
  \forall~N \in \mathbb{N}~, \\exists~n ≥
  N,\quad d(xn,a) \textless{} \epsilon
\item
  (ii) \forall~~V \in V (a), \n \in
  \mathbb{N}~∣xn \in V \ est
  infini.
\item
  (ii)' \forall~~\epsilon \textgreater{} 0,
  \n \in \mathbb{N}~∣d(xn,a)
  \textless{} \epsilon\ est infini.
\item
  (iii) a est limite d'une sous suite (x\phi(n)) de la suite
  (xn).
\end{itemize}

Démonstration (i)' n'est qu'une reformulation de (i) en termes de
boules, comme (ii)' vis à vis de (ii). Si \n \in
\mathbb{N}~∣xn \in V \ est
fini, il existe N \in \mathbb{N}~ tel que n ≥ N \rigtharrow~
xn∉V et donc (i) n'est pas vérifié.
Ceci montre que (i) \rigtharrow~(ii). De même, si \n \in
\mathbb{N}~∣xn \in V \ est
infini, il contient des éléments d'indices arbitrairement grands, donc
(ii) \rigtharrow~(i). Si a est limite de la sous suite (x\phi(n)) et V est
un voisinage de a, il existe N \in \mathbb{N}~ tel que n ≥ N \rigtharrow~ x\phi(n) \in V .
Donc \n \in \mathbb{N}~∣xn \in V
\ contient \phi({[}N,+\infty~{[}), il est donc infini, soit
(iii) \rigtharrow~(ii). Montrons maintenant que (i)' \rigtharrow~(iii), ce qui achèvera la
démonstration. On construit \phi(n) par récurrence de la manière suivante~:
on prend \epsilon = 1 \over n+1 et N = \left
\ \cases 0 &si n = 0
\cr \phi(n - 1) + 1&si n ≥ 1 \cr 
\right .~; il existe alors p ≥ \phi(n - 1) + 1 tel que
d(a,xp) \textless{} 1 \over n+1 ~; on pose
\phi(n) = p. On construit ainsi une fonction strictement croissante de \mathbb{N}~
dans \mathbb{N}~ vérifiant d(a,x\phi(n)) \textless{} 1
\over n+1 . D'où une sous suite de limite a.

Remarque~4.3.2 On a donc vu qu'une suite convergente admet une unique
valeur d'adhérence, sa limite. Il est clair qu'une valeur d'adhérence
d'une sous suite est encore une valeur d'adhérence de la suite.

\paragraph{4.3.3 Caractérisation des fermés d'un espace métrique}

Théorème~4.3.4 Soit E un espace métrique, A une partie de E et a \in E.
Alors a est dans l'adhérence de A si et seulement si~a est limite (dans
E) d'une suite d'éléments de A.

Démonstration Si a est limite d'une suite (an)n\in\mathbb{N}~
d'éléments de A, soit V un voisinage de a. Alors
\existsN \in \mathbb{N}~, n ≥ N \rigtharrow~ an~ \in V . En
particulier an \in V \bigcap A qui est donc non vide. Donc a
appartient à \overlineA. Inversement, soit a
\in\overlineA. Alors, pour tout \epsilon \textgreater{} 0, A \bigcap
B(a,\epsilon)\neq~\varnothing~. Pour \epsilon = 1 \over
n+1 on peut donc trouver an \in A tel que d(a,an)
\textless{} 1 \over n+1 . Donc a est limite de la
suite (an) d'éléments de A.

Corollaire~4.3.5 Soit E un espace métrique. Une partie A de E est fermée
si et seulement si~toute suite d'éléments de A qui converge dans E a une
limite qui appartient à A.

Démonstration Ceci traduit simplement l'inclusion
\overlineA \subset~ A.

{[}
{[}
{[}
{[}
