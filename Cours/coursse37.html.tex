\textbf{Warning: 
requires JavaScript to process the mathematics on this page.\\ If your
browser supports JavaScript, be sure it is enabled.}

\begin{center}\rule{3in}{0.4pt}\end{center}

{[}
{[}
{[}{]}
{[}

\subsubsection{7.3 Séries à termes réels positifs}

\paragraph{7.3.1 Convergence des séries à termes réels positifs}

Théorème~7.3.1 Soit \\\sum
 xn une série à termes réels positifs. Alors la suite des
sommes partielles est une suite croissante~; la série converge si et
seulement si~ses sommes partielles sont ma\\\\jmathmathmathmathorées~:
\existsM \in \mathbb{R}~, \\forall~~n \in \mathbb{N}~,
Sn \leq M.

Démonstration On a Sn - Sn-1 = xn ≥ 0 donc
la suite (Sn) est croissante~; par suite, elle converge si et
seulement si~elle est ma\\\\jmathmathmathmathorée.

Remarque~7.3.1 Si une série à termes positifs diverge, on a donc
nécessairement limSn~ = +\infty~ (puisque
la suite (Sn) est croissante).

Corollaire~7.3.2 Soit \\\sum
 xn et \\\sum
 yn deux séries à termes réels telles que 0 \leq xn
\leq yn. Alors

\begin{itemize}
\itemsep1pt\parskip0pt\parsep0pt
\item
  (i) si la série \\sum ~
  yn converge, la série
  \\sum  xn~
  converge également
\item
  (ii) si la série \\\sum
   xn diverge, la série
  \\sum  yn~
  diverge
\end{itemize}

Démonstration On a Sn(x) \leq Sn(y) donc tout ma\\\\jmathmathmathmathorant
de la suite (Sn(y)) est aussi un ma\\\\jmathmathmathmathorant de la suite
(Sn(x)), d'où (i). L'énoncé (ii) n'en est que la contraposée.

Remarque~7.3.2 Pour que l'énoncé précédent soit valable, il suffit
évidemment qu'il existe k \textgreater{} 0 et N \in \mathbb{N}~ tels que n ≥ N \rigtharrow~ 0 \leq
xn \leq kyn, c'est-à-dire que xn ≥ 0,
yn ≥ 0 et xn = O(yn).

\paragraph{7.3.2 Comparaison des séries à termes réels positifs}

Théorème~7.3.3 Soit \\\sum
 xn et \\\sum
 yn deux séries à termes réels positifs telles que
xn = O(yn) (resp. xn = o(yn)).
Alors

\begin{itemize}
\itemsep1pt\parskip0pt\parsep0pt
\item
  (i) si la série \\sum ~
  yn converge, la série
  \\sum  xn~
  converge également et Rn(x) = O(Rn(y)) (resp.
  Rn(x) = o(Rn(y)))
\item
  (ii) si la série \\\sum
   xn diverge, la série
  \\sum  yn~
  diverge et Sn(x) = O(Sn(y)) (resp. Sn(x)
  = o(Sn(y)))
\end{itemize}

Démonstration Les convergences et divergences résultent immédiatement de
la remarque qui suit le corollaire précédent et du fait que xn
= o(yn) \rigtharrow~ xn = O(yn). Montrons par exemple
les énoncés sur les relations de comparaison dans le cas xn =
o(yn) (des modifications évidentes de \epsilon en k ou 2k permettent
de traiter le cas xn = O(yn)).

(i) Soit \epsilon \textgreater{} 0~; il existe N \in \mathbb{N}~ tel que n ≥ N \rigtharrow~ 0 \leq
xn \leq \epsilonyn. Alors pour n ≥ N, on a 0
\leq\\sum ~
p=n+1^+\infty~xp \leq
\epsilon\\sum ~
p=n+1^+\infty~yp, soit 0 \leq Rn(x) \leq
\epsilonRn(y). On a donc Rn(x) = o(Rn(y)).

(ii) Soit \epsilon \textgreater{} 0~; il existe N \in \mathbb{N}~ tel que n ≥ N \rigtharrow~ 0 \leq
xn \leq \epsilon \over 2 yn. Alors pour n
\textgreater{} N, on a 0
\leq\\sum ~
p=N+1^nxp \leq \epsilon \over 2
 \\sum ~
p=N+1^nyp, soit Sn(x) -
SN(x) \leq \epsilon \over 2 (Sn(y) -
SN(y)) ou encore 0 \leq Sn(x) \leq \epsilon
\over 2 Sn(y) + (SN(x) - \epsilon
\over 2 SN(y)). Mais comme la série
\\sum  yn~ est
à termes positifs divergente, ses sommes partielles tendent vers + \infty~ et
donc il existe N' \in \mathbb{N}~ tel que n ≥ N' \rigtharrow~ \epsilon \over 2
Sn(y) ≥ SN(x) - \epsilon \over 2
SN(y). Alors pour n \textgreater{}\
max(N,N'), on a 0 \leq Sn(x) \leq \epsilon \over 2
Sn(y) + \epsilon \over 2 Sn(y) =
\epsilonSn(y) et donc Sn(x) = o(Sn(y)).

Corollaire~7.3.4 Soit \\\sum
 xn et \\\sum
 yn deux séries à termes réels strictement positifs telles
que

\existsN \in \mathbb{N}~, n ≥ N \rigtharrow~ xn+1~
\over xn \leq yn+1
\over yn

Alors xn = O(yn) et en particulier

\begin{itemize}
\itemsep1pt\parskip0pt\parsep0pt
\item
  (i) si la série \\sum ~
  yn converge, la série
  \\sum  xn~
  converge
\item
  (ii) si la série \\\sum
   xn diverge, la série
  \\sum  yn~
  diverge
\end{itemize}

Démonstration On vérifie immédiatement par récurrence que pour n ≥ N on
a xn \leq xN \over yN
yn et donc xn = O(yn).

Théorème~7.3.5 Soit \\\sum
 xn et \\\sum
 yn deux séries à termes réels telles que yn ≥ 0
et xn ∼ yn. Alors les deux séries sont de même
nature et

\begin{itemize}
\itemsep1pt\parskip0pt\parsep0pt
\item
  (i) si la série \\sum ~
  yn converge, la série
  \\sum  xn~
  converge également et Rn(x) ∼ Rn(y)
\item
  (ii) si la série \\\sum
   yn diverge, la série
  \\sum  xn~
  diverge et Sn(x) ∼ Sn(y)
\end{itemize}

Démonstration Soit \epsilon \textless{} 1. Il existe N \in \mathbb{N}~ tel que n ≥ N \rigtharrow~ (1 -
\epsilon)yn \leq xn \leq (1 + \epsilon)yn et donc xn
≥ 0 pour n ≥ N. On a à la fois xn = O(yn) et
yn = O(xn) ce qui d'après le théorème précédent
montre que les deux séries convergent ou divergent simultanément.
Supposons alors les séries convergentes. On a \textbar{}xn -
yn\textbar{} = o(yn), on en déduit donc la
convergence de \\sum ~
\textbar{}xn - yn\textbar{} et que
Rn(\textbar{}x - y\textbar{}) = o(Rn(y)). Mais
\textbar{}Rn(x) - Rn(y)\textbar{}\leq
Rn(\textbar{}x - y\textbar{}) donc \textbar{}Rn(x) -
Rn(y)\textbar{} = o(Rn(y)) et donc Rn(x) ∼
Rn(y). Supposons maintenant les séries divergentes. Alors,
soit la série \\sum ~
\textbar{}xn - yn\textbar{} converge et comme
limSn~(y) = +\infty~ on a
Sn(\textbar{}x - y\textbar{}) = o(Sn(y)), soit elle
diverge et le théorème précédent assure que Sn(\textbar{}x -
y\textbar{}) = o(Sn(y)). Mais alors \textbar{}Sn(x)
- Sn(y)\textbar{}\leq Sn(\textbar{}x - y\textbar{}) =
o(Sn(y)), soit Sn(x) ∼ Sn(y).

\paragraph{7.3.3 Séries de Riemann et de Bertrand}

Théorème~7.3.6 (séries de Riemann). Soit \alpha~ \in \mathbb{R}~. La série
\\sum ~  1
\over n^\alpha~ converge si et seulement si~\alpha~
\textgreater{} 1.

Si \alpha~ \textgreater{} 1, on a Rn ∼ 1 \over
\alpha~-1  1 \over n^\alpha~-1 ~; si \alpha~ \textless{}
1, on a Sn ∼ n^1-\alpha~ \over 1-\alpha~ ~;
si \alpha~ = 1, Sn ∼ log~ n.

Démonstration Soit \alpha~\neq~1. Posons xn
= 1 \over n^\alpha~ et yn = 1
\over n^\alpha~-1 - 1 \over
(n+1)^\alpha~-1 . On a

 yn \over xn = - (1 + 1
\over n )^1-\alpha~ - 1 \over  1
\over n 

qui admet pour limite l'opposé de la dérivée en 0 de
x\mapsto~(1 + x)^1-\alpha~ soit \alpha~ - 1. On a
donc xn ∼ 1 \over \alpha~-1 yn
\textgreater{} 0. Les deux séries sont donc de même nature. Or
Sn(y) = 1 - 1 \over (n+1)^\alpha~-1
admet une limite finie si et seulement si~\alpha~ \textgreater{} 1. Si \alpha~
\textgreater{} 1, on a Rn(x) ∼ 1 \over \alpha~-1
Rn(y) = 1 \over \alpha~-1  1
\over n^\alpha~-1 . Si \alpha~ \textless{} 1, on a
Sn(x) ∼ 1 \over \alpha~-1 Sn(y) = 1
\over 1-\alpha~ ((n + 1)^1-\alpha~ - 1) ∼
n^1-\alpha~ \over 1-\alpha~ . Enfin, si \alpha~ = 1, on
aboutit à une étude similaire avec yn
= log (n + 1) -\ log~
n = log (1 + 1 \over n~ )
∼ 1 \over n .

Corollaire~7.3.7 (séries de Bertrand). Soit \alpha~,\beta~ \in \mathbb{R}~. La série
\\sum  n≥2~ 1
\over n^\alpha~(log~
n)^\beta~ converge si et seulement si~\alpha~ \textgreater{} 1 ou \alpha~ =
1,\beta~ \textgreater{} 1.

Démonstration Soit xn = 1 \over
n^\alpha~(log n)^\beta~~ . Si \alpha~
\textgreater{} 1, soit \gamma tel que \alpha~ \textgreater{} \gamma \textgreater{} 1 et
yn = 1 \over n^\gamma . La série
\\sum  yn~
converge et  xn \over yn = 1
\over n^\alpha~-\gamma(log~
n)^\beta~ tend vers 0 car \alpha~ - \gamma \textgreater{} 0. On a donc
xn = o(yn) et la série
\\sum  xn~
converge. Si \alpha~ \textless{} 1, soit \gamma tel que \alpha~ \textless{} \gamma \textless{}
1 et yn = 1 \over n^\gamma . La série
\\sum  yn~
diverge et  yn \over xn =
(log n)^\beta~~ \over
n^\gamma-\alpha~ tend vers 0 car \gamma - \alpha~ \textgreater{} 0. On a donc
yn = o(xn) et la série
\\sum  xn~
converge. Le cas \alpha~ = 1 résulte facilement du paragraphe suivant.

\paragraph{7.3.4 Comparaison à des intégrales}

Théorème~7.3.8 Soit f : {[}0,+\infty~{[}\rightarrow~ \mathbb{R}~ continue par morceaux,
décroissante, positive. Posons wn =\\int
 n-1^nf(t) dt - f(n). Alors la série
\\sum  wn~ est
convergente.

Démonstration On a wn =\int ~
n-1^n(f(t) - f(n)) dt. Comme f est décroissante,
\forall~~t \in {[}n - 1,n{]}, f(t) ≥ f(n) et donc
wn ≥ 0. Mais d'autre part

0 \leq wn \leq\int  n-1^n~f(n
- 1) dt - f(n) = f(n - 1) - f(n)

On a \\sum ~
p=1^n(f(p - 1) - f(p)) = f(0) - f(n) qui admet une limite
quand p tend vers + \infty~ (car f admet une limite en + \infty~~: elle est
décroissante et positive). Ceci montre que la série
\\sum ~ (f(p - 1) - f(p))
converge. Il en est donc de même de la série
\\sum  wn~.

Corollaire~7.3.9 Soit f : {[}0,+\infty~{[}\rightarrow~ \mathbb{R}~ continue décroissante positive.
Alors la série \\sum ~
f(n) converge si et seulement si f est intégrable sur {[}0,+\infty~{[}.

Démonstration En effet, on déduit du théorème précédent que les deux
séries \\sum ~ f(n) et
\\sum ~
\int  n-1^n~f(t) dt convergent ou
divergent simultanément, car leur différence est une série convergente.
Mais on a \\sum ~
p=1^n\int ~
p-1^pf(t) dt =\int ~
0^nf(t) dt =\int ~
{[}0,n{]}f. Si f est intégrable, comme la suite
({[}0,n{]})n\in\mathbb{N}~ est une suite croissante de segments dont la
réunion est {[}0,+\infty~{[}, la suite (\int ~
{[}0,n{]}f) est convergente de limite
\int  {[}0,+\infty~{[}~f, donc la série
\\sum ~
\int  n-1^n~f(t) dt converge et
il en est de même de \\\sum
 f(n). Si \\sum ~
f(n) converge, il en est de même de
\\sum ~
\int  n-1^n~f(t) dt, et si
{[}a,b{]} est un segment contenu dans {[}0,+\infty~{[} les ma\\\\jmathmathmathmathorations

\int  {[}a,b{]}~f
\leq\int  0^{[}b{]}+1~f =
\sum p=0^{[}b{]}+1~
\\int  ~
p-1^pf(t) dt \leq\\sum
p=0^+\infty~\\\int
  p-1^pf(t) dt

et le fait que f soit positive, montrent que f est intégrable sur
{[}0,+\infty~{[}.

Remarque~7.3.3 Bien entendu, il suffit que la condition de décroissance
soit vérifiée sur un certain {[}t0,+\infty~{[}.

Dans le cas d'une série divergente, l'encadrement

\int  0^n+1~f(t) dt
\leq\sum p=0^n~f(p) \leq f(0) +
\\int  ~
0^nf(t) dt

permet souvent d'obtenir un équivalent de la somme partielle de la
série. Dans le cas d'une série convergente, on a de même

\int  n+1^+\infty~~f(t) dt
\leq\sum p=n+1^+\infty~~f(p)
\leq\\int  ~
n^+\infty~f(t) dt

ce qui permet souvent d'obtenir une ma\\\\jmathmathmathmathoration ou un équivalent du reste
de la série.

Exemple~7.3.1 Dans le cas limite des séries de Bertrand,
\\sum ~  1
\over n(log n)^\beta~~ ,
la fonction f(t) = 1 \over
t(log t)^\beta~~ est continue
décroissante (pour t assez grand) de limite 0. Donc la série est de même
nature que l'intégrale \int ~
3^+\infty~ dt \over
t(log t)^\beta~~ . Mais on a
\int  3^x~ dt
\over t(log t)^\beta~~
=\int  \log~
3^log x~ du \over
u^\beta~ (poser u = log~ t) qui admet
une limite finie quand x tend vers + \infty~ si et seulement si \beta~
\textgreater{} 1. Ceci achève la démonstration du critère de convergence
des séries de Bertrand.

{[}
{[}
{[}
{[}
