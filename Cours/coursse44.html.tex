\textbf{Warning: 
requires JavaScript to process the mathematics on this page.\\ If your
browser supports JavaScript, be sure it is enabled.}

\begin{center}\rule{3in}{0.4pt}\end{center}

{[}
{[}{]}
{[}

\subsubsection{8.1 Monotonie, continuité}

\paragraph{8.1.1 Limites et monotonie}

Proposition~8.1.1 Soit I un intervalle de \mathbb{R}~ et f : I \rightarrow~ \mathbb{R}~ croissante.
Alors

\begin{itemize}
\itemsep1pt\parskip0pt\parsep0pt
\item
  (i) f admet en tout point a de I (dans la mesure où cela a un sens)
  une limite à gauche f(a-) et une limite à droite f(a+) dans \mathbb{R}~ avec
  f(a-) \leq f(a) \leq f(a+)
\item
  (ii) f admet en l'extrémité droite b de I une limite si et seulement
  si~elle est ma\\\\jmathmathmathmathorée sur I~; dans le cas contraire
  limx\rightarrow~b,x\textless{}b~f(x) = +\infty~
\item
  (iii) f admet en l'extrémité gauche a de I une limite si et seulement
  si~elle est minorée sur I~; dans le cas contraire
  limx\rightarrow~a,x\textgreater{}a~f(x) = -\infty~
\end{itemize}

Démonstration (i) Supposons que a n'est pas l'extrémité gauche de I.
Pour x \textless{} a on a f(x) \leq f(a). Soit m
= supx\textless{}a~f(x) \leq f(a). Soit
\epsilon \textgreater{} 0. Il existe x0 \textless{} a tel que m - \epsilon
\textless{} f(x0) \leq m. Alors x \in{]}x0,a{[}\rigtharrow~ m - \epsilon
\textless{} f(x0) \leq f(x) \leq m et donc m
= limx\rightarrow~a,x\textless{}a~f(x). De même,
si a n'est pas l'extrémité gauche de I et si M
= inf x\textgreater{}a~f(x) ≥ f(a),
on a M =\
limx\rightarrow~a,x\textgreater{}af(x).

(ii) f admet de toute fa\ccon dans
\overline\mathbb{R}~ la limite
supx\inI~f(x) (comme ci dessus)~; cette
limite est dans \mathbb{R}~ si et seulement si~f est ma\\\\jmathmathmathmathorée sur I~; similaire
pour (iii).

Remarque~8.1.1 On a un résultat similaire pour les applications
décroissantes~:

Proposition~8.1.2 Soit I un intervalle de \mathbb{R}~ et f : I \rightarrow~ \mathbb{R}~ décroissante.
Alors (i) f admet en tout point a de I (dans la mesure où cela a un
sens) une limite à gauche f(a-) et une limite à droite f(a+) dans \mathbb{R}~ avec
f(a-) ≥ f(a) ≥ f(a+) (ii) f admet en l'extrémité droite b de I une
limite si et seulement si~elle est minorée sur I~; dans le cas contraire
limx\rightarrow~b,x\textless{}b~f(x) = -\infty~ (iii)
f admet en l'extrémité gauche a de I une limite si et seulement si~elle
est ma\\\\jmathmathmathmathorée sur I~; dans le cas contraire
limx\rightarrow~a,x\textgreater{}a~f(x) = +\infty~

\paragraph{8.1.2 Continuité et monotonie}

Lemme~8.1.3 Soit I un intervalle de \mathbb{R}~ et f : I \rightarrow~ \mathbb{R}~ monotone. Alors f est
continue si et seulement si~f(I) est un intervalle.

Démonstration La condition est évidemment nécessaire d'après le théorème
des valeurs intermédiaires. Inversement supposons que f(I) est un
intervalle et a \in I. On peut par exemple supposer que f est croissante.
Supposons que f(a) \textless{} f(a+) (ce qui sous entend que a n'est pas
l'extrémité droite de I). Soit x \textgreater{} a. On a alors f(a)
\textless{} f(a+) \leq f(x) (puisque f(a+) =\
inf t\textgreater{}af(t)). En particulier {]}f(a),f(a+){[}\subset~
{[}f(a),f(x){]} \subset~ f(I) (convexité des intervalles). Soit alors y
\in{]}f(a),f(a+){[}~; on peut poser y = f(t) pour t \in I. Mais si t \leq a, on
a y = f(t) \leq f(a) et si t \textgreater{} a on a y = f(t) ≥ f(a+). C'est
absurde. Donc f(a) = f(a+). On montre de même que si a n'est pas
l'extrémité gauche de I, f(a) = f(a-). Donc f est continue sur I.

Théorème~8.1.4 Soit I un intervalle de \mathbb{R}~ et f : I \rightarrow~ \mathbb{R}~ continue
strictement monotone. Alors J = f(I) est un intervalle de \mathbb{R}~ et f induit
un homéomorphisme de I sur J.

Démonstration On sait dé\\\\jmathmathmathmathà que J est un intervalle~; alors
f^-1 : J \rightarrow~ I est encore strictement monotone et
f^-1(J) = I est un intervalle, donc f^-1 est
continue. Donc f induit un homéomorphisme de I sur J.

Le théorème suivant montre que réciproquement, la condition de stricte
monotonie est une condition nécessaire pour un homéomorphisme d'un
intervalle sur un autre.

Théorème~8.1.5 Soit I un intervalle de \mathbb{R}~ et f : I \rightarrow~ \mathbb{R}~ continue. Alors f
est in\\\\jmathmathmathmathective si et seulement si~elle est strictement monotone.

Démonstration La condition est évidemment suffisante. Inversement,
supposons f continue et in\\\\jmathmathmathmathective. Soit X = \(x,y) \in I
\times I∣x \textless{} y\ et g :
X \rightarrow~ \mathbb{R}~ définie par g(x,y) = f(y) - f(x). Alors X est connexe (car
convexe) et g est continue. L'ensemble g(X) est donc un intervalle de \mathbb{R}~
et cet intervalle ne contient pas 0 car g est in\\\\jmathmathmathmathective. Donc soit g(X)
\subset~{]}0,+\infty~{[} (auquel cas f est strictement croissante), soit g(X) \subset~{]}
-\infty~,0{[} (auquel cas f est strictement décroissante).

{[}
{[}
