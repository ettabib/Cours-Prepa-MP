\textbf{Warning: 
requires JavaScript to process the mathematics on this page.\\ If your
browser supports JavaScript, be sure it is enabled.}

\begin{center}\rule{3in}{0.4pt}\end{center}

{[}
{[}
{[}{]}
{[}

\subsubsection{7.4 Séries absolument convergentes}

\paragraph{7.4.1 Notion de convergence absolue}

Définition~7.4.1 Soit E un espace vectoriel normé. On dit que la série
\\sum  xn~ est
absolument convergente si la série à termes réels positifs
\\sum ~
\\textbar{}xn\\textbar{}
converge.

Théorème~7.4.1 Soit E un espace vectoriel normé~complet. Alors toute
série absolument convergente à terme général dans E est convergente.

Démonstration On a
\\textbar{}\\\sum
 n=p^qxn\\textbar{}
\leq\\sum ~
n=p^q\\textbar{}xn\\textbar{}.
Si la série \\sum ~
\\textbar{}xn\\textbar{}
converge, elle vérifie le critère de Cauchy, il en est donc de même de
la série \\sum ~
xn et donc celle-ci converge.

Remarque~7.4.1 L'avantage est bien entendu de ramener l'étude à celle
d'une série à termes réels positifs.

\paragraph{7.4.2 Critères de convergence absolue}

Théorème~7.4.2 Soit \\\sum
 xn et \\\sum
 yn deux séries telles que xn = O(yn)
et \\sum  yn~
est absolument convergente. Alors
\\sum  xn~
converge absolument.

Démonstration On a xn = O(yn)
\Leftrightarrow
\\textbar{}xn\\textbar{} =
O(\\textbar{}yn\\textbar{}) et
il suffit d'appliquer le théorème de comparaison pour les séries à
termes réels positifs.

Remarque~7.4.2 Le théorème ci-dessus reste valable même si les termes
généraux xn et yn ne sont pas dans le même espace
vectoriel normé. En particulier, la série étalon
\\sum  yn~ sera
le plus souvent une série à termes réels positifs.

En ce qui concerne les équivalents, on a un résultat plus fort

Théorème~7.4.3 Soit \\\sum
 xn une série à terme général dans l'espace vectoriel
normé~E et \\sum ~
yn une série à termes réels positifs. On suppose qu'il existe
\ell \in E \diagdown\0\ tel que xn ∼
\ellyn. Alors les deux séries sont simultanément convergentes ou
divergentes.

Démonstration Si \\sum ~
yn converge, on a xn = O(yn) et
yn ≥ 0, donc la série
\\sum  xn~ est
absolument convergente. Inversement, supposons que la série
\\sum  xn~
converge. Puisque xn - \ellyn = o(\ellyn), il
existe N \in \mathbb{N}~ tel que n ≥ N \rigtharrow~\\textbar{} xn -
\ellyn\\textbar{} \leq 1 \over 2
\\textbar{}\ellyn\\textbar{} = 1
\over 2
\\textbar{}\ell\\textbar{}yn. En
sommant on obtient
\\textbar{}\\\sum
 n=p^qxn -
\ell\\sum ~
n=p^qyn\\textbar{} \leq 1
\over 2
\\textbar{}\ell\\textbar{}\\\sum
 n=p^qyn. On en déduit

\begin{align*}
\\textbar{}\ell\\textbar{}\\sum
n=p^qy n& =&
\\textbar{}\ell\\sum
n=p^qy n\\textbar{}
\leq\\textbar{} \ell\\sum
n=p^qy n -\\sum
n=p^qx n\\textbar{}
+\\textbar{} \\sum
n=p^qx n\\textbar{}\%&
\\ & \leq& 1 \over 2
\\textbar{}\ell\\textbar{}\\sum
n=p^qy n +\\textbar{}
\sum n=p^qx~
n\\textbar{} \%& \\
\end{align*}

d'où en définitive \\\sum
 n=p^qyn \leq 2 \over
\\textbar{}\ell\\textbar{}
\\textbar{}\
\sum ~
n=p^qxn\\textbar{}. La série
\\sum  xn~
converge, donc vérifie le critère de Cauchy. Il en est donc de même de
la série \\sum ~
yn, qui est par suite convergente.

\paragraph{7.4.3 Règles classiques}

Il suffit maintenant d'appliquer ces résultats à des séries étalons,
comme les séries de Riemann ou les séries géométriques.

Lemme~7.4.4 Soit a un nombre complexe. La série
\\sum  a^n~
converge si et seulement si~\textbar{}a\textbar{} \textless{} 1.

Démonstration La condition est évidemment nécessaire puisque le terme
général doit tendre vers 0. Supposons la vérifiée. On a
\\sum ~
p=0^na^p = 1-a^n+1
\over 1-a qui admet la limite  1 \over
1-a . Donc la série converge.

Théorème~7.4.5 (règle de d'Alembert). Soit E un espace vectoriel normé
complet. Soit \\sum ~
xn une série à termes dans E telle que pour tout n \in \mathbb{N}~,
xn\neq~0 et telle que la suite (
\\textbar{}xn+1\\textbar{}
\over
\\textbar{}xn\\textbar{} )
admet une limite \ell \in \mathbb{R}~ \cup\ + \infty~\. Alors

\begin{itemize}
\itemsep1pt\parskip0pt\parsep0pt
\item
  (i) si \ell \textless{} 1, la série converge absolument
\item
  (ii) si \ell \textgreater{} 1, la série diverge
\end{itemize}

Démonstration (i) Si \ell \textless{} 1, soit \rho tel que \ell \textless{} \rho
\textless{} 1~; il existe N \in \mathbb{N}~ tel que n ≥ N \rigtharrow~
\\textbar{}xn+1\\textbar{}
\over
\\textbar{}xn\\textbar{} \leq \rho
soit \\textbar{}xn+1\\textbar{}
\leq \rho\\textbar{}xn\\textbar{}. On
a donc alors par récurrence
\\textbar{}xn\\textbar{} \leq
\rho^n-N\\textbar{}xN\\textbar{}
= O(\rho^n). Comme la série
\\sum  \rho^n~
converge, la série \\\sum
 xn converge absolument.

(ii) Si \ell \textgreater{} 1, il existe N \in \mathbb{N}~ tel que n ≥ N \rigtharrow~
\\textbar{}xn+1\\textbar{}
\over
\\textbar{}xn\\textbar{}
\textgreater{} 1 soit
\\textbar{}xn+1\\textbar{}
\textgreater{}\\textbar{}
xn\\textbar{}. On a donc alors par récurrence
\\textbar{}xn\\textbar{}
\textgreater{}\\textbar{}
xN\\textbar{}. La suite (xn) ne peut
donc pas avoir 0 pour limite et la série diverge.

Remarque~7.4.3 Si \ell = 1 on ne peut rien conclure comme le montre
l'exemple des séries de Riemann. Lorsque la règle de d'Alembert
s'applique, elle conduit à des convergences rapides (de type
exponentielle) ou des divergences grossières (le terme général ne tend
pas vers 0).

Théorème~7.4.6 (règle de Riemann). Soit E un espace vectoriel normé.
Soit \\sum  xn~
une série à termes dans E.

\begin{itemize}
\itemsep1pt\parskip0pt\parsep0pt
\item
  (i) S'il existe \alpha~ \textgreater{} 1 tel que xn = O( 1
  \over n^\alpha~ ), alors la série converge
  absolument
\item
  (ii) S'il existe \alpha~ \in \mathbb{R}~ et \ell \in E \diagdown\0\
  tels que xn ∼ \ell \over n^\alpha~
  alors la série converge absolument si \alpha~ \textgreater{} 1 et diverge si
  \alpha~ \leq 1.
\item
  (iii) Si E = \mathbb{R}~ et xn ≥ 0, et s'il existe \alpha~ \leq 1 et \ell
  \textgreater{} 0 (y compris + \infty~) tel que
  limn^\alpha~xn~ = \ell, alors la
  série diverge.
\end{itemize}

Démonstration (i) et (ii) résultent de ce qui précède. Pour (iii), il
suffit de remarquer que les hypothèses impliquent que  1
\over n^\alpha~ = O(xn). Comme \alpha~ \leq 1, la
série \\sum ~  1
\over n^\alpha~ diverge et donc aussi la série
\\sum  xn~.

\paragraph{7.4.4 Règles complémentaires}

Théorème~7.4.7 (règle de Cauchy). Soit E un espace vectoriel normé
complet. Soit \\sum ~
xn une série à termes dans E telle que la suite
\left
(\rootn\of\\textbar{}xn\\textbar{}\right
) admet une limite \ell \in \mathbb{R}~ \cup\ + \infty~\.
Alors

\begin{itemize}
\itemsep1pt\parskip0pt\parsep0pt
\item
  (i) si \ell \textless{} 1, la série converge absolument
\item
  (ii) si \ell \textgreater{} 1, la série diverge
\end{itemize}

Démonstration (i) Si \ell \textless{} 1, soit \rho tel que \ell \textless{} \rho
\textless{} 1~; il existe N \in \mathbb{N}~ tel que n ≥ N
\rigtharrow~\rootn\of\\textbar{}xn\\textbar{}
\leq \rho soit
\\textbar{}xn\\textbar{} \leq
\rho^n. Comme la série
\\sum  \rho^n~
converge, la série \\\sum
 xn converge absolument.

(ii) Si \ell \textgreater{} 1, il existe N \in \mathbb{N}~ tel que n ≥ N
\rigtharrow~\rootn\of\\textbar{}xn\\textbar{}
\textgreater{} 1 soit
\\textbar{}xn\\textbar{}
\textgreater{} 1. La suite (xn) ne peut donc pas avoir 0 pour
limite et la série diverge.

Théorème~7.4.8 (règle de Duhamel). Soit
\\sum  xn~ une
série à termes dans \mathbb{R}~^+ telle que pour tout n \in \mathbb{N}~,
xn\neq~0 et telle que  xn+1
\over xn = 1 - \lambda~ \over n +
o( 1 \over n ) . Alors

\begin{itemize}
\itemsep1pt\parskip0pt\parsep0pt
\item
  (i) si \lambda~ \textgreater{} 1, la série converge
\item
  (ii) si \lambda~ \textless{} 1, la série diverge
\end{itemize}

Démonstration Posons yn = 1 \over
n^\alpha~ . On a  yn+1 \over
yn = 1 - \alpha~ \over n + o( 1
\over n ). On en déduit que si
\alpha~\neq~\lambda~,  xn+1 \over
xn - yn+1 \over yn
∼ \alpha~-\lambda~ \over n est pour n assez grand du signe de \alpha~ -
\lambda~. Si \lambda~ \textless{} 1, soit \alpha~ tel que \lambda~ \textless{} \alpha~ \textless{} 1. On
a donc pour n ≥ N,  xn+1 \over xn
≥ yn+1 \over yn et comme la série
\\sum  yn~
diverge (car \alpha~ \textless{} 1), la série
\\sum  xn~
diverge. Si \lambda~ \textgreater{} 1, soit \alpha~ tel que \lambda~ \textgreater{} \alpha~
\textgreater{} 1. On a donc pour n ≥ N,  xn+1
\over xn \leq yn+1
\over yn et comme la série
\\sum  yn~
converge (car \alpha~ \textgreater{} 1), la série
\\sum  xn~
converge.

\paragraph{7.4.5 Comparaison à une intégrale}

Théorème~7.4.9 Soit f : {[}0,+\infty~{[}\rightarrow~ \mathbb{C} de classe \mathcal{C}^1 telle que
f' soit intégrable sur {[}0,+\infty~{[}. Posons wn
=\int  n-1^n~f(t) dt - f(n).
Alors la série \\sum ~
wn est absolument convergente.

Démonstration On a par une intégration par parties

\begin{align*} \int ~
n-1^n(t - n + 1)f'(t) dt& =& \left
{[}(t - n + 1)f(t)\right {]} n-1^n
-\int  n-1^n~f(t) dt\%&
\\ & =& -wn \%&
\\ \end{align*}

On en déduit que

\textbar{}wn\textbar{}\leq\int ~
n-1^n(t - n + 1)\textbar{}f'(t)\textbar{} dt
\leq\int ~
n-1^n\textbar{}f'(t)\textbar{} dt

et donc

\sum p=1^n\textbar{}w~
p\textbar{}\leq\\int  ~
0^n\textbar{}f'(t)\textbar{} dt
\leq\\int  ~
0^+\infty~\textbar{}f'(t)\textbar{} dt

ce qui montre la convergence de la série à termes positifs
\\sum ~
\textbar{}wn\textbar{} et donc la convergence absolue de la
série.

Corollaire~7.4.10 Soit f : {[}0,+\infty~{[}\rightarrow~ \mathbb{C} de classe \mathcal{C}^1 telle
que f et f' soient intégrables sur {[}0,+\infty~{[}. Alors la série
\\sum ~ f(n) est
absolument convergente.

Démonstration En effet la série
\\sum ~
\int ~
n-1^n\textbar{}f(t)\textbar{} dt est convergente car

\sum p=1^n~
\\int  ~
n-1^n\textbar{}f(t)\textbar{} dt =
\\int  ~
0^n\textbar{}f(t)\textbar{} dt
\leq\\int  ~
0^+\infty~\textbar{}f(t)\textbar{} dt

et comme \left \textbar{}\int ~
n-1^nf(t) dt\right
\textbar{}\leq\int ~
n-1^n\textbar{}f(t)\textbar{} dt, la série
\\sum ~
\int  n-1^n~f(t) dt est
absolument convergente. Comme
\\sum  wn~ est
également absolument convergente, il en est de même de la série
\\sum ~ f(n).

{[}
{[}
{[}
{[}
