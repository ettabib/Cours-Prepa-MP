\textbf{Warning: 
requires JavaScript to process the mathematics on this page.\\ If your
browser supports JavaScript, be sure it is enabled.}

\begin{center}\rule{3in}{0.4pt}\end{center}

{[}
{[}
{[}{]}
{[}

\subsubsection{9.5 Intégration sur un intervalle quelconque~: fonctions
à valeurs réelles positives}

\paragraph{9.5.1 Fonctions intégrables à valeurs réelles positives}

Définition~9.5.1 Soit I un intervalle de \mathbb{R}~, f : I \rightarrow~ \mathbb{R}~ positive et
continue par morceaux. On dit que f est intégrable sur I s'il existe une
constante M ≥ 0 telle que, pour tout segment {[}a,b{]} \subset~ I, on ait
\int  a^b~f \leq M . On note alors
\int  I~f =\
sup{[}a,b{]}\subset~I\int ~
a^bf.

Proposition~9.5.1 Soit I un intervalle de \mathbb{R}~, f : I \rightarrow~ \mathbb{R}~ positive et
continue par morceaux, intégrable sur I. Alors f est intégrable sur tout
intervalle I' inclus dans I et \int ~
I'f \leq\int  I~f.

Démonstration En effet tout segment inclus dans I' est également un
segment inclus dans I, donc le même M convient comme ma\\\\jmathmathmathmathorant.

Proposition~9.5.2 Soit f,g : I \rightarrow~ \mathbb{R}~ positives et continues par morceaux
telles que 0 \leq f \leq g. Si g est intégrable sur I il en est de même de f
et \int  I~f
\leq\int  I~g.

Démonstration Evident d'après la définition.

Proposition~9.5.3 Soit I un intervalle de \mathbb{R}~, f : I \rightarrow~ \mathbb{R}~ positive et
continue par morceaux. Alors f est intégrable sur I si et seulement si
il existe une suite ({[}an,bn{]})n\in\mathbb{N}~
croissante de segments contenus dans I, dont la réunion est égale à I,
et une constante positive M telle que \forall~~n \in \mathbb{N}~,
\int  an^bn~f
\leq M. Dans ce cas, on a

\int  I~f =\
supn\int ~
an^bn f =\
limn\rightarrow~+\infty~\int ~
an^bn f

Démonstration La condition est bien évidemment nécessaire~: prendre
n'importe quelle suite ({[}an,bn{]}) vérifiant les
conditions voulues. Inversement supposons qu'il existe une telle suite
({[}an,bn{]}) et une constante M ≥ 0. Soit J =
{[}a,b{]} un segment inclus dans I et posons Jn =
{[}an,bn{]}. Si b = sup~I,
alors supI \in\cupJn~ et donc il existe N
\in \mathbb{N}~ tel que supI \in JN~ auquel cas
supI \in Jn~ pour tout n ≥ N. Si par
contre, b \textless{} sup~I
= limbn~, alors il existe N \in \mathbb{N}~ tel
que n ≥ N \rigtharrow~ bn \textgreater{} b. Dans les deux cas il existe N
\in \mathbb{N}~ tel que n ≥ N \rigtharrow~ bn ≥ b. De même, il existe N' \in \mathbb{N}~ tel que
n ≥ N' \rigtharrow~ an \leq a. Soit n = max~(N,N'),
on a alors J = {[}a,b{]} \subset~ {[}an,bn{]} =
Jn et donc

\int  a^b~f
\leq\int  an^bn~
f \leq M

ce qui montre que f est intégrable sur I.

La démonstration précédente montre clairement dans sa première partie
que supn~\\int
 an^bnf \leq\
sup{[}a,b{]}\subset~I\int ~
a^bf =\int  I~f et dans
sa deuxième partie que
sup{[}a,b{]}\subset~I~\\int
 a^bf \leq\
supn\int ~
an^bnf, et donc l'égalité
\int  I~f =\
supn\int ~
an^bnf. Mais comme la suite
\left (\int ~
an^bnf\right
)n\in\mathbb{N}~ est croissante ma\\\\jmathmathmathmathorée, sa borne supérieure est aussi sa
limite.

Proposition~9.5.4 Soit I = {[}a,b{]} un segment de \mathbb{R}~, f : I \rightarrow~ \mathbb{R}~ positive
et continue par morceaux. Alors f est intégrable sur I et
\int  I~f =\\int
 a^bf. De plus f est intégrable sur {]}a,b{[},
{[}a,b{[} et {]}a,b{]}, toutes ces intégrales étant égales.

Démonstration Si J = {[}c,d{]} est un segment inclus dans {[}a,b{]}, on
a \int  c^d~f
\leq\int  a^b~f, donc f est
intégrable et \int  I~f
\leq\int  a^b~f. Mais d'autre part,
{[}a,b{]} est lui même un segment inclus dans I, donc
\int  a^b~f
\leq\int  I~f, et donc l'égalité. On sait
alors que f est intégrable sur tout intervalle inclus dans I et en
particulier sur {]}a,b{[}, {[}a,b{[} et {]}a,b{]}. De plus, si
an = a + 1 \over n et bn = b -
1 \over n , Jn =
{[}an,bn{]} est une suite croissante de segments
dont la réunion est {]}a,b{[}, donc

\int  {]}a,b{[}~f
= lim\\int ~
an^bn f =\\int
 a^bf =\int ~
{[}a,b{]}f

par continuité de l'intégrale par rapport à ses bornes. Comme on a
{]}a,b{[}\subset~ {[}a,b{[}\subset~ {[}a,b{]}, on a aussi \\int
 {]}a,b{[}f \leq\int  {[}a,b{[}~f
\leq\int  {[}a,b{]}~f, d'où l'égalité des
trois nombres. Il en est de même de \int ~
{]}a,b{]}f.

Proposition~9.5.5 Soit f : I \rightarrow~ \mathbb{R}~ continue positive intégrable, telle que
\int  I~f = 0. Alors f = 0.

Démonstration Pour tout segment J \subset~ I, on a 0
\leq\int  J~f \leq\\int
 If = 0, donc \int  J~f = 0 ce
qui implique que f est nulle sur J. La fonction f est donc nulle sur
tout segment inclus dans I, donc elle est nulle.

Proposition~9.5.6 Soit f,g : I \rightarrow~ \mathbb{R}~ positives et continues par morceaux,
soit \alpha~ \in \mathbb{R}~^+. Si f et g sont intégrables sur I, il en est de
même de f + g et de \alpha~f et on a

\int  I~(f + g)
=\int  I~f +\\int
 Ig\text et \int ~
I(\alpha~f) = \alpha~\int  I~f

Démonstration L'intégrabilité est évidente à partir de la définition.
Pour les égalités, il suffit de prendre une suite (Jn)
croissante de segments de réunion I et de passer à la limite dans les
formules

\int  Jn~(f + g)
=\int  Jn~f
+\int ~
Jng\text et
\int  Jn~(\alpha~f) =
\alpha~\int  Jn~f

Proposition~9.5.7 Soit I un intervalle de \mathbb{R}~, f : I \rightarrow~ \mathbb{R}~ positive et
continue par morceaux. Soit a \in I^o. Alors f est intégrable
sur I si et seulement si elle est intégrable sur I\bigcap{]} -\infty~,a{]} et sur I
\bigcap {[}a,+\infty~{[}. Dans ce cas,

\int  I~f =\\int
 I\bigcap{]}-\infty~,a{]}f +\int ~
I\bigcap{[}a,+\infty~{[}

Démonstration Si f est intégrable sur I, elle est intégrable sur tout
sous intervalle de I et donc sur I\bigcap{]} -\infty~,a{]} et sur I \bigcap {[}a,+\infty~{[}.
Inversement, si f est intégrable sur ces deux sous intervalles, soit
M1 et M2 les ma\\\\jmathmathmathmathorants des intégrales sur les sous
segments de I\bigcap{]} -\infty~,a{]} et I \bigcap {[}a,+\infty~{[}. Si J est un segment inclus
dans I on a

\int  J~f \leq\left
\ \cases M1 &si
supJ \leq a \cr M1~ +
M2&si a \in J \cr M2 &si a
\leq inf J ~ \right .

et dans tous les cas \int  J~f \leq
M1 + M2. Donc f est intégrable sur I. Soit alors
Jn = {[}an,bn{]} une suite croissante de
segments de réunion I. Pour n assez grand, on a an \leq a \leq
bn car a est dans l'intérieur de I. Mais
({[}an,a{]}) est une suite croissante de segments de réunion
I\bigcap{]} -\infty~,a{]} et ({[}a,bn{]}) est une suite croissante de
segments de réunion I \bigcap {[}a,+\infty~{[}. On peut donc passer à la limite dans
la formule \int ~
{[}an,bn{]}f =\int ~
{[}an,a{]}f +\int ~
{[}a,bn{]}f, et on obtient

\int  I~f =\\int
 I\bigcap{]}-\infty~,a{]}f +\int ~
I\bigcap{[}a,+\infty~{[}

Proposition~9.5.8 Soit -\infty~ \textless{} a \textless{} b \leq +\infty~, et f :
{[}a,b{[}\rightarrow~ \mathbb{R}~ positive et continue par morceaux. Pour x \in {[}a,b{[},
posons F(x) =\int  a^x~f(t) dt.
Alors f est intégrable sur {[}a,b{[} si et seulement si F admet une
limite au point b. Dans ce cas, \int ~
{[}a,b{[}f = limx\rightarrow~b~F(x) -
F(a)

Démonstration Soit bn une suite croissante de {[}a,b{[} de
limite b. Alors {[}a,bn{]} est une suite croissante de
segments dont la réunion est {[}a,b{[}. Donc f est intégrable si et
seulement si la suite \int ~
a^bnf = F(bn) - F(a) admet une
limite, donc si et seulement si la suite (F(bn)) est
convergente. Mais comme F est croissante, ceci équivaut à l'existence de
la limite de F en b.

Remarque~9.5.1 Si f n'est pas intégrable sur {[}a,b{[}, alors F, qui est
croissante, admet + \infty~ comme limite au point b.

Remarque~9.5.2 De même, si -\infty~\leq a \textless{} b \textless{} +\infty~, et f
:{]}a,b{]} \rightarrow~ \mathbb{R}~ positive et continue par morceaux. Pour x \in{]}a,b{]},
posons F(x) =\int  x^b~f(t) dt.
Alors f est intégrable sur {]}a,b{]} si et seulement si F (qui est cette
fois décroissante) admet une limite au point a. Dans ce cas,
\int  {]}a,b{]}~f = F(b)
- limx\rightarrow~a~F(x)

\paragraph{9.5.2 Règles de comparaison}

Théorème~9.5.9 Soit f,g : {[}a,b{[}\rightarrow~ \mathbb{R}~ continues par morceaux positives.
On suppose qu'au voisinage de b on a f = O(g) (resp. f = o(g)). Alors
(i) si g est intégrable sur {[}a,b{[}, il en est de même de f et
\int  {[}x,b{[}~f(t) dt =
O(\int  {[}x,b{[}~g(t) dt) (resp.
\int  {[}x,b{[}~f(t) dt =
o(\int  {[}x,b{[}~g(t) dt)) (ii) si f
n'est pas intégrable sur {[}a,b{[}, g ne l'est pas non plus et
\int  a^x~f(t) dt =
O(\int  a^x~g(t) dt) (resp.
\int  a^x~f(t) dt =
o(\int  a^x~g(t) dt))

Démonstration Les convergences et divergences découlent immédiatement de
l'inégalité 0 \leq f \leq Kg qui est vraie sur {[}c,b{[} et du fait que f et g
sont intégrables sur {[}a,c{]} (car continues par morceaux sur ce
segment). De plus f = o(g) \rigtharrow~ f = O(g). En ce qui concerne la comparaison
des restes ou des intégrales partielles, la démonstration est tout à
fait similaire à celle du théorème analogue sur les séries. Nous allons
la faire dans le cas f = o(g), la démonstration étant analogue pour f =
O(g) en changeant \epsilon en K ou en 2K.

(i) Supposons f = o(g) et g intégrable. Soit \epsilon \textgreater{} 0. Il
existe c \in {[}a,b{[} tel que t ≥ c \rigtharrow~ 0 \leq f(t) \leq \epsilong(t). Alors pour x ≥ c,
on a (en intégrant l'inégalité de x à b), 0 \leq\\int
 {[}x,b{[}f(t) dt \leq \epsilon\int ~
{[}x,b{[}g(t) dt et donc \int ~
{[}x,b{[}f(t) dt = o(\int ~
{[}x,b{[}g(t) dt).

(ii) Supposons f = o(g) et f non intégrable sur {[}a,b{[}. Soit \epsilon
\textgreater{} 0. Il existe c \in {[}a,b{[} tel que t ≥ c \rigtharrow~ 0 \leq f(t) \leq \epsilon
\over 2 g(t). Alors pour x ≥ c, on a (en intégrant
l'inégalité de c à x), \int ~
c^xf(t) dt \leq \epsilon \over 2
\int  c^x~g(t) dt, soit encore à
l'aide de la relation de Chasles

0 \leq\int  a^x~f(t) dt \leq \epsilon
\over 2 \int ~
a^xg(t) dt + \left
(\int  a^c~f(t) dt - \epsilon
\over 2 \int ~
a^cg(t) dt\right )

Mais comme on sait que g n'est pas intégrable sur {[}a,b{[} et que g ≥
0, on a
limx\rightarrow~b\\int ~
a^xg(t) dt = +\infty~. Donc il existe c' \in {[}a,b{[} tel que x
≥ c' \rigtharrow~ \epsilon \over 2 \int ~
a^xg(t) dt \textgreater{}\int ~
a^cf(t) dt - \epsilon \over 2
\int  a^c~g(t) dt. Alors, pour x
≥ max~(c,c'), on a

0 \leq\int  a^x~f(t) dt \leq \epsilon
\over 2 \int ~
a^xg(t) dt + \epsilon \over 2
\int  a^x~g(t) dt =
\epsilon\int  a^x~g(t) dt

et donc \int  a^x~f(t) dt =
o(\int  a^x~g(t) dt).

Remarque~9.5.3 Il suffit pour appliquer le théorème précédent que la
condition de positivité de f et g soit vérifiée dans un voisinage de b.

Théorème~9.5.10 Soit f,g : {[}a,b{[}\rightarrow~ \mathbb{R}~ continues par morceaux. On
suppose que g est positive et que au voisinage de b, on a f ∼ g. Alors f
et g sont simultanément intégrables ou non intégrables sur {[}a,b{[}.
Plus précisément (i) Si g est intégrable sur {[}a,b{[}, alors f
également et \int  {[}x,b{[}~f(t) dt
∼\int  {[}x,b{[}~g(t) dt (ii) Si g est
non intégrable sur {[}a,b{[}, alors f également et
\int  a^x~f(t) dt
∼\int  a^x~g(t) dt.

Démonstration Puisque f(t) ∼ g(t), il existe c \in {[}a,b{[} tel que x
\textgreater{} c \rigtharrow~ 1 \over 2 g(t) \leq f(t) \leq 3
\over 2 g(t) ce qui montre que f est positive au
voisinage de b et que l'on a à la fois f = O(g) et g = O(f). Le théorème
précédent assure alors que f est intégrable sur {[}a,b{[} si et
seulement si~g l'est. Pla\ccons nous dans le cas
d'intégrabilité. On a \textbar{}f - g\textbar{} = o(g), on en déduit que
\textbar{}f - g\textbar{} est intégrable et que
\int  {[}x,b{[}~\textbar{}f(t) -
g(t)\textbar{} dt = o(\int ~
{[}x,b{[}g(t) dt). Mais bien évidemment \left
\textbar{}\int  {[}x,b{[}~f(t) dt
-\int  {[}x,b{[}~g(t)
dt\right \textbar{}\leq\int ~
{[}x,b{[}\textbar{}f(t) - g(t)\textbar{} dt. On a donc
\int  {[}x,b{[}~f(t) dt
-\int  {[}x,b{[}~g(t) dt =
o(\int  {[}x,b{[}~g(t) dt) et donc
\int  {[}x,b{[}~f(t) dt
∼\int  {[}x,b{[}~g(t) dt. Dans le cas de
non intégrabilité, deux cas se présentent. Si \textbar{}f - g\textbar{}
est non intégrable, le théorème précédent assure que
\int  a^x~\textbar{}f(t) -
g(t)\textbar{} dt = o(\int ~
a^xg(t) dt)~; si par contre elle est intégrable,
\int  a^x~\textbar{}f(t) -
g(t)\textbar{} dt admet une limite finie en b alors que
\int  a^x~g(t) dt tend vers + \infty~
et on a donc encore \int ~
a^x\textbar{}f(t) - g(t)\textbar{} dt =
o(\int  a^x~g(t) dt). L'inégalité
\left \textbar{}\int ~
a^xf(t) dt -\int ~
a^xg(t) dt\right
\textbar{}\leq\int ~
a^x\textbar{}f(t) - g(t)\textbar{} dt donne alors
\int  a^x~f(t) dt
-\int  a^x~g(t) dt =
o(\int  a^x~g(t) dt) et donc
\int  a^x~f(t) dt
∼\int  a^x~g(t) dt.

\paragraph{9.5.3 Exemples fondamentaux}

L'idée générale est d'obtenir une famille de fonctions étalons.

Proposition~9.5.11 La fonction
t\mapsto~t^\alpha~ est intégrable sur
{[}a,+\infty~{[} (avec a \textgreater{} 0) si et seulement si~\alpha~ \textgreater{}
1.

Démonstration On a

\int  1^x~ dt
\over t^\alpha~ = \left
\ \cases  1 \over
\alpha~-1 (1 - x^1-\alpha~)&si \alpha~\neq~1
\cr \cr log~ x
&si \alpha~ = 1 \cr  \right .

qui admet une limite finie en + \infty~ si et seulement si~\alpha~ \textgreater{} 1.

Exemple~9.5.1 Intégrales de Bertrand \int ~
e^+\infty~ dt \over
t^\alpha~(log t)^\beta~~ . Si \alpha~
\textgreater{} 1, soit \gamma tel que 1 \textless{} \alpha~ \textless{} \gamma. On a
alors  1 \over
t^\alpha~(log t)^\beta~~ = o( 1
\over t^\gamma ) et donc
t\mapsto~ 1 \over
t^\alpha~(log t)^\beta~~ est
intégrable sur {[}e,+\infty~{[}. Si \alpha~ \textless{} 1, soit \gamma tel que \alpha~
\textless{} \gamma \textless{} 1~; on a alors  1 \over
t^\gamma = o( 1 \over
t^\alpha~(log t)^\beta~~ ) et
comme t\mapsto~ 1 \over
t^\gamma n'est pas intégrable sur {[}e,+\infty~{[},
t\mapsto~ 1 \over
t^\alpha~(log t)^\beta~~ n'est pas
intégrable sur {[}e,+\infty~{[}. Si \alpha~ = 1, on a par le changement de variables
u = log~ t,

\begin{align*} \int ~
e^x dt \over
t(log t)^\beta~~ & =&
\int ~
1^log x~ du
\over u^\beta~ \%&
\\ & =& \left
\ \cases  1 \over
\beta~-1 (1 - (log x)^1-\beta~~)&si
\alpha~\neq~1 \cr \cr
log \log~ x &si \alpha~ = 1
 \right .\%&\\
\end{align*}

qui admet une limite en + \infty~ si et seulement si~\beta~ \textgreater{} 1. En
définitive t\mapsto~ 1 \over
t^\alpha~(log t)^\beta~~ est
intégrable sur {[}e,+\infty~{[} si et seulement si~\alpha~ \textgreater{} 1 ou (\alpha~ =
1 et \beta~ \textgreater{} 1).

Proposition~9.5.12 La fonction
t\mapsto~t^\alpha~ est intégrable sur {]}0,a{]}
(avec a \textgreater{} 0) si et seulement si~\alpha~ \textless{} 1.

Démonstration On a

\int  x^a~ dt
\over t^\alpha~ = \left
\ \cases  1 \over
1-\alpha~ (a^1-\alpha~ - x^1-\alpha~)&si
\alpha~\neq~1 \cr \cr
log a -\ log~ x&si \alpha~
= 1  \right .

qui admet une limite au point 0 si et seulement si~\alpha~ \textless{} 1.

Exemple~9.5.2 Intégrales de Bertrand \int ~
0^1\diagupet^\alpha~\textbar{}log~
t\textbar{}^\beta~ dt. Si \alpha~ \textgreater{} -1, soit \gamma tel que \alpha~
\textgreater{} \gamma \textgreater{} -1. On a alors en 0,
t^\alpha~\textbar{}log~
t\textbar{}^\beta~ = o(t^\gamma) (car 
t^\alpha~\textbar{} log~
t\textbar{}^\beta~ \over t^\gamma =
t^\alpha~-\gamma\textbar{}log~
t\textbar{}^\beta~ tend vers 0 quand t tend vers 0) et comme
t\mapsto~t^\gamma est intégrable sur
{]}0,1\diagupe{]}, il en est de même de
t\mapsto~t^\alpha~\textbar{}log~
t\textbar{}^\beta~. Si \alpha~ \textless{} -1, soit \gamma tel que \alpha~
\textless{} \gamma \textless{} -1. Alors t^\gamma =
o(t^\alpha~\textbar{}log~
t\textbar{}^\beta~) et comme
t\mapsto~t^\gamma n'est pas intégrable sur
{]}0,1\diagupe{]}, il en est de même de
t\mapsto~t^\alpha~\textbar{}log~
t\textbar{}^\beta~. Si \alpha~ = -1, le changement de variables u =
-log~ t conduit à

\int  x^1\diagupe~
\textbar{}log t\textbar{}^\beta~~
\over t dt =\int ~
1^- log xu^\beta~~ du

qui admet une limite quand x tend vers 0 si et seulement si~\beta~
\textless{} -1. En définitive,
t\mapsto~t^\alpha~\textbar{}log~
t\textbar{}^\beta~ est intégrable sur {[}0,1\diagupe{[} si et seulement
si~\alpha~ \textgreater{} -1 ou (\alpha~ = -1 et \beta~ \textless{} -1).

{[}
{[}
{[}
{[}
