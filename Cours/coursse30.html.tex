\textbf{Warning: 
requires JavaScript to process the mathematics on this page.\\ If your
browser supports JavaScript, be sure it is enabled.}

\begin{center}\rule{3in}{0.4pt}\end{center}

{[}
{[}
{[}{]}
{[}

\subsubsection{5.4 Compléments~: le théorème de Baire et ses
conséquences}

\paragraph{5.4.1 Le théorème de Baire}

Théorème~5.4.1 (Baire). Soit E un espace métrique complet et
(Un)n\in\mathbb{N}~ une suite d'ouverts denses dans E. Alors
\⋂ ~
n\in\mathbb{N}~Un est encore dense dans E.

Démonstration Rappelons qu'une partie est dense si et seulement si elle
rencontre tout ouvert non vide. Soit donc U un tel ouvert de E, soit
x0 \in U \bigcap U0 (qui est non vide par densité de
U0 et ouvert comme intersection de deux ouverts). Soit
r0 \textgreater{} 0 tel que B'(x0,r0) \subset~ U
\bigcap U0. Supposons xn et rn construits et
voyons comment nous allons construire xn+1 et rn+1.
Comme Un+1 est dense et B(xn,rn) est un
ouvert, Un+1 \bigcap B(xn,rn) est ouvert et non
vide~; soit donc xn+1 \in Un+1 \bigcap
B(xn,rn) et rn+1 \textless{}
rn \over 2 tel que
B'(xn+1,rn+1) \subset~ Un+1 \bigcap
B(xn,rn) . On construit ainsi une suite de boules
fermées B'(xn,rn) telles que
B'(xn+1,rn+1) \subset~ B'(xn,rn) avec
rn \textless{} r0 \over
2^n . Le théorème des fermés emboîtés nous garantit que
\⋂ ~
n\in\mathbb{N}~B'(xn,rn)\neq~\varnothing~
(car \delta(B'(xn,rn)) \textless{} 2rn tend
vers 0). Mais on a B'(x0,r0) \subset~ U \bigcap U0 et
pour n ≥ 1, B'(xn,rn) \subset~ Un. On en déduit
que U \bigcap\⋂ ~
n\in\mathbb{N}~Un\neq~\varnothing~, ce qui achève la
démonstration.

En passant au complémentaire, on obtient une version équivalente

Théorème~5.4.2 (Baire). Soit E un espace métrique complet et
(Fn)n\in\mathbb{N}~ une suite de fermés d'intérieurs vides de E.
Alors \⋃ ~
n\in\mathbb{N}~Fn est encore d'intérieur vide dans E.

Exemple~5.4.1 On montre facilement qu'un sous-espace vectoriel de E
distinct de E est d'intérieur vide (exercice). On en déduit que, si E,
espace vectoriel normé~de dimension infinie, est complet, E (qui n'est
pas d'intérieur vide) ne peut pas être réunion dénombrable de
sous-espaces vectoriels de dimension finie (dont on sait qu'ils sont
fermés). En particulier E ne peut pas admettre de base dénombrable.
C'est ainsi que \mathbb{R}~{[}X{]} (qui admet une base dénombrable) n'est complet
pour aucune norme.

\paragraph{5.4.2 Les grands théorèmes}

Nous en citerons trois qui concernent tous des applications linéaires
dans des espaces vectoriels normés complets.

Théorème~5.4.3 (Banach-Steinhaus). Soit E un espace vectoriel
normé~complet et F un espace vectoriel normé. Soit H un ensemble
d'applications linéaires continues telles que

\forall~~x \in E,
\existsKx~ ≥ 0,
\forall~~u \in H,\quad
\\textbar{}u(x)\\textbar{} \leq Kx

Alors il existe K ≥ 0 tel que \forall~~u \in H,
\\textbar{}u\\textbar{} \leq K.

Démonstration Posons pour x \in E, p(x) =\
supu\inH\\textbar{}u(x)\\textbar{}(\leq
Kx) et considérons En = \x \in
E∣p(x) \leq n\. Remarquons tout
d'abord que En est fermé~: en effet si (xq) est une
suite d'éléments de En qui converge vers x \in E, on a pour tout
u dans H, \forall~~q \in \mathbb{N}~,
\\textbar{}u(xq)\\textbar{} \leq
n~; en faisant tendre q vers + \infty~ et en utilisant la continuité de u, on
a encore \\textbar{}u(x)\\textbar{} \leq n et
donc x \in En. Maintenant notre hypothèse implique que chaque x
de E appartient à l'un des En (par exemple pour n =
E(Kx) + 1). Donc E qui est d'intérieur évidemment non vide est
réunion d'une famille de fermés. Le théorème de Baire implique que l'un
des En est d'intérieur non vide~: soit donc N \in \mathbb{N}~,
x0 \in E et r \textgreater{} 0 tel que B'(x0,r) \subset~
EN. Prenons alors x \in B'(0,1) et u \in H. Alors x0 +
rx \in B'(x0,r) et donc \\textbar{}u(x0
+ rx)\\textbar{} \leq N. Mais alors
\\textbar{}u(x)\\textbar{} = 1
\over r \\textbar{}u(x0 + rx)
- u(x0)\\textbar{} \leq 1 \over
r (N +\\textbar{}
u(x0\\textbar{}) = K. On a donc
\forall~~u \in H,
\\textbar{}u\\textbar{} \leq K.

Remarque~5.4.1 Sous les mêmes hypothèses, on montre alors facilement
qu'une limite simple d'applications linéaires continues est encore
continue (attention à l'hypothèse E complet)~; en effet le théorème de
Banach Steinhaus implique que la suite est équicontinue (le module de
continuité en x0, \eta(\epsilon,x0), ne dépend pas de n) et on
montre simplement qu'une limite simple d'une suite équicontinue est
continue.

Théorème~5.4.4 (théorème de Banach). Soit E et F deux espaces vectoriels
normés complets, et u : E \rightarrow~ F linéaire, continue, bi\\\\jmathmathmathmathective. Alors
u^-1 est encore continue.

Démonstration On va montrer que u(B'(0,1)) \subset~ F contient une boule de
centre 0 dans F, B'(0,r1). On aura alors B'(0,r1) \subset~
u(B'(0,1)), soit u^-1(B'(0,r1)) \subset~ B'(0,1) et donc
si y \in F avec \\textbar{}y\\textbar{} \leq 1,
on aura u^-1( r1 \over 2 y) \in
B'(0,1) soit encore
\\textbar{}u^-1(y)\\textbar{}
\leq 2 \over r1 ce qui montrera que
u^-1 est continue.

Soit r \textgreater{} 0. On a E =\
⋃  n\in\mathbb{N}~~nB'(0,r), on en déduit que F =
u(E) = \⋃ ~
n\in\mathbb{N}~nu(B'(0,r)) et a fortiori F =\
⋃ ~
n\in\mathbb{N}~n\overlineu(B'(0,r)). L'espace vectoriel
normé complet F qui est son propre intérieur est réunion d'une famille
dénombrable de fermés~; donc l'un d'entre eux (Baire) est d'intérieur
non vide. Mais si n\overlineu(B'(0,r)) est
d'intérieur non vide, il en est de même de
\overlineu(B'(0,r)). Soit donc y0 \in F et \rho
\textgreater{} 0 tel que B'(y0,\rho)
\subset~\overlineu(B'(0,r)). On a aussi (puisque
l'application x\mapsto~ - x laisse invariante
B'(0,r)), B'(-y0,\rho) \subset~\overlineu(B'(0,r)),
et alors, si y \in B'(0,\rho),

2y = (y - y0) + (y + y0) \in B'(-y0,\rho) +
B(y0,\rho0)

or

B'(-y0,\rho) + B(y0,\rho0)
\subset~\overlineu(B'(0,r)) +
\overlineu(B'(0,r))
\subset~\overlineu(B'(0,2r))

(facile) et donc y \in\overlineu(B'(0,r)). On a donc
trouvé, pour tout r \textgreater{} 0 un \rho \textgreater{} 0 tel que
B'(0,\rho) \subset~\overlineu(B'(0,r)). Les translations étant
des homéomorphismes, on a évidemment pour tout x \in E, B'(u(x),\rho)
\subset~\overlineu(B'(x,r)).

Montrons alors que sous ces hypothèses B'(0,\rho) \subset~ u(B'(0,2r)). Soit en
effet y \in B'(0,\rho). Soit \rhon le réel associé à  r
\over 2^n par la propriété ci dessus. Quitte
à remplacer les \rhon par des réels plus petits, on peut supposer
que \rhon tend vers 0. On va construire un élément xn
de E par récurrence de manière à vérifier
\\textbar{}xn+1 -
xn\\textbar{} \leq r \over
2^n et \\textbar{}y -
u(xn)\\textbar{} \leq \rhon. On pose
x0 = 0~; supposons xn construit. On a donc y \in
B'(u(xn),\rhon)
\subset~\overlineu(B'(xn, r \over
2^n )) et donc on peut trouver un point xn+1 \in
B'(xn, r \over 2^n ) tel que
\\textbar{}y -
u(xn+1)\\textbar{} \leq \rhon+1, soit y \in
B'(u(xn+1),\rhon+1), ce qui achève la construction par
récurrence. On a donc pour tout n,
\\textbar{}xn+1 -
xn\\textbar{} \leq r \over
2^n et \\textbar{}y -
u(xn)\\textbar{} \leq \rhon. On a
\\textbar{}xn+p -
xn\\textbar{} \leq r \over
2^n + r \over 2^n+1 +
\\ldots~ + r
\over 2^n+p-1 \leq r \over
2^n-1 , ce qui montre que la suite (xn) est une
suite de Cauchy. Comme E est complet, elle converge. Soit x sa limite.
On a \\textbar{}x -
x0\\textbar{} \leq 2r d'après l'inégalité ci
dessus pour n = 0 et p tendant vers + \infty~. D'autre part l'inégalité
\\textbar{}y - u(xn)\\textbar{}
\leq \rhon et la continuité de u nous montrent que y = u(x), donc y
appartient à u(B'(0,2r)).

On a alors aussi B'(0, \rho \over 2r ) \subset~ u(B'(0,1)), ce
qui montre comme on l'a remarqué, que u^-1 est continue.

Théorème~5.4.5 (théorème du graphe fermé). Soit E et F deux espaces
vectoriels normés complets, et u : E \rightarrow~ F linéaire. Alors u est continue
si et seulement si~son graphe est fermé dans E \times F.

Démonstration Supposons tout d'abord que u est continue et soit
(xn,u(xn)) une suite du graphe qui converge vers
(x,y) \in E \times F. Alors limxn~ = x et
par continuité de u, limu(xn~) =
u(x)~; mais alors l'unicité de la limite nécessite y = u(x), donc (x,y)
est encore dans le graphe de u, ce qui montre bien que le graphe est
fermé (il s'agit là d'une propriété tout à fait générale des espaces
métriques, mais la réciproque est fausse en général). Supposons
maintenant que u est linéaire de graphe \Gamma fermé. Alors \Gamma est un
sous-espace vectoriel fermé de E \times F, donc il est complet. L'application
\Gamma \rightarrow~ E, (x,u(x))\mapsto~x est linéaire continue et
bi\\\\jmathmathmathmathective. D'après le théorème de Banach, sa réciproque
x\mapsto~(x,u(x)) est continue et donc
x\mapsto~u(x) aussi.

Remarque~5.4.2 Il s'agit d'une technique importante~; il est en effet
considérablement plus facile de montrer qu'un graphe est fermé plutôt
qu'une continuité~; si (xn) est une suite de limite x, il
s'agit de montrer non plus que la suite u(xn) converge vers
u(x) mais plutôt que la suite u(xn) ne peut pas avoir d'autre
limite que u(x)~; un exemple typique d'application linéaire de graphe
fermé est la dérivation pour la topologie de la convergence uniforme~:
le théorème de dérivation des suites uniformément convergentes ne fait
que traduire la fermeture du graphe (si la suite des dérivées converge
uniformément, alors c'est vers la dérivée de la limite)~; attention
cependant que la dérivation n'est pas continue pour la topologie de la
convergence uniforme (le théorème du graphe fermé ne s'applique pas car
l'espace des applications \mathcal{C}^1 n'est pas complet).

{[}
{[}
{[}
{[}
