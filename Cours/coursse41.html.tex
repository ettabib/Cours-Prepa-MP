\textbf{Warning: 
requires JavaScript to process the mathematics on this page.\\ If your
browser supports JavaScript, be sure it is enabled.}

\begin{center}\rule{3in}{0.4pt}\end{center}

{[}
{[}
{[}{]}
{[}

\subsubsection{7.7 Séries doubles}

En anticipant un peu sur le chapitre concernant les séries de fonctions,
nous ferons appel au lemme suivant pour la démonstration du théorème
fondamental sur les séries doubles.

Lemme~7.7.1 (Weierstrass~: théorème de convergence dominée pour les
séries) Soit (xn,q)(n,q)\in\mathbb{N}~\times\mathbb{N}~ une famille de nombres
réels ou complexes indexée qar \mathbb{N}~ \times \mathbb{N}~. On fait les hypothèses suivantes

\begin{itemize}
\itemsep1pt\parskip0pt\parsep0pt
\item
  il existe une séries à termes réels positifs
  \\sum  \alpha~n~
  convergente telle que \forall~~q \in
  \mathbb{N}~,\textbar{}xn,q\textbar{}\leq \alpha~n
\item
  pour chaque n \in \mathbb{N}~,
  limq\rightarrow~+\infty~xn,q~ existe (on
  appelle yn cette limite)
\end{itemize}

Alors, pour chaque q \in \mathbb{N}~, la série
\\sum ~
nxn,q est absolument convergente ainsi que la série
\\sum ~
nyn, la suite \left
(\\sum ~
n=0^+\infty~xn,q\right )q\in\mathbb{N}~
admet une limite quand q tend vers + \infty~ et on a

limq\rightarrow~+\infty~~\\sum
n=0^+\infty~x n,q = \\sum
n=0^+\infty~y n

autrement dit

limq\rightarrow~+\infty~~\\sum
n=0^+\infty~x n,q = \\sum
n=0^+\infty~lim q\rightarrow~+\infty~xn,q

(interversion de la limite et du signe somme)

Démonstration L'inégalité \textbar{}xn,q\textbar{}\leq
\alpha~n, celle qui s'en déduit par passage à la limite
\textbar{}yn\textbar{}\leq \alpha~n et la convergence de la
série \\sum ~
\alpha~n montrent les convergences absolues des séries
\\sum ~
nxn,q et
\\sum ~
nyn. Prenons donc \epsilon \textgreater{} 0 et choisissons M
tel que \\sum ~
n=M+1^+\infty~\alpha~n \textless{} \epsilon\over
4. On a alors

\begin{align*} \left
\textbar{}\sum n=0^+\infty~y~
n -\sum n=0^+\infty~x~
n,q\right \textbar{}& \leq& \\sum
n=0^+\infty~\textbar{}y n -
xn,q\textbar{}\leq\\sum
n=0^M\textbar{}y n - xn,q\textbar{}
+ \\sum
n=M+1^+\infty~(\textbar{}y n\textbar{} +
\textbar{}xn,q\textbar{})\%& \\
& \leq& \\sum
n=0^M\textbar{}y n - xn,q\textbar{}
+ 2\sum n=M+1^+\infty~\alpha~ n~
\leq\sum n=0^M\textbar{}y~
n - xn,q\textbar{} + \epsilon\over 2
\%&\\ \end{align*}

Maintenant, on a
limq\rightarrow~+\infty~~\\\sum
 n=0^M\textbar{}yn -
xn,q\textbar{} = 0 (chacun des termes de cette somme admet 0
pour limite), et donc il existe N \in \mathbb{N}~ tel que q ≥ N
\rigtharrow~\\sum ~
n=0^M\textbar{}yn - xn,q\textbar{}
\textless{} \epsilon\over 2. On a donc

q ≥ N \rigtharrow~\left \textbar{}\\sum
n=0^+\infty~y n -\\sum
n=0^+\infty~x n,q\right \textbar{}\leq
\epsilon\over 2 + \epsilon\over 2 = \epsilon

ce qui montre que la suite \left
(\\sum ~
n=0^+\infty~xn,q\right )q\in\mathbb{N}~
admet la limite \\sum ~
n=0^+\infty~yn quand q tend vers + \infty~.

Remarque~7.7.1 Le lecteur qui a dé\\\\jmathmathmathmathà des connaissances sur les séries de
fonctions, remarquera qu'il s'agit là tout simplement du théorème
d'interversion des limites dans le cas de convergence normale (donc
uniforme) d'une série de fonctions.

Nous pouvons maintenant démontrer le théorème d'interversion des signes
somme dans les séries doubles.

Théorème~7.7.2 Soit u = (un,p)(n,p)\in\mathbb{N}~\times\mathbb{N}~ une famille
de nombres réels ou complexes indexée par \mathbb{N}~ \times \mathbb{N}~. On suppose que

\begin{itemize}
\itemsep1pt\parskip0pt\parsep0pt
\item
  pour tout entier n la série
  \\sum ~
  pun,p est absolument convergente
\item
  la série \\sum ~
  n \\sum ~
  p=0^+\infty~\textbar{}un,p\textbar{} est
  convergente
\end{itemize}

Alors les séries \\\sum
 n\left
(\\sum ~
p=0^+\infty~un,p\right ) et
\\sum ~
p\left
(\\sum ~
n=0^+\infty~un,p\right ) sont
convergentes et on a

\sum n=0^+\infty~~\left
(\sum p=0^+\infty~u~
n,p\right ) = \\sum
p=0^+\infty~\left (\\sum
n=0^+\infty~u n,p\right )

Démonstration Nous allons appliquer le lemme précédent en posant
xn,q =\ \\sum
 p=0^qun,p et \alpha~n
= \\sum ~
p=0^+\infty~\textbar{}un,p\textbar{} et bien entendu
yn = \\sum ~
p=0^+\infty~un,p =\
limq\rightarrow~+\infty~xn,q. Les hypothèses du lemme étant
évidemment vérifiées, on sait que
\\sum ~
n=0^+\infty~xn,q admet la limite
\\sum ~
n=0^+\infty~yn quand q tend vers + \infty~. Mais, puisque
l'on a l'égalité
\textbar{}un,p\textbar{}\leq\\\sum
 p=0^+\infty~\textbar{}un,p\textbar{}, la série
\\sum ~
nun,p est absolument convergente pour tout p \in \mathbb{N}~ et
donc, par linéarité de la somme,

\sum n=0^+\infty~x n,q~ =
\\sum
n=0^+\infty~\\sum
p=0^qu n,p = \\sum
p=0^q \\sum
n=0^+\infty~u n,p

L'existence de
limq\rightarrow~+\infty~~\\\sum
 n=0^+\infty~xn,q montre donc que la série
\\sum ~
p \\sum ~
n=0^+\infty~un,p est convergente et a pour somme
\\sum ~
n=0^+\infty~yn =\
\sum ~
n=0^+\infty~\\\sum
 p=0^+\infty~un,p autrement dit que

\sum n=0^+\infty~~\left
(\sum p=0^+\infty~u~
n,p\right ) = \\sum
p=0^+\infty~\left (\\sum
n=0^+\infty~u n,p\right )

Remarque~7.7.2 En appliquant le théorème à la suite u' =
(\textbar{}un,p\textbar{})(n,p)\in\mathbb{N}~\times\mathbb{N}~, on constate que
la série \\sum ~
p\left
(\\sum ~
n=0^+\infty~\textbar{}un,p\textbar{}\right
) est convergente, ce qui implique la convergence absolue de la série
\\sum ~
p \\sum ~
n=0^+\infty~un,p.

Remarque~7.7.3 On pourra retenir le théorème précédent sous la forme
suivante

\sum n=0^+\infty~~\left
(\sum p=0^+\infty~\textbar{}u~
n,p\textbar{}\right ) \textless{}
+\infty~\rigtharrow~\\sum
n=0^+\infty~\left (\\sum
p=0^+\infty~u n,p\right ) =
\sum p=0^+\infty~~\left
(\sum n=0^+\infty~u~
n,p\right )

{[}
{[}
{[}
{[}
