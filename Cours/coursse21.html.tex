\textbf{Warning: 
requires JavaScript to process the mathematics on this page.\\ If your
browser supports JavaScript, be sure it is enabled.}

\begin{center}\rule{3in}{0.4pt}\end{center}

{[}
{[}
{[}{]}
{[}

\subsubsection{4.4 Limites de fonctions}

Définition~4.4.1 Si E et F sont deux ensembles, on appellera fonction de
E vers F toute application d'une partie X de E (le domaine de définition
de la fonction) dans F. On notera Def~ (f) le
domaine de définition de la fonction f.

\paragraph{4.4.1 Notion de limite suivant une partie}

Définition~4.4.2 Soit E et F deux espaces métriques, A \subset~ F , a
\in\overlineA. Soit f une fonction de E vers F telle
que A \subset~ Def~ (f). On dit que f admet une limite
en a suivant A s'il existe \ell \in F vérifiant les conditions équivalentes

\begin{itemize}
\itemsep1pt\parskip0pt\parsep0pt
\item
  (i) \forall~V \in V (\ell), \\exists~U
  \in V (a),\quad f(U \bigcap A) \subset~ V
\item
  (ii) \forall~~\epsilon \textgreater{} 0,
  \exists~\eta \textgreater{} 0,\quad (x
  \in A\text et d(x,a) \textless{} \eta) \rigtharrow~ d(f(x),\ell)
  \textless{} \epsilon.
\end{itemize}

Démonstration De nouveau, (ii) n'est qu'une reformulation de (i) en
termes de boules~: toute boule est un voisinage, tout voisinage contient
une boule.

Remarque~4.4.1 Sans la condition a \in\overlineA, on
pourrait avoir U \bigcap A = \varnothing~ et la notion deviendrait triviale, tout élément
\ell vérifiant la condition.

Proposition~4.4.1 Si la fonction f admet une limite en a suivant A,
l'élément \ell de E est unique~; on l'appelle la limite de la fonction en a
suivant A. On pose \ell =\
limx\rightarrow~a,x\inAf(x).

Démonstration Si \ell et \ell' conviennent avec \ell\neq~\ell', il existe d'après la
propriété de séparation V ouvert contenant \ell et V ' ouvert contenant \ell'
tels que V \bigcap V ' = \varnothing~. Mais \exists~U \in V (a), f(U \bigcap
A) \subset~ V et \exists~U' \in V (a), f(U' \bigcap A) \subset~ V '. On a
alors f(U \bigcap U' \bigcap A) \subset~ V \bigcap V ' = \varnothing~ alors que U \bigcap U' \bigcap
A\neq~\varnothing~ puisque U \bigcap U' est un voisinage de a et
que a \in\overlineA. C'est absurde.

Remarque~4.4.2 On prendra garde à ne pas introduire le symbole
limx\rightarrow~a,x\inA~f(x) de manière opératoire
avant d'avoir démontré son existence. On remarquera d'autre part que les
notions de convergence et de limites sont purement topologiques
puisqu'on peut les exprimer en terme de voisinages~; elles sont donc
inchangées par changement de distance topologiquement équivalente.

Théorème~4.4.2 Soit U0 un ouvert contenant a. Alors f admet
une limite en a suivant A si et seulement si il admet une limite suivant
U0 \bigcap A et dans ce cas la limite est la même (on dit que la
notion de limite est une notion locale).

Démonstration Si f(U \bigcap A) \subset~ V , on a à fortiori f(U \bigcap U0 \bigcap A)
\subset~ V . Inversement, si f(U \bigcap U0 \bigcap A) \subset~ V , U' = U \bigcap
U0 est un voisinage de a tel que f(U' \bigcap A) \subset~ V .

Exemple~4.4.1

\begin{enumerate}
\itemsep1pt\parskip0pt\parsep0pt
\item
  Pour E = \overline\mathbb{R}~, A = \mathbb{N}~, a = +\infty~ et f(n) =
  xn on retrouve le cas particulier des suites.
\item
  Pour A ={]}a,+\infty~{[}\bigcapDef~ (f) on trouve le cas
  particulier d'une limite à droite (si cela a un sens, c'est à dire si
  a est dans l'adhérence de cet ensemble)~; de même pour les limites à
  gauche.
\item
  Pour A = Def~ (f)
  \diagdown\a\ on trouve le cas important de
  limite quand x tend vers a en étant distinct de a.
\item
  Pour a \in Def~ (f) et A
  = Def~ (f), la seule limite possible est f(a)
  (facile).
\end{enumerate}

\paragraph{4.4.2 Propriétés élémentaires}

Proposition~4.4.3 Si f admet \ell pour limite en a suivant A, alors \ell
\in\overlinef(A).

Démonstration Si V \in V (\ell), il existe U \in V (a) tel que f(U \bigcap A) \subset~ V
(\bigcapf(A)) et comme U \bigcap A\neq~\varnothing~, on a V \bigcap
f(A)\neq~\varnothing~. Donc \ell
\in\overlinef(A).

Remarque~4.4.3 Si f admet \ell pour limite en a suivant A et si A' \subset~ A est
tel que a \in\overlineA', il est clair que f(U \bigcap A') \subset~
f(U \bigcap A) et donc f admet encore \ell comme limite en a suivant A'. La
réciproque est évidemment fausse mais on a

Théorème~4.4.4 Soit A et A' deux parties de E telles que A \cup A'
\subset~ Def (f) et a \in\overlineA~
\bigcap\overlineA'. Alors on a équivalence de

\begin{itemize}
\itemsep1pt\parskip0pt\parsep0pt
\item
  (i) f admet une limite en a suivant A \cup A'
\item
  (ii) f admet une limite suivant A, une limite suivant A' et ces
  limites sont égales.
\end{itemize}

Démonstration D'après la remarque précédente, on a (i) \rigtharrow~(ii).
Inversement supposons que \ell =\
limAf(x) = limA'~f(x). Soit
V un voisinage de \ell. Soit U \in V (a) tel que f(U \bigcap A) \subset~ V et U' \in V (a)
tel que f(U' \bigcap A') \subset~ V . Alors U \bigcap U' est un voisinage de a et f((U \bigcap
U') \bigcap (A \cup A')) \subset~ V (facile). Donc \ell est limite suivant A \cup A'.

Exemple~4.4.2 Une suite (xn) converge si et seulement si~les
deux sous suites (x2n) et (x2n+1) convergent et ont
la même limite. De même, une fonction admet une limite en a si et
seulement si~elle a une limite à gauche et une limite à droite et ces
limites sont égales (à condition que tout cela ait un sens).

\paragraph{4.4.3 Composition des limites}

Théorème~4.4.5 Soit E,F et G trois espaces métriques, f fonction de E
vers F, g une fonction de F vers G. Soit A une partie de E et B une
partie de F. On suppose que A \subset~ Def~ (f), B
\subset~ Def~ (g) et f(A) \subset~ B (si bien que g \cdot f est
définie sur A). Si f admet une limite b en a suivant A et si g admet une
limite \ell en b suivant B, alors g \cdot f admet \ell pour limite en a suivant A.

Démonstration Remarquons que b \in\overlinef(A)
\subset~\overlineB. Soit alors W \in V (\ell). Il existe V \in V
(b) tel que g(V \bigcap B) \subset~ W. Pour ce voisinage V , il existe U \in V (a) tel
que f(U \bigcap A) \subset~ V . Mais on a f(U \bigcap A) \subset~ f(A) \subset~ B et donc f(U \bigcap A) \subset~ V \bigcap
B soit g \cdot f(U \bigcap A) \subset~ W, ce qui achève la démonstration.

Proposition~4.4.6 Soit
E,F1,\\ldots,Fp~
des espaces métriques, F = F1 \times⋯ \times
Fp, pi la pro\\\\jmathmathmathmathection de F sur Fi définie
par
pi(y1,\\ldots,yp~)
= yi. Soit f une fonction de E vers F, fi =
pi \cdot f si bien que f(x) =
(f1(x),\\ldots,fp~(x)).
Alors f admet une limite \ell en a suivant A si et seulement si~chacune des
fi admet une limite \elli en a suivant A. dans ce cas \ell
=
(\ell1,\\ldots,\ellp~).

Lemme~4.4.7 Pour tout b \in F on a
limy\rightarrow~bpi~(y) =
pi(b).

Démonstration Soit V i un voisinage ouvert de bi =
pi(b). Alors U = F1 \times⋯ \times
V i \times⋯ \times Fp est un ouvert
contenant b tel que pi(U) \subset~ V .

Démonstration La condition est nécessaire d'après le théorème de
composition des limites~: si f admet \ell pour limite, alors pi \cdot
f admet pour limite pi(\ell) que l'on nomme \elli. On a
alors bien entendu, \ell =
(\ell1,\\ldots,\ellp~).
Réciproquement, supposons que chacune des fi admet
\elli pour limite en a suivant A et soit \ell =
(\ell1,\\ldots,\ellp~).
Soit V un voisinage de \ell. Il existe alors un ouvert élémentaire V
1 \times⋯ \times V p tel que
(\ell1,\\ldots,\ellp~)
\subset~ V 1 \times⋯ \times V p \subset~ V . Pour
chaque i, il existe Ui voisinage de a tel que
fi(Ui \bigcap A) \subset~ V i. Soit U = U1
\bigcap\\ldots~ \bigcap
Up. On a alors f(U \bigcap A) \subset~ V 1
\times⋯ \times V p \subset~ V , ce qui montre que f
a \ell pour limite en a suivant A.

\paragraph{4.4.4 Limites et suites}

Théorème~4.4.8 Soit E et F deux espaces métriques. Alors f admet \ell pour
limite en a suivant A si et seulement si~pour toute suite (an)
d'éléments de A de limite a, la suite (f(an))n\in\mathbb{N}~
admet \ell pour limite.

Démonstration Le fait que la condition soit nécessaire résulte du
théorème de composition des limites~: soit V voisinage de \ell~; il existe
U \in V (a) tel que f(U \bigcap A) \subset~ V ~; il existe N \in \mathbb{N}~ tel que n ≥ N \rigtharrow~
an \in U(\bigcapA)~; alors, pour n ≥ N, on a f(an) \in V ,
donc \ell est limite de la suite (f(an)). Supposons maintenant
que f n'admet pas \ell pour limite en a suivant A~; ceci signifie que

\exists~\epsilon \textgreater{} 0,
\forall~~\eta \textgreater{}
0\exists~a' \in A\text tel que
d(a,a') \textless{} \eta\text et d(f(x),\ell) ≥ \epsilon

Pour \eta = 1 \over n+1 , on a donc an \in A tel
que d(a,an) \textless{} 1 \over n+1 avec
d(\ell,f(an)) ≥ \epsilon, d'où une suite d'éléments de A de limite a
telle que la suite (f(an)) n'admet pas \ell pour limite. Ceci
démontre la réciproque par contraposition.

Corollaire~4.4.9 Avec les mêmes notations, f admet une limite en a
suivant A si et seulement si~pour toute suite (an) d'éléments
de A de limite a, la suite (f(an)) converge.

Démonstration La condition est évidemment nécessaire. Pour la
réciproque, il suffit de montrer que la limite de la suite
(f(an)) ne dépend pas de la suite (an)~; or si
(an) et (bn) sont deux telles suites, on définit
(cn) par c2n = an et c2n+1 =
bn~; cette suite converge vers a, donc la suite
(f(cn)) converge et donc ses deux sous suites
(f(an)) et (f(bn)) ont la même limite.

Remarque~4.4.4 Ce corollaire permet d'assurer l'existence d'une limite
sans expliciter celle-ci à condition d'avoir un critère de convergence
des suites (par exemple le critère de Cauchy).

{[}
{[}
{[}
{[}
