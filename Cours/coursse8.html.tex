\textbf{Warning: 
requires JavaScript to process the mathematics on this page.\\ If your
browser supports JavaScript, be sure it is enabled.}

\begin{center}\rule{3in}{0.4pt}\end{center}

{[}
{[}
{[}{]}
{[}

\subsubsection{2.2 Bases et dimension}

\paragraph{2.2.1 Existence de bases}

Lemme~2.2.1 fondamental. Soit E un K-espace vectoriel et \mathcal{E} une famille
de E. On a équivalence de (i) \mathcal{E} est une base de E (ii) \mathcal{E} est une famille
libre maximale (toute surfamille stricte est liée) (iii) \mathcal{E} est une
famille génératrice minimale (toute sous-famille stricte est non
génératrice).

Démonstration Si \mathcal{E} est une base, on ne peut lui ad\\\\jmathmathmathmathoindre aucun vecteur
x sans obtenir une famille liée puisque x est combinaison linéaire de
\mathcal{E}~; on ne peut lui retirer aucun vecteur ei0 sans
obtenir une famille non génératrice puisque ei0
n'est pas combinaison linéaire des autres vecteurs. Donc une base est
une famille libre maximale et une famille génératrice minimale.

Inversement, soit \mathcal{E} une famille libre maximale~; si on lui ad\\\\jmathmathmathmathoint un
vecteur x, la famille devient liée, c'est donc que x est combinaison
linéaire de \mathcal{E} et donc \mathcal{E} est également génératrice, donc c'est une base.

De même soit \mathcal{E} une famille génératrice minimale~; si elle était liée,
l'un des vecteurs serait combinaison linéaire des autres vecteurs et la
famille obtenue en retirant ce vecteur serait encore génératrice, ce qui
n'est pas~; la famille est donc libre, donc c'est une base.

Théorème~2.2.2 (de la base incomplète). Soit E un K-espace vectoriel ,
(ei)i\inI une famille génératrice de E. Soit L une
partie de I telle que la famille (ei)i\inL soit libre.
Alors il existe une partie J avec L \subset~ J \subset~ I telle que
(ei)i\inJ est une base de E.

Démonstration On considère X =
\J∣L \subset~ J \subset~
I\text et
(ei)i\inJ\text
libre\ ordonné par l'inclusion. Si
(Js)s\inS est une famille totalement ordonnée de X,
alors J = \cups\inSJs est encore dans X (facile car si on
a une sous-famille finie de la famille (e\\\\jmathmathmathmath)\\\\jmathmathmathmath\inJ, ils
sont tous dans un même Js, donc forment une famille libre) et
c'est un ma\\\\jmathmathmathmathorant de la famille. Le théorème de Zorn garantit que
l'ensemble X admet un élément maximal. Soit J un tel élément maximal. La
famille (ei)i\inJ est libre. D'autre part, pour tout \\\\jmathmathmathmath
\in I \diagdown J, la famille
(ei)i\inI\cup\\\\\jmathmathmathmath\ est
liée, donc e\\\\jmathmathmathmath est combinaison linéaire de la famille
(ei)i\inJ. On en déduit que la famille est aussi
génératrice.

Corollaire~2.2.3 Tout K-espace vectoriel admet des bases.

Démonstration Prendre la famille (x)x\inE comme famille
génératrice et la famille vide comme sous-famille libre.

Corollaire~2.2.4 Soit E un K-espace vectoriel et F un sous-espace
vectoriel de E. Alors F admet des supplémentaires dans E.

Démonstration Il suffit de prendre une base de F, de la compléter en une
base de E et de prendre pour supplémentaire le sous-espace vectoriel
engendré par les vecteurs de la base introduits en supplément.

\paragraph{2.2.2 Espaces vectoriels de dimension finie. Dimension}

Définition~2.2.1 On dit qu'un espace vectoriel E est de dimension finie,
s'il admet une famille génératrice finie.

Lemme~2.2.5 Soit E un espace vectoriel de dimension finie,
(x1,\\ldots,xn~)
une famille génératrice finie. Alors toute famille libre a un cardinal
inférieur ou égal à n.

Démonstration Il suffit de démontrer par récurrence sur n que si n + 1
vecteurs
y1,\\ldots,yn+1~
sont combinaisons linéaires de n vecteurs
x1,\\ldots,xn~,
alors la famille
(y1,\\ldots,yn+1~)
est liée (c'est évident pour n = 1). Pour cela on écrit y\\\\jmathmathmathmath
= \\sum ~
i=1^nai,\\\\jmathmathmathmathx\\\\jmathmathmathmath. Si, pour tout i,
ai,n = 0 alors
y1,\\ldots,yn~
sont combinaisons linéaires de
x1,\\ldots,xn-1~,
donc forment une famille liée~; il en est de même a fortiori de la
famille complète. Sinon par exemple
an+1,n\neq~0. On pose alors
\forall~i \in {[}1,n{]}, zi = yi~
- ai,n \over an+1,n
yn+1 =\ \\sum
 \\\\jmathmathmathmath=1^n-1bi,\\\\jmathmathmathmathx\\\\jmathmathmathmath. Par récurrence,
la famille
z1,\\ldots,zn~
est liée (combinaisons linéaires de
x1,\\ldots,xn-1~),
donc il existe
\alpha~1,\\ldots,\alpha~n~
non tous nuls tels que \alpha~1z1 +
\\ldots~ +
\alpha~nzn = 0 ce qui donne après remplacement
\alpha~1y1 +
\\ldots~ +
\alpha~nyn + \beta~yn+1 = 0 et montre que la famille
est liée. Ceci achève la démonstration.

Remarque~2.2.1 Ceci permet de redémontrer dans ce cas le théorème de la
base incomplète sans faire appel au théorème de Zorn, l'existence d'un
sous-ensemble J de X maximal étant garanti par la limitation sur son
cardinal induite par le lemme précédent. Le lemme précédent montre aussi
que le cardinal d'une famille libre est nécessairement fini et que le
cardinal d'une famille libre est inférieur au cardinal d'une famille
génératrice. Comme les bases sont à la fois libres et génératrices, on a
le théorème suivant~:

Théorème~2.2.6 Soit E un K-espace vectoriel de dimension finie. Alors
toutes les bases de E ont un même cardinal fini, appelé la dimension de
E. Toute famille libre est finie et a un cardinal inférieur ou égal à
dim~ E avec égalité si et seulement si c'est
une base de E. Toute famille génératrice a un cardinal supérieur ou égal
à dim~ E avec égalité si et seulement si c'est
une base de E.

\paragraph{2.2.3 Résultats sur la dimension}

Théorème~2.2.7 Deux espaces sont isomorphes si et seulement si ils ont
la même dimension.

Démonstration Deux espaces de même dimension sont isomorphes~: prendre
deux bases et envoyer l'une sur l'autre. Inversement, l'image d'une base
par un isomorphisme étant une base, deux espaces isomorphes ont même
dimension.

Théorème~2.2.8

\begin{itemize}
\itemsep1pt\parskip0pt\parsep0pt
\item
  dim (E \times F) =\ dim~
  E + dim~ F
\item
  dim (F1~
  \oplus~⋯ \oplus~ Fk)
  = \\sum ~
  dim Fi~
\item
   dim (E\diagupF) =\ dim~ E
  - dim~ F
\item
  f : E \rightarrow~ F linéaire, dim~ E
  = dim~
  \mathrmKer~f
  + dim~
  \mathrmIm~f (théorème du
  rang)
\item
  dim (F + G) =\ dim~
  F + dim G -\ dim~ F
  \bigcap G
\item
  F \subset~ E,\quad dim~ F
  \leq dim~ E avec égalité si et seulement si F =
  E
\item
  dim L(E,F) =\ dim~
  E.dim~ F
\end{itemize}

Démonstration

\begin{itemize}
\item
  Si (ei)i\inI et (f\\\\jmathmathmathmath)\\\\jmathmathmathmath\inJ sont des
  bases respectives de E et F, on vérifie immédiatement que la famille
  \left ((ei,0)\right
  )i\inI \cup\left
  ((0,f\\\\jmathmathmathmath)\right )\\\\jmathmathmathmath\inJ) est une base de E
  \times F en écrivant (x,y) = (x,0) + (0,y).
\item
  F1 \oplus~⋯ \oplus~ Fk est
  isomorphe à F1 \times⋯ \times Fk
  (prendre
  (x1,\\ldots,xk)\mapsto~x1~
  + \\ldots~ +
  xk) qui est de dimension
  \\sum ~
  dim Fi~ par récurrence à partir du
  résultat précédent.
\item
  L'espace E\diagupF est isomorphe à tout supplémentaire de F dans E qui est
  de dimension dim~ E -\
  dim F d'après le résultat précédent.
\item
  Il existe un isomorphisme \overlinef de
  E\diagup\mathrmKer~f sur
  \mathrmIm~f, d'où
  dim~
  \mathrmIm~f
  = dim~
  E\diagup\mathrmKer~f
  = dim E -\ dim~
  \mathrmKer~f.
\item
  L'application linéaire f : F \times G \rightarrow~ E,
  (x,y)\mapsto~x + y a pour image F + G et pour
  noyau \(x,-x)∣x \in F \bigcap
  G\ qui est naturellement isomorphe à F \bigcap G. Il suffit
  donc d'appliquer le résultat précédent à f.
\item
  D'après le théorème de la base incomplète, toute base de F est une
  famille libre de E, donc peut être complétée en une base de E, d'où
  dim F \leq\ dim~ E avec
  égalité si et seulement si F = E
\item
  On peut soit utiliser le résultat analogue sur les matrices que l'on
  verra plus loin, soit montrer que si (ei)i\inI et
  (f\\\\jmathmathmathmath)\\\\jmathmathmathmath\inJ sont des bases respectives de E et F,
  alors les applications linéaires \phi\\\\jmathmathmathmath,i \in L(E,F) définies par

   \phi\\\\jmathmathmathmath,i(ek) = \left
  \\cases f\\\\jmathmathmathmath&si k = i
  \cr 0 &si k\neq~i 
  \right .

  forment une base de L(E,F), ce qui est laissé en exercice.
\end{itemize}

Théorème~2.2.9 Soit F et G deux sous-espaces vectoriels de E de
dimension finie. On a équivalence de (i) F et G sont supplémentaires
dans E (ii) E = F + G et dim~ E
= dim F +\ dim~ G
(iii) F \bigcap G = \0\ et
dim E =\ dim~ F
+ dim~ G

Démonstration Elémentaire.

Remarque~2.2.2 On sait que si F est un sous-espace vectoriel de E, la
restriction à tout supplémentaire G de F dans E de la pro\\\\jmathmathmathmathection
canonique de E sur E\diagupF est un isomorphisme d'espace vectoriel. Ceci
\\\\jmathmathmathmathustifie la définition suivante~:

Définition~2.2.2 Soit E un K espace vectoriel et F un sous-espace
vectoriel de E. On appelle codimension de F la dimension (éventuellement
infinie) de l'espace vectoriel quotient E\diagupF, qui est encore la dimension
de tout supplémentaire de F dans E.

{[}
{[}
{[}
{[}
