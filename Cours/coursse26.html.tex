\textbf{Warning: 
requires JavaScript to process the mathematics on this page.\\ If your
browser supports JavaScript, be sure it is enabled.}

\begin{center}\rule{3in}{0.4pt}\end{center}

{[}
{[}
{[}{]}
{[}

\subsubsection{4.9 Espaces et parties connexes}

\paragraph{4.9.1 Notion de connexe}

Définition~4.9.1 Soit E un espace topologique. On dit que E n'est pas
connexe s'il vérifie les conditions équivalentes

\begin{itemize}
\itemsep1pt\parskip0pt\parsep0pt
\item
  (i) E est réunion de deux ouverts non vides dis\\\\jmathmathmathmathoints
\item
  (ii) E est réunion de deux fermés non vides dis\\\\jmathmathmathmathoints
\item
  (iii) il existe une partie de E distincte de \varnothing~ et de E qui est à la
  fois ouverte et fermée dans E.
\end{itemize}

Démonstration (i) \rigtharrow~(ii). Si E = U1 \cup U2 avec
U1 et U2 ouverts non vides dis\\\\jmathmathmathmathoints, U1
et U2 sont aussi fermés puisque U1 = cU2
et U2 = cU1, d'où la propriété (ii). On montre de
même que (ii) \rigtharrow~(i).

(i) \rigtharrow~(iii). Si E = U1 \cup U2 avec U1 et
U2 ouverts non vides dis\\\\jmathmathmathmathoints, alors U1 est ouvert,
fermé (car U1 = cU2), distinct de \varnothing~ et de E.

(iii) \rigtharrow~(i). Si A est à la fois ouverte et fermée, distincte de \varnothing~ et de
E, on écrit E = A \cupcA, avec cA ouvert (complémentaire d'un fermé), les
deux étant non vides et dis\\\\jmathmathmathmathoints.

Définition~4.9.2 Soit E un espace topologique et F une partie de E. On
dit que F est connexe si F muni de la topologie induite est connexe.

\paragraph{4.9.2 Propriétés des connexes}

Théorème~4.9.1 Soit E,F deux espaces topologiques, f : E \rightarrow~ F continue.
Si E est connexe, alors f(E) est une partie connexe de F.

Démonstration En effet si f(E) est réunion de deux ouverts non vides
dis\\\\jmathmathmathmathoints U1 et U2 de f(E), alors E est réunion des
deux ouverts f^-1(U1) et
f^-1(U2) qui sont encore dis\\\\jmathmathmathmathoints et non vides

Corollaire~4.9.2 Soit E,F deux espaces topologiques, f : E \rightarrow~ F continue
et A une partie connexe de E. Alors f(A) est une partie connexe de F.

Démonstration En effet la restriction de f à A est encore continue, et
on peut lui appliquer le théorème précédent.

Proposition~4.9.3 Soit A une partie connexe de E. Alors toute partie B
telle que A \subset~ B \subset~\overlineA est connexe.

Démonstration En effet si B est réunion de deux ouverts de B non vides
dis\\\\jmathmathmathmathoints U1 et U2, alors A est réunion des deux
ouverts de A dis\\\\jmathmathmathmathoints~: U1 \bigcap A et U2 \bigcap A~; or ces
deux ouverts sont non vides car, A étant dense dans B, tout ouvert non
vide de B rencontre A. C'est absurde. Donc \overlineA
est connexe.

Proposition~4.9.4 Soit (Ai)i\inI une famille de
parties connexes de E telle que
\⋂ ~
i\inIAi\neq~\varnothing~. Alors
\⋃ ~
i\inIAi est connexe.

Démonstration Soit a
\in\⋂ ~
i\inIAi. Si A =\
⋃  i\inIAi = U1~ \cup
U2 avec U1 et U2 ouverts dis\\\\jmathmathmathmathoints de A,
alors chacun des Ai doit être contenu soit dans U1,
soit dans U2, sinon Ai serait réunion des deux
ouverts non vides dis\\\\jmathmathmathmathoints Ai \bigcap U1 et Ai
\bigcap U2. Comme Ai contient a, il est forcément contenu
dans celui des deux ouverts U1 et U2 qui contient a.
Mais alors A lui-même est contenu dans cet ouvert, et donc l'autre est
vide.

Corollaire~4.9.5 Soit E un espace topologique et a un point de E. Alors
l'ensemble des connexes contenant a a un plus grand élément appelé la
composante connexe de a dans E~; deux composantes connexes sont soit
dis\\\\jmathmathmathmathointes soit confondues~; toute composante connexe est fermée.

Démonstration La composante connexe de a est bien entendu
\⋃  ~a\inA,
A\textconnexeA~; c'est un connexe d'après la
proposition précédente et c'est bien entendu le plus grand~; si A est la
composante connexe de a, B celle de b et si A \bigcap
B\neq~\varnothing~, alors A \cup B est connexe et donc A \cup B \subset~
A, soit B \subset~ A et de même A \subset~ B soit A = B. D'autre part,
\overlineA est encore connexe contenant a, donc
\overlineA \subset~ A et donc A est fermé.

Proposition~4.9.6 Soit
E1,\\ldots,Ek~
des espaces connexes. Alors E1 \times⋯ \times
Ek est connexe.

Démonstration Il suffit évidemment de montrer le résultat pour k = 2.
Soit E1 \times E2 = U1 \cup U2 avec
U1 et U2 ouverts dis\\\\jmathmathmathmathoints. Remarquons que si a \in
E1, l'application y\mapsto~(a,y) est un
homéomorphisme de E2 sur \a\
\times E2 qui est donc aussi connexe. Donc V doit être contenu soit
dans U1 soit dans U2 (sinon il serait réunion des
deux ouverts non vides dis\\\\jmathmathmathmathoints (\a\
\times E2) \bigcap U1 et (\a\
\times E2) \bigcap U2). Soit b \in E2~; pour la même
raison, on a par exemple E1
\times\b\ \subset~ U1. Alors, pour tout
a \in E1, comme (a,b) \in\a\ \times
E2 et E1 \times\b\ \subset~
U1, on a nécessairement \a\
\times E2 \subset~ U1 et donc E1 \times E2 \subset~
U1, soit encore U2 = \varnothing~.

\paragraph{4.9.3 Connexes de \mathbb{R}~}

Définition~4.9.3 Une partie A de \mathbb{R}~ est dite convexe si
\forall~~a,b \in \mathbb{R}~, {[}a,b{]} \subset~ \mathbb{R}~.

Proposition~4.9.7 Les parties convexes de \mathbb{R}~ sont les intervalles.

Démonstration Il est clair que tout intervalle est convexe. L'ensemble
vide est bien entendu un intervalle. Soit donc A une partie convexe non
vide, m = inf~ A \in \mathbb{R}~
\cup\-\infty~\ et M =\
supA \in \mathbb{R}~ \cup\ + \infty~\. On a A \subset~
{[}m,M{]}. Pour montrer que A est un intervalle, il suffit de montrer
que {]}m,M{[}\subset~ A. Or, soit x \in{]}m,M{[}. Il existe a \in A tel que m \leq a
\textless{} x (propriété caractéristique de la borne inférieure) et de
même, il existe b \in A tel que x \textless{} b \leq M. On a donc x
\in{]}a,b{[}\subset~ A~; d'où l'inclusion et le résultat.

Théorème~4.9.8 Les parties connexes de \mathbb{R}~ sont les intervalles.

Démonstration Soit A une partie connexe~; si A n'était pas convexe, il
existerait a,b \in A tel que {]}a,b{[}⊄A (car a,b sont dans A)~; soit x
\in{]}a,b{[} tel que x∉A. On a alors A = (A\bigcap{]}
-\infty~,x{[}) \cup (A\bigcap{]}x,+\infty~{[}), réunion de deux ouverts de A non vides et
dis\\\\jmathmathmathmathoints~; c'est absurde. Donc A est convexe et donc un intervalle.

Inversement, soit I un intervalle~; alors il existe J intervalle ouvert
tel que J \subset~ I \subset~\overlineJ donc il suffit de montrer
que les intervalles ouverts sont connexes.

Soit I ={]}a,b{[} un intervalle ouvert de \mathbb{R}~ et A une partie ouverte et
fermée, non vide et distincte de I. Soit x \in I \diagdown A. Alors A =
(A\bigcap{]}a,x{[}) \cup (A\bigcap{]}x,b{[})~; au moins une des deux parties est non
vide, par exemple B = A\bigcap{]}x,b{[}= A \bigcap {[}x,b{[}. Cette partie est à la
fois ouverte et fermée dans I (intersection de deux ouverts de I et
aussi de deux fermés de I). Soit m = inf~ B ≥
x. On a m \in I et comme B est fermé dans I, on a m \in B. Mais alors
\exists~\epsilon \textgreater{} 0, {]}m - \epsilon,m + \epsilon{[}\subset~ B, ce
qui contredit la définition de la borne inférieure. C'est absurde, donc
I est connexe.

Corollaire~4.9.9 (théorème des valeurs intermédiaires). Soit E un espace
topologique connexe et f : E \rightarrow~ \mathbb{R}~ continue. Alors
\mathrmIm~f est un
intervalle de \mathbb{R}~.

Démonstration f(E) est connexe, donc un intervalle.

\paragraph{4.9.4 Connexité par arcs}

Définition~4.9.4 Soit E un espace topologique, a,b \in E. On appelle
chemin d'origine a et d'extrémité b dans E toute application continue \gamma
: {[}0,1{]} \rightarrow~ E telle que \gamma(0) = a et \gamma(1) = b.

Proposition~4.9.10 Soit E un espace topologique. La relation ''il existe
un chemin d'origine a et d'extrémité b'' est une relation d'équivalence
sur E.

Démonstration Cette relation est clairement réflexive (prendre \gamma
constant) et symétrique (prendre \gamma1(t) = \gamma(1 - t)). Pour la
transitivité, soit \gamma1 une chemin de a à b et \gamma2 un
chemin de b à c. On définit \gamma : {[}0,1{]} \rightarrow~ E par \gamma(t) =
\left \ \cases
\gamma1(2t) &si t \in {[}0,1\diagup2{]} \cr \gamma2(2t
- 1)&si t \in {[}1\diagup2,1{]} \cr  \right ..
Alors \gamma est un chemin de a à c.

Définition~4.9.5 On dit que E est connexe par arcs si, pour tout couple
(a,b) \in E^2 il existe un chemin de a à b dans E.

Proposition~4.9.11 Tout espace topologique connexe par arcs est connexe.

Démonstration Soit a \in E et pour x \in E soit \gammax un chemin
d'origine a et d'extrémité x. Les \gammax({[}0,1{]}) sont des
images de connexes par une application continue, ils sont donc connexes.
Leur intersection contient a, et donc leur réunion est connexe. Mais on
a évidemment E =\ \⋃
 x\inE\gammax({[}0,1{]}) (une réunion de parties de E est
contenue dans E et de plus tout élément x de E appartient à
\gammax({[}0,1{]})). Donc E est connexe.

{[}
{[}
{[}
{[}
