
\subsubsection{Notion de série entière}
\label{sec:notion-de-serie}


%\textbf{Warning: 
%requires JavaScript to process the mathematics on this page.\\ If your
%browser supports JavaScript, be sure it is enabled.}
%
%\begin{center}\rule{3in}{0.4pt}\end{center}
%
%{[}
%{[}{]}
%{[}
%
%\subsubsection{11.1 Convergence des séries entières}
%
%\paragraph{11.1.1 Notion de série entière}
%
\begin{de}
Soit $(a_n)_n \in \mathbb{N}$ une suite de l'espace
 vectoriel normé complet E. On appelle série entière associée à la suite
 (an) la série de fonctions de $\mathbb{C}$ (resp. $\mathbb{R}$~) dans E,
 $\sum_{n \geq 0} u_n$, où l'on pose $u_n(z) = a_n z^n$; on notera simplement
 $\sum_{n \geq 0} a_n z^n $ cette série de fonctions de la
 variable z.
 \end{de}
%
%Remarque~11.1.1 Dans le cas où E = \mathbb{R}~ ou E = \mathbb{C}, la série entière est
%associée à une unique série formelle
%\sum ~
%n=0^+\infty~anX^n \in K{[}{[}X{]}{]}.
%
%\paragraph{11.1.2 Rayon de convergence}
%
\begin{lem}[Abel]
Soit E un K-espace vectoriel normé complet,
$\sum a_n z^n$ une série entière à coefficients dans E. Soit
$z_0 \in K^*$ tel que la suite
$(a_n z_0^n)$ soit bornée. Alors la série
$\sum_{a_n} z^n$ converge absolument pour tout $z \in K$ tel que
$|z| < |z_0|$; la
série entière converge même normalement dans tout disque fermé $D'(0,r) =
\{ z \in K,|z|\leq r\} $
pour tout nombre réel r tel que $r <
|z_0|$.
\end{lem}
%Démonstration Soit M \geq 0 tel que \forall~~n \in \mathbb{N}~,
%\|anz0^n\|
%\leq M et soit z \in K tel que |z| \leq
%|z0|. On a alors
%\|anz^n\|
%=\|
%anz0^n\|
%\left | z \over z0
%\right |^n \leq M\left
%| z \over z0 \right
%|^n. Comme \left | z
%\over z0 \right |
%\leq 1, la série géométrique est convergente et donc la série
%\sum ~
%anz^n converge absolument. Pour z \in D'(0,r), on a
%de la même fa\ccon
%\|anz^n\|
%\leq M\left | r \over
%z0 \right |^n qui est une
%série convergente indépendante de z, donc la série converge normalement
%sur D'(0,r).
%
\begin{thm}
Soit E un K-espace vectoriel normé complet,
$\sum a_n z^n$ une série entière à coefficients dans E.
Posons $R1 =\sup ||z|| \sum
a_n z^n \text{ converge}
 \in \mathbb{R}^+ \cup +
\infty $ et $R2 =\sup ||z||(a_n z^n)\text{
est bornée } \in \mathbb{R}^+ \cup \ +
\infty $. On a $R_1 = R_2$. En notant R la
valeur commune, la série converge absolument dans $D(0,R) =
\left{ z \in K \| |z| \leq R \right}$ et converge normalement dans tout disque fermé $D'(0,r) = \left {z \in K, |z| \leq r \right}$
tel que $r \leq R$.
\end{thm}
%Démonstration Soit r \in {[}0,R1{[}~; d'après la propriété
%caractéristique de la borne supérieure, il existe z \in K tel que la série
%\sum ~
%anz^n converge avec r \leq
%|z|\leq R1. Mais alors
%limanz^n~ = 0, donc la
%suite (anz^n) est bornée et a fortiori la suite
%(anr^n) est bornée~; donc r \in {[}0,R2{]},
%soit {[}0,R1{[}\subset~ {[}0,R2{]} et donc R1 \leq
%R2. Soit r \in {[}0,R2{[}~; d'après la propriété
%caractéristique de la borne supérieure, il existe z \in K tel que la suite
%(anz^n) soit bornée avec r \leq
%|z|\leq R2. Mais alors, d'après le lemme
%d'Abel, la série \sum ~
%anr^n converge absolument, r \in
%{[}0,R1{]}, soit {[}0,R2{[}\subset~ {[}0,R1{]} et
%donc R2 \leq R1.
%
%Soit alors z \in D(0,R)~; il existe z0 \in K tel que la suite
%(anz0^n) soit bornée avec
%|z| \leq |z0|\leq R.
%Mais alors, d'après le lemme d'Abel, la série
%\sum ~
%anz^n converge absolument. De même, soit r
%\leq R~; il existe z0 \in K tel que la suite
%(anz0^n) soit bornée avec r \leq
%|z0|\leq R. Mais alors, d'après le lemme
%d'Abel, la série \sum ~
%anz^n converge normalement sur D'(0,r).
%
%Remarque~11.1.2 Le lecteur prendra garde au fait qu'en général la série
%ne converge pas normalement sur D(0,R) ni même uniformément.
%
%Définition~11.1.2 R est appelé le rayon de convergence de la série
%entière, D(0,R) son disque ouvert de convergence, C(0,R) =
%\z \in K∣|z|
%= R\ son cercle de convergence.
%
%Remarque~11.1.3 Par définition même au vu des résultats précédents, la
%série converge absolument dans le disque ouvert de convergence (et même
%uniformément dans tout disque fermé inclus dans le disque ouvert de
%convergence)~; pour |z| \textgreater{} R la série
%diverge et en fait, la suite (anz^n) n'est même pas
%bornée. La nature de la série sur le disque fermé de convergence dépend
%de la série et du point considéré.
%
%Exemple~11.1.1 Soit \alpha~ \in \mathbb{R}~~; la série entière
%\sum   z^n~
%\over n^\alpha~ a pour rayon de convergence 1~;
%pour |z| = 1 la nature de la série dépend à la fois de
%\alpha~ et de z.
%
%\begin{itemize}
%\itemsep1pt\parskip0pt\parsep0pt
%\item
%  (i) Pour \alpha~ \textgreater{} 1, la série converge pour tout z tel que
%  |z| = 1
%\item
%  (ii) Pour \alpha~ \leq 0, la série diverge pour tout z tel que
%  |z| = 1 (le terme général ne tend pas vers 0)
%\item
%  (iii) Pour 0 \leq \alpha~ \leq 1, la série diverge en z = 1 mais
%  converge pout tout point z tel que |z| = 1 et
%  z\neq~1 (appliquer le critère d'Abel).
%\end{itemize}
%
\paragraph{Recherche du rayon de convergence}
%
Les deux remarques suivantes, qui découlent immédiatement des résultats
précédents peuvent rendre de grands services dans la détermination du
rayon de convergence

\begin{itemize}
\item
   si $z \in K$ est tel que la série
  $\sum a_n z^n$ converge, alors $|z|\leq R$
\item
   si $z \in K$ est tel que la série
 $\sum a_n z^n$ diverge, alors $R \leq|z|$
\end{itemize}

On pourra éventuellement, pour cette recherche, trouver refuge dans l'un
des théorèmes suivants
%
\begin{thm}[règle de d'Alembert]
 On suppose que $|a_n| > 0$ . Si la suite
   $\frac{||a_{n+1}||}{||a_n||}$ admet
 une limite  $l \in \mathbb{R}^{+}  \cup\ +  \infty$ alors le  rayon de
 convergence de la série entière $\sum a_n z^n$ est $\frac{1}{ l}$ .

\end{thm}

%Démonstration Il suffit d'appliquer la règle de d'Alembert pour les
%séries numériques en remarquant que 
%\|an+1z^n+1\|
%\over
%\|anz^n\|
%admet la limite \ell|z| et que donc la série converge
%absolument pour \ell|z| \leq 1 et diverge pour
%\ell|z| \textgreater{} 1.
%
%Remarque~11.1.4 On prendra garde à la condition
%\forall~~n \in \mathbb{N}~,
%an\neq~0. En particulier, on ne tentera
%pas d'appliquer cette règle à des séries comportant une infinité de
%termes nuls comme les séries entières du type
%\sum ~
%anz^n^2 ~; c'est ainsi qu'une
%application imprudente de la règle de d'Alembert à la série entière
%\sum ~
%3^nz^2n pourrait faire croire que le rayon de
%convergence est  1 \over 3 alors que l'écriture
%|3^nz^2n| =
%\left
%(3|z|^2\right )^n
%montre qu'il vaut  1 \over \sqrt3
%.
%
%Exemple~11.1.2 Pour \\sum
%  z^n \over n! , on a R = +\infty~~; pour
%\sum   z^n~
%\over n^\alpha~ , on a R = 1~; pour
%\sum  n!z^n~,
%on a R = 0 (la série diverge pour tout z\neq~0).
%
\begin{thm}[règle d'Hadamard]
  Le rayon de convergence de la série entière $\sum a_n z^n$ est égal à
\[
\frac{1}{\limsup ||a_n||^{1 \over n}} \in \mathbb{R}^+ \cup\ + \infty~\
\]
\end{thm}

\begin{rem}
  En particulier si $||a_n||^{1 \over n}\rightarrow l$ alors $R=\frac{1}{l}$
\end{rem}
%Démonstration Posons \ell =\
%limsup\rootn\of\|an\|
%\in \mathbb{R}~^+ \cup\ + \infty~\. Soit z \in K
%tel que |z| \leq 1 \over \ell .
%On a alors \ell \leq 1 \over
%|z| . Soit donc \rho tel que \ell \leq \rho
%\leq 1 \over |z| . D'après
%la propriété de la limite supérieure, il existe N \in \mathbb{N}~ tel que n \geq N
%\rigtharrow~\rootn\of\|an\|
%\leq \rho soit encore
%\|an\| \leq
%\rho^n et donc
%\|anz^n\|
%\leq (\rho|z|)^n. Mais \rho|z|
%(\rho|z|)^n converge. On en déduit que la
%série \sum ~
%anz^n converge absolument, soit R \geq 1
%\over \ell . De plus, si |z|
%\textgreater{} 1 \over \ell , on a \ell \textgreater{} 1
%\over |z| . Comme \ell est valeur
%d'adhérence de la suite
%(\rootn\of\|an\|),
%il existe une infinité de n tels que
%\rootn\of\|an\|
%\textgreater{} 1 \over |z| soit
%\|anz^n\|
%\textgreater{} 1~; la suite (anz^n) ne tend pas
%vers 0, donc la série diverge~; ceci montre que R \leq 1
%\over \ell , ce qui achève la démonstration.
%
%Remarque~11.1.5 En particulier, si la suite
%(\rootn\of\|an\|)
%converge vers \ell, on a R = 1 \over \ell .
%
\paragraph{Opérations sur les séries entières}
~
%Proposition~11.1.5 Soit
%\sum ~
%anz^n et
%\sum ~
%bnz^n deux séries entières à coefficients dans E de
%rayons de convergence respectifs R1 et R2, \alpha~ et \beta~
%des scalaires. Alors la série entière
%\sum  (\alpha~an~ +
%\beta~bn)z^n a un rayon de convergence supérieur ou égal
%à min(R1,R2~) et
%
%|z| \leq\
%min(R1,R2) \rigtharrow~\sum
%n=0^+\infty~(\alpha~a n + \beta~bn)z^n =
%\alpha~\sum n=0^+\infty~a~
%nz^n + \beta~\sum
%n=0^+\infty~b nz^n
%
%Démonstration En effet, si |z|
%\leq min(R1,R2~),
%les deux séries \sum ~
%anz^n et
%\sum ~
%bnz^n sont convergentes, et donc la série
%\sum  (\alpha~an~ +
%\beta~bn)z^n converge également, soit R
%\geq min(R1,R2~). La formule
%découle immédiatement du résultat similaire sur les séries numériques.
%
%Remarque~11.1.6 L'exemple bn = -an, \alpha~ = \beta~ = 1,
%montre que l'on peut avoir R \textgreater{}\
%min(R1,R2)
%
%Proposition~11.1.6 Soit
%\sum ~
%anz^n et
%\sum ~
%bnz^n deux séries entières à coefficients dans K de
%rayons de convergence respectifs R1 et R2. Posons
%cn = \sum ~
%k=0^nakbn-k
%= \sum ~
%p+q=napbq (série entière produit). Alors la
%série entière \sum ~
%cnz^n a un rayon de convergence supérieur ou égal à
%min(R1,R2~) et
%
%|z| \leq\
%min(R1,R2) \rigtharrow~\sum
%n=0^+\infty~c nz^n =
%\left (\sum
%n=0^+\infty~a nz^n\right
%)\left (\sum
%n=0^+\infty~b nz^n\right
%)
%
%Démonstration En effet, si |z|
%\leq min(R1,R2~),
%les deux séries \sum ~
%anz^n et
%\sum ~
%bnz^n sont absolument convergentes, et donc la
%série produit de Cauchy est également absolument convergente. Mais on a
%\sum ~
%p+q=n(apz^p)(bqz^q) =
%z^n \sum ~
%p+q=napbq = cnz^n. On a
%donc R \geq min(R1,R2~). La
%formule découle immédiatement du résultat similaire sur les séries
%numériques (la somme du produit de Cauchy est le produit des sommes des
%séries).
%
%Proposition~11.1.7 Soit
%\sum ~
%anz^n une série entière à coefficients dans K avec
%a0\neq~0. Il existe alors une unique
%série entière \sum ~
%bnz^n tel que le produit des deux séries entières
%soit la constante 1. Si
%\sum ~
%anz^n a un rayon de convergence non nul R, il en
%est de même du rayon de convergence R' de la série entière
%\sum ~
%bnz^n. On a
%
%|z| \leq min~(R,R')
%\rigtharrow~\sum n=0^+\infty~b~
%nz^n = 1 \over \sum
%n=0^+\infty~anz^n
%
%Démonstration On doit avoir a0b0 = 1 et pour n \geq 1,
%\sum ~
%k=0^nan-kbk = 0. La suite
%(bn) est donc définie par b0 = 1
%\over a0 et pour n \geq 1, bn = - 1
%\over a0 \
%\sum ~
%k=0^n-1an-kbk ce qui définit
%parfaitement la suite par récurrence. Remarquons que si l'on multiplie
%tous les an par \lambda~\neq~0, tous les
%bn sont divisés par \lambda~, et les rayons de convergence ne sont
%pas modifiés. Sans nuire à la généralité, on peut donc supposer que
%a0 = 1. On a alors b0 = 1 et bn =
%-\sum ~
%k=0^n-1an-kbk. Supposons donc R
%\textgreater{} 0 et soit r \leq R. La suite
%(anr^n) est donc bornée. Soit M tel que
%\forall~~n,
%|an|r^n \leq M soit
%|an|\leq Mr^-n. On va montrer que
%\forall~n \geq 1, |bn~|\leq
%M(M + 1)^n-1r^-n par récurrence sur n. Pour n = 1,
%on a b1 = -a1, soit |b1|
%= |a1|\leq Mr^-1 ce qui est bien
%l'inégalité souhaitée. Supposons l'inégalité vérifiée de 1 à n - 1. On a
%alors (compte tenu de b0 = 1)
%
%\begin{align*} |bn|&
%\leq& |an| + \sum
%k=1^n-1|a
%n-k||bk| \%&
%\\ & \leq& Mr^-n +
%\sum k=1^n-1Mr^k-n~M(M
%+ 1)^k-1r^-k \%& \\
%& =& Mr^-n\left (1 +
%M\sum k=1^n-1~(M +
%1)^k-1\right ) \%&
%\\ & =&
%Mr^-n\left (1 + M (M + 1)^n-1 - 1
%\over (M + 1) - 1 \right ) = M(M +
%1)^n-1r^-n\%& \\
%\end{align*}
%
%ce qui achève la récurrence. Alors
%|bnz^n|\leq M \over
%M+1 \left ( (M+1)|z|
%\over r \right )^n ce qui
%montre que la série \\sum
% bnz^n converge pour |z|
%\leq r \over M+1 , soit R' \geq r
%\over M+1 \textgreater{} 0.
%
%La formule découle immédiatement de la proposition précédente.
%
%Remarque~11.1.7 L'exemple a0 = 1, a1 = -1,
%an = 0 pour n \geq 2 (c'est-à-dire de la série entière 1 - z),
%pour laquelle la série inverse est la série
%\sum  z^n~
%pour laquelle R' = 1, montre qu'on ne peut pas dire grand chose de la
%valeur effective de R'. On peut avoir aussi bien R' \leq R que R' \geq R
%(échanger le rôle de \\sum
% anz^n et
%\sum ~
%bnz^n).
%
%{[}
%{[}
