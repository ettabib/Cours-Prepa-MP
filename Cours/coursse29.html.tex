\textbf{Warning: 
requires JavaScript to process the mathematics on this page.\\ If your
browser supports JavaScript, be sure it is enabled.}

\begin{center}\rule{3in}{0.4pt}\end{center}

{[}
{[}
{[}{]}
{[}

\subsubsection{5.3 Espaces vectoriels normés de dimensions finies}

\paragraph{5.3.1 Equivalence des normes}

Lemme~5.3.1 Toutes les normes sur \mathbb{R}~^n sont équivalentes.

Démonstration Posons
\\textbar{}x\\textbar{}
= max\textbar{}xi~\textbar{} et
montrons que toute autre norme N est équivalente à cette norme. Soit
(e1,\\ldotsen~)
la base canonique de \mathbb{R}~^n et x \in \mathbb{R}~^n. On a N(x) =
N(\\sum ~
xiei)
\leq\\sum ~
\textbar{}xi\textbar{}N(ei)
\leq\
max\textbar{}xi\textbar{}\\\sum
 iN(ei) =
\beta~\\textbar{}x\\textbar{}. On en déduit que
\textbar{}N(x) - N(y)\textbar{}\leq N(x - y) \leq \beta~\\textbar{}x
- y\\textbar{} ce qui démontre que l'application N :
(\mathbb{R}~^n,\\textbar{}.\\textbar{}) \rightarrow~
\mathbb{R}~ est continue. Soit S = \x \in
\mathbb{R}~^n∣\\textbar{}x\\textbar{}
= 1\~; S est une partie compacte de
(\mathbb{R}~^n,\\textbar{}.\\textbar{})
(fermée bornée), donc l'application N y atteint sa borne inférieure.
Soit \alpha~ = inf x\inS~N(x) =
N(x0). On a x0\neq~0 (car
x0 \in S) donc \alpha~ \textgreater{} 0. Alors, si x \in \mathbb{R}~^n,
x\neq~0, on a  x \over
\\textbar{}x\\textbar{} \in S soit N( x
\over
\\textbar{}x\\textbar{} ) ≥ \alpha~ soit
encore N(x) ≥ \alpha~\\textbar{}x\\textbar{}. On
a donc trouvé \alpha~ et \beta~ strictement positifs tels que
\forall~x \in \mathbb{R}~^n~,
\alpha~\\textbar{}x\\textbar{} \leq N(x) \leq
\beta~\\textbar{}x\\textbar{}, ce qu'il fallait
démontrer.

Théorème~5.3.2 Sur un K-espace vectoriel normé~de dimension finie toutes
les normes sont équivalentes.

Démonstration Tout \mathbb{C}-espace vectoriel normé~étant aussi un \mathbb{R}~-espace
vectoriel normé, il suffit de le montrer lorsque le corps de base est \mathbb{R}~.
Soit N1 et N2 deux normes sur E~; soit \mathcal{E} =
(e1,\\ldots,en~)
une base de E et u : \mathbb{R}~^n \rightarrow~ E définie par
u(x1,\\ldots,xn~)
= \\sum ~
xiei (u est un isomorphisme d'espaces vectoriels).
Alors N1 \cdot u et N2 \cdot u sont deux normes sur
\mathbb{R}~^n (facile), elles sont donc équivalentes, et donc il existe
\alpha~ et \beta~ strictement positifs tels que \forall~~x \in
\mathbb{R}~^n, \alpha~N1(u(x)) \leq N2(u(x)) \leq
\beta~N1(u(x)). Mais tout élément de E s'écrivant sous la forme
u(x), on a, \forall~y \in E, \alpha~N1~(y) \leq
N2(y) \leq \beta~N1(y), ce qu'il fallait démontrer.

\paragraph{5.3.2 Propriétés topologiques et métriques des espaces
vectoriels normés de dimension finie}

Remarque~5.3.1 Tout \mathbb{C}-espace vectoriel normé~étant aussi un \mathbb{R}~-espace
vectoriel normé, il suffit de considérer le cas où le corps de base est
\mathbb{R}~. Soit (E,\\textbar{}.\\textbar{}) un
espace vectoriel normé~de dimension finie, soit \mathcal{E} =
(e1,\\ldots,en~)
une base de E et u : \mathbb{R}~^n \rightarrow~ E définie par
u(x1,\\ldots,xn~)
= \\sum ~
xiei (u est un isomorphisme d'espaces vectoriels).
Alors N :
x\mapsto~\\textbar{}u(x)\\textbar{}
est une norme sur \mathbb{R}~^n qui est équivalente à la norme
\\textbar{}.\\textbar{}\infty~~; de
plus l'application u : (\mathbb{R}~^n,N) \rightarrow~
(E,\\textbar{}.\\textbar{}) est une
isométrie~; on en déduit que
(E,\\textbar{}.\\textbar{}) a, en tant
qu'espace vectoriel normé, les mêmes propriétés que
(\mathbb{R}~^n,\\textbar{}.\\textbar{}\infty~)
c'est-à-dire

Théorème~5.3.3 Tout espace vectoriel normé de dimension finie est
complet~; les parties compactes en sont les fermés bornés.

Corollaire~5.3.4 Tout sous-espace vectoriel de dimension finie d'un
espace vectoriel normé~est fermé.

Démonstration Muni de la restriction de la norme, il est complet, donc
fermé.

\paragraph{5.3.3 Continuité des applications linéaires}

Théorème~5.3.5 Soit E et F deux espaces vectoriels normés, E étant
supposé de dimension finie. Alors toute application linéaire de E dans F
est continue.

Démonstration Soit \mathcal{E} =
(e1,\\ldots,en~)
une base de E~; comme toute les normes sur E sont équivalentes, on peut
prendre la norme définie par
\\textbar{}x\\textbar{}
= sup\textbar{}xi~\textbar{} si x
= \\sum ~
xiei. On a alors

\begin{align*}
\\textbar{}u(x)\\textbar{}& =&
\\textbar{}\\sum
xiu(ei)\\textbar{}
\leq\\sum
\textbar{}xi\textbar{}\,\\textbar{}u(ei)\\textbar{}\%&
\\ & \leq&
\\textbar{}x\\textbar{}\\sum
\\textbar{}u(ei)\\textbar{} =
K\\textbar{}x\\textbar{} \%&
\\ \end{align*}

Ceci montre la continuité de l'application linéaire u.

Remarque~5.3.2 Ce résultat s'étend sans difficulté aux applications
bilinéaires de E1 \times E2 dans F à condition que
E1 et E2 soient de dimensions finies~; de même pour
des applications p-linéaires.

{[}
{[}
{[}
{[}
