\textbf{Warning: 
requires JavaScript to process the mathematics on this page.\\ If your
browser supports JavaScript, be sure it is enabled.}

\begin{center}\rule{3in}{0.4pt}\end{center}

{[}
{[}
{[}{]}
{[}

\subsubsection{4.6 Continuité uniforme}

\paragraph{4.6.1 Applications uniformément continues}

La continuité de f : E \rightarrow~ F s'exprime par

\begin{align*} \forall~~a \in
E,\forall~~\epsilon \textgreater{} 0,
\exists~\eta(a,\epsilon) \textgreater{} 0,&& \%&
\\ & & d(x,a) \textless{} \eta(a,\epsilon) \rigtharrow~
d(f(x),f(a)) \textless{} \epsilon\%& \\
\end{align*}

Remarque~4.6.1 En général, \eta dépend de \epsilon mais aussi de a. On dira que f
est uniformément continue sur E si on peut choisir un \eta ne dépendant pas
de a. Ceci se traduit par

Définition~4.6.1 Soit E et F deux espaces métriques. On dit que f : E \rightarrow~
F est uniformément continue si on a

\begin{align*} \forall~~\epsilon
\textgreater{} 0,\exists~\eta \textgreater{}
0,\quad \forall~~x,x' \in E,&& \%&
\\ & & d(x,x') \textless{} \eta \rigtharrow~
d(f(x),f(x')) \textless{} \epsilon\%& \\
\end{align*}

Remarque~4.6.2 Toute application uniformément continue est donc
continue. Il s'agit d'une notion métrique et non topologique (elle ne
peut pas se traduire en termes d'ouverts).

Proposition~4.6.1 La composée de deux applications uniformément
continues est uniformément continue.

Démonstration Evident.

Remarque~4.6.3 Le lemme suivant peut rendre des services pour montrer
que certaines applications ne sont pas uniformément continues~:

Lemme~4.6.2 Soit E et F deux espaces métriques et f : E \rightarrow~ F. Alors f est
uniformément continue si et seulement si~pour tout couple de suites
(an),(bn) de points de E tels que
limd(an,bn~) = 0, on a
limd(f(an),f(bn~)) = 0.

Démonstration (i) \rigtharrow~(ii) Soit \epsilon \textgreater{} 0. Alors
\exists~\eta \textgreater{} 0,\quad
\forall~~x,x' \in E, d(x,x') \textless{} \eta \rigtharrow~
d(f(x),f(x')) \textless{} \epsilon. Pour ce \eta, il existe N \in \mathbb{N}~ tel que n ≥ N \rigtharrow~
d(an,bn) \textless{} \eta. Alors n ≥ N \rigtharrow~
d(f(an),f(bn)) \textless{} \epsilon ce qui montre que
limd(f(an),f(bn~)) = 0

(ii) \rigtharrow~(i) Nous allons montrer la contraposée. Supposons f non
uniformément continue. Alors

\exists~\epsilon \textgreater{} 0,
\forall~~\eta \textgreater{} 0,\quad
\exists~a,b \in E, d(a,b) \textless{}
\eta\text et d(f(a),f(b)) ≥ \epsilon

en prenant \eta = 1 \over n+1 , on trouve an
et bn tels que d(an,bn) \textless{} 1
\over n+1 alors que d(f(an),f(bn))
≥ \epsilon, et donc (ii) n'est pas vérifiée.

Exemple~4.6.1 L'application f : \mathbb{R}~ \rightarrow~ \mathbb{R}~,
x\mapsto~x^2 est continue, mais par
uniformément continue~: pour an = n,bn = n + 1
\over n , on a
lim\textbar{}an~ -
bn\textbar{} = 0, mais
lim\textbar{}an^2~ -
bn^2\textbar{} = 2.

\paragraph{4.6.2 Applications lipschitziennes}

Définition~4.6.2 Soit E et F deux espaces métriques. On dit que f : E \rightarrow~
F est lipschitzienne de rapport k ≥ 0 si

\forall~~x,x' \in E,\quad d(f(x),f(x')) \leq
kd(x,x')

Théorème~4.6.3 Toute application lipschitzienne est uniformément
continue.

Démonstration Si k = 0, f est constante et sinon

d(x,x') \textless{} \epsilon \over k \rigtharrow~ d(f(x),f(x'))
\textless{} \epsilon

Remarque~4.6.4 On montrera souvent qu'une application est lipschitzienne
par application d'un théorème des accroissements finis.

{[}
{[}
{[}
{[}
