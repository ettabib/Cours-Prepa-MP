\textbf{Warning: 
requires JavaScript to process the mathematics on this page.\\ If your
browser supports JavaScript, be sure it is enabled.}

\begin{center}\rule{3in}{0.4pt}\end{center}

{[}
{[}{]}
{[}

\subsubsection{6.1 Relations de comparaison}

\paragraph{6.1.1 Notations}

Soit A \subset~ \mathbb{R}~ et a \in\overline\mathbb{R}~ tel que a
\in\overlineA (adhérence dans
\overline\mathbb{R}~). Si E est un espace vectoriel normé, on
notera ℱa,A(E) l'ensemble des fonctions de \mathbb{R}~ vers E telles
qu'il existe V \in V (a) tel que f soit définie sur V \bigcap A (autrement dit f
est définie sur A au voisinage de a).

Exemple~6.1.1 Si A = \mathbb{N}~ et a = +\infty~, ℱa,A(E) est l'ensemble des
suites (xn)n≥n0 d'éléments de E.

\paragraph{6.1.2 Domination, prépondérance}

Définition~6.1.1 Soit f \inℱa,A(E) et g \inℱa,A(F).

\begin{itemize}
\item
  (i) On dit que f est dominée par g au voisinage de a suivant A et on
  note f = O(g) si

  \existsK ≥ 0, \\exists~V \in V
  (a), \forall~~t \in V \bigcap A,
  \\textbar{}f(t)\\textbar{} \leq
  K\\textbar{}g(t)\\textbar{}
\item
  (ii) On dit que g est prépondérante devant f (ou que f est négligeable
  devant g) au voisinage de a suivant A et on note f = o(g) si

  \forall~~\epsilon \textgreater{} 0,
  \existsV \in V (a), \\forall~~t \in V
  \bigcap A, \\textbar{}f(t)\\textbar{} \leq
  \epsilon\\textbar{}g(t)\\textbar{}
\end{itemize}

Remarque~6.1.1 Il est clair que f = o(g) \rigtharrow~ f = O(g). De plus, si f =
O(g) et si X est choisi comme ci dessus, on voit que g(t) = 0 \rigtharrow~ f(t) =
0~; on constate donc que

\begin{itemize}
\itemsep1pt\parskip0pt\parsep0pt
\item
  (i) aux indéterminations près de type  0 \over 0 ,
  f = O(g) \Leftrightarrow
  \\textbar{}f\\textbar{}
  \over
  \\textbar{}g\\textbar{} est bornée (au
  voisinage de a suivant A)
\item
  (ii) aux indéterminations près de type  0 \over 0 ,
  f = o(g) \Leftrightarrow
  limt\rightarrow~a,t\inA~
  \\textbar{}f(t)\\textbar{}
  \over
  \\textbar{}g(t)\\textbar{} = 0
\end{itemize}

Exemple~6.1.2

\begin{itemize}
\itemsep1pt\parskip0pt\parsep0pt
\item
  f = O(1) \Leftrightarrow f est bornée au voisinage de a
  suivant A~;
\item
  f = o(1) \Leftrightarrow
  limt\rightarrow~a,t\inA~f(t) = 0.
\end{itemize}

Proposition~6.1.1

\begin{itemize}
\item
  (i) f1 = O(g)\text et f2 = O(g)
  \rigtharrow~ \alpha~f1 + \muf2 = O(g)
\item
  (ii) f1 = o(g)\text et f2 =
  o(g) \rigtharrow~ \alpha~f1 + \muf2 = o(g)
\item
  (iii) soit \phi,\psi \inℱa,A(K) et f,g \inℱa,A(E), alors

  \begin{align*} \phi = O(\psi)\text et
  f = O(g)& \rigtharrow~& \phif = O(\psig)\%& \\ \phi =
  o(\psi)\text et f = O(g)& \rigtharrow~& \phif = o(\psig) \%&
  \\ \phi = O(\psi)\text et f
  = o(g)& \rigtharrow~& \phif = o(\psig) \%& \\
  \end{align*}
\item
  (iv) f \inℱa,A(E),g \inℱa,A(F),h \inℱa,A(G)~;
  alors

  \begin{align*} f = O(g)\text et
  g = O(h)& \rigtharrow~& f = O(h)\%& \\ f =
  O(g)\text et g = o(h)& \rigtharrow~& f = o(h) \%&
  \\ f = o(g)\text et g
  = O(h)& \rigtharrow~& f = o(h) \%& \\
  \end{align*}
\end{itemize}

Démonstration Facile

\paragraph{6.1.3 Equivalence}

Lemme~6.1.2 Soit f,g \inℱa,A(E). Alors f - g = o(g) \rigtharrow~ g = O(f).

Démonstration Il existe V \in V (a) tel que \forall~~t \in
V \bigcap A, \\textbar{}f(t) - g(t)\\textbar{}
\leq 1 \over 2
\\textbar{}g(t)\\textbar{}. Pour t \in V \bigcap
A, on a donc \\textbar{}g(t)\\textbar{}
=\\textbar{} g(t) - f(t) + f(t)\\textbar{}
\leq\\textbar{} g(t) - f(t)\\textbar{}
+\\textbar{} f(t)\\textbar{} \leq 1
\over 2
\\textbar{}g(t)\\textbar{}
+\\textbar{} f(t)\\textbar{} soit encore
\\textbar{}g(t)\\textbar{} \leq
2\\textbar{}f(t)\\textbar{}, et donc g =
O(f).

Théorème~6.1.3 Pour f,g \inℱa,A(E), on pose f ∼ g si f - g =
o(g). Il s'agit d'une relation d'équivalence appelée l'équivalence des
fonctions (au voisinage de a suivant A).

Démonstration La réflexivité est claire puisque f - f = 0 = o(f). Si f ∼
g, on a f - g = o(g) et aussi d'après le lemme, g = O(f), d'où f - g =
o(f) et donc aussi g - f = o(f), soit g ∼ f. La relation est donc
symétrique. Si f ∼ g et g ∼ h, on a f - g = o(g) et g - h = o(h). Mais
on a h ∼ g, soit h - g = o(g) soit g = O(h). Alors f - g = o(g) et g =
O(h) donne f - g = o(h) et donc f - h = (f - g) + (g - h) = o(h), d'où f
∼ h, ce qui démontre la transitivité.

Proposition~6.1.4

\begin{itemize}
\itemsep1pt\parskip0pt\parsep0pt
\item
  (i) f ∼ g \rigtharrow~ f = O(g)\text et g = O(f)
\item
  (ii) \phi,\psi \inℱa,A(K), f,g \inℱa,A(E), alors \phi ∼
  \psi\text et f ∼ g \rigtharrow~ \phif ∼ \psig
\end{itemize}

Démonstration (i) est évident d'après le lemme ci dessus et la symétrie
de la relation. Pour (ii), on écrit \phif - \psig = (\phi - \psi)f + \psi(f - g). On a
\phi - \psi = o(\psi)\text et f = O(g) \rigtharrow~ (\phi - \psi)f = o(\psig) et f
- g = o(g) \rigtharrow~ \psi(f - g) = o(\psig), d'où \phif - \psig = o(\psig) et \phif ∼ \psig.

Remarque~6.1.2 La relation d'équivalence est donc compatible avec la
multiplication~; par contre, elle n'est pas compatible avec l'addition~:
f1 ∼ g1\text et f2 ∼
g2\rigtharrow~̸f1 + f2 ∼ g1 + g2
comme le montre l'exemple a = 0, f1(t) = 1 + t,g1(t)
= 1 + t^2,f2(t) = g2(t) = -1~; on a
f1 ∼ g1\text et f2 =
g2, pourtant f1(t) + f2(t) = t et
g1(t) + g2(t) = t^2 ne sont pas
équivalentes au voisinage de 0.

Lemme~6.1.5 Soit f,g \inℱa,A(K). Alors on a équivalence de

\begin{itemize}
\itemsep1pt\parskip0pt\parsep0pt
\item
  (i) f ∼ g
\item
  (ii) il existe \phi \inℱa,A(K) telle que f = g\phi et
  limt\rightarrow~a,t\inA~\phi(t) = 1
\end{itemize}

Démonstration (ii) \rigtharrow~(i). On écrit f - g = g(\phi - 1) avec
limt\rightarrow~a,t\inA~(\phi(t) - 1) = 0, d'où f - g
= o(g) et f ∼ g.

(i) \rigtharrow~(ii). On a f = O(g). D'après une remarque précédente, il existe V \in
V (a) tel que \forall~~t \in V \bigcap A, g(t) = 0 \rigtharrow~ f(t) = 0.
Définissons \phi sur V \bigcap A de la manière suivante~: \phi(t) =
\left \ \cases  f(t)
\over g(t) &si g(t)\neq~0
\cr 1 &si g(t) = 0  \right .~; si
g(t)\neq~0, on a f(t) = \phi(t)g(t) de manière
évidente et cela reste vrai si g(t) = 0 puisque alors on a aussi f(t) =
0. Montrons que limt\rightarrow~a,t\inA~\phi(t) = 1.
Soit \epsilon \textgreater{} 0~; il existe V 0 \in V (a) tel que
\forall~t \in V 0~ \bigcap A, \textbar{}f(t) -
g(t)\textbar{}\leq \epsilon\textbar{}g(t)\textbar{} soit encore pour t \in V
0 \bigcap V \bigcap A, \textbar{}\phi(t) -
1\textbar{}\,\textbar{}g(t)\textbar{}\leq
\epsilon\textbar{}g(t)\textbar{}. Si g(t)\neq~0 on a
donc \textbar{}\phi(t) - 1\textbar{}\leq \epsilon mais cela reste vrai si g(t) = 0
puisqu'alors \phi(t) = 1. On a donc bien
limt\rightarrow~a,t\inA~\phi(t) = 1.

Théorème~6.1.6

\begin{itemize}
\item
  (i) si f,g \inℱa,A(K) et n \in \mathbb{N}~, alors f ∼ g \rigtharrow~ f^n ∼
  g^n
\item
  (ii) si f,g \inℱa,A(\mathbb{R}~) et s'il existe V \in V (a) tel que
  \forall~~t \in V \bigcap A, g(t) ≥ 0 (resp. \textgreater{}
  0) alors

  \forall~\alpha~ \in \mathbb{R}~^+~\text
  (resp. \$\forall~\alpha~ \in \mathbb{R}~\$) f ∼ g \rigtharrow~ f^\alpha~~
  ∼ g^\alpha~
\end{itemize}

Démonstration Résulte immédiatement du lemme précédent en remarquant
pour (ii) que si \phi tend vers 1, elle est strictement positive au
voisinage de a et que lim\phi^\alpha~~ = 1.

Remarque~6.1.3 La relation d'équivalence est donc compatible avec les
puissances entières ou réelles~; par contre elle n'est pas compatible
avec l'exponentielle~: en fait on a e^f ∼ e^g
\Leftrightarrow lim~(f - g) = 0.

Le théorème suivant \\\\jmathmathmathmathustifie l'intérêt de l'utilisation des équivalents
pour les recherches de limites

Théorème~6.1.7 Soit f,g \inℱa,A(E) telles que f ∼ g. Si g admet
une limite \ell en a suivant A, f admet la même limite en a suivant A.

Démonstration Puisque limt\rightarrow~a,t\inA~g(t)
= \ell, il existe V 0 \in V (a) tel que t \in V 0 \bigcap A
\rigtharrow~\\textbar{} g(t) - \ell\\textbar{}
\textless{} 1 soit
\\textbar{}g(t)\\textbar{} \leq 1
+\\textbar{} \ell\\textbar{}. Soit alors \epsilon
\textgreater{} 0. Il existe V \in V (a) tel que
\forall~~t \in V \bigcap A, \\textbar{}g(t) -
\ell\\textbar{} \textless{} \epsilon \over 2 et
il existe V ' \in V (a) tel que \forall~~t \in V ' \bigcap A,
\\textbar{}f(t) - g(t)\\textbar{} \leq \epsilon
\over
2(1+\\textbar{}\ell\\textbar{})
\\textbar{}g(t)\\textbar{}. Pour t \in V
0 \bigcap V \bigcap V ' \bigcap A on a

\begin{align*} \\textbar{}f(t) -
\ell\\textbar{}& \leq& \\textbar{}f(t) -
g(t)\\textbar{} +\\textbar{} g(t) -
\ell\\textbar{}\%& \\ &
\leq& \epsilon \over 2(1 +\\textbar{}
\ell\\textbar{}) (1 +\\textbar{}
\ell\\textbar{}) + \epsilon \over 2 \%&
\\ & =& \epsilon \%&
\\ \end{align*}

et donc f admet \ell pour limite en a suivant A.

\paragraph{6.1.4 Changement de variables}

Soit A,B \subset~ \mathbb{R}~ et a,b \in\overline\mathbb{R}~ tel que a
\in\overlineA et b \in\overlineB.

Soit \phi une fonction de \mathbb{R}~ vers \mathbb{R}~ telle que \phi(A) \subset~ B et
limt\rightarrow~a,t\inA~\phi(t) = b. Par définition
de la notion de limite on a aussitôt

Lemme~6.1.8 \forall~~V `\in V (b),
\exists~V \in V (a), \phi(V \bigcap A) \subset~ V' \bigcap B.

Il en découle immédiatement le théorème suivant

Théorème~6.1.9 Soit f,g \inℱb,B(E). Alors

\begin{itemize}
\itemsep1pt\parskip0pt\parsep0pt
\item
  (i) f = Ob,B(g) \rigtharrow~ f \cdot \phi = Oa,A(g \cdot \phi)
\item
  (i) f = ob,B(g) \rigtharrow~ f \cdot \phi = oa,A(g \cdot \phi)
\item
  (i) f ∼b,Bg \rigtharrow~ f \cdot \phi ∼a,Ag \cdot \phi
\end{itemize}

autrement dit on peut faire tout changement de variable raisonnable dans
des relations de comparaison.

{[}
{[}
