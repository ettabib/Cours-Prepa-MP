
\chapter{Calcul différentiel}
\section{Dérivées partielles}
%
\paragraph{Notion de dérivée partielle}
~ 
%Définition~15.1.1 Soit E et F deux espaces vectoriels normés. Soit U un
%ouvert de E, f : U \longrightarrow F, a \in U. Soit v \in E
%∖\0\. On dit que f admet au point a
%une dérivée partielle suivant le vecteur v si l'application
%t\mathrel↦f(a + tv) (définie sur un voisinage de 0)
%est dérivable au point 0.
%
%Remarque~15.1.1 L'existence de la dérivée partielle en a suivant le
%vecteur v est donc équivalente à l'existence de
%:gmathoplim}_t\longrightarrow0{ f(a+tv)−f(a)
%\over t} = \partial_vf(a). Remarquons que si v' = λv,
%λ\mathrel≠0, alors { f(a+tv')−f(a)
%\over t} = λ f(a+uv)−f(a) \over u
%avec u = λt ce qui montre que f admet en a une dérivée partielle selon v
%si et seulement si f admet une dérivée partielle suivant λv et qu'alors
%\partial_λv}f(a) = λ{\partial_vf(a).
%
%Exemple~15.1.1 Soit f : \mathbb{R}}\^{}{2 \longrightarrow \mathbb{R} définie par f(x,y) ={
%x}\^{}{2} \over y si
%y\mathrel≠0 et f(x,0) = 0. Soit v =
%(a,b)\mathrel≠(0,0). On a { f((0,0)+tv)−f(0,0)
%\over t} = \left \{
%\cases{ 0 \&si b = 0 \cr {
%a}\^{}{2} \over b \&si
%b\mathrel≠}0  \right .. On en déduit
%que f admet une dérivée partielle suivant tout vecteur v et que
%\partial_vf(0,0) = \left \{
%\cases{ 0 \&si b = 0 \cr {
%a}\^{}{2} \over b \&si
%b\mathrel≠}0  \right .. Remarquons
%que l'application v\mathrel↦}{\partial_vf(0,0) n'est
%pas linéaire. Remarquons également que f n'est pas continue en (0,0)
%(puisque :gmathoplim}_t\longrightarrow0}f(t,{t}\^{}{2) =
%1\mathrel≠f(0,0)). L'existence de dérivée partielle
%suivant tout vecteur n'implique donc pas la continuité.
%
%Proposition~15.1.1 On a les propriétés évidentes de la dérivation de
%t\mathrel↦f(a + tv) à savoir (i) si f et g admettent
%en a une dérivée partielle suivant le vecteur v, il en est de même de αf
%+ βg et \partial_v}(αf + βg)(a) = α{\partial_vf(a) +
%β\partial_vg(a). (ii) si f et g (à valeurs scalaires) admettent en a
%une dérivée partielle suivant le vecteur v, il en est de même de fg et
%\partial_v}(fg)(a) = g(a){\partial_v}f(a) + f(a){\partial_vg(a).
%
%Remarque~15.1.2 Par contre, on n'a pas de théorème général de
%composition des dérivées partielles. En reprenant l'exemple ci dessus, f
%: \mathbb{R}}\^{}{2} \longrightarrow \mathbb{R} définie par f(x,y) ={ {x}\^{}{2
%\over y} si y\mathrel≠0 et f(x,0) =
%0, l'application f admet en (0,0) une dérivée partielle suivant tout
%vecteur, l'application t\mathrel↦}(t,{t}\^{}{2)
%est dérivable en 0 et pourtant
%t\mathrel↦}f(t,{t}\^{}{2) n'est pas dérivable en
%0 (elle n'y est même pas continue).
%
%Définition~15.1.2 Soit E un espace vectoriel normé de dimension finie, ℰ
%=
%(e_1},+{\ldots{}}},{e_n)
%une base de E, F un espace vectoriel normé. Soit U un ouvert de E, f : U
%\longrightarrow F, a \in U. On dit que f admet au point a une dérivée partielle d'indice
%i (suivant la base ℰ) si elle admet une dérivée partielle suivant le
%vecteur e_i. On note alors { \partialf \over
%\partialx_i}} (a) = {\partial_{e_i}f(a).
%
%Exemple~15.1.2 Si E = \mathbb{R}}\^{}{n et si ℰ est la base canonique de
%\mathbb{R}}\^{}{n, l'existence d'une dérivée partielle d'indice i au point
%a =
%(a_1},+{\ldots{}}},{a_n)
%équivaut à la dérivabilité au point a_i de l'application
%partielle
%x_i}\mathrel{↦}f({a_1},+{\ldots{}}},{a_i−1},{x_i},{a_i+1},\mathop{\mathop{\ldots{}}},{a_n).
%On retrouve bien la notion habituelle de dérivée partielle~: dérivée
%suivant la variable x_i, toutes les autres étant considérées
%comme constantes.
%
%\paragraph{15.1.2 Composition des dérivées partielles}
%
%On a vu précédemment qu'on n'avait pas de théorème de composition des
%dérivées partielles en toute généralité. On va introduire une notion
%d'application de classe C}\^{}{1.
%
%Définition~15.1.3 Soit U un ouvert de \mathbb{R}}\^{}{n, f : U \longrightarrow F. On dit
%que f est de classe C}\^{}{1 au point a si, sur un certain
%voisinage V de a, f admet des dérivées partielles de tout indice i \in
%{[}1,n{]} et si ces dérivées partielles x\mathrel↦{
%\partialf \over \partialx_i} (x) sont continues au point a.
%
%Lemme~15.1.2 Soit F un espace vectoriel de dimension finie, V un ouvert
%de \mathbb{R}}\^{}{n, f : V \longrightarrow F. Soit I un intervalle de \mathbb{R}, t \in I et φ =
%(φ_1},+{\ldots{}}},{φ_n)
%: I \longrightarrow V . On suppose que φ est dérivable au point t et que f est de
%classe C}\^{}{1 au point φ(t). Alors f ∘ φ est dérivable au point
%t et (f ∘ φ)'(t) =:gmathop{\sum
%}} }_i=1}\^{}{n}{ \partialf \over \partial{x_i}
%(φ(t))φ_i'(t).
%
%Démonstration Sans nuire à la généralité, en prenant une base (sur \mathbb{R}) de
%F et en travaillant composante par composante, on peut supposer que f
%est à valeurs réelles. On écrit
%
%\begineqnarray* f(φ(t + h)) − f(φ(t))\&\& \%\&
%\\ \& =\& f(φ_1(t +
%h),+\ldots{}}},{φ_n(t
%+ h)) −
%f(φ_1}(t),+{\ldots{}}},{φ_n(t))
%\%\& \\ \& =\& :gmathop{\sum
%}}_i=1}\^{}{n}(f(\mathop{\ldots{}},{φ_{
%i−1}(t),φ_i}(t + h),{φ_i+1(t +
%h),\mathop\ldots{})\%\&
%\\ \& \& \qquad −
%f(+\ldots{}}},{φ_i−1}(t),{φ_i}(t),{φ_i+1(t
%+ h),+\ldots{}})) \%\&
%\\ \endeqnarray*
%
%Mais φ est continue au point t et donc pour h assez petit, tous les
%(φ_1}(t),+{\ldots{}}},{φ_i−1}(t),{φ_i(t
%+ h),φ_i+1(t +
%h),+\ldots{}}},{φ_n(t
%+ h)) se trouvent à l'intérieur d'une boule de centre a sur laquelle les
%dérivées partielles de f de tout indice existent. En particulier,
%l'application
%x_i}\mathrel{↦}f({φ_1}(t),+{\ldots{}}},{φ_i−1}(t),{x_i},{φ_i+1(t
%+
%h),+\ldots{}}},{φ_n(t
%+ h)) est dérivable sur le segment {[}φ_i}(t),{φ_i(t +
%h){]} et on peut appliquer le théorème des accroissements finis. On
%obtient l'existence d'un ξ_i \in
%{[}φ_i}(t),{φ_i(t + h){]} tel que
%
%\begineqnarray*
%f(φ_1}(t),+{\ldots{}}},{φ_i−1}(t),{φ_i(t
%+ h),φ_i+1(t +
%h),+\ldots{}}},{φ_n(t
%+ h))\&\& \%\& \\ \& \&
%−f(φ_1}(t),+{\ldots{}}},{φ_i−1}(t),{φ_i}(t),{φ_i+1(t
%+
%h),+\ldots{}}},{φ_n(t
%+ h))\%\& \\ \& =\& (φ_i(t + h) −
%φ_i(t)) \%\& \\ \& \&
%\quad  \partialf \over \partial{x_i}
%(φ_1}(t),+{\ldots{}}},{φ_i−1}(t),{ξ_i},{φ_i+1(t
%+
%h),+\ldots{}}},{φ_n(t
%+ h)) \%\& \\
%\endeqnarray*
%
%Comme φ est continue au point t et ξ_i \in
%{[}φ_i}(t),{φ_i(t + h){]}, on a
%
%\begineqnarray*
%:gmathoplim}_h\longrightarrow0}({φ_1}(t),\mathop{\mathop{\ldots{}}},{φ_i−1}(t),{ξ_i},{φ_i+1(t
%+
%h),+\ldots{}}},{φ_n(t
%+ h))\&\&\%\& \\ \& =\&
%(φ_1}(t),+{\ldots{}}},{φ_i−1}(t),{φ_i}(t),{φ_i+1}(t),\mathop{\mathop{\ldots{}}},{φ_n(t))\%\&
%\\ \& =\& φ(t) \%\&
%\\ \endeqnarray*
%
%et comme  \partialf \over \partial{x_i} est continue au
%point φ(t), on a
%
%\begineqnarray*{ \partialf \over
%\partialx_i} (φ(t))\&\& \%\& \\ \& =\&
%:gmathoplim}_h\longrightarrow0{ \partialf \over
%\partialx_i}
%(φ_1}(t),+{\ldots{}}},{φ_i−1}(t),{ξ_i},{φ_i+1(t
%+
%h),+\ldots{}}},{φ_n(t
%+ h))\%\& \\
%\endeqnarray*
%
%Il suffit alors de diviser par h et de faire tendre h vers 0 pour voir
%que
%
%:gmathoplim}_h\longrightarrow0{ f(φ(t + h)) − f(φ(t))
%\over h} ={\sum
%}}_i=1}\^{}{n}{ \partialf \over \partial{x_i}
%(φ(t))φ_i'(t)
%
%ce qui achève la démonstration.
%
%Appliquant ce lemme à φ : t\mathrel↦g(a + tv) au
%point t = 0, on obtient le théorème suivant
%
\begin{thm}
Soit F un espace vectoriel de dimension finie, U un
ouvert de $\mathbb{R}^{n}, f : U \longrightarrow F$. Soit E un espace vectoriel normé, V
un ouvert de E et $g = (g_1},\ldots{},g_n) : V \longrightarrow U \subset \mathbb{R}^n$. Soit $a \in V$ et $v \in E-{0}$\\
 Si g admet en a une dérivée
partielle suivant le vecteur $v$ et si $f$ est de classe $C^1$ au
point a, alors $f \circ g$ admet en a une dérivée partielle suivant le vecteur
v et on a
\[
\partial_v(f \circ g)(a) =\sum_{i=1}^{n}{ \partial f \over \partial{x_i}}
(g(a)) \partial_v g_i(a)
\]
\end{thm}
Dans le cas particulier où $E = \mathbb{R}^p $et où on prend pour v le
j-ième vecteur de la base canonique, on obtient la version suivante (on
a changé le nom des variables pour les appeler
$y_1,+\ldots,y_n$
dans $\mathbb{R}^n$.
%
\begin{cor}
  Soit F un espace vectoriel de dimension finie, U un
  ouvert de $\mathbb{R}^{n},$ $f : U \longrightarrow F.$ Soit V un ouvert de
  $\mathbb{R}^{p}$
  et $g =
(g_1,\ldots,g_n)
  : V \longrightarrow U \subset \mathbb{R}^{n}.$ Soit $a \in V $et $j \in
  [1,p].$ Si g admet en a
une dérivée partielle d'indice j et si f est de classe C}\^{}{1 au
point a, alors $f ∘ g $ admet en a une dérivée partielle d'indice j et on a

\[ \frac{\partial (f \circ g)}{\partial x_j} (a) =
  \sum_{i=1}^n \frac{ \partial f}{\partial y_i} (g(a)) \frac{\partial g_i}{\partial x_j} (a)
\]
\end{cor}
%Remarque~15.1.3 On en déduit immédiatement que la composée de deux
%applications de classe C}\^{}{1 est encore de classe
%C}\^{}{1.
%
%Citons aussi le corollaire suivant du lemme, où l'on prend φ(t) = a + tv
%
%Corollaire~15.1.5 Soit F un espace vectoriel de dimension finie, V un
%ouvert de \mathbb{R}}\^{}{n, f : V \longrightarrow F. Soit a \in V . Si f est de classe
%C}\^{}{1 au point a, alors elle admet en a des dérivées partielles
%suivant tout vecteur et on a
%
%\partial_vf(a) ={\sum
%}}_i=1}\^{}{n}{v_ i{ \partialf \over
%\partialx_i}} (a)\qquad \text{ si v =
%(v_1},\mathop{\ldots{}},{v_n)
%
%Remarque~15.1.4 On voit que dans ce cas
%v\mathrel↦}{\partial_vf(a) est linéaire. Cette
%remarque nous conduira à la définition de la différentielle dans la
%section suivante.
%
%\paragraph{15.1.3 Théorème des accroissements finis et applications}
%
%Théorème~15.1.6 Soit U un ouvert de \mathbb{R}}\^{}{n, f : U \longrightarrow \mathbb{R} de classe
%C}\^{}{1}. Soit a \in U et h \in {\mathbb{R}}\^{}{n tel que {[}a,a + h{]} \subset
%U. Alors, il existe \theta \in{]}0,1{[} tel que
%
%f(a + h) − f(a) ={\sum
%}}_i=1}\^{}{n}{h_ i{ \partialf \over
%\partialx_i} (a + \thetah)
%
%Démonstration Soit ψ : {[}0,1{]} \longrightarrow \mathbb{R} définie par ψ(t) = f(a + th). Le
%lemme du paragraphe précédent montre que ψ est dérivable sur {[}0,1{]}
%et que ψ'(t) =:gmathop\sum }
%}_i=1}\^{}{n}{h_i{ \partialf \over
%\partialx_i} (a + th). Le théorème des accroissements finis assure
%qu'il existe \theta \in{]}0,1{[} tel que ψ(1) − ψ(0) = (1 − 0)ψ'(\theta), ce qui
%n'est autre que la formule ci dessus.
%
%Corollaire~15.1.7 Soit U un ouvert de \mathbb{R}}\^{}{n, F un espace
%vectoriel de dimension finie, f : U \longrightarrow F de classe C}\^{}{1. Alors
%f est continue.
%
%Démonstration En prenant une base (sur \mathbb{R}) de F et en travaillant
%composante par composante, on peut supposer que f est à valeurs réelles.
%Puisque les dérivées partielles sont continues au point a, il existe η
%\textgreater{} 0 tel que B(a,η) \subset U et∀}x \in
%B(0,η), \textbar{} \partialf \over \partial{x_i} (x) −{ \partialf
%\over \partialx_i} (a)\textbar{}≤ 1, d'où \textbar{}{
%\partialf \over \partialx_i} (x)\textbar{}≤ 1 + \textbar{}{
%\partialf \over \partialx_i} (a)\textbar{}. Pour
%\\textbar{}h\\textbar{} \textless{} η, on
%a alors {[}a,a + h{]} \subset B(0,η) et donc \textbar{}f(a + h) −
%f(a)\textbar{}≤:gmathop\sum
%}_i=1}\^{}{n}\textbar{}{h_i\textbar{}\,\left
%(\left \textbar{}{ \partialf \over
%\partialx_i} (a)\right \textbar{} +
%1\right ), ce qui montre la continuité de f au point a.
%
\begin{cor}
  Soit U un ouvert connexe de $\mathbb{R}^n,$ F un
espace vectoriel de dimension finie, $f : U \longrightarrow F.$ Alors f est constante si
et seulement si~elle est de classe $C^1$ et toutes ses dérivées
partielles sont nulles.
\end{cor}
%Démonstration Si f est constante, il est clair qu'elle est de classe
%C}\^{}{1 et que toutes ses dérivées partielles sont nulles. Pour
%la réciproque, en prenant une base (sur \mathbb{R}) de F et en travaillant
%composante par composante, on peut supposer que f est à valeurs réelles.
%Soit x_0 \in U et soit X = \{x \in
%U\mathrel∣}f(x) = f({x_0})\.
%Puisque f est continue (d'après le corollaire précédent), X est un fermé
%de U, évidemment non vide. Montrons que X est également ouvert dans U~;
%soit en effet x_1 dans X et soit η \textgreater{} 0 tel que
%B(x_1,η) \subset U. Pour
%\\textbar{}h\\textbar{} \textless{} η, on
%a {[}x_1},{x_1} + h{]} \subset B({x_1,η) \subset U et le
%théorème des accroissements finis nous donne f(x_1 + h) =
%f(x_1}) = f({x_0), les dérivées partielles étant
%supposées nulles. On a donc B(x_1,η) \subset X, et donc X est ouvert.
%Comme X est à la fois ouvert et fermé, non vide dans U connexe, on a X =
%U et donc f est constante.
%
\paragraph{ Dérivées partielles successives}
%
%On définit la notion de dérivées partielles successives de manière
%récursive de la manière suivante
%
%Définition~15.1.4 Soit U un ouvert de \mathbb{R}}\^{}{n, a \in U et f : U \longrightarrow
%E. Soit
%(i_1},+{\ldots{}}},{i_k)
%\in {[}1,n{]}}\^{}{k}. On dit que { {\partial}\^{}{kf
%\over
%\partialx_{i_1}}+{\ldots{}}}\partial{x_{i_k}}
%(a) existe s'il existe un ouvert V tel que a \in V \subset U et sur lequel {
%\partial}\^{}{k−1f \over
%\partialx_{i_2}}+{\ldots{}}}\partial{x_{i_k}}
%(x) existe et si l'application x\mathrel↦{
%\partial}\^{}{k−1f \over
%\partialx_{i_2}}+{\ldots{}}}\partial{x_{i_k}}
%(x) admet une dérivée partielle d'indice i_1. On pose alors
%
% {\partial}\^{}{kf \over
%\partialx_{i_1}}+{\ldots{}}}\partial{x_{i_k}}
%(a) = \partial \over \partial{x_{i_1}}
%\left ( {\partial}\^{}{k−1f \over
%\partialx_{i_2}}+{\ldots{}}}\partial{x_{i_k}}
%\right )(a)
%
%Définition~15.1.5 Soit U un ouvert de \mathbb{R}}\^{}{n et f : U \longrightarrow E. On
%dit que f est de classe C}\^{}{k sur U si,
%\mathop∀}({i_1},+{\ldots{}}},{i_k)
%\in {[}1,n{]}}\^{}{k}, l'application x\mathrel{↦{
%\partial}\^{}{kf \over
%\partialx_{i_1}}+{\ldots{}}}\partial{x_{i_k}}
%(x) est définie et continue sur U.
%
%Remarque~15.1.5 Comme on a vu que toute application de classe
%C}\^{}{1 est continue, on en déduit immédiatement que toute
%application de classe C}\^{}{k est aussi de classe
%C}\^{}{k−1. On dira bien entendu que f est de classe
%C}\^{}{∞} si elle est de classe {C}\^{}{k pour tout k. Une
%récurrence évidente sur k montre que la composée de deux applications de
%classe C}\^{}{k} est encore de classe {C}\^{}{k et que donc la
%composée de deux applications de classe C}\^{}{∞ est encore de
%classe C}\^{}{∞.
%
%Lemme~15.1.9 Soit U un ouvert de \mathbb{R}}\^{}{2, f : U \longrightarrow \mathbb{R} de classe
%C}\^{}{2}. Alors, { {\partial}\^{}{2f \over
%\partialx_1}\partial{x_2}} ={ {\partial}\^{}{2f \over
%\partialx_2}\partial{x_1} .
%
%Démonstration Soit (a_1},{a_2) \in U et soit
%
%\begineqnarray*} φ({h_1},{h_2)\& =\&{
%1 \over h_1}{h_2}} (f({a_1 +
%h_1},{a_2} + {h_2}) − f({a_1 +
%h_1},{a_2)\%\& \\ \& \&
%\quad \quad \quad −
%f(a_1},{a_2} + {h_2) +
%f(a_1},{a_2)) \%\& \\
%\endeqnarray*
%
%définie pour h_1} et {h_2 non nuls et assez petits. On a
%φ(h_1},{h_2) ={ 1 \over
%h_1}{h_2}} {ψ_1}({a_1} + {h_1) −
%ψ_1}({a_1}) avec {ψ_1}({x_1) =
%f(x_1},{a_2} + {h_2) −
%f(x_1},{a_2}). Or {ψ_1 est dérivable sur
%{[}a_1},{a_1} + {h_1{]} avec
%ψ_1}'({x_1}) ={ \partialf \over \partial{x_1}
%(x_1},{a_2} + {h_2) −{ \partialf \over
%\partialx_1}} ({x_1},{a_2). On peut donc appliquer le
%théorème des accroissements finis, et donc il existe ξ_1 \in
%{[}a_1},{a_1} + {h_1{]} tel que
%
%\begineqnarray*} φ({h_1},{h_2)\& =\&{
%1 \over h_2}} {ψ_1}'({ξ_1) \%\&
%\\ \& =\&{ 1 \over
%h_2} \left ({ \partialf \over
%\partialx_1}} ({ξ_1},{a_2} + {h_2) −{ \partialf
%\over \partialx_1}
%(ξ_1},{a_2)\right )\%\&
%\\ \& =\& {\partial}\^{}{2f
%\over \partialx_2}\partial{x_1}
%(ξ_1},{ξ_2) \%\& \\
%\endeqnarray*
%
%avec ξ_2} \in {[}{a_2},{a_2} + {h_2{]} en
%appliquant le théorème des accroissements finis à
%x_2}\mathrel{↦{ \partialf \over
%\partialx_1}} ({ξ_1},{x_2) qui est dérivable sur
%{[}a_2},{a_2} + {h_2{]}, de dérivée {
%\partial}\^{}{2}f \over \partial{x_2}\partial{x_1}
%(ξ_1},{x_2}). Quand {h_1} et {h_2 tendent
%vers 0, ξ_1} et {ξ_2 tendent respectivement vers
%a_1} et {a_2} et la continuité de { {\partial}\^{}{2f
%\over \partialx_2}\partial{x_1} montre que
%:gmathoplim}_({h_1},{h_2})\longrightarrow(0,0)}φ({h_1},{h_2)
%= {\partial}\^{}{2}f \over \partial{x_2}\partial{x_1}
%(a_1},{a_2). Comme les deux variables jouent un rôle
%symétrique dans la définition de φ, en posant ψ_2}({x_2)
%= f(a_1} + {h_1},{x_2) −
%f(a_1},{x_2) et en appliquant deux fois le théorème des
%accroissements finis, on obtient
%:gmathoplim}_({h_1},{h_2})\longrightarrow(0,0)}φ({h_1},{h_2)
%= {\partial}\^{}{2}f \over \partial{x_1}\partial{x_2}
%(a_1},{a_2}), ce qui démontre que { {\partial}\^{}{2f
%\over \partialx_1}\partial{x_2}
%(a_1},{a_2}) ={ {\partial}\^{}{2f \over
%\partialx_2}\partial{x_1}} ({a_1},{a_2).
%
%Théorème~15.1.10 (Schwarz). Soit U un ouvert de \mathbb{R}}\^{}{n et f : U
%\longrightarrow E (espace vectoriel normé de dimension finie) de classe
%C}\^{}{2}. Alors∀(i,j) \in
%{[}1,n{]}}\^{}{2,
%
% {\partial}\^{}{2}f \over \partial{x_i}\partial{x_j} ={
%\partial}\^{}{2}f \over \partial{x_j}\partial{x_i}
%
%Démonstration En prenant une base de E, on peut se contenter de montrer
%le résultat lorsque E = \mathbb{R}. Si i = j, le résultat est évident. Supposons
%i \textless{} j et soit
%(a_1},+{\ldots{}}},{a_n)
%\in \mathbb{R}}\^{}{n. On applique le lemme précédent à l'application de
%classe C}\^{}{2, définie sur un ouvert contenant
%(a_i},{a_j),
%
%g(x_i},{x_j) =
%f(a_1},+{\ldots{}}},{a_i−1},{x_i},{a_i+1},\mathop{\mathop{\ldots{}}},{a_j−1},{x_j},{a_j+1},\mathop{\mathop{\ldots{}}},{a_n)
%
%qui est de classe C}\^{}{2 (composée d'applications de classe
%C}\^{}{2}). On a donc { {\partial}\^{}{2g \over
%\partialx_i}\partial{x_j}} ({a_i},{a_j) ={
%\partial}\^{}{2}g \over \partial{x_j}\partial{x_i}
%(a_i},{a_j), soit encore
%
% {\partial}\^{}{2}f \over \partial{x_i}\partial{x_j}
%(a_1},+{\ldots{}}},{a_n)
%= {\partial}\^{}{2}f \over \partial{x_j}\partial{x_i}
%(a_1},+{\ldots{}}},{a_n)
%
%Corollaire~15.1.11 Soit U un ouvert de \mathbb{R}}\^{}{n et f : U \longrightarrow E de
%classe C}\^{}{k. Soit
%(i_1},+{\ldots{}}},{i_k)
%\in {[}1,n{]}}\^{}{k. Pour toute permutation σ de {[}1,k{]} on a
%
% {\partial}\^{}{kf \over
%\partialx_{i_σ(1)}}+{\ldots{}}}\partial{x_{i_σ(k)}}
%= {\partial}\^{}{kf \over
%\partialx_{i_1}}+{\ldots{}}}\partial{x_{i_k}}
%
%Démonstration D'après le théorème de Schwarz, le résultat est vrai
%lorsque σ = τ_j,j+1 est la transposition qui échange j et j +
%1. Mais toute permutation de {[}1,k{]} est un produit de telles
%transpositions (facile) ce qui démontre le corollaire.
%
%Notation définitive Soit
%(i_1},+{\ldots{}}},{i_k)
%\in {[}1,n{]}}\^{}{k}. Pour j \in {[}1,n{]}, soit {k_j le
%nombre de i_q qui sont égaux à j. On a donc à une permutation
%près, la famille
%(i_1},+{\ldots{}}},{i_k)
%qui est égale à
%(\overbrace1,+{\ldots{}}},1}{k_1
%fois,+\ldots{}}},\overbrace{j,\mathop{\mathop{\ldots{}}},j
%k_j
%fois,+\ldots{}}},\overbrace{n,\mathop{\mathop{\ldots{}}},n}{k_n
%fois), chaque j étant compté k_j fois. En notant
%\partialx_j}\^{}{{k_j} à la place de
%\overbrace\partial{x_j}+{\ldots{}}}\partial{x_j}
%k_j fois, on obtient
%
% {\partial}\^{}{kf \over
%\partialx_{i_1}}+{\ldots{}}}\partial{x_{i_k}}
%= {\partial}\^{}{kf \over
%\partialx_1}\^{}{{k_1}}+{\ldots{}}}\partial{x_n}\^{}{{k_n}}
%
%\paragraph{15.1.5 Formules de Taylor}
%
%Lemme~15.1.12 Soit U un ouvert de \mathbb{R}}\^{}{n et f : U \longrightarrow E de classe
%C}\^{}{k}. Soit a \in U et h \in {\mathbb{R}}\^{}{n tel que {[}a,a + h{]} \subset
%U. Posons φ(t) = f(a + th), définie et de classe C}\^{}{k sur
%{[}0,1{]}. Alors, pour tout t \in {[}0,1{]},
%
%\begineqnarray*}{ φ}\^{}{(k)(t) ={
%\mathop{\sum
%}_k_1}+\mathop{\ldots{}}+{k_n}=k{
%k! \over
%k_1}!\mathop{\ldots{}}{k_n}!
%h_1}\^{}{{k_1
%}\mathop\ldots{}}{h_n}\^{}{{k_n} {
%\partial}\^{}{kf \over
%\partialx_1}\^{}{{k_1}}\mathop{\ldots{}}\partial{x_n}\^{}{{k_n}}
%(a + th)\& \& \%\& \\
%\endeqnarray*
%
%Démonstration Par récurrence sur k. Pour k = 1, ce n'est qu'une autre
%formulation du résultat
%
%\begineqnarray* φ'(t)\& =\&
%:gmathop\sum }_i=1}\^{}{n}{h_ i{ \partialf
%\over \partialx_i} (a + th) \%\&
%\\ \& =\& :gmathop{\sum
%}_k_1}+\mathop{\ldots{}}+{k_n}=1}{h_1}\^{}{{k_1
%}\mathop\ldots{}}{h_n}\^{}{{k_n} {
%\partialf \over
%\partialx_1}\^{}{{k_1}}\mathop{\ldots{}}\partial{x_n}\^{}{{k_n}}
%(a + th)\%\& \\
%\endeqnarray*
%
%en posant k_i} = 1 et {k_j = 0 pour
%i\mathrel≠j.
%
%Supposons le résultat démontré pour k − 1. On a donc
%
%\begineqnarray*}{ φ}\^{}{(k−1)(t) =\&\& \%\&
%\\ \& \& :gmathop{\sum
%}_k_1}+\mathop{\ldots{}}+{k_n}=k−1{
%(k − 1)! \over
%k_1}!\mathop{\ldots{}}{k_n}!
%h_1}\^{}{{k_1
%}\mathop\ldots{}}{h_n}\^{}{{k_n} {
%\partial}\^{}{k−1f \over
%\partialx_1}\^{}{{k_1}}\mathop{\ldots{}}\partial{x_n}\^{}{{k_n}}
%(a + th)\%\& \\
%\endeqnarray*
%
%On en déduit que
%
%\begineqnarray*}{ φ}\^{}{(k)(t) =\&\& \%\&
%\\ \& \& :gmathop{\sum
%}_k_1}+\mathop{\ldots{}}+{k_n}=k−1{
%(k − 1)! \over
%k_1}!\mathop{\ldots{}}{k_n}!
%h_1}\^{}{{k_1
%}\mathop\ldots{}}{h_n}\^{}{{k_n} {
%d \over dt} \left ( {\partial}\^{}{k−1f
%\over
%\partialx_1}\^{}{{k_1}}\mathop{\ldots{}}\partial{x_n}\^{}{{k_n}}
%(a + th)\right )\%\& \\
%\endeqnarray*
%
%soit encore
%
%\begineqnarray*}{ φ}\^{}{(k)(t)\& =\&
%:gmathop{\sum
%}_k_1}+\mathop{\ldots{}}+{k_n}=k−1{
%(k − 1)! \over
%k_1}!\mathop{\ldots{}}{k_n}!
%h_1}\^{}{{k_1
%}\mathop\ldots{}}{h_n}\^{}{{k_n} 
%\%\& \\ \& \& \quad
%\quad {\sum
%}}_i=1}\^{}{n}{h_ i}{ {\partial}\^{}{kf
%\over
%\partialx_1}\^{}{{k_1}}\mathop{\ldots{}}\partial{x_i}\^{}{{k_i}+1}\mathop{\ldots{}}\partial{x_n}\^{}{{k_n}}
%(a + th)\%\& \\
%\endeqnarray*
%
%En intervertissant les deux signes de somme on obtient
%
%\begineqnarray*}{ φ}\^{}{(k)(t)\& =\&
%:gmathop\sum }_i=1}\^{}{n{
%\mathop{\sum
%}_k_1}+\mathop{\ldots{}}+{k_n}=k−1{
%(k − 1)!(k_i + 1) \over
%k_1}!\mathop{\ldots{}}({k_i +
%1)!\mathop\ldots{}}{k_n}! \%\&
%\\ \& \& \quad
%\quad h_1}\^{}{{k_1
%}+\ldots{}}}{h_i}\^{}{{k_i}+1}\mathop{\mathop{\ldots{}}}{h_{
%n}\^{}{k_n} }{ {\partial}\^{}{kf \over
%\partialx_1}\^{}{{k_1}}+{\ldots{}}}\partial{x_i}\^{}{{k_i}+1}\mathop{\mathop{\ldots{}}}\partial{x_n}\^{}{{k_n}}
%(a + th)\%\& \\
%\endeqnarray*
%
%et en faisant un changement d'indice
%
%\begineqnarray*}{ φ}\^{}{(k)(t)\& =\&
%:gmathop\sum }_i=1}\^{}{n{
%\mathop\sum _{
%k_1}+\mathop{\ldots{}}+{k_n=k
%\atop k_i}≥1} }{ (k − 1)!{k_i
%\over
%k_1}!\mathop{\ldots{}}{k_n}! \%\&
%\\ \& \& \quad
%\quad h_1}\^{}{{k_1
%}+\ldots{}}}{h_i}\^{}{{k_i
%}+\ldots{}}}{h_n}\^{}{{k_n
%} {\partial}\^{}{kf \over
%\partialx_1}\^{}{{k_1}}+{\ldots{}}}\partial{x_i}\^{}{{k_i}}\mathop{\mathop{\ldots{}}}\partial{x_n}\^{}{{k_n}}
%(a + th)\%\& \\
%\endeqnarray*
%
%Réintroduisons les termes pour k_i = 0 qui sont nuls puisqu'ils
%contiennent le facteur (k − 1)!k_i, on obtient
%
%\begineqnarray*}{ φ}\^{}{(k)(t)\& =\&
%:gmathop\sum }_i=1}\^{}{n{
%\mathop{\sum
%}_k_1}+\mathop{\ldots{}}+{k_n}=k{
%(k − 1)!k_i \over
%k_1}!\mathop{\ldots{}}{k_n}!
%h_1}\^{}{{k_1
%}\mathop\ldots{}}{h_i}\^{}{{k_i
%}\mathop\ldots{}}{h_n}\^{}{{k_n
%}\%\& \\ \& \& \quad
%\quad \quad  {\partial}\^{}{kf
%\over
%\partialx_1}\^{}{{k_1}}+{\ldots{}}}\partial{x_i}\^{}{{k_i}}\mathop{\mathop{\ldots{}}}\partial{x_n}\^{}{{k_n}}
%(a + th) \%\& \\
%\endeqnarray*
%
%Ceci nous permet de réintervertir les deux sommations, soit encore,
%après mise en facteur
%
%\begineqnarray*}{ φ}\^{}{(k)(t)\& =\&
%:gmathop{\sum
%}_k_1}+\mathop{\ldots{}}+{k_n}=k{
%(k − 1)!:gmathop\sum }_i=1}\^{}{n}{k_i
%\over
%k_1}!\mathop{\ldots{}}{k_n}!
%h_1}\^{}{{k_1
%}\mathop\ldots{}}{h_i}\^{}{{k_i
%}\mathop\ldots{}}{h_n}\^{}{{k_n
%}\%\& \\ \& \& \quad
%\quad \quad  {\partial}\^{}{kf
%\over
%\partialx_1}\^{}{{k_1}}+{\ldots{}}}\partial{x_i}\^{}{{k_i}}\mathop{\mathop{\ldots{}}}\partial{x_n}\^{}{{k_n}}
%(a + th) \%\& \\
%\endeqnarray*
%
%soit encore
%
%\begineqnarray*}{ φ}\^{}{(k)(t)\& =\&
%:gmathop{\sum
%}_k_1}+\mathop{\ldots{}}+{k_n}=k{
%k! \over
%k_1}!\mathop{\ldots{}}{k_n}!
%h_1}\^{}{{k_1
%}\mathop\ldots{}}{h_n}\^{}{{k_n} {
%\partial}\^{}{kf \over
%\partialx_1}\^{}{{k_1}}\mathop{\ldots{}}\partial{x_n}\^{}{{k_n}}
%(a + th)\%\& \\
%\endeqnarray*
%
%ce qui achève la récurrence.
%
%Remarque~15.1.6 Cette formule est tout à fait analogue à la formule du
%binôme généralisée
%
%({X_1 +
%+\ldots{}} +
%X_n})}\^{}{k ={\sum
%}_k_1}+\mathop{\ldots{}}+{k_n}=k{
%k! \over
%k_1}!\mathop{\ldots{}}{k_n}!
%X_1}\^{}{{k_1
%}\mathop\ldots{}}{X_n}\^{}{{k_n} 
%
%Cette remarque nous conduira à une notation plus compacte. Introduisons
%un produit symbolique sur les expressions du type
%h_1}\^{}{{k_1}}+{\ldots{}}}{h_n}\^{}{{k_n}{
%\partial}\^{}{k \over
%\partialx_1}\^{}{{k_1}}+{\ldots{}}}\partial{x_n}\^{}{{k_n}}
%en posant
%
%\begineqnarray* \left
%(h_1}\^{}{{k_1
%}+\ldots{}}}{h_n}\^{}{{k_n
%} {\partial}\^{}{k \over
%\partialx_1}\^{}{{k_1}}+{\ldots{}}}\partial{x_n}\^{}{{k_n}}
%\right ) ∗\left
%(h_1}\^{}{{l_1
%}+\ldots{}}}{h_n}\^{}{{l_n
%} {\partial}\^{}{l \over
%\partialx_1}\^{}{{l_1}}+{\ldots{}}}\partial{x_n}\^{}{{l_n}}
%\right ) =\&\&\%\& \\ \& \&
%h_1}\^{}{{k_1}+{l_1
%}+\ldots{}}}{h_n}\^{}{{k_n}+{l_n
%} {\partial}\^{}{k+l \over
%\partialx_1}\^{}{{k_1}+{l_1}}+{\ldots{}}}\partial{x_n}\^{}{{k_n}+{l_n}}
%\quad \quad \quad \%\&
%\\ \endeqnarray*
%
%Ce produit est commutatif, et
%
%\begineqnarray* :gmathop{\sum
%}_k_1}+\mathop{\ldots{}}+{k_n}=k{
%k! \over
%k_1}!\mathop{\ldots{}}{k_n}!
%h_1}\^{}{{k_1
%}\mathop\ldots{}}{h_n}\^{}{{k_n} {
%\partial}\^{}{k \over
%\partialx_1}\^{}{{k_1}}\mathop{\ldots{}}\partial{x_n}\^{}{{k_n}}
%\&\&\%\& \\ \& \& ={ \left
%(h_1}{ \partial \over \partial{x_1} +
%+\ldots{}} +
%h_n}{ \partial \over \partial{x_n}
%\right )}\^{}k∗\quad
%\quad \quad \%\&
%\\ \endeqnarray*
%
%où la notation }\^{}{k∗ désigne la puissance k-ième pour ce
%produit commutatif. La formule s'écrit alors de manière plus agréable
%sous la forme
%
%φ}\^{}{(k)}(t) ={ \left ({h_ 1{ \partial
%\over \partialx_1} +
%+\ldots{}} +
%h_n}{ \partial \over \partial{x_n}
%\right )}\^{}k∗f(a + th)
%
%Ces puissances se développent de la manière évidente en respectant la
%règle de calcul pour le produit ∗.
%
%Exemple~15.1.3 φ'(t) = \left (h_1{ \partial
%\over \partialx_1} +
%+\ldots{}} +
%h_n}{ \partial \over \partial{x_n}
%\right )f(a + th)
%
%\begineqnarray* φ''(t)\& =\&{ \left
%(h_1}{ \partial \over \partial{x_1} +
%+\ldots{}} +
%h_n}{ \partial \over \partial{x_n}
%\right )}\^{}2∗f(a + th) \%\&
%\\ \& =\& :gmathop{\sum
%}}_i=1}\^{}{n}{h_ i}\^{}{2}{ {\partial}\^{}{2f
%\over \partialx_i}\^{}{2} (a + th) +
%2:gmathop{\sum
%}}_i\textless{}j}{h_i}{h_j}{ {\partial}\^{}{2f
%\over \partialx_i}\partial{x_j} (a + th)\%\&
%\\ \endeqnarray*
%
%et ainsi de suite.
%
%Théorème~15.1.13 (formule de Taylor avec reste intégral). Soit U un
%ouvert de \mathbb{R}}\^{}{n} et f : U \longrightarrow E de classe {C}\^{}{k+1. Soit a
%\in U et h \in \mathbb{R}}\^{}{n tel que {[}a,a + h{]} \subset U. Alors
%
%\begineqnarray* f(a + h)\& =\& f(a) +{
%\mathop\sum }_p=1}\^{}{k{ 1
%\over p!} :gleft (h_1{ \partial
%\over \partialx_1} +
%\mathop\ldots{}} + {h_n{ \partial
%\over \partialx_n} \right
%)}\^{}p∗f(a)\%\& \\
%+:gmathop∫ } _0}\^{}{1{ {(1 −
%t)}\^{}k} \over k! { \left
%(h_1}{ \partial \over \partial{x_1} +
%+\ldots{}} +
%h_n}{ \partial \over \partial{x_n}
%\right )}\^{}(k+1)∗f(a + th) dt\&\&\%\&
%\\ \endeqnarray*
%
%Démonstration C'est simplement la formule de Taylor avec reste intégral
%pour la fonction φ~:
%
%φ(1) = φ(0) +\sum }_p=1}\^{}{k{ 1
%\over p!} φ}\^{}{(p)(0) +{
%+∫ } 
%}_0}\^{}{1}{ {(1 − t)}\^{}{k} \over k!
%φ}\^{}{(k+1)(t) dt
%
%Remarque~15.1.7 On utilisera le plus souvent cette formule pour k = 1~;
%dans cas d'une fonction définie sur un ouvert de \mathbb{R}}\^{}{2 on
%obtiendra par exemple
%
%\begineqnarray*} f(a + h)\& =\& f(a) + {h_1{
%\partialf \over \partialx_1}} (a) + {h_2{ \partialf
%\over \partialx_2} (a) \%\&
%\\ \& \& \quad +
%h_1}\^{}{2}:gmathop{∫ } _0}\^{}{1(1
%− t) {\partial}\^{}{2}f \over \partial{x_1}\^{}{2} (a
%+ th) dt \%\& \\ \& \&
%\quad + h_2}\^{}{2:gmathop{∫
%} }_0}\^{}{1}(1 − t){ {\partial}\^{}{2f \over
%\partialx_2}\^{}{2} (a + th) dt \%\&
%\\ \& \& \quad +
%2h_1}{h_2}:gmathop{∫ 
%}_0}\^{}{1}(1 − t){ {\partial}\^{}{2f \over
%\partialx_1}\partial{x_2} (a + th) dt\%\&
%\\ \endeqnarray*
%

\begin{thm}[formule de Taylor-Lagrange]
	Soit U un ouvert de
$\mathbb{R}^n et $f : U \longrightarrow \mathbb{R}$ de classe $C^{k+1}$. Soit $a \in U$ et $h \in \mathbb{R}^n$ tel que $[a,a + h] \in  U$. Alors, il existe $\theta \in ]0,1[$ tel que

\begin{eqnarray*} 
f(a + h) & = & f(a) +\sum_{p=1}{k} { 1
\over p!} \left(h_{1} { \partial
\over \partial{x_1}} +\ldots{} + h_{n}{ \partial
\over \partial x_n} \right)}^{p*}f(a) & \\
& & +{ 1 \over (k + 1)!} \left(h_1 { \partial
\over \partial x_1} +
+\ldots{} + h_n { \partial \over \partial{x_n}} \right)^{(k+1)*} f(a + \thetah)\

\end{eqnarray*}
\end{thm}
%Démonstration C'est simplement la formule de Taylor Lagrange pour la
%fonction φ~:
%
%φ(1) = φ(0) +\sum }_p=1}\^{}{k{ 1
%\over p!} φ}\^{}{(p)(0) +{ 1 \over
%(k + 1)!} φ}\^{}{(k+1)(\theta)
%
%Théorème~15.1.15 (formule de Taylor-Young). Soit U un ouvert de
%\mathbb{R}}\^{}{n et f : U \longrightarrow E (espace vectoriel normé de dimension finie)
%de classe C}\^{}{k. Soit a \in U. Alors, quand h tend vers 0 on a
%
%f(a + h) = f(a) +\sum }_p=1}\^{}{k{ 1
%\over p!} :gleft (h_1{ \partial
%\over \partialx_1} +
%\mathop\ldots{}} + {h_n{ \partial
%\over \partialx_n} \right
%)}\^{}p∗f(a) +
%o(\\textbar{}h\\textbar{}}\^{}{k)
%
%Démonstration Quitte à prendre une base de E et à travailler composante
%par composante, on peut supposer que E = \mathbb{R}~; toutes les normes sur
%\mathbb{R}}\^{}{n étant équivalentes, on peut supposer que
%\\textbar{}h\\textbar{} =
%\textbar{}h_1\textbar{} +
%+\ldots{}} +
%\textbar{}h_n\textbar{}. Soit ρ \textgreater{} 0 tel que B(a,ρ)
%\subset U et soit h tel que
%\\textbar{}h\\textbar{} \textless{} ρ. On
%a alors {[}a,a + h{]} \subset B(a,ρ) \subset U~; on peut donc appliquer la formule
%de Taylor-Lagrange à l'ordre k − 1 qui nous donne
%
%\begineqnarray* f(a + h)\& −\& f(a)
%−:gmathop\sum }_p=1}\^{}{k{ 1
%\over p!} :gleft (h_1{ \partial
%\over \partialx_1} +
%\mathop\ldots{}} + {h_n{ \partial
%\over \partialx_n} \right
%)}\^{}p∗f(a)\%\& \\ \& =\&{ 1
%\over k!} :gleft (h_1{ \partial
%\over \partialx_1} +
%+\ldots{}} +
%h_n}{ \partial \over \partial{x_n}
%\right )}\^{}k∗f(a + \thetah) \%\&
%\\ \& −\& 1 \over k!
%:gleft (h_1{ \partial \over
%\partialx_1} +
%+\ldots{}} +
%h_n}{ \partial \over \partial{x_n}
%\right )}\^{}k∗f(a) \%\&
%\\ \endeqnarray*
%
%Mais les dérivées partielles de f sont continues. Soit ε \textgreater{}
%0~; il existe η \textgreater{} 0 tel que
%
%\begineqnarray*
%\\textbar{}h\\textbar{} \textless{} η\&
%⇒\&
%\mathop∀}({k_1},+{\ldots{}}},{k_n)\text{
%tel que }k_1 +
%+\ldots{}} +
%k_n} = k,∀t \in {[}0,1{]} \%\&
%\\ \& \& \left \textbar{}{
%\partial}\^{}{kf \over
%\partialx_1}\^{}{{k_1}}+{\ldots{}}}\partial{x_n}\^{}{{k_n}}
%(a + th)\right . −\left .{
%\partial}\^{}{kf \over
%\partialx_1}\^{}{{k_1}}+{\ldots{}}}\partial{x_n}\^{}{{k_n}}
%(a)\right \textbar{} \textless{} ε\%\&
%\\ \endeqnarray*
%
%Pour \\textbar{}h\\textbar{} \textless{}
%η, on a alors (en développant les deux puissances symboliques)
%
%\begineqnarray* \big
%\textbar{}:gleft (\mathop{\sum
%}h_i}{ \partial \over \partial{x_i}
%\right )}\^{}k∗f(a + \thetah) −:gleft
%(\mathop\sum }{h_i{ \partial \over
%\partialx_i} \right
%)}\^{}k∗f(a)\big \textbar{}\&\&\%\&
%\\ \& \textless{}\&
%ε:gmathop{\sum
%}_k_1}+\mathop{\ldots{}}+{k_n}=k{
%k! \over
%k_1}!\mathop{\ldots{}}{k_n}!
%\textbar{}h_1}{\textbar{}}\^{}{{k_1
%}\mathop\ldots{}}\textbar{}{h_n}{\textbar{}}\^{}{{k_n
%}\%\& \\ \& =\&
%ε(\textbar{}{h_1\textbar{} +
%+\ldots{}} +
%\textbar{}h_n}\textbar{})}\^{}{k =
%ε\\textbar{}h\\textbar{}}\^{}{k \%\&
%\\ \endeqnarray*
%
%ce qui démontre le résultat.
%
%\paragraph{15.1.6 Application aux extremums de fonctions de plusieurs
%variables}
%
%Soit U un ouvert de \mathbb{R}}\^{}{n et f : U \longrightarrow \mathbb{R}. Nous allons rechercher
%les extremums de la fonction f à l'aide des résultats qui suivent.
%
%Proposition~15.1.16 Soit U un ouvert de \mathbb{R}}\^{}{n et f : U \longrightarrow \mathbb{R} de
%classe C}\^{}{1. Soit a \in U. Si f admet en a un extremum local, on
%a∀}i \in {[}1,n{]},{ \partialf \over
%\partialx_i} (a) = 0.
%
%Démonstration Il suffit de remarquer que la fonction
%t\mathrel↦}f(a + t{e_i) (définie sur un
%voisinage de 0) admet en 0 un extremum local. On a donc
%
% \partialf \over \partial{x_i} (a) ={ d
%\over dt} :gleft (f(a +
%te_i})\right )_t=0 = 0
%
%Dans le cas des fonctions d'une variable, la condition ci dessus n'est
%déjà pas suffisante (considérer
%x\mathrel↦}{x}\^{}{3 au point 0). Il est clair
%qu'il en est de même a fortiori pour une fonction de plusieurs
%variables. Pour obtenir des résultats plus précis et en particulier des
%conditions suffisantes d'extremums, nous allons introduire une forme
%quadratique sur \mathbb{R}}\^{}{n
%
%Définition~15.1.6 Soit U un ouvert de \mathbb{R}}\^{}{n et f : U \longrightarrow \mathbb{R} de
%classe C}\^{}{2. Soit a \in U. On appelle différentielle seconde au
%point a la forme quadratique sur \mathbb{R}}\^{}{n,
%
%\begineqnarray* h\& =\&
%(h_1},+{\ldots{}}},{h_n})\mathrel{↦{\left
%(h_1}{ \partial \over \partial{x_1} +
%+\ldots{}} +
%h_n}{ \partial \over \partial{x_n}
%\right )}\^{}2∗f(a) \%\&
%\\ \& \& ={\sum
%}}_i=1}\^{}{n}{h_ i}\^{}{2}{ {\partial}\^{}{2f
%\over \partialx_i}\^{}{2} (a) +
%2:gmathop{\sum
%}}_i\textless{}j}{h_i}{h_j}{ {\partial}\^{}{2f
%\over \partialx_i}\partial{x_j} (a)\%\&
%\\ \endeqnarray*
%
%Théorème~15.1.17 Soit U un ouvert de \mathbb{R}}\^{}{n et f : U \longrightarrow \mathbb{R} de
%classe C}\^{}{2}. Soit a \in U tel que∀i \in
%{[}1,n{]}, \partialf \over \partial{x_i} (a) = 0 et soit Φ
%la forme quadratique différentielle seconde au point a. Alors (i) si Φ
%est définie positive, c'est-à-dire si h\mathrel≠0 ⇒
%Φ(h) \textgreater{} 0, alors f admet en a un minimum local strict (ii)
%si Φ est définie négative, c'est-à-dire si
%h\mathrel≠0 ⇒ Φ(h) \textless{} 0, alors f admet en a
%un maximum local strict (iii) si Φ n'est ni positive ni négative, alors
%f n'admet pas d'extremum en a (on dit dans ce cas que a est un point
%selle ou point col de a, par analogie avec une selle de cheval ou un col
%de montagne).
%
%Démonstration (i). Utilisons la formule de Taylor Young à l'ordre 2. On
%a donc, en tenant compte de  \partialf \over \partial{x_i}
%(a) = 0, f(a + h) = f(a) + 1 \over 2 Φ(h)
%+\\textbar{}
%h\\textbar{}}\^{}{2ε(h), avec
%:gmathoplim}_h\longrightarrow0ε(h) = 0. Pour démontrer (i),
%nous allons utiliser le lemme suivant
%
%Lemme~15.1.18 Soit Φ une forme quadratique définie positive sur
%\mathbb{R}}\^{}{n (ou tout espace vectoriel normé de dimension finie).
%Alors∃}α \textgreater{} 0,
%\mathop∀}h \in {\mathbb{R}}\^{}{n, Φ(h) ≥
%α\\textbar{}h\\textbar{}}\^{}{2.
%
%Démonstration Soit S la sphère unité de \mathbb{R}}\^{}{n. Comme Φ est
%continue sur S qui est compact, Φ atteint sur S sa borne inférieure α.
%Soit donc x_0} \in S tel que Φ({x_0) = α
%=:gmathop inf} _x\inSΦ(x). Comme
%x_0}\mathrel{≠0, on a α \textgreater{} 0. De
%plus, si h\mathrel≠0, on a { h \over
%\\textbar{}h\\textbar{}} \in S, soit Φ({ h
%\over
%\\textbar{}h\\textbar{}} ) ≥ α soit {
%Φ(h) \over
%\\textbar{}h\\textbar{}}\^{}{2} ≥
%α, soit encore Φ(h) ≥
%α\\textbar{}h\\textbar{}}\^{}{2.
%
%Puisque :gmathoplim}_h\longrightarrow0ε(h) = 0, il existe η
%\textgreater{} 0 tel que
%\\textbar{}h\\textbar{} \textless{} η
%⇒\textbar{}ε(h)\textbar{}≤ α \over 4 . Pour
%\\textbar{}h\\textbar{} \textless{} η, on
%a donc
%
%\begineqnarray* f(a + h) − f(a)\& =\&{ 1
%\over 2} Φ(h) +\\textbar{}
%h\\textbar{}}\^{}{2ε(h) \%\&
%\\ \& ≥\& α \over 2
%\\textbar{}h\\textbar{}}\^{}{2 −{ α
%\over 4}
%\\textbar{}h\\textbar{}}\^{}{2 ={ α
%\over 4}
%\\textbar{}h\\textbar{}}\^{}{2
%\textgreater{} 0\%\& \\
%\endeqnarray*
%
%pour h\mathrel≠0. Donc f admet en a un minimum local
%strict.
%
%Pour démontrer (ii) à partir de (i), il suffit de changer f en − f.
%
%(iii). Si Φ n'est ni positive, ni négative, il existe v_1 \in
%\mathbb{R}}\^{}{n} tel que Φ({v_1) \textless{} 0 et il existe
%v_2} \in {\mathbb{R}}\^{}{n} tel que Φ({v_2) \textgreater{} 0.
%On a alors, d'après la même formule de Taylor, en posant h =
%tv_i}, f(a + t{v_i) = f(a) +{ 1 \over
%2} Φ(tv_i) +
%t}\^{}{2}\\textbar{}{v{_{
%i}\\textbar{}}\^{}2}ε(t{v_ i) = f(a) +{
%t}\^{}{2} \over 2} Φ({v_i) +
%t}\^{}{2}{ε_ i(t) avec
%:gmathoplim}_t\longrightarrow0}{ε_i(t) = 0. On en
%déduit qu'il existe un η \textgreater{} 0 tel que \textbar{}t\textbar{}
%\textless{} η ⇒ f(a + tv_1) \textless{}
%f(a)\text et }f(a + t{v_2) \textgreater{}
%f(a). Donc f n'a ni minimum, ni maximum en a.
%
%Remarque~15.1.8 Dans le cas où Φ est soit positive, soit négative, mais
%non définie (c'est-à-dire que Φ(h) peut être nul sans que h soit nul),
%on ne peut pas conclure en général et il faut utiliser une formule de
%Taylor à un ordre supérieur.
%
%Exemple~15.1.4 n = 2~; soit U un ouvert de \mathbb{R}}\^{}{2 et f : U \longrightarrow \mathbb{R},
%(x,y)\mathrel↦f(x,y). Soit (a,b) \in U. Une condition
%nécessaire pour que f admette en (a,b) un extremum est que { \partialf
%\over \partialx} (a,b) = \partialf \over \partialy (a,b) =
%0. Posons r = {\partial}\^{}{2}f \over \partial{x}\^{}{2}
%(a,b), s = {\partial}\^{}{2}f \over \partialx\partialy (a,b), t ={
%\partial}\^{}{2}f \over \partial{y}\^{}{2} (a,b) (notations
%de Monge). La forme quadratique Φ est
%(h,k)\mathrel↦}r{h}\^{}{2 + 2shk +
%tk}\^{}{2. Considérons suivant le cas le rapport { h
%\over k} ou le rapport  k \over h ,
%on constate immédiatement à l'aide de l'étude du signe d'un trinome du
%second degré que si (i) rt − s}\^{}{2 \textgreater{} 0 et r
%\textgreater{} 0, alors Φ est définie positive et f a en a un minimum
%local strict (ii) rt − s}\^{}{2 \textgreater{} 0 et r \textless{}
%0, alors Φ est définie négative et f a en a un maximum local strict
%(iii) rt − s}\^{}{2 \textless{} 0, alors f a en a un point selle
%(pas d'extremum local en a) (iv) rt − s}\^{}{2 = 0, alors on ne
%peut pas conclure.
%
%Le lecteur comparera les surfaces z = f(x,y) ainsi que lignes de niveau
%de ces surfaces dans les trois exemples ci dessous (correspondant
%respectivement à un minimum local, un point selle et un point de type
%(iv))
%
%\includegraphics{cours8x.png}
%
%{[}\href{coursse83.html}{next}{]} {[}\href{coursse82.html}{front}{]}
%{[}\href{coursch16.html\#coursse82.html}{up}{]}
