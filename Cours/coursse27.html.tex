\textbf{Warning: 
requires JavaScript to process the mathematics on this page.\\ If your
browser supports JavaScript, be sure it is enabled.}

\begin{center}\rule{3in}{0.4pt}\end{center}

{[}
{[}{]}
{[}

\subsubsection{5.1 Notion d'espace vectoriel normé}

\paragraph{5.1.1 Norme et distance associée}

Définition~5.1.1 Soit E un K-espace vectoriel . On appelle norme sur E
toute application
x\mapsto~\\textbar{}x\\textbar{}
de E dans \mathbb{R}~^+ vérifiant

\begin{itemize}
\itemsep1pt\parskip0pt\parsep0pt
\item
  (i) \\textbar{}x\\textbar{} = 0
  \Leftrightarrow x = 0 (séparation)
\item
  (ii) \\textbar{}\lambda~x\\textbar{} =
  \textbar{}\lambda~\textbar{}\\textbar{}x\\textbar{}
  (homogénéité)
\item
  (iii) \\textbar{}x + y\\textbar{}
  \leq\\textbar{} x\\textbar{}
  +\\textbar{} y\\textbar{} (inégalité
  triangulaire)
\end{itemize}

On appelle espace vectoriel normé un couple
(E,\\textbar{}.\\textbar{}) d'un K-espace
vectoriel et d'une norme sur E.

Exemple~5.1.1 Sur K^n, on définit trois normes usuelles,
\\textbar{}x\\textbar{}1
= \\sum ~
\textbar{}xi\textbar{},
\\textbar{}x\\textbar{}2 =
\sqrt\\\sum
 \textbar{}xi\textbar{}^2,
\\textbar{}x\\textbar{}\infty~
= sup\textbar{}xi~\textbar{}. De la
même fa\ccon, on définit sur l'espace vectoriel des
fonctions continues de {[}0,1{]} dans K, C({[}0,1{]},K), trois normes
usuelles,
\\textbar{}f\\textbar{}1
=\int ~
0^1\textbar{}f(t)\textbar{} dt,
\\textbar{}f\\textbar{}2 =
\sqrt\int ~
0^1\textbar{}f(t)\textbar{}^2 dt,
\\textbar{}f\\textbar{}\infty~
=\
supx\in{[}0,1{]}\textbar{}f(x)\textbar{}.

Proposition~5.1.1 Soit E un K-espace vectoriel normé. L'application d :
E \times E \rightarrow~ \mathbb{R}~^+ définie par d(x,y) =\\textbar{} x
- y\\textbar{} est une distance sur E appelée distance
associée à la norme. La topologie associée à cette distance est appelée
topologie définie par la norme.

Remarque~5.1.1 Si E est un espace vectoriel normé, on dispose de deux
familles importantes d'homéomorphismes de E sur lui même~: les
translations tv : x\mapsto~x + v et les
homothéties h\lambda~ : x\mapsto~\lambda~x
(\lambda~\neq~0). On constate que tous les points ont
les mêmes propriétés topologiques et que deux boules ouvertes sont
tou\\\\jmathmathmathmathours homéomorphes.

Définition~5.1.2 On appelle espace de Banach un espace vectoriel normé
complet (pour la distance associée).

Définition~5.1.3 Soit
(E1,\\textbar{}.\\textbar{}1),\\ldots,(Ek,\\textbar{}.\\textbar{}k~)
des espaces vectoriels normés. On définit une norme sur le produit E =
E1 \times⋯ \times Ek en posant
\\textbar{}x\\textbar{}
=\
max\\textbar{}xi\\textbar{}i.
L'espace vectoriel
normé~(E,\\textbar{}.\\textbar{}) est
appelé l'espace vectoriel normé produit. Il est complet si chacun des
Ei est complet.

\paragraph{5.1.2 Convexes, connexes}

Définition~5.1.4 Soit E un espace vectoriel normé, a,b \in E. On pose
{[}a,b{]} = \ta + (1 - t)b∣t
\in {[}0,1{]}\. On dit qu'une partie A de E est convexe
si \forall~~a,b \in A,\quad {[}a,b{]} \subset~
A.

Remarque~5.1.2 Le théorème d'associativité des barycentres montre
immédiatement par récurrence que si A est convexe,
a1,\\ldots,an~
\in A et
\lambda~1,\\ldots,\lambda~n~
sont des réels positifs de somme 1, alors \lambda~1a1 +
\\ldots~ +
\lambda~nan est encore dans A.

Proposition~5.1.2 Toute partie convexe est connexe par arcs (et donc
connexe).

Démonstration \gamma(t) = (1 - t)a + tb est un chemin continu (l'application
est \\textbar{}b -
a\\textbar{}-lipschitzienne) d'origine a et d'extrémité
b.

Proposition~5.1.3 Dans un espace vectoriel normé, les boules sont
convexes (et donc connexes).

Démonstration Montrons le par exemple pour une boule ouverte B(a,r).
Soit x,y \in B(a,r) et t \in {[}0,1{]}. On a alors

\begin{align*} \\textbar{}tx + (1 -
t)y - a\\textbar{}& =& \\textbar{}t(x -
a) + (1 - t)(y - a)\\textbar{}\%&
\\ & \leq& t\\textbar{}x -
a\\textbar{} + (1 - t)\\textbar{}y -
a\\textbar{} \%& \\ &
\textless{}& tr + (1 - t)r = r \%& \\
\end{align*}

car t ≥ 0,1 - t ≥ 0 et soit t, soit 1 - t est non nul.

Théorème~5.1.4 Dans un espace vectoriel normé, tout ouvert connexe est
connexe par arcs.

Démonstration Soit U un ouvert connexe. Soit \mathcal{R} la relation d'équivalence
sur U~: a\mathcal{R}b s'il existe un chemin d'origine a et d'extrémité b. Soit
C(a) la classe d'équivalence de a et montrons que C(a) est ouverte. Pour
cela soit b \in C(a) \subset~ U. Il existe r \textgreater{} 0 tel que B(b,r) \subset~ U.
Mais la boule, étant convexe, est connexe par arcs et donc pour tout x
de B(b,r) on a x \in C(b) = C(a), soit B(b,r) \subset~ C(a). Les classes
d'équivalences sont donc ouvertes dans U. Mais on a alors
^cC(a) =\ \⋃
 x∉C(a)C(x) est ouvert dans U et
donc C(a) est fermé dans U. Comme U est connexe, les seules parties
ouvertes et fermées dans U sont \varnothing~ et U, soit C(a) = U et donc U est
connexe par arcs.

\paragraph{5.1.3 Continuité des opérations algébriques}

Théorème~5.1.5 Soit E un espace vectoriel normé. L'application s : E \times E
\rightarrow~ E, (x,y)\mapsto~x + y est uniformément continue,
et l'application p : K \times E \rightarrow~ E, (\lambda~,x)\mapsto~\lambda~x est
continue.

Démonstration On a \\textbar{}s(x,y)
-s(x',y')\\textbar{} =\\textbar{} (x-x') +
(y -y')\\textbar{} \leq\\textbar{}
x-x'\\textbar{} +\\textbar{} y
-y'\\textbar{} \leq
2max~(\\textbar{}x-x'\\textbar{},\\textbar{}y
-y'\\textbar{}) = 2\\textbar{}(x,y) -
(x',y')\\textbar{}, ce qui montre que s est
2-lipschitzienne.

Soit (\lambda~0,x0) \in K \times E, \lambda~ \in K et x \in E. On a

\begin{align*} \\textbar{}p(\lambda~,x) -
p(\lambda~0,x0)\\textbar{}& =&
\\textbar{}\lambda~x -
\lambda~0x0\\textbar{} \%&
\\ & =& \\textbar{}\lambda~(x -
x0) + (\lambda~ -
\lambda~0)x0\\textbar{}\%&
\\ & \leq&
\textbar{}\lambda~\textbar{}\,\\textbar{}x -
x0\\textbar{} + \textbar{}\lambda~ -
\lambda~0\textbar{}\,\\textbar{}x0\\textbar{}
\%& \\ \end{align*}

Pour \eta \leq 1, on a \textbar{}\lambda~ - \lambda~0\textbar{} \textless{} \eta
\rigtharrow~\textbar{}\lambda~\textbar{}\leq\textbar{}\lambda~0\textbar{} + 1. Soit donc \epsilon
\textgreater{} 0 et \eta = max~(1, \epsilon
\over 2(1+\textbar{}\lambda~0\textbar{}) , \epsilon
\over
2(1+\\textbar{}x0\\textbar{})
). Alors

\begin{align*} \\textbar{}(\lambda~,x) -
(\lambda~0,x0)\\textbar{}
= max~(\textbar{}\lambda~ -
\lambda~0\textbar{},\\textbar{}x -
x0\\textbar{}) \textless{} \eta&&\%&
\\ & \rigtharrow~& \\textbar{}p(\lambda~,x)
- p(\lambda~0,x0)\\textbar{} \leq \epsilon
\over 2 + \epsilon \over 2 = \epsilon\%&
\\ \end{align*}

ce qui montre la continuité de p au point (\lambda~0,x0).

Corollaire~5.1.6 Soit X un espace métrique, E un espace vectoriel normé,
A une partie de X et a \in\overlineA. (i) Si f et g
sont deux fonctions de X vers E telles que A \subset~\
Def (f) \bigcap Def~ (g), si f et g ont toutes deux
des limites en a suivant A et si \alpha~,\beta~ \in K, alors \alpha~f + \beta~g a une limite en
a suivant A et on a

limx\rightarrow~a,x\inA~(\alpha~f(x) + \beta~g(x)) =
\alpha~limx\rightarrow~a,x\inA~f(x) +
\beta~limx\rightarrow~a,x\inA~g(x)

(ii) Si \phi est une fonction de X vers K et f une fonction de X vers E
telles que A \subset~ Def~ (f)
\bigcap Def~ (\phi), si f et \phi ont toutes deux des
limites en a suivant A, alors \phif a une limite en a suivant A et on a

limx\rightarrow~a,x\inA~(\phi(x)f(x))
=\
limx\rightarrow~a,x\inA\phi(x)limx\rightarrow~a,x\inA~f(x)

Remarque~5.1.3 Dans le cas de E = \mathbb{R}~, les opérations algébriques sur \mathbb{R}~ \times
\mathbb{R}~ ne peuvent pas s'étendre de manière continue à
\overline\mathbb{R}~ \times\overline\mathbb{R}~, ce qui
fait que certaines opérations sur les limites ne sont pas valides en
général. On a cependant le théorème suivant qui permet d'étendre les
opérations sur les limites sauf dans les cas d'indéterminations ''\infty~-\infty~''
et ''0 \times\infty~''

Théorème~5.1.7 (i) L'application s : \mathbb{R}~ \times \mathbb{R}~ \rightarrow~ \mathbb{R}~,
(x,y)\mapsto~x + y s'étend en une application
continue de \overline\mathbb{R}~ \times\overline\mathbb{R}~
\diagdown\(-\infty~,+\infty~),(+\infty~,-\infty~)\ dans
\overline\mathbb{R}~ en posant x + (+\infty~) = +\infty~ si
x\neq~ -\infty~ et x + (-\infty~) = -\infty~ si
x\neq~ + \infty~. (ii) L'application p : \mathbb{R}~ \times \mathbb{R}~ \rightarrow~ \mathbb{R}~,
(x,y)\mapsto~xy s'étend en une application continue
de

\overline\mathbb{R}~ \times\overline\mathbb{R}~
\diagdown\(0,+\infty~),(+\infty~,0),(0,-\infty~),(-\infty~,0)\

dans \overline\mathbb{R}~ en posant x \times (+\infty~)
= sgn(x)\infty~ si x\mathrel\neq~~0 et
x \times (-\infty~) = -sgn~(x)\infty~ si
x\neq~0.

Démonstration La vérification de la continuité est tout à fait
élémentaire. Remarquons que puisque \mathbb{R}~ \times \mathbb{R}~ est dense dans
\overline\mathbb{R}~ \times\overline\mathbb{R}~, ces
prolongements sont uniques.

{[}
{[}
