\textbf{Warning: 
requires JavaScript to process the mathematics on this page.\\ If your
browser supports JavaScript, be sure it is enabled.}

\begin{center}\rule{3in}{0.4pt}\end{center}

{[}
{[}
{[}{]}
{[}

\subsubsection{1.6 Polynômes à plusieurs variables}

\paragraph{1.6.1 Généralités}

Définition~1.6.1 Soit A un anneau commutatif.

\begin{align*}
A{[}X1,\\ldots,Xn~{]}
=& \\\sum
(k1,\ldots,kn)\in\mathbb{N}~^nak1,\\ldots,knX1^k1\\ldotsXn^kn\mathrel∣~&
\%& \\ & \textnombre
fini de
ak1,\\ldots,kn~\text
non nuls\ & \%& \\
\end{align*}

Remarque~1.6.1 On a bien entendu un isomorphisme naturel entre
A{[}X1,\\ldots,Xn~{]}
et
A{[}X1,\\ldots,Xn-1{]}{[}Xn~{]}
qui montre que si A est intègre, il en est de même de
A{[}X1,\\ldots,Xn~{]}.

Proposition~1.6.1 (règle de substitution). Soit A et B deux anneaux
commutatifs et \phi:A \rightarrow~ B un morphisme d'anneaux. Soit
\beta~1,\\ldots,\beta~n~
\in B. Alors l'application
T\phi,\beta~1,\\ldots,\beta~n:A{[}X1,\\\ldots,Xn~{]}
\rightarrow~ B,

\begin{align*} \\sum
(k1,\ldots,kn)\in\mathbb{N}~^nak1,\\ldots,knX1^k1~
\ldotsXn^kn~
&& \%& \\ & &
\mapsto~\\sum
(k1,\ldots,kn)\in\mathbb{N}~^n\phi(ak1,\\ldots,kn)\beta~1^k1~
\ldots\beta~n^kn~
\%& \\ \end{align*}

est un morphisme d'anneaux.

\paragraph{1.6.2 Dérivées partielles, formule de Taylor}

Définition~1.6.2 Soit P \in
A{[}X1,\\ldots,Xn~{]}.
On note  \partial~P \over \partial~Xi la dérivée de P dans
(A{[}X1,\\ldots,Xi-1,Xi+1,\\\ldots,Xn{]}){[}Xi~{]}.

Par simple calcul sur les monômes on montre alors

Lemme~1.6.2 (Schwarz) Soit P \in
A{[}X1,\\ldots,Xn~{]},
i,\\\\jmathmathmathmath \in {[}1,n{]}. Alors  \partial~ \over \partial~Xi ( \partial~P
\over \partial~X\\\\jmathmathmathmath ) = \partial~ \over
\partial~X\\\\jmathmathmathmath ( \partial~P \over \partial~Xi ).

Ceci permet de définir des dérivées partielles itérées 
\partial~^kP \over
\partial~X1^k1\\ldots\partial~Xn^kn~
si k = k1 +
\\ldots~ +
kn. On a alors

Théorème~1.6.3 (Formule de Taylor) Soit K un corps de caractéristique 0,
P \in
K{[}X1,\\ldots,Xn~{]}
et
(a1,\\ldots,an~)
\in K^n. Alors

\begin{align*} P(X1 +
a1,\\ldots,Xn~
+ an)&& \%& \\ & & =
\\sum
k1,\ldots,kn~
1 \over
k1!\ldotskn!~ 
\partial~^k1+\ldots+kn~P
\over
\partial~X1^k1\ldots\partial~Xn^kn~
(a1,\ldots,an)X1^k1~
\ldotsXn^kn~
\%& \\ \end{align*}

\paragraph{1.6.3 Degré total, polynômes homogènes}

Définition~1.6.3 On définit le degré d'un monôme non nul
aX1^k1\\ldotsXn^kn~
comme étant l'entier k1 +
\\ldots~ +
kn. On appelle degré d'un polynôme P non nul, le plus grand
des degrés de ses monômes non nuls. On dit qu'un polynôme P \in
K{[}X1,\\ldots,Xn~{]}
est homogène de degré p si tous ses monômes non nuls ont le même degré p
ou s'il est nul. On notera
Hp(X1,\\ldots,Xn~)
l'espace vectoriel des polynômes homogènes de degré p.

Théorème~1.6.4 On a Hp.Hq \subset~ Hp+q et
K{[}X1,\\ldots,Xn~{]}
=
\oplus~p\in\mathbb{N}~Hp(X1,\\ldots,Xn~)
(c'est-à-dire que tout polynôme s'écrit de manière unique comme somme
finie de polynômes homogènes de degrés distincts).

Démonstration La décomposition correspond tout simplement au
regroupement des termes de même degré au sein d'un polynôme homogène.
Cela montre à la fois l'existence et l'unicité de la décomposition.

Corollaire~1.6.5 Soit K un corps. On a deg~ PQ
= deg P +\ deg~ Q.

Démonstration On décompose P et Q en polynômes homogènes P =
Pm +
\\ldots~ et Q =
Qn +
\\ldots~ où
Pm et Qn sont les parties homogènes de plus haut
degré. Alors PmQn\neq~0 et
c'est la partie homogène de plus haut degré de PQ, d'où le résultat.

Théorème~1.6.6 (Euler). Soit K un corps commutatif de caractéristique 0
et P \in
K{[}X1,\\ldots,Xn~{]}.
On a équivalence de (i) P est homogène de degré p (ii)
\\sum ~
i=1^nXi \partial~P \over
\partial~Xi = pP.

Démonstration On pose D =\
\sum  i=1^nXi~ \partial~
\over \partial~Xi . On démontre (i) \rigtharrow~(ii) en
calculant sur les monômes et en utilisant la linéarité de D. On démontre
(ii) \rigtharrow~(i) en décomposant P en somme de polynômes homogènes~: si P =
Pm +
\\ldots~ +
P0, on a pPm +
\\ldots~ +
pP0 = pP = DP = DPm +
\\ldots~ +
DP0 = mPm +
\\ldots~ +
0P0. Par unicité de la décomposition en polynômes homogènes,
on a pour tout k, pPk = kPk ce qui exige
Pk = 0 si k\neq~p. Finalement P =
Pp est homogène de degré p.

\paragraph{1.6.4 Polynômes symétriques}

Définition~1.6.4 On dit que P \in
K{[}X1,\\ldots,Xn~{]}
est symétrique si pour toute permutation \sigma on a

P(X\sigma(1),\\ldots,X\sigma(n)~)
=
P(X1,\\ldots,Xn~)

Exemple~1.6.1 Pour 1 \leq k \leq n, on définit les polynômes symétriques
élémentaires à n variables
\sigmak(X1,\\ldots,Xn~)
= \\sum ~
1\leqi1\textless{}⋯\textless{}ik\leqnXi1\\ldotsXin~
(homogène de degré k, Cn^k monômes). Ces polynômes
symétriques vérifient la formule de récurrence (séparer les termes ne
contenant pas Xn de ceux contenant Xn)~:

\sigmak(X1,\\ldots,Xn~)
=
\sigmak(X1,\\ldots,Xn-1~)
+
\sigmak-1(X1,\\ldots,Xn-1)Xn~

Théorème~1.6.7

\begin{align*} \∏
i=1^n(T - X i)&& \%&
\\ & =& T^n
-\\sum
k=1^n(-1)^k\sigma
k(X1,\ldots,Xn)T^n-k~
\%& \\ & =& T^n - \sigma
1(X1,\\ldots,Xn)T^n-1~
+ \\ldots~ +
(-1)^n\sigma
n(X1,\\ldots,Xn~)\%&
\\ \end{align*}

Démonstration Par récurrence sur n en utilisant la formule de récurrence
vérifiée par les \sigmak.

Corollaire~1.6.8 Soit P \in K{[}X{]} scindé sur K. On peut donc écrire

P(X) = anX^n +
\\ldots + a~
0 = an ∏ i=1^n~(X
- \alpha~ i)

Alors on a, \forall~~k \in {[}1,n{]},
\sigmak(\alpha~1,\\ldots,\alpha~n~)
= (-1)^k an-k \over an
.

On admettra le résultat suivant

Théorème~1.6.9 Soit P \in
K{[}X1,\\ldots,Xn~{]}
un polynôme symétrique. Il existe un unique polynôme Q \in
K{[}X1,\\ldots,Xn~{]}
tel que P =
Q(\sigma1,\\ldots,\sigman~).

{[}
{[}
{[}
{[}
