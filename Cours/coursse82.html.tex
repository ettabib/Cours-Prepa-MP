% \textbf{Warning: 
% requires JavaScript to process the mathematics on this page.\\ If your
% browser supports JavaScript, be sure it is enabled.}

% \begin{center}\rule{3in}{0.4pt}\end{center}

% {[}
% {[}{]}
% {[}

% \subsubsection{15.1 Dérivées partielles}

% \paragraph{15.1.1 Notion de dérivée partielle}

% Définition~15.1.1 Soit E et F deux espaces vectoriels normés. Soit U un
% ouvert de E, f : U \rightarrow~ F, a \in U. Soit v \in E
% \diagdown\0\. On dit que f admet au point a
% une dérivée partielle suivant le vecteur v si l'application
% t\mapsto~f(a + tv) (définie sur un voisinage de 0)
% est dérivable au point 0.

% Remarque~15.1.1 L'existence de la dérivée partielle en a suivant le
% vecteur v est donc équivalente à l'existence de
% limt\rightarrow~0~ f(a+tv)-f(a)
% \over t = \partial~vf(a). Remarquons que si v' = \lambda~v,
% \lambda~\neq~0, alors  f(a+tv')-f(a)
% \over t = \lambda~ f(a+uv)-f(a) \over u
% avec u = \lambda~t ce qui montre que f admet en a une dérivée partielle selon v
% si et seulement si f admet une dérivée partielle suivant \lambda~v et qu'alors
% \partial~\lambda~vf(a) = \lambda~\partial~vf(a).

% Exemple~15.1.1 Soit f : \mathbb{R}~^2 \rightarrow~ \mathbb{R}~ définie par f(x,y) =
% x^2 \over y si
% y\neq~0 et f(x,0) = 0. Soit v =
% (a,b)\neq~(0,0). On a  f((0,0)+tv)-f(0,0)
% \over t = \left \
% \cases 0 &si b = 0 \cr 
% a^2 \over b &si
% b\neq~0  \right .. On en déduit
% que f admet une dérivée partielle suivant tout vecteur v et que
% \partial~vf(0,0) = \left \
% \cases 0 &si b = 0 \cr 
% a^2 \over b &si
% b\neq~0  \right .. Remarquons
% que l'application v\mapsto~\partial~vf(0,0) n'est
% pas linéaire. Remarquons également que f n'est pas continue en (0,0)
% (puisque limt\rightarrow~0f(t,t^2~) =
% 1\neq~f(0,0)). L'existence de dérivée partielle
% suivant tout vecteur n'implique donc pas la continuité.

% Proposition~15.1.1 On a les propriétés évidentes de la dérivation de
% t\mapsto~f(a + tv) à savoir (i) si f et g admettent
% en a une dérivée partielle suivant le vecteur v, il en est de même de \alpha~f
% + \beta~g et \partial~v(\alpha~f + \beta~g)(a) = \alpha~\partial~vf(a) +
% \beta~\partial~vg(a). (ii) si f et g (à valeurs scalaires) admettent en a
% une dérivée partielle suivant le vecteur v, il en est de même de fg et
% \partial~v(fg)(a) = g(a)\partial~vf(a) + f(a)\partial~vg(a).

% Remarque~15.1.2 Par contre, on n'a pas de théorème général de
% composition des dérivées partielles. En reprenant l'exemple ci dessus, f
% : \mathbb{R}~^2 \rightarrow~ \mathbb{R}~ définie par f(x,y) = x^2
% \over y si y\neq~0 et f(x,0) =
% 0, l'application f admet en (0,0) une dérivée partielle suivant tout
% vecteur, l'application t\mapsto~(t,t^2)
% est dérivable en 0 et pourtant
% t\mapsto~f(t,t^2) n'est pas dérivable en
% 0 (elle n'y est même pas continue).

% Définition~15.1.2 Soit E un espace vectoriel normé de dimension finie, \mathcal{E}
% =
% (e1,\\ldots,en~)
% une base de E, F un espace vectoriel normé. Soit U un ouvert de E, f : U
% \rightarrow~ F, a \in U. On dit que f admet au point a une dérivée partielle d'indice
% i (suivant la base \mathcal{E}) si elle admet une dérivée partielle suivant le
% vecteur ei. On note alors  \partial~f \over
% \partial~xi (a) = \partial~eif(a).

% Exemple~15.1.2 Si E = \mathbb{R}~^n et si \mathcal{E} est la base canonique de
% \mathbb{R}~^n, l'existence d'une dérivée partielle d'indice i au point
% a =
% (a1,\\ldots,an~)
% équivaut à la dérivabilité au point ai de l'application
% partielle
% xi\mapsto~f(a1,\\ldots,ai-1,xi,ai+1,\\\ldots,an~).
% On retrouve bien la notion habituelle de dérivée partielle~: dérivée
% suivant la variable xi, toutes les autres étant considérées
% comme constantes.

% \paragraph{15.1.2 Composition des dérivées partielles}

% On a vu précédemment qu'on n'avait pas de théorème de composition des
% dérivées partielles en toute généralité. On va introduire une notion
% d'application de classe \mathcal{C}^1.

% Définition~15.1.3 Soit U un ouvert de \mathbb{R}~^n, f : U \rightarrow~ F. On dit
% que f est de classe \mathcal{C}^1 au point a si, sur un certain
% voisinage V de a, f admet des dérivées partielles de tout indice i \in
% {[}1,n{]} et si ces dérivées partielles x\mapsto~
% \partial~f \over \partial~xi (x) sont continues au point a.

% Lemme~15.1.2 Soit F un espace vectoriel de dimension finie, V un ouvert
% de \mathbb{R}~^n, f : V \rightarrow~ F. Soit I un intervalle de \mathbb{R}~, t \in I et \phi =
% (\phi1,\\ldots,\phin~)
% : I \rightarrow~ V . On suppose que \phi est dérivable au point t et que f est de
% classe \mathcal{C}^1 au point \phi(t). Alors f \cdot \phi est dérivable au point
% t et (f \cdot \phi)'(t) =\ \\sum
%  i=1^n \partial~f \over \partial~xi
% (\phi(t))\phii'(t).

% Démonstration Sans nuire à la généralité, en prenant une base (sur \mathbb{R}~) de
% F et en travaillant composante par composante, on peut supposer que f
% est à valeurs réelles. On écrit

% \begin{align*} f(\phi(t + h)) - f(\phi(t))&& \%&
% \\ & =& f(\phi1(t +
% h),\\ldots,\phin~(t
% + h)) -
% f(\phi1(t),\\ldots,\phin~(t))
% \%& \\ & =& \\sum
% i=1^n(f(\ldots,\phi~
% i-1(t),\phii(t + h),\phii+1(t +
% h),\ldots~)\%&
% \\ & & \qquad -
% f(\\ldots,\phii-1(t),\phii(t),\phii+1~(t
% + h),\\ldots~)) \%&
% \\ \end{align*}

% Mais \phi est continue au point t et donc pour h assez petit, tous les
% (\phi1(t),\\ldots,\phii-1(t),\phii~(t
% + h),\phii+1(t +
% h),\\ldots,\phin~(t
% + h)) se trouvent à l'intérieur d'une boule de centre a sur laquelle les
% dérivées partielles de f de tout indice existent. En particulier,
% l'application
% xi\mapsto~f(\phi1(t),\\ldots,\phii-1(t),xi,\phii+1~(t
% +
% h),\\ldots,\phin~(t
% + h)) est dérivable sur le segment {[}\phii(t),\phii(t +
% h){]} et on peut appliquer le théorème des accroissements finis. On
% obtient l'existence d'un \xi~i \in
% {[}\phii(t),\phii(t + h){]} tel que

% \begin{align*}
% f(\phi1(t),\\ldots,\phii-1(t),\phii~(t
% + h),\phii+1(t +
% h),\\ldots,\phin~(t
% + h))&& \%& \\ & &
% -f(\phi1(t),\\ldots,\phii-1(t),\phii(t),\phii+1~(t
% +
% h),\\ldots,\phin~(t
% + h))\%& \\ & =& (\phii(t + h) -
% \phii(t)) \%& \\ & &
% \quad  \partial~f \over \partial~xi
% (\phi1(t),\\ldots,\phii-1(t),\xi~i,\phii+1~(t
% +
% h),\\ldots,\phin~(t
% + h)) \%& \\
% \end{align*}

% Comme \phi est continue au point t et \xi~i \in
% {[}\phii(t),\phii(t + h){]}, on a

% \begin{align*}
% limh\rightarrow~0(\phi1(t),\\\ldots,\phii-1(t),\xi~i,\phii+1~(t
% +
% h),\\ldots,\phin~(t
% + h))&&\%& \\ & =&
% (\phi1(t),\\ldots,\phii-1(t),\phii(t),\phii+1(t),\\\ldots,\phin~(t))\%&
% \\ & =& \phi(t) \%&
% \\ \end{align*}

% et comme  \partial~f \over \partial~xi est continue au
% point \phi(t), on a

% \begin{align*} \partial~f \over
% \partial~xi (\phi(t))&& \%& \\ & =&
% limh\rightarrow~0~ \partial~f \over
% \partial~xi
% (\phi1(t),\\ldots,\phii-1(t),\xi~i,\phii+1~(t
% +
% h),\\ldots,\phin~(t
% + h))\%& \\
% \end{align*}

% Il suffit alors de diviser par h et de faire tendre h vers 0 pour voir
% que

% limh\rightarrow~0~ f(\phi(t + h)) - f(\phi(t))
% \over h = \\sum
% i=1^n \partial~f \over \partial~xi
% (\phi(t))\phii'(t)

% ce qui achève la démonstration.

% Appliquant ce lemme à \phi : t\mapsto~g(a + tv) au
% point t = 0, on obtient le théorème suivant

% Théorème~15.1.3 Soit F un espace vectoriel de dimension finie, U un
% ouvert de \mathbb{R}~^n, f : U \rightarrow~ F. Soit E un espace vectoriel normé, V
% un ouvert de E et g =
% (g1,\\ldots,gn~)
% : V \rightarrow~ U \subset~ \mathbb{R}~^n. Soit a \in V et v \in E
% \diagdown\0\. Si g admet en a une dérivée
% partielle suivant le vecteur v et si f est de classe \mathcal{C}^1 au
% point a, alors f \cdot g admet en a une dérivée partielle suivant le vecteur
% v et on a

% \partial~v(f \cdot g)(a) = \\sum
% i=1^n \partial~f \over \partial~xi
% (g(a))\partial~vgi(a)

% Dans le cas particulier où E = \mathbb{R}~^p et où on prend pour v le
% \\\\jmathmathmathmath-ième vecteur de la base canonique, on obtient la version suivante (on
% a changé le nom des variables pour les appeler
% y1,\\ldots,yn~
% dans \mathbb{R}~^n).

% Corollaire~15.1.4 Soit F un espace vectoriel de dimension finie, U un
% ouvert de \mathbb{R}~^n, f : U \rightarrow~ F. Soit V un ouvert de \mathbb{R}~^p
% et g =
% (g1,\\ldots,gn~)
% : V \rightarrow~ U \subset~ \mathbb{R}~^n. Soit a \in V et \\\\jmathmathmathmath \in {[}1,p{]}. Si g admet en a
% une dérivée partielle d'indice \\\\jmathmathmathmath et si f est de classe \mathcal{C}^1 au
% point a, alors f \cdot g admet en a une dérivée partielle d'indice \\\\jmathmathmathmath et on a

%  \partial~(f \cdot g) \over \partial~x\\\\jmathmathmathmath (a) =
% \sum i=1^n~ \partial~f
% \over \partial~yi (g(a)) \partial~gi
% \over \partial~x\\\\jmathmathmathmath (a)

% Remarque~15.1.3 On en déduit immédiatement que la composée de deux
% applications de classe \mathcal{C}^1 est encore de classe
% \mathcal{C}^1.

% Citons aussi le corollaire suivant du lemme, où l'on prend \phi(t) = a + tv

% Corollaire~15.1.5 Soit F un espace vectoriel de dimension finie, V un
% ouvert de \mathbb{R}~^n, f : V \rightarrow~ F. Soit a \in V . Si f est de classe
% \mathcal{C}^1 au point a, alors elle admet en a des dérivées partielles
% suivant tout vecteur et on a

% \partial~vf(a) = \\sum
% i=1^nv i \partial~f \over
% \partial~xi (a)\qquad \text si v =
% (v1,\ldots,vn~)

% Remarque~15.1.4 On voit que dans ce cas
% v\mapsto~\partial~vf(a) est linéaire. Cette
% remarque nous conduira à la définition de la différentielle dans la
% section suivante.

% \paragraph{15.1.3 Théorème des accroissements finis et applications}

% Théorème~15.1.6 Soit U un ouvert de \mathbb{R}~^n, f : U \rightarrow~ \mathbb{R}~ de classe
% \mathcal{C}^1. Soit a \in U et h \in \mathbb{R}~^n tel que {[}a,a + h{]} \subset~
% U. Alors, il existe \theta \in{]}0,1{[} tel que

% f(a + h) - f(a) = \\sum
% i=1^nh i \partial~f \over
% \partial~xi (a + \thetah)

% Démonstration Soit \psi : {[}0,1{]} \rightarrow~ \mathbb{R}~ définie par \psi(t) = f(a + th). Le
% lemme du paragraphe précédent montre que \psi est dérivable sur {[}0,1{]}
% et que \psi'(t) = \\sum ~
% i=1^nhi \partial~f \over
% \partial~xi (a + th). Le théorème des accroissements finis assure
% qu'il existe \theta \in{]}0,1{[} tel que \psi(1) - \psi(0) = (1 - 0)\psi'(\theta), ce qui
% n'est autre que la formule ci dessus.

% Corollaire~15.1.7 Soit U un ouvert de \mathbb{R}~^n, F un espace
% vectoriel de dimension finie, f : U \rightarrow~ F de classe \mathcal{C}^1. Alors
% f est continue.

% Démonstration En prenant une base (sur \mathbb{R}~) de F et en travaillant
% composante par composante, on peut supposer que f est à valeurs réelles.
% Puisque les dérivées partielles sont continues au point a, il existe \eta
% \textgreater{} 0 tel que B(a,\eta) \subset~ U et \forall~~x \in
% B(0,\eta), \textbar{} \partial~f \over \partial~xi (x) - \partial~f
% \over \partial~xi (a)\textbar{}\leq 1, d'où \textbar{}
% \partial~f \over \partial~xi (x)\textbar{}\leq 1 + \textbar{}
% \partial~f \over \partial~xi (a)\textbar{}. Pour
% \\textbar{}h\\textbar{} \textless{} \eta, on
% a alors {[}a,a + h{]} \subset~ B(0,\eta) et donc \textbar{}f(a + h) -
% f(a)\textbar{}\leq\\sum ~
% i=1^n\textbar{}hi\textbar{}\,\left
% (\left \textbar{} \partial~f \over
% \partial~xi (a)\right \textbar{} +
% 1\right ), ce qui montre la continuité de f au point a.

% Corollaire~15.1.8 Soit U un ouvert connexe de \mathbb{R}~^n, F un
% espace vectoriel de dimension finie, f : U \rightarrow~ F. Alors f est constante si
% et seulement si~elle est de classe \mathcal{C}^1 et toutes ses dérivées
% partielles sont nulles.

% Démonstration Si f est constante, il est clair qu'elle est de classe
% \mathcal{C}^1 et que toutes ses dérivées partielles sont nulles. Pour
% la réciproque, en prenant une base (sur \mathbb{R}~) de F et en travaillant
% composante par composante, on peut supposer que f est à valeurs réelles.
% Soit x0 \in U et soit X = \x \in
% U∣f(x) = f(x0)\.
% Puisque f est continue (d'après le corollaire précédent), X est un fermé
% de U, évidemment non vide. Montrons que X est également ouvert dans U~;
% soit en effet x1 dans X et soit \eta \textgreater{} 0 tel que
% B(x1,\eta) \subset~ U. Pour
% \\textbar{}h\\textbar{} \textless{} \eta, on
% a {[}x1,x1 + h{]} \subset~ B(x1,\eta) \subset~ U et le
% théorème des accroissements finis nous donne f(x1 + h) =
% f(x1) = f(x0), les dérivées partielles étant
% supposées nulles. On a donc B(x1,\eta) \subset~ X, et donc X est ouvert.
% Comme X est à la fois ouvert et fermé, non vide dans U connexe, on a X =
% U et donc f est constante.

% \paragraph{15.1.4 Dérivées partielles successives}

% On définit la notion de dérivées partielles successives de manière
% récursive de la manière suivante

% Définition~15.1.4 Soit U un ouvert de \mathbb{R}~^n, a \in U et f : U \rightarrow~
% E. Soit
% (i1,\\ldots,ik~)
% \in {[}1,n{]}^k. On dit que  \partial~^kf
% \over
% \partial~xi1\\ldots\partial~xik~
% (a) existe s'il existe un ouvert V tel que a \in V \subset~ U et sur lequel 
% \partial~^k-1f \over
% \partial~xi2\\ldots\partial~xik~
% (x) existe et si l'application x\mapsto~
% \partial~^k-1f \over
% \partial~xi2\\ldots\partial~xik~
% (x) admet une dérivée partielle d'indice i1. On pose alors

%  \partial~^kf \over
% \partial~xi1\\ldots\partial~xik~
% (a) = \partial~ \over \partial~xi1
% \left ( \partial~^k-1f \over
% \partial~xi2\\ldots\partial~xik~
% \right )(a)

% Définition~15.1.5 Soit U un ouvert de \mathbb{R}~^n et f : U \rightarrow~ E. On
% dit que f est de classe C^k sur U si,
% \forall~(i1,\\\ldots,ik~)
% \in {[}1,n{]}^k, l'application x\mapsto~
% \partial~^kf \over
% \partial~xi1\\ldots\partial~xik~
% (x) est définie et continue sur U.

% Remarque~15.1.5 Comme on a vu que toute application de classe
% \mathcal{C}^1 est continue, on en déduit immédiatement que toute
% application de classe C^k est aussi de classe
% C^k-1. On dira bien entendu que f est de classe
% C^\infty~ si elle est de classe C^k pour tout k. Une
% récurrence évidente sur k montre que la composée de deux applications de
% classe C^k est encore de classe C^k et que donc la
% composée de deux applications de classe C^\infty~ est encore de
% classe C^\infty~.

% Lemme~15.1.9 Soit U un ouvert de \mathbb{R}~^2, f : U \rightarrow~ \mathbb{R}~ de classe
% C^2. Alors,  \partial~^2f \over
% \partial~x1\partial~x2 = \partial~^2f \over
% \partial~x2\partial~x1 .

% Démonstration Soit (a1,a2) \in U et soit

% \begin{align*} \phi(h1,h2)& =&
% 1 \over h1h2 (f(a1 +
% h1,a2 + h2) - f(a1 +
% h1,a2)\%& \\ & &
% \quad \quad \quad -
% f(a1,a2 + h2) +
% f(a1,a2)) \%& \\
% \end{align*}

% définie pour h1 et h2 non nuls et assez petits. On a
% \phi(h1,h2) = 1 \over
% h1h2 \psi1(a1 + h1) -
% \psi1(a1) avec \psi1(x1) =
% f(x1,a2 + h2) -
% f(x1,a2). Or \psi1 est dérivable sur
% {[}a1,a1 + h1{]} avec
% \psi1'(x1) = \partial~f \over \partial~x1
% (x1,a2 + h2) - \partial~f \over
% \partial~x1 (x1,a2). On peut donc appliquer le
% théorème des accroissements finis, et donc il existe \xi~1 \in
% {[}a1,a1 + h1{]} tel que

% \begin{align*} \phi(h1,h2)& =&
% 1 \over h2 \psi1'(\xi~1) \%&
% \\ & =& 1 \over
% h2 \left ( \partial~f \over
% \partial~x1 (\xi~1,a2 + h2) - \partial~f
% \over \partial~x1
% (\xi~1,a2)\right )\%&
% \\ & =& \partial~^2f
% \over \partial~x2\partial~x1
% (\xi~1,\xi~2) \%& \\
% \end{align*}

% avec \xi~2 \in {[}a2,a2 + h2{]} en
% appliquant le théorème des accroissements finis à
% x2\mapsto~ \partial~f \over
% \partial~x1 (\xi~1,x2) qui est dérivable sur
% {[}a2,a2 + h2{]}, de dérivée 
% \partial~^2f \over \partial~x2\partial~x1
% (\xi~1,x2). Quand h1 et h2 tendent
% vers 0, \xi~1 et \xi~2 tendent respectivement vers
% a1 et a2 et la continuité de  \partial~^2f
% \over \partial~x2\partial~x1 montre que
% lim(h1,h2)\rightarrow~(0,0)\phi(h1,h2~)
% = \partial~^2f \over \partial~x2\partial~x1
% (a1,a2). Comme les deux variables \\\\jmathmathmathmathouent un rôle
% symétrique dans la définition de \phi, en posant \psi2(x2)
% = f(a1 + h1,x2) -
% f(a1,x2) et en appliquant deux fois le théorème des
% accroissements finis, on obtient
% lim(h1,h2)\rightarrow~(0,0)\phi(h1,h2~)
% = \partial~^2f \over \partial~x1\partial~x2
% (a1,a2), ce qui démontre que  \partial~^2f
% \over \partial~x1\partial~x2
% (a1,a2) = \partial~^2f \over
% \partial~x2\partial~x1 (a1,a2).

% Théorème~15.1.10 (Schwarz). Soit U un ouvert de \mathbb{R}~^n et f : U
% \rightarrow~ E (espace vectoriel normé de dimension finie) de classe
% C^2. Alors \forall~~(i,\\\\jmathmathmathmath) \in
% {[}1,n{]}^2,

%  \partial~^2f \over \partial~xi\partial~x\\\\jmathmathmathmath =
% \partial~^2f \over \partial~x\\\\jmathmathmathmath\partial~xi

% Démonstration En prenant une base de E, on peut se contenter de montrer
% le résultat lorsque E = \mathbb{R}~. Si i = \\\\jmathmathmathmath, le résultat est évident. Supposons
% i \textless{} \\\\jmathmathmathmath et soit
% (a1,\\ldots,an~)
% \in \mathbb{R}~^n. On applique le lemme précédent à l'application de
% classe C^2, définie sur un ouvert contenant
% (ai,a\\\\jmathmathmathmath),

% g(xi,x\\\\jmathmathmathmath) =
% f(a1,\\ldots,ai-1,xi,ai+1,\\\ldots,a\\\\jmathmathmathmath-1,x\\\\jmathmathmathmath,a\\\\jmathmathmathmath+1,\\\ldots,an~)

% qui est de classe C^2 (composée d'applications de classe
% C^2). On a donc  \partial~^2g \over
% \partial~xi\partial~x\\\\jmathmathmathmath (ai,a\\\\jmathmathmathmath) =
% \partial~^2g \over \partial~x\\\\jmathmathmathmath\partial~xi
% (ai,a\\\\jmathmathmathmath), soit encore

%  \partial~^2f \over \partial~xi\partial~x\\\\jmathmathmathmath
% (a1,\\ldots,an~)
% = \partial~^2f \over \partial~x\\\\jmathmathmathmath\partial~xi
% (a1,\\ldots,an~)

% Corollaire~15.1.11 Soit U un ouvert de \mathbb{R}~^n et f : U \rightarrow~ E de
% classe C^k. Soit
% (i1,\\ldots,ik~)
% \in {[}1,n{]}^k. Pour toute permutation \sigma de {[}1,k{]} on a

%  \partial~^kf \over
% \partial~xi\sigma(1)\\ldots\partial~xi\sigma(k)~
% = \partial~^kf \over
% \partial~xi1\\ldots\partial~xik~

% Démonstration D'après le théorème de Schwarz, le résultat est vrai
% lorsque \sigma = \tau\\\\jmathmathmathmath,\\\\jmathmathmathmath+1 est la transposition qui échange \\\\jmathmathmathmath et \\\\jmathmathmathmath +
% 1. Mais toute permutation de {[}1,k{]} est un produit de telles
% transpositions (facile) ce qui démontre le corollaire.

% Notation définitive Soit
% (i1,\\ldots,ik~)
% \in {[}1,n{]}^k. Pour \\\\jmathmathmathmath \in {[}1,n{]}, soit k\\\\jmathmathmathmath le
% nombre de iq qui sont égaux à \\\\jmathmathmathmath. On a donc à une permutation
% près, la famille
% (i1,\\ldots,ik~)
% qui est égale à
% (\overbrace1,\\ldots,1k1~
% fois,\\ldots,\overbrace\\\\jmathmathmathmath,\\\ldots,\\\\jmathmathmathmath~
% k\\\\jmathmathmathmath
% fois,\\ldots,\overbracen,\\\ldots,nkn~
% fois), chaque \\\\jmathmathmathmath étant compté k\\\\jmathmathmathmath fois. En notant
% \partial~x\\\\jmathmathmathmath^k\\\\jmathmathmathmath à la place de
% \overbrace\partial~x\\\\jmathmathmathmath\\ldots\partial~x\\\\jmathmathmathmath~
% k\\\\jmathmathmathmath fois, on obtient

%  \partial~^kf \over
% \partial~xi1\\ldots\partial~xik~
% = \partial~^kf \over
% \partial~x1^k1\\ldots\partial~xn^kn~

% \paragraph{15.1.5 Formules de Taylor}

% Lemme~15.1.12 Soit U un ouvert de \mathbb{R}~^n et f : U \rightarrow~ E de classe
% C^k. Soit a \in U et h \in \mathbb{R}~^n tel que {[}a,a + h{]} \subset~
% U. Posons \phi(t) = f(a + th), définie et de classe C^k sur
% {[}0,1{]}. Alors, pour tout t \in {[}0,1{]},

% \begin{align*} \phi^(k)(t) =
% \\sum
% k1+\ldots+kn=k~
% k! \over
% k1!\ldotskn!~
% h1^k1
% \ldotshn^kn ~
% \partial~^kf \over
% \partial~x1^k1\ldots\partial~xn^kn~
% (a + th)& & \%& \\
% \end{align*}

% Démonstration Par récurrence sur k. Pour k = 1, ce n'est qu'une autre
% formulation du résultat

% \begin{align*} \phi'(t)& =&
% \sum i=1^nh i~ \partial~f
% \over \partial~xi (a + th) \%&
% \\ & =& \\sum
% k1+\ldots+kn=1h1^k1~
% \ldotshn^kn ~
% \partial~f \over
% \partial~x1^k1\ldots\partial~xn^kn~
% (a + th)\%& \\
% \end{align*}

% en posant ki = 1 et k\\\\jmathmathmathmath = 0 pour
% i\neq~\\\\jmathmathmathmath.

% Supposons le résultat démontré pour k - 1. On a donc

% \begin{align*} \phi^(k-1)(t) =&& \%&
% \\ & & \\sum
% k1+\ldots+kn=k-1~
% (k - 1)! \over
% k1!\ldotskn!~
% h1^k1
% \ldotshn^kn ~
% \partial~^k-1f \over
% \partial~x1^k1\ldots\partial~xn^kn~
% (a + th)\%& \\
% \end{align*}

% On en déduit que

% \begin{align*} \phi^(k)(t) =&& \%&
% \\ & & \\sum
% k1+\ldots+kn=k-1~
% (k - 1)! \over
% k1!\ldotskn!~
% h1^k1
% \ldotshn^kn ~
% d \over dt \left ( \partial~^k-1f
% \over
% \partial~x1^k1\ldots\partial~xn^kn~
% (a + th)\right )\%& \\
% \end{align*}

% soit encore

% \begin{align*} \phi^(k)(t)& =&
% \\sum
% k1+\ldots+kn=k-1~
% (k - 1)! \over
% k1!\ldotskn!~
% h1^k1
% \ldotshn^kn ~
% \%& \\ & & \quad
% \quad  \\sum
% i=1^nh i \partial~^kf
% \over
% \partial~x1^k1\ldots\partial~xi^ki+1\\ldots\partial~xn^kn~
% (a + th)\%& \\
% \end{align*}

% En intervertissant les deux signes de somme on obtient

% \begin{align*} \phi^(k)(t)& =&
% \sum i=1^n~
% \\sum
% k1+\ldots+kn=k-1~
% (k - 1)!(ki + 1) \over
% k1!\ldots(ki~ +
% 1)!\ldotskn!~ \%&
% \\ & & \quad
% \quad h1^k1
% \\ldotshi^ki+1\\\ldotsh~
% n^kn  \partial~^kf \over
% \partial~x1^k1\\ldots\partial~xi^ki+1\\\ldots\partial~xn^kn~
% (a + th)\%& \\
% \end{align*}

% et en faisant un changement d'indice

% \begin{align*} \phi^(k)(t)& =&
% \sum i=1^n~
% \sum ~
% k1+\ldots+kn~=k
% \atop ki≥1  (k - 1)!ki
% \over
% k1!\ldotskn!~ \%&
% \\ & & \quad
% \quad h1^k1
% \\ldotshi^ki~
% \\ldotshn^kn~
%  \partial~^kf \over
% \partial~x1^k1\\ldots\partial~xi^ki\\\ldots\partial~xn^kn~
% (a + th)\%& \\
% \end{align*}

% Réintroduisons les termes pour ki = 0 qui sont nuls puisqu'ils
% contiennent le facteur (k - 1)!ki, on obtient

% \begin{align*} \phi^(k)(t)& =&
% \sum i=1^n~
% \\sum
% k1+\ldots+kn=k~
% (k - 1)!ki \over
% k1!\ldotskn!~
% h1^k1
% \ldotshi^ki~
% \ldotshn^kn~
% \%& \\ & & \quad
% \quad \quad  \partial~^kf
% \over
% \partial~x1^k1\\ldots\partial~xi^ki\\\ldots\partial~xn^kn~
% (a + th) \%& \\
% \end{align*}

% Ceci nous permet de réintervertir les deux sommations, soit encore,
% après mise en facteur

% \begin{align*} \phi^(k)(t)& =&
% \\sum
% k1+\ldots+kn=k~
% (k - 1)!\sum i=1^nki~
% \over
% k1!\ldotskn!~
% h1^k1
% \ldotshi^ki~
% \ldotshn^kn~
% \%& \\ & & \quad
% \quad \quad  \partial~^kf
% \over
% \partial~x1^k1\\ldots\partial~xi^ki\\\ldots\partial~xn^kn~
% (a + th) \%& \\
% \end{align*}

% soit encore

% \begin{align*} \phi^(k)(t)& =&
% \\sum
% k1+\ldots+kn=k~
% k! \over
% k1!\ldotskn!~
% h1^k1
% \ldotshn^kn ~
% \partial~^kf \over
% \partial~x1^k1\ldots\partial~xn^kn~
% (a + th)\%& \\
% \end{align*}

% ce qui achève la récurrence.

% Remarque~15.1.6 Cette formule est tout à fait analogue à la formule du
% binôme généralisée

% (X1 +
% \\ldots~ +
% Xn)^k = \\sum
% k1+\ldots+kn=k~
% k! \over
% k1!\ldotskn!~
% X1^k1
% \ldotsXn^kn ~

% Cette remarque nous conduira à une notation plus compacte. Introduisons
% un produit symbolique sur les expressions du type
% h1^k1\\ldotshn^kn~
% \partial~^k \over
% \partial~x1^k1\\ldots\partial~xn^kn~
% en posant

% \begin{align*} \left
% (h1^k1
% \\ldotshn^kn~
%  \partial~^k \over
% \partial~x1^k1\\ldots\partial~xn^kn~
% \right ) ∗\left
% (h1^l1
% \\ldotshn^ln~
%  \partial~^l \over
% \partial~x1^l1\\ldots\partial~xn^ln~
% \right ) =&&\%& \\ & &
% h1^k1+l1
% \\ldotshn^kn+ln~
%  \partial~^k+l \over
% \partial~x1^k1+l1\\ldots\partial~xn^kn+ln~
% \quad \quad \quad \%&
% \\ \end{align*}

% Ce produit est commutatif, et

% \begin{align*} \\sum
% k1+\ldots+kn=k~
% k! \over
% k1!\ldotskn!~
% h1^k1
% \ldotshn^kn ~
% \partial~^k \over
% \partial~x1^k1\ldots\partial~xn^kn~
% &&\%& \\ & & = \left
% (h1 \partial~ \over \partial~x1 +
% \\ldots~ +
% hn \partial~ \over \partial~xn
% \right )^k∗\quad
% \quad \quad \%&
% \\ \end{align*}

% où la notation ^k∗ désigne la puissance k-ième pour ce
% produit commutatif. La formule s'écrit alors de manière plus agréable
% sous la forme

% \phi^(k)(t) = \left (h 1 \partial~
% \over \partial~x1 +
% \\ldots~ +
% hn \partial~ \over \partial~xn
% \right )^k∗f(a + th)

% Ces puissances se développent de la manière évidente en respectant la
% règle de calcul pour le produit ∗.

% Exemple~15.1.3 \phi'(t) = \left (h1 \partial~
% \over \partial~x1 +
% \\ldots~ +
% hn \partial~ \over \partial~xn
% \right )f(a + th)

% \begin{align*} \phi''(t)& =& \left
% (h1 \partial~ \over \partial~x1 +
% \\ldots~ +
% hn \partial~ \over \partial~xn
% \right )^2∗f(a + th) \%&
% \\ & =& \\sum
% i=1^nh i^2 \partial~^2f
% \over \partial~xi^2 (a + th) +
% 2\\sum
% i\textless{}\\\\jmathmathmathmathhih\\\\jmathmathmathmath \partial~^2f
% \over \partial~xi\partial~x\\\\jmathmathmathmath (a + th)\%&
% \\ \end{align*}

% et ainsi de suite.

% Théorème~15.1.13 (formule de Taylor avec reste intégral). Soit U un
% ouvert de \mathbb{R}~^n et f : U \rightarrow~ E de classe C^k+1. Soit a
% \in U et h \in \mathbb{R}~^n tel que {[}a,a + h{]} \subset~ U. Alors

% \begin{align*} f(a + h)& =& f(a) +
% \sum p=1^k~ 1
% \over p! \left (h1 \partial~
% \over \partial~x1 +
% \ldots + hn~ \partial~
% \over \partial~xn \right
% )^p∗f(a)\%& \\
% +\int  0^1~ (1 -
% t)^k \over k!  \left
% (h1 \partial~ \over \partial~x1 +
% \\ldots~ +
% hn \partial~ \over \partial~xn
% \right )^(k+1)∗f(a + th) dt&&\%&
% \\ \end{align*}

% Démonstration C'est simplement la formule de Taylor avec reste intégral
% pour la fonction \phi~:

% \phi(1) = \phi(0) + \sum p=1^k~ 1
% \over p! \phi^(p)(0) +
% \\int  ~
% 0^1 (1 - t)^k \over k!
% \phi^(k+1)(t) dt

% Remarque~15.1.7 On utilisera le plus souvent cette formule pour k = 1~;
% dans cas d'une fonction définie sur un ouvert de \mathbb{R}~^2 on
% obtiendra par exemple

% \begin{align*} f(a + h)& =& f(a) + h1
% \partial~f \over \partial~x1 (a) + h2 \partial~f
% \over \partial~x2 (a) \%&
% \\ & & \quad +
% h1^2\int  0^1~(1
% - t) \partial~^2f \over \partial~x1^2 (a
% + th) dt \%& \\ & &
% \quad + h2^2\\int
%  0^1(1 - t) \partial~^2f \over
% \partial~x2^2 (a + th) dt \%&
% \\ & & \quad +
% 2h1h2\int ~
% 0^1(1 - t) \partial~^2f \over
% \partial~x1\partial~x2 (a + th) dt\%&
% \\ \end{align*}

% Théorème~15.1.14 (formule de Taylor-Lagrange). Soit U un ouvert de
% \mathbb{R}~^n et f : U \rightarrow~ \mathbb{R}~ de classe C^k+1. Soit a \in U et h
% \in \mathbb{R}~^n tel que {[}a,a + h{]} \subset~ U. Alors, il existe \theta
% \in{]}0,1{[} tel que

% \begin{align*} f(a + h)& =& f(a) +
% \sum p=1^k~ 1
% \over p! \left (h1 \partial~
% \over \partial~x1 +
% \ldots + hn~ \partial~
% \over \partial~xn \right
% )^p∗f(a) \%& \\ & & + 1
% \over (k + 1)! \left (h1 \partial~
% \over \partial~x1 +
% \\ldots~ +
% hn \partial~ \over \partial~xn
% \right )^(k+1)∗f(a + \thetah)\%&
% \\ \end{align*}

% Démonstration C'est simplement la formule de Taylor Lagrange pour la
% fonction \phi~:

% \phi(1) = \phi(0) + \sum p=1^k~ 1
% \over p! \phi^(p)(0) + 1 \over
% (k + 1)! \phi^(k+1)(\theta)

% Théorème~15.1.15 (formule de Taylor-Young). Soit U un ouvert de
% \mathbb{R}~^n et f : U \rightarrow~ E (espace vectoriel normé de dimension finie)
% de classe C^k. Soit a \in U. Alors, quand h tend vers 0 on a

% f(a + h) = f(a) + \sum p=1^k~ 1
% \over p! \left (h1 \partial~
% \over \partial~x1 +
% \ldots + hn~ \partial~
% \over \partial~xn \right
% )^p∗f(a) +
% o(\\textbar{}h\\textbar{}^k)

% Démonstration Quitte à prendre une base de E et à travailler composante
% par composante, on peut supposer que E = \mathbb{R}~~; toutes les normes sur
% \mathbb{R}~^n étant équivalentes, on peut supposer que
% \\textbar{}h\\textbar{} =
% \textbar{}h1\textbar{} +
% \\ldots~ +
% \textbar{}hn\textbar{}. Soit \rho \textgreater{} 0 tel que B(a,\rho)
% \subset~ U et soit h tel que
% \\textbar{}h\\textbar{} \textless{} \rho. On
% a alors {[}a,a + h{]} \subset~ B(a,\rho) \subset~ U~; on peut donc appliquer la formule
% de Taylor-Lagrange à l'ordre k - 1 qui nous donne

% \begin{align*} f(a + h)& -& f(a)
% -\sum p=1^k~ 1
% \over p! \left (h1 \partial~
% \over \partial~x1 +
% \ldots + hn~ \partial~
% \over \partial~xn \right
% )^p∗f(a)\%& \\ & =& 1
% \over k! \left (h1 \partial~
% \over \partial~x1 +
% \\ldots~ +
% hn \partial~ \over \partial~xn
% \right )^k∗f(a + \thetah) \%&
% \\ & -& 1 \over k!
% \left (h1 \partial~ \over
% \partial~x1 +
% \\ldots~ +
% hn \partial~ \over \partial~xn
% \right )^k∗f(a) \%&
% \\ \end{align*}

% Mais les dérivées partielles de f sont continues. Soit \epsilon \textgreater{}
% 0~; il existe \eta \textgreater{} 0 tel que

% \begin{align*}
% \\textbar{}h\\textbar{} \textless{} \eta&
% \rigtharrow~&
% \forall~(k1,\\\ldots,kn~)\text
% tel que k1 +
% \\ldots~ +
% kn = k, \forall~~t \in {[}0,1{]} \%&
% \\ & & \left \textbar{}
% \partial~^kf \over
% \partial~x1^k1\\ldots\partial~xn^kn~
% (a + th)\right . -\left .
% \partial~^kf \over
% \partial~x1^k1\\ldots\partial~xn^kn~
% (a)\right \textbar{} \textless{} \epsilon\%&
% \\ \end{align*}

% Pour \\textbar{}h\\textbar{} \textless{}
% \eta, on a alors (en développant les deux puissances symboliques)

% \begin{align*} \big
% \textbar{}\left (\\sum
% hi \partial~ \over \partial~xi
% \right )^k∗f(a + \thetah) -\left
% (\sum hi~ \partial~ \over
% \partial~xi \right
% )^k∗f(a)\big \textbar{}&&\%&
% \\ & \textless{}&
% \epsilon\\sum
% k1+\ldots+kn=k~
% k! \over
% k1!\ldotskn!~
% \textbar{}h1\textbar{}^k1
% \ldots\textbar{}hn\textbar{}^kn~
% \%& \\ & =&
% \epsilon(\textbar{}h1\textbar{} +
% \\ldots~ +
% \textbar{}hn\textbar{})^k =
% \epsilon\\textbar{}h\\textbar{}^k \%&
% \\ \end{align*}

% ce qui démontre le résultat.

% \paragraph{15.1.6 Application aux extremums de fonctions de plusieurs
% variables}

% Soit U un ouvert de \mathbb{R}~^n et f : U \rightarrow~ \mathbb{R}~. Nous allons rechercher
% les extremums de la fonction f à l'aide des résultats qui suivent.

% Proposition~15.1.16 Soit U un ouvert de \mathbb{R}~^n et f : U \rightarrow~ \mathbb{R}~ de
% classe \mathcal{C}^1. Soit a \in U. Si f admet en a un extremum local, on
% a \forall~~i \in {[}1,n{]}, \partial~f \over
% \partial~xi (a) = 0.

% Démonstration Il suffit de remarquer que la fonction
% t\mapsto~f(a + tei) (définie sur un
% voisinage de 0) admet en 0 un extremum local. On a donc

%  \partial~f \over \partial~xi (a) = d
% \over dt \left (f(a +
% tei)\right )t=0 = 0

% Dans le cas des fonctions d'une variable, la condition ci dessus n'est
% dé\\\\jmathmathmathmathà pas suffisante (considérer
% x\mapsto~x^3 au point 0). Il est clair
% qu'il en est de même a fortiori pour une fonction de plusieurs
% variables. Pour obtenir des résultats plus précis et en particulier des
% conditions suffisantes d'extremums, nous allons introduire une forme
% quadratique sur \mathbb{R}~^n


% Démonstration (i). Utilisons la formule de Taylor Young à l'ordre 2. On
% a donc, en tenant compte de  \partial~f \over \partial~xi
% (a) = 0, f(a + h) = f(a) + 1 \over 2 \Phi(h)
% +\\textbar{}
% h\\textbar{}^2\epsilon(h), avec
% limh\rightarrow~0~\epsilon(h) = 0. Pour démontrer (i),
% nous allons utiliser le lemme suivant

% Lemme~15.1.18 Soit \Phi une forme quadratique définie positive sur
% \mathbb{R}~^n (ou tout espace vectoriel normé de dimension finie).
% Alors \exists~\alpha~ \textgreater{} 0,
% \forall~h \in \mathbb{R}~^n~, \Phi(h) ≥
% \alpha~\\textbar{}h\\textbar{}^2.

% Démonstration Soit S la sphère unité de \mathbb{R}~^n. Comme \Phi est
% continue sur S qui est compact, \Phi atteint sur S sa borne inférieure \alpha~.
% Soit donc x0 \in S tel que \Phi(x0) = \alpha~
% = inf x\inS~\Phi(x). Comme
% x0\neq~0, on a \alpha~ \textgreater{} 0. De
% plus, si h\neq~0, on a  h \over
% \\textbar{}h\\textbar{} \in S, soit \Phi( h
% \over
% \\textbar{}h\\textbar{} ) ≥ \alpha~ soit 
% \Phi(h) \over
% \\textbar{}h\\textbar{}^2 ≥
% \alpha~, soit encore \Phi(h) ≥
% \alpha~\\textbar{}h\\textbar{}^2.

% Puisque limh\rightarrow~0~\epsilon(h) = 0, il existe \eta
% \textgreater{} 0 tel que
% \\textbar{}h\\textbar{} \textless{} \eta
% \rigtharrow~\textbar{}\epsilon(h)\textbar{}\leq \alpha~ \over 4 . Pour
% \\textbar{}h\\textbar{} \textless{} \eta, on
% a donc

% \begin{align*} f(a + h) - f(a)& =& 1
% \over 2 \Phi(h) +\\textbar{}
% h\\textbar{}^2\epsilon(h) \%&
% \\ & ≥& \alpha~ \over 2
% \\textbar{}h\\textbar{}^2 - \alpha~
% \over 4
% \\textbar{}h\\textbar{}^2 = \alpha~
% \over 4
% \\textbar{}h\\textbar{}^2
% \textgreater{} 0\%& \\
% \end{align*}

% pour h\neq~0. Donc f admet en a un minimum local
% strict.

% Pour démontrer (ii) à partir de (i), il suffit de changer f en - f.

% (iii). Si \Phi n'est ni positive, ni négative, il existe v1 \in
% \mathbb{R}~^n tel que \Phi(v1) \textless{} 0 et il existe
% v2 \in \mathbb{R}~^n tel que \Phi(v2) \textgreater{} 0.
% On a alors, d'après la même formule de Taylor, en posant h =
% tvi, f(a + tvi) = f(a) + 1 \over
% 2 \Phi(tvi) +
% t^2\\textbar{}v
% i\\textbar{}^2\epsilon(tv i) = f(a) +
% t^2 \over 2 \Phi(vi) +
% t^2\epsilon i(t) avec
% limt\rightarrow~0\epsiloni~(t) = 0. On en
% déduit qu'il existe un \eta \textgreater{} 0 tel que \textbar{}t\textbar{}
% \textless{} \eta \rigtharrow~ f(a + tv1) \textless{}
% f(a)\text et f(a + tv2) \textgreater{}
% f(a). Donc f n'a ni minimum, ni maximum en a.

% Remarque~15.1.8 Dans le cas où \Phi est soit positive, soit négative, mais
% non définie (c'est-à-dire que \Phi(h) peut être nul sans que h soit nul),
% on ne peut pas conclure en général et il faut utiliser une formule de
% Taylor à un ordre supérieur.

% Exemple~15.1.4 n = 2~; soit U un ouvert de \mathbb{R}~^2 et f : U \rightarrow~ \mathbb{R}~,
% (x,y)\mapsto~f(x,y). Soit (a,b) \in U. Une condition
% nécessaire pour que f admette en (a,b) un extremum est que  \partial~f
% \over \partial~x (a,b) = \partial~f \over \partial~y (a,b) =
% 0. Posons r = \partial~^2f \over \partial~x^2
% (a,b), s = \partial~^2f \over \partial~x\partial~y (a,b), t =
% \partial~^2f \over \partial~y^2 (a,b) (notations
% de Monge). La forme quadratique \Phi est
% (h,k)\mapsto~rh^2 + 2shk +
% tk^2. Considérons suivant le cas le rapport  h
% \over k ou le rapport  k \over h ,
% on constate immédiatement à l'aide de l'étude du signe d'un trinome du
% second degré que si (i) rt - s^2 \textgreater{} 0 et r
% \textgreater{} 0, alors \Phi est définie positive et f a en a un minimum
% local strict (ii) rt - s^2 \textgreater{} 0 et r \textless{}
% 0, alors \Phi est définie négative et f a en a un maximum local strict
% (iii) rt - s^2 \textless{} 0, alors f a en a un point selle
% (pas d'extremum local en a) (iv) rt - s^2 = 0, alors on ne
% peut pas conclure.

% Le lecteur comparera les surfaces z = f(x,y) ainsi que lignes de niveau
% de ces surfaces dans les trois exemples ci dessous (correspondant
% respectivement à un minimum local, un point selle et un point de type
% (iv))

% \includegraphics{cours8x.png}

% {[}
% {[}
