\textbf{Warning: 
requires JavaScript to process the mathematics on this page.\\ If your
browser supports JavaScript, be sure it is enabled.}

\begin{center}\rule{3in}{0.4pt}\end{center}

{[}
{[}
{[}{]}
{[}

\subsubsection{16.2 Théorie de Cauchy-Lipschitz}

\paragraph{16.2.1 Unicité de solutions, solutions maximales}

Définition~16.2.1 Soit E un espace vectoriel normé de dimension finie, U
un ouvert de \mathbb{R}~ \times E et F : U \rightarrow~ E. On dira que F vérifie la condition
d'unicité du problème de Cauchy Lipschitz si pour toutes solutions (I,\phi)
et (J,\psi) de l'équation différentielle y' = F(t,y) qui coïncident en un
point t0 \in I \bigcap J (c'est-à-dire que \phi(t0) =
\psi(t0)), on a

\forall~~t \in I \bigcap J, \phi(t) = \psi(t)

Définition~16.2.2 Soit E un espace vectoriel normé de dimension finie, U
un ouvert de \mathbb{R}~ \times E et F : U \rightarrow~ E. On dira que F vérifie la condition
d'existence au problème de Cauchy-Lipschitz, si pour tout
(t0,y0) \in U, il existe \eta \textgreater{} 0 et une
solution ({]}t0 - \eta,t0 + \eta{[},\phi) de l'équation
différentielle y' = F(t,y) vérifiant la condition \phi(t0) =
y0.

Définition~16.2.3 Soit (I,\phi) et (J,\psi) deux solutions de l'équation
différentielle y' = F(t,y). On dira que (J,\psi) est un prolongement de
(I,\phi), et on notera (I,\phi) \prec~ (J,\psi) si I \subset~ J et \phi est la restriction de \psi
à I.

Remarque~16.2.1 Il est clair qu'il s'agit d'une relation d'ordre partiel
sur l'ensemble des solutions de l'équation différentielle.

Définition~16.2.4 On appelle solution maximale de l'équation
différentielle y' = F(t,y) toute solution (I,\phi) qui est maximale pour la
relation d'ordre \prec~.

Remarque~16.2.2 Ceci signifie donc qu'il n'existe aucune solution
définie sur un intervalle I' contenant strictement I et qui prolonge \phi.

Théorème~16.2.1 (existence et unicité d'une solution maximale à
condition initiale donnée). On suppose que F vérifie les conditions
d'existence et d'unicité au problème de Cauchy Lipschitz. Soit
(t0,y0) \in U~; alors il existe une unique solution
maximale (I0,\phi0) de l'équation différentielle y' =
F(t,y) qui vérifie \phi0(t0) = y0. Pour toute
solution (J,\psi) de l'équation différentielle vérifiant \psi(t0) =
y0, on a~:

\text\$J \subset~ I0\$ et \$\psi\$ est la restriction
de \$\phi0\$ à \$J\$.

Démonstration Unicité~: Soit (I0,\phi0) et
(I1,\phi1) deux solutions maximales vérifiant
\phi0(t0) = \phi1(t0) = y0.
Définissons I2 = I0 \cup I1 et soit
\phi2 l'application de I2 dans E définie par
\phi2(t) = \left \
\cases \phi0(t)&si t \in I0
\cr \phi1(t)&si t \in I1\\ 
\right .. Comme \phi0 et \phi1 coïncident
sur I0 \bigcap I1, \phi2 est bien définie. On
vérifie facilement qu'elle est de classe \mathcal{C}^1 et solution de
l'équation différentielle y' = F(t,y). La maximalité de
(I0,\phi0) et (I1,\phi1) exige alors
I2 = I0 = I1 et \phi2 =
\phi0 = \phi1, ce qui montre l'unicité de la solution
maximale.

Existence Soit \left
((I\\\\jmathmathmathmath,\psi\\\\jmathmathmathmath)\right )\\\\jmathmathmathmath\inℱ la
famille de toutes les solutions de l'équation différentielle y' = F(t,y)
définies sur un intervalle I\\\\jmathmathmathmath non réduit à un point contenant
t0 et vérifiant \psi\\\\jmathmathmathmath(t0) = y0~;
cette famille est non vide puisque la fonction F vérifie la condition
d'existence au problème de Cauchy-Lipschitz. Posons I0
= \⋃ ~
\\\\jmathmathmathmath\inℱI\\\\jmathmathmathmath et définissons \phi0 : I0 \rightarrow~ E
par \phi0(t) = \psi\\\\jmathmathmathmath(t) si t \in I\\\\jmathmathmathmath. Cette
définition est bien cohérente car si t \in I\\\\jmathmathmathmath \bigcap Ik,
alors \psi\\\\jmathmathmathmath et \psik coïncident sur I\\\\jmathmathmathmath \bigcap
Ik, et en particulier \psi\\\\jmathmathmathmath(t) = \psik(t). On
vérifie facilement que la fonction \phi0 est de classe
\mathcal{C}^1 et si t \in I\\\\jmathmathmathmath, on a \phi'0(t) =
\psi\\\\jmathmathmathmath'(t) = F(t,\psi\\\\jmathmathmathmath(t)) = F(t,\phi0(t)) ce qui
montre que (I0,\phi0) est bien une solution de
l'équation différentielle~; cette solution vérifie bien entendu
\phi0(t0) = y0. De plus, si
(I0,\phi0) \prec~ (I1,\phi1), on a
\phi1(t0) = \phi0(t0) = y0,
ce qui montre que (I1,\phi1) est l'une des
(I\\\\jmathmathmathmath,\psi\\\\jmathmathmathmath) et que donc I1 \subset~ I0~; on
a donc finalement I0 = I1 et \phi0 =
\phi1 ce qui montre que cette solution est maximale.

Si (J,\psi) est une solution de l'équation différentielle vérifiant
\psi(t0) = y0, alors (J,\psi) est l'une des
(I\\\\jmathmathmathmath,\psi\\\\jmathmathmathmath) ce qui montre que J \subset~ I0 et que \psi
= \psi\\\\jmathmathmathmath est la restriction de \phi0 à J = I\\\\jmathmathmathmath.
Ceci achève la démonstration.

Remarque~16.2.3 On constate que du point de vue de la relation \prec~, la
solution maximale vérifiant la condition \phi0(t0) =
y0 est un plus grand élément de l'ensemble des solutions
vérifiant cette condition initiale, ce qui en explique d'ailleurs
l'unicité. Il est clair, d'après la condition d'existence, que
t0 est un point intérieur à I0, intervalle de
définition de la solution maximale~; nous allons d'ailleurs préciser ce
point dans la proposition suivante

Théorème~16.2.2 On suppose que F vérifie la condition d'existence et
d'unicité au problème de Cauchy Lipschitz. Alors toute solution maximale
de l'équation différentielle y' = F(t,y) est définie sur un intervalle
ouvert.

Démonstration Soit (I,\phi) une solution maximale et soit a
\in\overline\mathbb{R}~ une borne de I (par exemple la borne
supérieure). Supposons que a \in I si bien que (a,\phi(a)) \in U. D'après la
condition d'existence il existe \eta \textgreater{} 0 et une solution ({]}a
- \eta,a + \eta{[},\psi) vérifiant la condition initiale \psi(a) = \phi(a). D'après la
condition d'unicité, \phi et \psi qui coïncident au point a, coïncident
également sur l'intersection de leurs intervalles de définition, ce qui
permet de définir I1 = I\cup{]}a - \eta,a + \eta{[} et \phi1 :
I1 \rightarrow~ E par \phi1(t) = \left
\ \cases \phi0(t)&si t \in I
\cr \psi(t) &si t \in{]}a - \eta,a + \eta{[} 
\right .. Le couple (I1,\phi1) est une
solution de l'équation différentielle qui prolonge strictement (I,\phi) ce
qui contredit le caractère maximal de cette solution.

Remarque~16.2.4 On aurait pu aussi dire que si a \in I et si \phi(a) = b,
(I,\phi) est une solution maximale pour la condition initiale \phi(a) = b, ce
qui montre que a appartient à l'intérieur de I comme on l'a dé\\\\jmathmathmathmathà
remarqué. Nous avons cependant pensé que la démonstration précédente
était plus constructive.

{[}
{[}
{[}
{[}
