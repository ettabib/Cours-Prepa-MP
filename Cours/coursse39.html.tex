\textbf{Warning: 
requires JavaScript to process the mathematics on this page.\\ If your
browser supports JavaScript, be sure it is enabled.}

\begin{center}\rule{3in}{0.4pt}\end{center}

{[}
{[}
{[}{]}
{[}

\subsubsection{7.5 Séries semi-convergentes}

\paragraph{7.5.1 Séries alternées}

Théorème~7.5.1 (convergence des séries alternées). Soit (an)
une suite de nombres réels, décroissante, de limite 0. Alors la série
\\sum ~
(-1)^nan converge~; le reste d'ordre n est du signe
de son premier terme (c'est-à-dire (-1)^n+1) et sa valeur
absolue est ma\\\\jmathmathmathmathorée par la valeur absolue de ce premier terme
(c'est-à-dire an+1).

Démonstration On a S2n+2 - S2n = a2n+2 -
a2n+1 \leq 0 et S2n+3 - S2n+1 =
a2n+2 - a2n+3 ≥ 0. La suite (S2n) est donc
décroissante, la suite S2n+1 est croissante~; comme
S2n - S2n+1 = a2n+1 est positif et tend
vers 0, ces deux suites forment un couple de suites ad\\\\jmathmathmathmathacentes~; elles
admettent donc une limite commune S qui est limite de la suite
Sn. On a pour tout n, S2n-1 \leq S2n+1 \leq S \leq
S2n. Ceci nous montre que 0 \leq-R2n = S2n -
S \leq S2n - S2n+1 = a2n+1 et que 0 \leq
R2n-1 = S - S2n-1 \leq S2n - S2n-1
= a2n d'où les assertions sur le reste.

\paragraph{7.5.2 Etude de séries semi-convergentes}

Les théorèmes de comparaison ne s'appliquent pas aux séries quelconques.
Ainsi on a  (-1)^n \over
\sqrtn ∼ (-1)^n \over
\sqrtn + 1 \over n alors que la
première est convergente et la deuxième divergente (somme d'une série
convergente et d'une série divergente). Pour une série à termes réels,
on peut envisager le plan suivant

(i) regarder si le critère de convergence des séries alternées
s'applique (an =
(-1)^n\textbar{}an\textbar{} avec
\textbar{}an\textbar{} décroissant de limite 0).

(ii) si an = (-1)^n\textbar{}an\textbar{}
mais si on ne peut pas appliquer le critère de convergence des séries
alternées, on peut essayer de trouver une série alternée
\\sum  bn~ qui
relève de ce critère telle que an ∼ bn~; alors,
comme la série \\sum ~
bn converge, la nature de la série
\\sum  an~ sera
la même que celle de la série
\\sum  (an~ -
bn), avec an - bn = o(an)~; on
essayera de poursuivre le processus \\\\jmathmathmathmathusqu'à tomber soit sur une série
divergente, soit sur une série absolument convergente

(iii) si an n'est pas alterné en signes, on peut utiliser une
sommation par paquets (cf plus loin)~: en regroupant les termes
consécutifs de même signe, on aboutira à une série alternée en signe à
laquelle on pourra appliquer l'une des méthodes précédentes

Enfin, pour une série à termes non réels ou qui ne relève pas d'une des
méthodes précédentes, on pourra utiliser un théorème d'Abel comme le
suivant

Théorème~7.5.2 Soit (an) une suite de nombres réels et
(xn) une suite de l'espace vectoriel normé~complet E telles
que

\begin{itemize}
\itemsep1pt\parskip0pt\parsep0pt
\item
  (i) \existsM ≥ 0, \\forall~~n \in
  \mathbb{N}~,
  \\textbar{}\\\sum
   p=0^nxp\\textbar{} \leq M
\item
  (ii) la suite (an) tend vers 0 en décroissant.
\end{itemize}

Alors la série \\sum ~
anxn converge.

Démonstration On a

\begin{align*} \\sum
n=p^qa nxn& =&
\sum n=p^qa~
n(Sn(x) - Sn-1(x)) \%&
\\ & =& \\sum
n=p^qa nSn(x)
-\sum n=p^qa~
nSn-1(x) \%& \\ & =&
\sum n=p^qa~
nSn(x) -\\sum
n=p-1^q-1a n+1Sn(x) \%&
\\ \text (changement
d'indices \$n - 1\mapsto~n\$)&& \%&
\\ & & \%&
\\ & =& aqSq(x) -
apSp-1(x) + \\sum
n=p^q-1(a n -
an+1)Sn(x)\%& \\
\end{align*}

On a effectué ici une transformation dite transformation d'Abel.

Comme \forall~~n,
\\textbar{}Sn(x)\\textbar{} \leq M
on a

\\textbar{}\\sum
n=p^qa
nxn\\textbar{} \leq
M(\textbar{}aq\textbar{} + \textbar{}ap\textbar{}
+ \\sum
n=p^q-1\textbar{}a n -
an+1\textbar{}) = 2Map

en tenant compte de an ≥ 0 et an - an+1 ≥
0. Comme limap~ = 0, la série
\\sum ~
anxn vérifie le critère de Cauchy, donc elle
converge.

{[}
{[}
{[}
{[}
