\textbf{Warning: 
requires JavaScript to process the mathematics on this page.\\ If your
browser supports JavaScript, be sure it is enabled.}

\begin{center}\rule{3in}{0.4pt}\end{center}

{[}
{[}
{[}{]}
{[}

\subsubsection{7.6 Opérations sur les séries}

\paragraph{7.6.1 Combinaisons linéaires}

Proposition~7.6.1 Soit E un espace vectoriel normé,
\\sum  an~ et
\\sum  bn~ deux
séries à termes dans E, \alpha~ et \beta~ deux scalaires. Si
\\sum  an~ et
\\sum  bn~ sont
convergentes (resp. absolument convergentes), il en est de même de la
série \\sum ~
(\alpha~an + \beta~bn) et alors

\sum n=0^+\infty~(\alpha~a n~ +
\beta~bn) = \alpha~\\sum
n=0^+\infty~a n + \beta~\\sum
n=0^+\infty~b n

Démonstration Le résultat a dé\\\\jmathmathmathmathà été vu pour la convergence~; pour la
convergence absolue, il résulte de
\\textbar{}\alpha~an +
\beta~bn\\textbar{}
\leq\textbar{}\alpha~\textbar{}\,\\textbar{}an\\textbar{}
+
\textbar{}\beta~\textbar{}\,\\textbar{}bn\\textbar{}

Corollaire~7.6.2 Soit (zn) une suite de nombres complexes,
zn = xn + iyn, xn,yn \in
\mathbb{R}~. Alors la série \\sum ~
zn est convergente (resp. absolument convergente) si et
seulement si~les deux séries
\\sum  xn~ et
\\sum  yn~ le
sont.

Démonstration Le sens direct résulte de xn = 1
\over 2 (zn +
\overlinezn) et yn = 1
\over 2i (zn
-\overlinezn). La réciproque est évidente.

\paragraph{7.6.2 Sommation par paquets}

Théorème~7.6.3 (Sommation par paquets) Soit E un espace vectoriel normé,
\\sum  xn~ une
série à termes dans E, \phi une application strictement croissante de \mathbb{N}~
dans \mathbb{N}~. On pose y0 =\
\sum  k=0^\phi(0)xk~ et
pour n ≥ 1, yn =\
\sum ~
k=\phi(n-1)+1^\phi(n)xk. Alors

\begin{itemize}
\itemsep1pt\parskip0pt\parsep0pt
\item
  (i) si la série \\sum ~
  xn converge, la série
  \\sum  yn~
  converge et a même somme
\item
  (ii) la réciproque est vraie dans les deux cas suivants

  \begin{itemize}
  \itemsep1pt\parskip0pt\parsep0pt
  \item
    (a) la suite xn tend vers 0 et la suite \phi(n + 1) - \phi(n)
    (la taille des ''paquets'') est ma\\\\jmathmathmathmathorée
  \item
    (b) E = \mathbb{R}~ et à l'intérieur de chaque ''paquet'' (k \in {[}\phi(n - 1) +
    1,\phi(n){]}), tous les xk, sont de même signe.
  \end{itemize}
\end{itemize}

Démonstration On a d'abord

Sn(y) = \\sum
p=0^n(\\sum
k=\phi(n-1)+1^\phi(n)x k) =
\sum k=0^\phi(n)x k~ =
S\phi(n)(x)

(en convenant que \phi(-1) = -1). La suite Sn(y) est donc une
sous suite de la suite Sn(x), ce qui montre l'assertion (i).

(ii.a) Soit S = \\sum ~
n=0^+\infty~yn et K tel que
\forall~~n, \phi(n + 1) - \phi(n) \leq K. Soit n \in \mathbb{N}~ et p
l'unique entier tel que \phi(p - 1) \textless{} n \leq \phi(p). On a alors

Sp(y) - Sn(x) = S\phi(p)(x) - Sn(x)
= \sum k=n+1^\phi(p)x k~

Soit alors \epsilon \textgreater{} 0 et N \in \mathbb{N}~ tel que n ≥ N
\rigtharrow~\\textbar{} xn\\textbar{}
\textless{} \epsilon \over 2K . Alors pour n ≥ N, on a
\\textbar{}Sp(y) -
Sn(x)\\textbar{}
\leq\\sum ~
k=n+1^\phi(p)\\textbar{}xk\\textbar{}
\leq (\phi(p) - n) \epsilon \over 2K \leq \epsilon \over
2 . Mais il existe N' tel que q ≥ N' \rigtharrow~\\textbar{} S -
Sq(y)\\textbar{} \textless{} \epsilon
\over 2 . Si on choisit n ≥\
max(N,\phi(N')), on a p ≥ N' et donc

\\textbar{}S - Sn(x)\\textbar{}
\leq\\textbar{} S -
Sp(y)\\textbar{} +\\textbar{}
Sp(y) - Sn(x)\\textbar{} \textless{}
\epsilon

ce qui montre que la série
\\sum  xn~
converge.

(ii.b) La démonstration est similaire mais on remarque que

\begin{align*} \textbar{}Sp(y) -
Sn(x)\textbar{}& =& \textbar{}\\sum
k=n+1^\phi(p)x k\textbar{} =
\sum k=n+1^\phi(p)\textbar{}x~
k\textbar{} \%& \\ & \leq&
\\sum
k=\phi(p-1)+1^\phi(p)\textbar{}x k\textbar{} =
\textbar{}\\sum
k=\phi(p-1)+1^\phi(p)x k\textbar{}\%&
\\ & =& \textbar{}yp\textbar{}
\%& \\ \end{align*}

(car tous les xk sont de même signe). Comme la série
\\sum  yq~
converge, pour q ≥ N on a \textbar{}yq\textbar{} \textless{}
\epsilon \over 2 . Alors pour n ≥ \phi(N), on a p ≥ N et donc
\textbar{}Sp(y) -
Sn(x)\textbar{}\leq\textbar{}yp\textbar{} \textless{}
\epsilon \over 2 . On achève alors la démonstration comme dans
le cas précédent.

Remarque~7.6.1 La réciproque de (i) n'est pas valable en toute
généralité comme le montre l'exemple de la série
\\sum  (-1)^n~
et de \phi(n) = 2n. On a alors yn = 0, la série
\\sum  yn~
converge alors que la série
\\sum  xn~ est
divergente. La réciproque (ii.b) est particulièrement intéressante pour
le cas de séries de nombres réels qui ne sont pas de signe constant~; en
regroupant ensemble les termes consécutifs de même signe, on obtient une
série de même nature que la série initiale et dont les termes sont
alternés en signe.

\paragraph{7.6.3 Modification de l'ordre des termes}

Nous allons ici étudier l'effet d'une permutation sur les termes d'une
série convergente. Pour cela nous aurons besoin du lemme suivant.

Théorème~7.6.4 Soit \\\sum
 xn une série à termes réels ou complexes absolument
convergente et soit \sigma : \mathbb{N}~ \rightarrow~ \mathbb{N}~ bi\\\\jmathmathmathmathective, une permutation de \mathbb{N}~. Alors la
série \\sum ~
x\sigma(n) est absolument convergente et
\\sum ~
n=0^+\infty~x\sigma(n) =\
\sum  n=0^+\infty~xn~.

Démonstration Premier cas~: série à termes réels positifs. Pour n \in \mathbb{N}~,
soit Nn le plus grand élément de \sigma({[}0,n{]}). On a alors

\sum k=0^nx \sigma(k)~
\leq\sum p=0^Nn~
xp \leq\\sum
p=0^+\infty~x p

ce qui montre que la série à termes réels positifs
\\sum  x\sigma(k)~
converge et que \\sum ~
n=0^+\infty~x\sigma(n)
\leq\\sum ~
n=0^+\infty~xn. Mais les deux séries \\\\jmathmathmathmathouent un rôle
symétrique puisque xn = x\sigma^-1(\sigma(n)), et
donc on a aussi \\sum ~
n=0^+\infty~xn
\leq\\sum ~
n=0^+\infty~x\sigma(n) ce qui nous donne l'égalité.

Deuxième cas~: séries à termes réels On introduit, comme d'habitude,
pour x \in \mathbb{R}~, x^+ = max~(x,0) \in
\mathbb{R}~^+ et x^- = max~(-x,0) \in
\mathbb{R}~^+ si bien que x = x^+ - x^-,
\textbar{}x\textbar{} = x^+ + x^-. On a alors 0 \leq
xn^+ \leq\textbar{}xn\textbar{} et 0 \leq
xn^-\leq\textbar{}xn\textbar{}, ce qui montre
que les deux séries à termes positifs
\\sum ~
xn^+ et
\\sum ~
xn^- sont convergentes. D'après le premier cas de la
démonstration, les deux séries
\\sum ~
x\sigma(n)^+ et
\\sum ~
x\sigma(n)^- sont convergentes et on a

\sum n=0^+\infty~x~
\sigma(n)^+ = \\sum
n=0^+\infty~x n^+,\quad
\sum n=0^+\infty~x~
\sigma(n)^- = \\sum
n=0^+\infty~x n^-

Comme \textbar{}x\sigma(n)\textbar{} = x\sigma(n)^+ +
x\sigma(n)^-, la série
\\sum ~
\textbar{}x\sigma(n)\textbar{} converge, donc la série
\\sum  x\sigma(n)~
est absolument convergente, et comme x\sigma(n) =
x\sigma(n)^+ - x\sigma(n)^-, on a

\sum n=0^+\infty~x \sigma(n)~ =
\sum n=0^+\infty~x~
\sigma(n)^+-\\sum
n=0^+\infty~x \sigma(n)^- =
\sum n=0^+\infty~x~
n^+-\\sum
n=0^+\infty~x n^- =
\sum n=0^+\infty~x n~

Troisième cas~: séries à termes complexes On travaille de la même
fa\ccon avec les parties réelles et parties
imaginaires. On a 0
\leq\textbar{}\mathrmRe(xn)\textbar{}\leq\textbar{}xn~\textbar{}
et 0
\leq\textbar{}\mathrmIm(xn)\textbar{}\leq\textbar{}xn~\textbar{},
ce qui montre que les deux séries
\\sum ~
\mathrmRe(xn~) et
\\sum ~
\mathrmIm(xn~)
sont absolument convergentes. D'après le deuxième cas de la
démonstration, les deux séries
\\sum ~
\mathrmRe(x\sigma(n)~)
et \\sum ~
\mathrmIm(x\sigma(n)~)
sont absolument convergentes et on a

\\sum
n=0^+\infty~\mathrmRe(x \sigma(n))
= \\sum
n=0^+\infty~\mathrmRe(x
n),\quad \\sum
n=0^+\infty~\mathrmIm(x \sigma(n))
= \\sum
n=0^+\infty~\mathrmIm(x n)

Comme
\textbar{}x\sigma(n)\textbar{}\leq\textbar{}\mathrmRe(x\sigma(n)~)\textbar{}
+
\textbar{}\mathrmIm(x\sigma(n)~)\textbar{},
la série \\sum ~
\textbar{}x\sigma(n)\textbar{} converge, donc la série
\\sum  x\sigma(n)~
est absolument convergente, et comme x\sigma(n)
=\
\mathrmRe(x\sigma(n)) +
i\mathrmRe(x\sigma(n)~),
on a

\sum n=0^+\infty~x \sigma(n)~ =
\\sum
n=0^+\infty~\mathrmRe(x
\sigma(n))+i\\sum
n=0^+\infty~\mathrmRe(x \sigma(n))
= \\sum
n=0^+\infty~\mathrmRe(x
n)+i\\sum
n=0^+\infty~\mathrmIm(x n) =
\sum n=0^+\infty~x n~

Remarque~7.6.2 La condition de convergence absolue est indispensable à
la validité du théorème. Considérons la série semi convergente
\\sum  xn~ avec
xn = (-1)^n-1 \over n et soit S
sa somme (on peut montrer que S = log~ 2). Soit
\phi : \mathbb{N}~^∗\rightarrow~ \mathbb{N}~^∗ définie par \phi(3k + 1) = 2k + 1, \phi(3k
+ 2) = 4k + 2 et \phi(3k + 3) = 4k + 4. On vérifie facilement que \phi est une
bi\\\\jmathmathmathmathection de \mathbb{N}~ dans \mathbb{N}~ (la bi\\\\jmathmathmathmathection réciproque est définie par des
congruences modulo 4). Sommons alors par paquets de 3 la série
\\sum  x\phi(n)~.
On a

\begin{align*} x\phi(3k+1) +
x\phi(3k+2) + x\phi(3k+3)&& \%&
\\ & =& 1 \over 2k +
1 - 1 \over 4k + 2 - 1 \over 4k +
4 = 1 \over 4k + 2 - 1 \over 4k +
4 \%& \\ & =& 1
\over 2 \left (x2k+1 +
x2k+2\right ) \%&
\\ \end{align*}

Ceci montre (réciproque du théorème de sommation par paquets, la taille
des paquets étant bornée et le terme général tendant vers 0) que la
nouvelle série converge encore, mais que sa somme est la moitié de la
somme de la série initiale.

\paragraph{7.6.4 Produit de Cauchy}

Définition~7.6.1 Soit \\\sum
 an et \\\sum
 bn deux séries à termes réels ou complexes. On appelle
produit de Cauchy (ou produit de convolution) des deux séries, la série
\\sum  cn~ avec

\forall~n \in \mathbb{N}~, cn~ =
\sum k=0^na~
kbn-k = \\sum
p+q=napbq

Théorème~7.6.5 Soit \\\sum
 an et \\\sum
 bn deux séries à termes réels ou complexes, absolument
convergentes. Alors leur produit de Cauchy
\\sum  cn~ est
une série absolument convergente et on a

\sum n=0^+\infty~c n~ =
\left (\\sum
n=0^+\infty~a n\right
)\left (\\sum
n=0^+\infty~b n\right )

Démonstration Cas particulier~: les deux séries sont à termes réels
positifs. Notons Kn = {[}0,n{]} \times {[}0,n{]} \subset~ \mathbb{N}~^2
et Tn = \(p,q) \in
\mathbb{N}~^2∣p + q \leq n\.
On a évidemment Tn \subset~ Kn \subset~ T2n. On a alors

\begin{align*} \\sum
k=0^nc k& =& \\sum
k=0^n \\sum
p+q=kapbq = \\sum
(p,q)\inTnapbq
\leq\\sum
(p,q)\inKnapbq\%&
\\ & =& \\sum
p=0^na p \\sum
q=0^nb q \leq\\sum
p=0^+\infty~a p \\sum
q=0^+\infty~b q \%&
\\ \end{align*}

La série \\sum ~
cn est une série à termes réels positifs dont les sommes
partielles sont ma\\\\jmathmathmathmathorées, donc elle converge. De plus les inclusions
Tn \subset~ Kn \subset~ T2n se traduisent par
Sn(c) \leq Sn(a)Sn(b) \leq S2n(c) et
en faisant tendre n vers + \infty~, on obtient S(c) = S(a)S(b) ce qui est la
formule souhaitée.

Cas général Posons an' = \textbar{}an\textbar{},
bn' = \textbar{}bn\textbar{} et cn'
= \\sum ~
p+q=n\textbar{}ap\textbar{}\textbar{}bq\textbar{}
leur produit de Cauchy, et désignons par
Sn(a'),Sn(b') et Sn(c') les sommes
partielles d'indice n de ces trois séries. Puisque les séries
\\sum  an~' et
\\sum  bn~'
sont convergentes, le cas particulier ci dessus montre que la série
\\sum  cn~' est
convergente et que sa somme est le produit des sommes de ces deux
séries. Mais, comme \textbar{}cn\textbar{}\leq cn', on
en déduit la convergence absolue de la série
\\sum  cn~. On
a alors

\begin{align*} \left
\textbar{}Sn(a)Sn(b) -
Sn(c)\right \textbar{}& =&
\left \textbar{}\\sum
(p,q)\inKnapbq
-\\sum
(p,q)\inTnapbq\right
\textbar{} = \left \textbar{}\\sum
(p,q)\inKn\diagdownTnapbq\right
\textbar{} \%& \\ & \leq&
\\sum
(p,q)\inKn\diagdownTn\textbar{}ap\textbar{}\textbar{}bq\textbar{}
= \\sum
(p,q)\inKn\textbar{}ap\textbar{}\textbar{}bq\textbar{}-\\sum
(p,q)\inTn\textbar{}ap\textbar{}\textbar{}bq\textbar{}
= Sn(a')Sn(b') -
Sn(c')\%&\\
\end{align*}

Puisque la somme de la série
\\sum  cn~' est
le produit des sommes des deux séries
\\sum  an~' et
\\sum  bn~', on
a
limn\rightarrow~+\infty~(Sn(a')Sn~(b')
- Sn(c')) = 0 et donc par la ma\\\\jmathmathmathmathoration ci-dessus
limn\rightarrow~+\infty~(Sn(a)Sn~(b)
- Sn(c)) = 0, ce qui montre que la somme de la série
\\sum  cn~ est
le produit des sommes des deux séries
\\sum  an~ et
\\sum  bn~ et
achève la démonstration.

Remarque~7.6.3 On aurait pu passer aussi du cas réel positif au cas
complexe en utilisant, comme dans le théorème de permutation des termes,
les parties positives x^+ et x^- d'un réel x, puis
les parties réelle et imaginaire d'un nombre complexe, mais la
démonstration n'aurait pas pu se généraliser comme nous le ferons
ci-dessous au cas d'une application bilinéaire plus générale.

Remarque~7.6.4 Le théorème ci dessus n'est plus valable pour des séries
convergentes~: posons an = bn = (-1)^n
\over \sqrtn+1 . On a
\textbar{}cn\textbar{} =\
\sum  k=0^n~ 1
\over \sqrt(k+1)(n-k+1) . Mais pour
k \in {[}0,n{]}, (k + 1)(n - k + 1) \leq ( n \over 2 +
1)^2 (facile). Donc \textbar{}cn\textbar{}≥ n+1
\over  n \over 2 +1 qui tend vers
2~; donc la suite (cn) ne tend pas vers 0 et la série
\\sum  cn~
diverge.

On a une généralisation du théorème précédent sous la forme suivante qui
nous sera utile quand nous considérerons des séries d'endomorphismes.

Théorème~7.6.6 Soit E, F et G sont trois espaces vectoriels normés, u :
E \times F \rightarrow~ G une application bilinéaire continue,
\\sum  an~ une
série à termes dans E absolument convergente,
\\sum  bn~ une
série à termes dans F absolument convergente, et si l'on pose
cn = \\sum ~
p+q=nu(ap,bq), alors la série
\\sum  cn~ est
absolument convergente et on a

\sum n=0^+\infty~c n~ =
u\left (\\sum
n=0^+\infty~a n,\\sum
n=0^+\infty~b n\right )

Démonstration La démonstration est tout à fait similaire~: utiliser
l'existence d'un réel positif K tel que
\\textbar{}u(x,y)\\textbar{} \leq
K\\textbar{}x\\textbar{}
\\textbar{}y\\textbar{} pour montrer que
\left \textbar{}Sn(a)Sn(b) -
Sn(c)\right \textbar{}\leq K\left
(Sn(a')Sn(b') -
Sn(c')\right ) en posant an'
=\\textbar{} an\\textbar{},
bn' =\\textbar{}
bn\\textbar{} et cn'
= \\sum ~
p+q=n\\textbar{}ap\\textbar{}\\textbar{}bq\\textbar{}

{[}
{[}
{[}
{[}
