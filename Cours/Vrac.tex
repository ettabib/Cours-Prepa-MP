%% LyX 2.0.7.1 created this file.  For more info, see http://www.lyx.org/.
%% Do not edit unless you really know what you are doing.
\documentclass[french]{article}
\usepackage[T1]{fontenc}
\usepackage[latin9]{inputenc}
\usepackage{amstext}
\usepackage{babel}
\makeatletter
\addto\extrasfrench{%
   \providecommand{\og}{\leavevmode\flqq~}%
   \providecommand{\fg}{\ifdim\lastskip>\z@\unskip\fi~\frqq}%
}

\makeatother
\begin{document}

\subsubsection{Formes quadratiques en dimension finie}

Soit E un K-espace vectoriel de dimension finie, $\Phi\in Q(E)$ de
forme polaire $\phi$.


\paragraph{Theoreme }

Soit $\mathcal{E}$ une base de E. Alors $Mat(\phi,\mathcal{E})$
est l\textquoteright{}unique matrice $\Omega\in M_{K}(n)$ qui est
sym�trique et qui v�rifie :

\begin{eqnarray*}
\forall x\in E, & \Phi(x)= & ^{t}X\Omega X\\
\end{eqnarray*}



\paragraph{Remarque}

On prendra garde � la condition de sym�trie de $\Omega$. Il est en
effet clair que l\textquoteright{}on peut remplacer, dans la condition
$\Phi(x)=^{t}X\Omega X$, la matrice $\Omega$ par une matrice $\Omega'=\Omega+A$
o� $A$ est antisym�trique, puisque dans ce cas $^{t}XAX=0$. On aura
alors $\Phi(x)=^{t}X\Omega X$ bien que $\Omega'$ ne soit pas la
matrice de $\Phi$ dans la base $\mathcal{E}$.

\begin{eqnarray*}
\text{\ensuremath{\phi}}(x,x) & = & \sum_{i}\omega_{i}x_{i}^{2}+2\sum_{i<j}\omega_{ij}x_{i}x_{j}=^{t}X\Omega X\\
\text{\ensuremath{\phi}}(x,y) & = & \sum_{i}\omega_{i}x_{i}y_{i}+\sum_{i<j}\omega_{ij}(x_{i}y_{j}+y_{i}x_{j})
\end{eqnarray*}

\end{document}
