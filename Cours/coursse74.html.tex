\textbf{Warning: 
requires JavaScript to process the mathematics on this page.\\ If your
browser supports JavaScript, be sure it is enabled.}

\begin{center}\rule{3in}{0.4pt}\end{center}

{[}
{[}
{[}{]}
{[}

\subsubsection{13.2 Formes sesquilinéaires}

\paragraph{13.2.1 Généralités}

Définition~13.2.1 Soit E un \mathbb{C}-espace vectoriel . On appelle forme
sesquilinéaire sur E toute application \phi : E \times E \rightarrow~ \mathbb{C} telle que

\begin{itemize}
\itemsep1pt\parskip0pt\parsep0pt
\item
  (i) \forall~~x \in E,
  y\mapsto~\phi(x,y) est linéaire
\item
  (ii) \forall~~y \in E,
  x\mapsto~\phi(x,y) est semilinéaire
\end{itemize}

Remarque~13.2.1 On a en particulier \forall~~y \in E,
\phi(y,0) = \phi(0,y) = 0~; de plus \phi(x,\lambda~y) = \lambda~\phi(x,y), \phi(\lambda~x,y) =
\overline\lambda~\phi(x,y). Plus généralement
\phi(\\sum ~
\lambda~ixi,\\\sum
 \mu\\\\jmathmathmathmathy\\\\jmathmathmathmath) =\
\sum ~
i,\\\\jmathmathmathmath\overline\lambda~i\mu\\\\jmathmathmathmath\phi(xi,y\\\\jmathmathmathmath).

Il est clair que si \phi et \psi sont deux formes sesquilinéaires sur E, il en
est de même de \alpha~\phi + \beta~\psi, d'où la proposition

Proposition~13.2.1 L'ensemble L3\diagup2(E) des formes
sesquilinéaires sur E est un sous-espace vectoriel de l'espace
\mathbb{C}^E\timesE des applications de E \times E dans \mathbb{C}.

Remarque~13.2.2 Soit \phi une forme sesquilinéaire sur E. Pour chaque x \in
E, l'application y\mapsto~\phi(x,y) est une forme
linéaire sur E donc un élément, noté g\phi(x), du dual
E^∗ de E. De même, pour chaque y \in E, l'application
x\mapsto~\overline\phi(x,y) est une
forme linéaire sur E, donc un élément, noté d\phi(y), de
E^∗. La relation

\begin{align*} \left
{[}g\phi(\alpha~x + \beta~x')\right {]}(y)& =& \phi(\alpha~x +
\beta~x',y) = \overline\alpha~\phi(x,y) +
\overline\beta~\phi(x',y)\%&
\\ & =& \left
{[}\overline\alpha~g\phi(x) +
\overline\beta~g\phi(x')\right
{]}(y) \%& \\
\end{align*}

montre clairement que g\phi :
x\mapsto~g\phi(x) est une application
semilinéaire de E dans E^∗. Il en est de même de d\phi
: y\mapsto~d\phi(y).

Définition~13.2.2 L'application g\phi : E \rightarrow~ E^∗ (resp.
d\phi) est appelée l'application semilinéaire gauche (resp.
droite) associée à la forme sesquilinéaire \phi.

\paragraph{13.2.2 Formes sesquilinéaires hermitiennes, antihermitiennes}

Définition~13.2.3 Soit \phi \in L3\diagup2(E). On dit que \phi est
hermitienne (resp. antihermitienne) si \forall~~x,y \in
E, \phi(y,x) = \overline\phi(x,y) (resp. =
-\overline\phi(x,y)).

Proposition~13.2.2 \phi est hermitienne si et seulement si~i\phi est
antihermitienne.

Démonstration Evident

Remarque~13.2.3 Ceci nous permettra par la suite de ne considérer que le
cas des formes hermitiennes.

Proposition~13.2.3 Soit \phi \in L3\diagup2(E). Alors \phi est hermitienne
si et seulement si~d\phi = g\phi.

Démonstration En effet \phi(x,y) =\big
{[}g\phi(x)\big {]}(y) et
\overline\phi(y,x) =\big
{[}d\phi(x)\big {]}(y). Alors

\begin{align*} \forall~~x,y \in E,
\overline\phi(y,x) = \phi(x,y)&& \%&
\\ & \Leftrightarrow &
\forall~~x,y \in E, \big
{[}g\phi(x)\big {]}(y) = \epsilon\big
{[}d\phi(x)\big {]}(y)\%&
\\ & \Leftrightarrow &
\forall~x \in E, g\phi(x) = d\phi~(x)
\Leftrightarrow g\phi = d\phi \%&
\\ \end{align*}

Proposition~13.2.4 L'ensemble H(E) des formes sesquilinéaires
hermitiennes est un \mathbb{R}~-sous-espace vectoriel de L3\diagup2(E) (mais
pas un \mathbb{C} sous-espace vectoriel). On a L3\diagup2(E) = H(E) \oplus~ iH(E).

Démonstration La première affirmation est laissée aux soins du lecteur.
On a clairement H(E) \bigcap iH(E) = \0\ et
la relation \phi = \psi + i\theta avec \psi(x,y) = 1 \over 2
(\phi(x,y) + \phi(y,x)), \theta(x,y) = 1 \over 2i (\phi(x,y) -
\phi(y,x)) qui sont toutes deux hermitiennes montre que L3\diagup2(E) =
H(E) \oplus~ iH(E).

\paragraph{13.2.3 Matrice d'une forme sesquilinéaire}

Supposons que E est de dimension finie et soit \mathcal{E} =
(e1,\\ldots,en~)
une base de E.

Définition~13.2.4 Soit \phi \in L3\diagup2(E). On appelle matrice de \phi
dans la base \mathcal{E} la matrice

\mathrmMat~ (\phi,\mathcal{E}) =
(\phi(ei,e\\\\jmathmathmathmath))1\leqi,\\\\jmathmathmathmath\leqn \in M\mathbb{C}(n)

Proposition~13.2.5
\mathrmMat~ (\phi,\mathcal{E}) est
l'unique matrice \Omega \in M\mathbb{C}(n) vérifiant

\forall~(x,y) \in E \times E, \phi(x,y) = X^∗~\OmegaY

où X (resp. Y ) désigne le vecteur colonne des coordonnées de x (resp.
y) dans la base \mathcal{E}.

Démonstration Si \Omega = (\omegai,\\\\jmathmathmathmath), on a

X^∗\OmegaY = \\sum
i=1^n\overlinex i(\OmegaY
)i = \\sum
i=1^n\overlinex i
\sum \\\\jmathmathmathmath=1^n\omega~
i,\\\\jmathmathmathmathy\\\\jmathmathmathmath = \\sum
i,\\\\jmathmathmathmath\omegai,\\\\jmathmathmathmath\overlinexiy\\\\jmathmathmathmath

Mais d'autre part \phi(x,y) =
\phi(\\sum ~
i=1^nxiei,\\\sum
 \\\\jmathmathmathmath=1^ny\\\\jmathmathmathmathe\\\\jmathmathmathmath)
= \\sum ~
i,\\\\jmathmathmathmath\phi(ei,e\\\\jmathmathmathmath)\overlinexiy\\\\jmathmathmathmath
en utilisant la sesquilinéarité de \phi. Ceci montre que
\mathrmMat~ (\phi,\mathcal{E}) vérifie
bien la relation voulue. Inversement, si \Omega vérifie cette formule, on a
\phi(ek,el) = Ek^∗\OmegaEl
= \\sum ~
i,\\\\jmathmathmathmath\omegai,\\\\jmathmathmathmath\deltai^k\delta\\\\jmathmathmathmath^l =
\omegak,l ce qui montre que \Omega =\
\mathrmMat (\phi,\mathcal{E}).

Théorème~13.2.6 L'application
\phi\mapsto~\mathrmMat~
(\phi,\mathcal{E}) est un isomorphisme d'espaces vectoriels de L3\diagup2(E) sur
M\mathbb{C}(n).

Démonstration Les détails sont laissés aux soins du lecteur.
L'application réciproque est bien entendu l'application qui à \Omega \in
M\mathbb{C}(n) associe \phi : E \times E \rightarrow~ \mathbb{C} définie par \phi(x,y) =
X^∗\OmegaY qui est clairement sesquilinéaire.

Théorème~13.2.7 Soit E de dimension finie, \mathcal{E} =
(e1,\\ldots,en~)
une base de E, \mathcal{E}^∗ =
(e1^∗,\\ldots,en^∗~)
la base duale. Soit \phi \in L3\diagup2(E). Alors

\mathrmMat~ (\phi,\mathcal{E}) =
\overline\mathrmMat~
(d\phi,\mathcal{E},\mathcal{E}^∗) =
^t \mathrmMat~
(g \phi,\mathcal{E},\mathcal{E}^∗)

Démonstration Notons \Omega =\
\mathrmMat (\phi,\mathcal{E}), A =\
\mathrmMat (d\phi,\mathcal{E},\mathcal{E}^∗) et B
= \mathrmMat~
(g\phi,\mathcal{E},\mathcal{E}^∗). On a

\begin{align*}
\overline\omegai,\\\\jmathmathmathmath& =&
\overline\phi(ei,e\\\\jmathmathmathmath) =
\left (d\phi(e\\\\jmathmathmathmath)\right
)(ei) \%& \\ & =&
\left (\\sum
k=1^na
k,\\\\jmathmathmathmathek^∗\right )(e i) =
ai,\\\\jmathmathmathmath\%& \\
\end{align*}

compte tenu de ek^∗(ei) =
\deltak^i~; de même

\begin{align*} \omegai,\\\\jmathmathmathmath& =&
\phi(ei,e\\\\jmathmathmathmath) = \left
(g\phi(ei)\right )(e\\\\jmathmathmathmath) \%&
\\ & =& \left
(\sum k=1^nb~
k,iek^∗\right )(e \\\\jmathmathmathmath) =
b\\\\jmathmathmathmath,i\%& \\
\end{align*}

ce qui démontre le résultat.

Corollaire~13.2.8 La forme sesquilinéaire \phi est hermitienne si et
seulement si~sa matrice dans la base \mathcal{E} est hermitienne.

Le rang de \mathrmMat~
(d\phi,\mathcal{E},\mathcal{E}^∗) est indépendant du choix de la base \mathcal{E}~;
il en est donc de même du rang de
\mathrmMat~ (\phi,\mathcal{E}). Ceci
conduit à la définition suivante

Définition~13.2.5 Soit E de dimension finie et \phi \in L3\diagup2(E). On
appelle rang de E le rang de sa matrice dans n'importe quelle base de E.
On a

\mathrmrg~\phi
= \mathrmrgd\phi~
= \mathrmrgg\phi~
=\
\mathrmrg\mathrmMat~
(\phi,\mathcal{E})

\paragraph{13.2.4 Changements de bases}

Théorème~13.2.9 Soit E un espace vectoriel de dimension finie, \mathcal{E} et \mathcal{E}'
deux bases de E, P = P\mathcal{E}^\mathcal{E}' la matrice de passage de \mathcal{E} à
\mathcal{E}'. Soit \phi \in L3\diagup2(E), \Omega =\
\mathrmMat (\phi,\mathcal{E}) et \Omega' =\
\mathrmMat (\phi,\mathcal{E}'). Alors

\Omega' = P^∗\OmegaP

Démonstration Si X (resp. Y ) désigne le vecteur colonne des coordonnées
de x (resp. y) dans la base \mathcal{E} et X' (resp. Y ') désigne le vecteur
colonne des coordonnées de x (resp. y) dans la base \mathcal{E}', on a X = PX', Y
= PY ', d'où

\phi(x,y) = (PX')^∗\Omega(PY `) = X'^∗(P^∗\OmegaP)Y
'

Comme \Omega' est l'unique matrice vérifiant \forall~~(x,y)
\in E \times E, \phi(x,y) = X'^∗\Omega'Y ', on a \Omega' = P^∗\OmegaP.

\paragraph{13.2.5 Orthogonalité}

Soit E un \mathbb{C}-espace vectoriel ~et \phi une forme sesquilinéaire hermitienne
sur E.

Définition~13.2.6 On dit que x est orthogonal à y (relativement à \phi), et
on pose x \bot y, si \phi(x,y) = 0.

Remarque~13.2.4 \phi étant supposée hermitienne, il s'agit visiblement
d'une relation symétrique

Définition~13.2.7 Soit A une partie de E. On pose A^\bot =
\x \in
E∣\forall~~a \in A, \phi(a,x) =
0\

Remarque~13.2.5 Notons A^\bot^∗  l'orthogonal de A
dans le dual E^∗ de E, c'est-à-dire l'espace vectoriel des
formes linéaires sur E qui sont nulles sur A. On a

\begin{align*} x \in A^\bot&
\Leftrightarrow & \forall~~a \in A, \phi(a,x)
= 0 \%& \\ &
\Leftrightarrow & \forall~~a \in A,
\big {[}d\phi(x)\big {]}(a) = 0
\%& \\ & \Leftrightarrow &
d\phi(x) \in A^\bot^∗ 
\Leftrightarrow x \in
d\phi^-1(A^\bot^∗ )\%&
\\ \end{align*}

On en déduit que A^\bot =
d\phi^-1(A^\bot^∗ ) =
g\phi^-1(A^\bot^∗ ).

Proposition~13.2.10 Soit A une partie de E~; alors

\begin{itemize}
\itemsep1pt\parskip0pt\parsep0pt
\item
  (i)A^\bot est un sous-espace vectoriel de E
\item
  (ii)A^\bot =\
  \mathrmVect(A)^\bot
\item
  (iii) A \subset~ (A^\bot)^\bot
\item
  (iv) A \subset~ B \rigtharrow~ B^\bot\subset~ A^\bot.
\end{itemize}

Démonstration (i) découle immédiatement de la sesquilinéarité de \phi ou de
la remarque précédente. Il en est de même pour (ii) puisqu'un vecteur x
est orthogonal à tout vecteur de A si et seulement si il est orthogonal
à toute combinaison linéaire de vecteurs de A, c'est à dire à
\mathrmVect~(A). En ce qui
concerne (iii), il suffit de remarquer que tout vecteur a de A est
orthogonal à tout vecteur qui est orthogonal à tout vecteur de A. Pour
(iv), un vecteur x qui est orthogonal à tout vecteur de B est évidemment
orthogonal à tout vecteur de A.

\paragraph{13.2.6 Formes non dégénérées}

En règle générale on posera

Définition~13.2.8 Soit E un \mathbb{C}-espace vectoriel , \phi une forme
sesquilinéaire hermitienne sur E. On appelle noyau de \phi le sous-espace

\mathrmKer~\phi =
\x \in
E∣\forall~~y \in E, \phi(x,y) =
0\ = E^\bot =\
\mathrmKerd \phi

Définition~13.2.9 Soit E un \mathbb{C}-espace vectoriel , \phi une forme
sesquilinéaire hermitienne sur E. On dit que \phi est non dégénérée si elle
vérifie les conditions équivalentes

\begin{itemize}
\itemsep1pt\parskip0pt\parsep0pt
\item
  (i) \mathrmKer~\phi =
  E^\bot = \0\
\item
  (ii) pour x \in E on a \left
  (\forall~~y \in E, \phi(x,y) = 0\right ) \rigtharrow~
  x = 0
\item
  (iii) d\phi (resp. g\phi) est une application
  semilinéaire in\\\\jmathmathmathmathective de E dans E^∗.
\end{itemize}

L'équivalence entre ces trois propriétés est évidente.

Si E est un espace vectoriel de dimension finie, on sait que
dim E^∗~ =\
dim E. Si g\phi est in\\\\jmathmathmathmathective, elle est nécessairement
bi\\\\jmathmathmathmathective et on obtient

Théorème~13.2.11 Soit E un \mathbb{C}-espace vectoriel ~de dimension finie, \phi une
forme sesquilinéaire hermitienne non dégénérée sur E. Alors
l'application semilinéaire gauche g\phi est un isomorphisme
d'espace vectoriel de E sur E^∗~; autrement dit, pour toute
forme linéaire f sur E, il existe un unique vecteur vf \in E tel
que \forall~x \in E, f(x) = \phi(vf~,x).

Corollaire~13.2.12 Soit E un \mathbb{C}-espace vectoriel ~de dimension finie, \phi
une forme sesquilinéaire hermitienne non dégénérée sur E. Soit A un
sous-espace vectoriel de E. Alors dim~ A
+ dim A^\bot~ =\
dim E et A = A^\bot\bot.

Démonstration On a en effet

dim A^\bot~ =\
dim g \phi^-1(A^\bot^∗ )
= dim A^\bot^∗ ~
= dim E -\ dim~ A

puisque g\phi est un isomorphisme d'espaces vectoriels. On sait
d'autre part que A \subset~ A^\bot\bot et que
dim A^\bot\bot~ =\
dim E - dim A^\bot~
= dim~ A, d'où l'égalité.

Remarque~13.2.6 Il ne faudrait pas en déduire abusivement que A et
A^\bot sont supplémentaires~; en effet, en général A \bigcap
A^\bot\neq~\0\.
Nous nous intéresserons plus particulièrement à ce point dans le
paragraphe suivant.

Si \mathcal{E} est une base de E, alors \Omega =\
\mathrmMat (\phi,\mathcal{E}) =
^t \mathrmMat~
(g\phi,\mathcal{E},\mathcal{E}^∗) et
\mathrmrg~\phi
= \mathrmrg~\Omega. On en déduit

Théorème~13.2.13 Soit E un \mathbb{C}-espace vectoriel ~de dimension finie n, \phi
une forme sesquilinéaire hermitienne sur E, \mathcal{E} une base de E et \Omega
= \mathrmMat~ (\phi,\mathcal{E}). Alors
les propositions suivantes sont équivalentes

\begin{itemize}
\itemsep1pt\parskip0pt\parsep0pt
\item
  (i) \phi est non dégénérée
\item
  (ii) \Omega est une matrice inversible
\item
  (iii) \mathrmrg~\phi = n.
\end{itemize}

Remarque~13.2.7 En général,
\mathrmKer~\phi
= \mathrmKerg\phi~,
\mathrmrg~\phi
= \mathrmrgg\phi~, si
bien que le théorème du rang devient

Proposition~13.2.14 Soit E un \mathbb{C}-espace vectoriel ~de dimension finie n,
\phi une forme sesquilinéaire hermitienne sur E, \mathcal{E} une base de E. Alors
dim~ E =\
\mathrmrg\phi + dim~
\mathrmKer~\phi.

{[}
{[}
{[}
{[}
