\textbf{Warning: 
requires JavaScript to process the mathematics on this page.\\ If your
browser supports JavaScript, be sure it is enabled.}

\begin{center}\rule{3in}{0.4pt}\end{center}

{[}
{[}
{[}{]}
{[}

\subsubsection{6.2 Développements limités}

\paragraph{6.2.1 Notion de développement limité}

Définition~6.2.1 Soit I un intervalle de \mathbb{R}~ et a \in I. Soit f : I \rightarrow~ E et n
\in \mathbb{N}~. On dit que f admet en a un développement limité à l'ordre n s'il
existe
a0,a1,\\ldots,an~
\in E tels que, au voisinage de a, f(t) = a0 + a1(t -
a) + \\ldots~ +
an(t - a)^n + o((t - a)^n).

Remarque~6.2.1 On notera aussi f(t) = P(t - a) + o((t - a)^n)
et on parlera un peu abusivement du polynôme P.

Proposition~6.2.1 Si f admet en a un développement limité à l'ordre n,
alors celui ci est unique.

Démonstration Supposons que l'on ait deux développements distincts~:
f(t) = a0 + a1(t-a) +
\\ldots~ +
an(t-a)^n + o((t-a)^n) = b0 +
b1(t-a) +
\\ldots~ +
bn(t-a)^n + o((t-a)^n)et soit p
=\
min\k∣ak\mathrel\neq~bk\.
Alors on a par soustraction (ap - bp)(t -
a)^p = o((t - a)^n) ce qui est absurde.

Proposition~6.2.2 Si f admet en a un développement limité à l'ordre n,
alors f est continue en a. Si n ≥ 1, alors f est dérivable en a.

Démonstration On a bien entendu f(a) = a0 et
limt\rightarrow~af(t) = a0~ d'où la
continuité. Si n ≥ 1, on a  f(t)-f(a) \over t-a =
f(t)-a0 \over t-a = a1 + o(1) de
limite a1 quand t tend vers a.

Remarque~6.2.2 Ceci ne s'étend pas à des ordres supérieurs~; la fonction
f(t) = t^100 sin~ ( 1
\over t^100 ) si
t\neq~0, f(0) = 0 admet en 0 un développement
limité à l'ordre 99 puisque f(t) = o(t^99) (la fonction
sin~ étant bornée)~; pourtant f n'est pas 2
fois dérivable en 0 puisque sa dérivée est définie par f'(0) = 0 et
f'(x) = 100t^99 sin~ ( 1
\over t^100 ) - 100 \over
t  cos~ ( 1 \over
t^100 )~; elle n'est pas continue en 0, donc pas dérivable.
Par contre on a

Théorème~6.2.3 Si f : I \rightarrow~ E est n fois dérivable au point a, alors f
admet en a le développement limité à l'ordre n

f(t) = f(a) + \sum k=1^n~
f^(k)(a) \over k! (t - a)^k +
o((t - a)^n)

Démonstration C'est la formule de Taylor Young, démontrée dans le
chapitre sur les fonctions d'une variable réelle.

Remarque~6.2.3 Ce théorème permet, en connaissant les dérivées
successives de la fonction f (ce qui est finalement assez rare), de
calculer un développement limité~; mais cela permet également en
connaissant un développement limité à l'ordre n de la fonction f en a
(par exemple à l'aide des méthodes du paragraphe suivant), d'en déduire
les dérivées successives de la fonction f en a.

\paragraph{6.2.2 Opérations sur les développements limités}

Proposition~6.2.4 Si f,g : I \rightarrow~ E admettent en a des développements
limités à l'ordre n, f(t) = P(t - a) + o((t - a)^n),g(t) =
Q(t - a) + o((t - a)^n), alors \alpha~f + \beta~g admet en a le
développement limité à l'ordre n, (\alpha~f + \beta~g)(t) = (\alpha~P + \beta~Q)(t - a) +
o((t - a)^n).

Démonstration Découle immédiatement des propriétés de la relation de
prépondérance.

Proposition~6.2.5 Si f,g : I \rightarrow~ K admettent en a des développements
limités à l'ordre n, f(t) = P(t - a) + o((t - a)^n),g(t) =
Q(t - a) + o((t - a)^n), alors fg admet en a le développement
limité à l'ordre n, f(t)g(t) = R(t - a) + o((t - a)^n), où R
est le polynôme obtenu en tronquant à l'ordre n le polynôme PQ.

Démonstration On a f(t)g(t) =
P(t-a)Q(t-a)+P(t-a)(t-a)^n\epsilon2(t-a)+Q(t-a)(t-a)^n\epsilon1(t-a)+(t-a)^2n\epsilon1(t-a)\epsilon2(t-a)
avec limt\rightarrow~a\epsiloni~(t - a) = 0.
On a donc f(t)g(t) = P(t - a)Q(t - a) + o((t - a)^n). Mais on
a P(X)Q(X) = R(X) + X^n+1S(X), d'où P(t - a)Q(t - a) = R(t -
a) + o((t - a)^n), et donc f(t)g(t) = R(t - a) + o((t -
a)^n).

Proposition~6.2.6 Si f,g : I \rightarrow~ K admettent en a des développements
limités à l'ordre n, f(t) = P(t - a) + o((t - a)^n),g(t) =
Q(t - a) + o((t - a)^n), et si
g(a)\neq~0, alors  f \over g
admet en a le développement limité à l'ordre n,  f(t)
\over g(t) = R(t - a) + o((t - a)^n), où R
est le quotient de la division suivant les puissances croissantes à
l'ordre n du polynôme P(X) par le polynôme R(X).

Démonstration Remarquons que g(a) = Q(0), donc
Q(0)\neq~0. On a

\begin{align*} f(t) \over g(t)
- P(t - a) \over Q(t - a) && \%&
\\ & =& (f(t) - P(t - a))Q(t - a) +
P(t - a)(Q(t - a) - g(t)) \over Q(t - a)g(t) \%&
\\ & =& o((t - a)^n) \%&
\\ \end{align*}

puisque f(t) - P(t - a) = o((t - a)^n), Q(t - a) = O(1), g(t)
- Q(t - a) = o((t - a)^n), P(t - a) = O(1) et
limt\rightarrow~a~ 1 \over
Q(t-a)g(t) = 1 \over g(a)^2 . Ecrivons
alors P(X) = Q(X)R(X) + X^n+1S(X) (division suivant les
puissances croissantes de P par Q à l'ordre n, possible car
Q(0)\neq~0). On a alors  P(t-a)
\over Q(t-a) = R(t - a) + (t - a)^n+1
S(t-a) \over Q(t-a) = R(t - a) + o((t -
a)^n) puisque limt\rightarrow~a~
S(t-a) \over Q(t-a) = S(0) \over
Q(0) . En définitive  f(t) \over g(t) = R(t - a) +
o((t - a)^n).

Le théorème suivant sera uniquement formulé en 0 pour des raisons de
commodité~; on se ramène immédiatement à cette situation par des
translations sur les variables.

Théorème~6.2.7 Soit I,J deux intervalles de \mathbb{R}~ contenant 0, \phi : I \rightarrow~ J
vérifiant \phi(0) = 0 et admettant en 0 un développement limité à l'ordre
n, \phi(t) = P(t) + o(t^n)~; soit f : J \rightarrow~ E admettant en 0 un
développement limité à l'ordre n, f(u) = Q(u) + o(u^n). Alors
f \cdot \phi admet en 0 un développement limité à l'ordre n, f \cdot \phi(t) = R(t) +
o(t^n) où R(X) est le polynôme obtenu en tronquant à l'ordre
n le polynôme Q(P(X)).

Démonstration On écrit f(\phi(t)) = a0 + a1\phi(t) +
\\ldots~ +
an\phi(t)^n + \phi(t)^n\epsilon(\phi(t)). Mais chacune
des fonctions \phi(t)^i admet d'après la proposition précédente
un développement \phi(t)^i = P(t)^i +
o(t^n). On a donc f(\phi(t)) = a0 + a1P(t) +
\\ldots~ +
anP(t)^n + o(t^n) +
\phi(t)^n\epsilon(\phi(t)). Mais comme \phi admet en 0 un développement
limité à l'ordre 1 et que \phi(0) = 0, on a \phi(t) = O(t) et donc
\phi(t)^n\epsilon(\phi(t)) = o(t^n). On obtient donc f \cdot \phi(t) =
Q(P(t)) + o(t^n) = R(t) + t^n+1S(t) +
o(t^n) = R(t) + o(t^n).

Les deux résultats suivants découlent immédiatement de la formule de
Taylor-Young et de l'unicité du développement limité

Proposition~6.2.8 Soit f : I \rightarrow~ E une fonction n fois dérivable au point
a \in I, admettant en a le développement limité à l'ordre n, f(t) =
a0 + a1(t - a) +
\\ldots~ +
an(t - a)^n + o((t - a)^n). Soit F : I \rightarrow~
E une fonction dérivable telle que F' = f. Alors F admet en a le
développement limité à l'ordre n + 1, F(t) = F(a) + a0(t - a)
+ a1 \over 2 (t - a)^2 +
\\ldots~ +
an \over n+1 (t - a)^n+1 + o((t
- a)^n+1).

Proposition~6.2.9 Soit f : I \rightarrow~ \mathbb{R}~ une fonction continue strictement
monotone, n fois dérivable au point 0 telle que f(0) = 0 et
f'(0)\neq~0. Soit J l'intervalle f(I). Alors g =
f^-1 : J \rightarrow~ \mathbb{R}~ admet en 0 un développement limité à l'ordre n~:
g(t) = b1t +
\\ldots~ +
bnt^n + o(t^n)~; on obtient ce
développement limité en identifiant le développement limité de g(f(t))
au polynôme t, ce qui conduit à un système triangulaire en les inconnues
b1,\\ldots,bn~.

\paragraph{6.2.3 Développements limités classiques}

On part d'un certain nombre de développements limités classiques obtenus
par la formule de Taylor-Young et on en déduit d'autres par changements
de variables et intégration. On obtient les développements suivants en 0

\begin{align*} e^t& =& 1 + t +
t^2 \over 2 +
\\ldots~ +
t^n \over n! + o(t^n) \%&
\\ cos~ t& =& 1
- t^2 \over 2! +
\\ldots~ +
(-1)^n t^2n \over (2n)! +
o(t^2n+1) \%& \\
sin t& =& t - t^3~
\over 3! +
\\ldots~ +
(-1)^n t^2n+1 \over (2n + 1)! +
o(t^2n+2)\%& \\
\mathrmch~ t& =& 1 +
t^2 \over 2! +
\\ldots~ +
t^2n \over (2n)! + o(t^2n+1) \%&
\\
\mathrmsh~ t& =& t +
t^3 \over 3! +
\\ldots~ +
t^2n+1 \over (2n + 1)! +
o(t^2n+2) \%& \\ (1 +
t)^\alpha~& =& 1 + \alpha~t + \alpha~(\alpha~ - 1) \over 2!
t^2 +
\\ldots~ \%&
\\ & \text & + \alpha~(\alpha~
- 1)\\ldots~(\alpha~ - n +
1) \over n! t^n + o(t^n) \%&
\\  1 \over 1 + t &
=& 1 - t + t^2 +
\\ldots~ +
(-1)^nt^n + o(t^n) \%&
\\  1 \over 1 - t &
=& 1 + t + t^2 +
\\ldots~ +
t^n + o(t^n) \%& \\
log (1 + t)& =& t - t^2~
\over 2 +
\\ldots~ +
(-1)^n t^n \over n +
o(t^n) \%& \\
log (1 - t)& =& -t - t^2~
\over 2
-\\ldots~ -
t^n \over n + o(t^n) \%&
\\
\mathrmarctg~ t& =& t -
t^3 \over 3 +
\\ldots~ +
(-1)^n t^2n+1 \over 2n + 1 +
o(t^2n+2) \%& \\
arg~
\mathrmth~ t& =& t +
t^3 \over 3 +
\\ldots~ +
t^2n+1 \over 2n + 1 + o(t^2n+2)
\%& \\ arcsin~
t& =& t + t^3 \over 6 +
\\ldots~ \%&
\\ & \text & +
1.3\\ldots~(2n - 1)
\over
2.4\\ldots(2n)~ 
t^2n+1 \over 2n + 1 + o(t^2n+2)
\%& \\ arg~
\mathrmsh~ t& =& t -
t^3 \over 6 +
\\ldots~ \%&
\\ & \text &
+(-1)^n
1.3\\ldots~(2n - 1)
\over
2.4\\ldots(2n)~ 
t^2n+1 \over 2n + 1 + o(t^2n+2)
\%& \\ \end{align*}

{[}
{[}
{[}
{[}
