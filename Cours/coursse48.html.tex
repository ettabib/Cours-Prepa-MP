\textbf{Warning: 
requires JavaScript to process the mathematics on this page.\\ If your
browser supports JavaScript, be sure it is enabled.}

\begin{center}\rule{3in}{0.4pt}\end{center}

{[}
{[}
{[}{]}
{[}

\subsubsection{8.5 Fonctions classiques}

\paragraph{8.5.1 Fonctions circulaires réciproques}

Le lecteur démontrera sans difficulté les résultats suivants qui
découlent immédiatement des caractérisations des homéomorphismes et des
difféomorphismes d'un intervalle sur un autre intervalle de \mathbb{R}~.

Théorème~8.5.1 (i)
x\mapsto~cos~ x est un
homéomorphisme décroissant de {[}0,\pi~{]} sur {[}-1,1{]}~;
l'homéomorphisme réciproque est noté arccos~ :
{[}-1,1{]} \rightarrow~ {[}0,\pi~{]}~; arccos~ est
C^\infty~ sur {]} - 1,1{[} et arccos~ '(x)
= - 1 \over \sqrt1-x^2
. (ii) x\mapsto~sin~ x est
un homéomorphisme croissant de {[}-\pi~\diagup2,\pi~\diagup2{]} sur {[}-1,1{]}~;
l'homéomorphisme réciproque est noté arcsin~ :
{[}-1,1{]} \rightarrow~ {[}-\pi~\diagup2,\pi~\diagup2{]}~; arcsin~ est
C^\infty~ sur {]} - 1,1{[} et arcsin~ '(x)
= 1 \over \sqrt1-x^2 .
(iii) x\mapsto~tan~ x est
un C^\infty~ difféomorphisme croissant de {]} - \pi~\diagup2,\pi~\diagup2{[} sur {]}
-\infty~,+\infty~{[}~; le difféomorphisme réciproque est noté
arctan~ :{]} -\infty~,+\infty~{[}\rightarrow~{]} - \pi~\diagup2,\pi~\diagup2{[} et
arctan~ '(x) = 1 \over
1+x^2 .

Remarque~8.5.1 Pour t \in {[}-1,1{]},

x = arccos t \mathrel\Leftrightarrow~ t
= cos x\text et ~x \in
{[}0,\pi~{]}

x = arcsin t \mathrel\Leftrightarrow~ t
= sin x\text et ~x \in
{[}-\pi~\diagup2,\pi~\diagup2{]}

Pour t \in \mathbb{R}~,

x = arctan t \mathrel\Leftrightarrow~ t
= tan x\text et ~x \in{]} -
\pi~\diagup2,\pi~\diagup2{[}

cos (\arccos~ t) = t,
sin (\arccos~ t) =
\sqrt1 - t^2,
tan (\arccos~ t) =
\\ldots~

cos (\arcsin~ t) =
\sqrt1 - t^2,
sin (\arcsin~ t) = t,
tan (\arcsin~ t) =
\\ldots~

cos (\arctan~ t) = 1
\over \sqrt1 + t^2 ,
sin (\arctan~ t) =
\\ldots~,
tan (\arctan~ t) = t

\paragraph{8.5.2 Fonctions hyperboliques directes}

\mathrmch~ x = 1
\over 2 (e^x + e^-x),
\mathrmsh~ x = 1
\over 2 (e^x - e^-x),
\mathrmth~ x =
\mathrmsh~ x
\over
\mathrmch x~ =
e^2x-1 \over e^2x+1

 \mathrmch~
`= \mathrmsh~ ,
\mathrmsh~'
= \mathrmch~ ,
\mathrmth~ ' = 1
-\mathrmth ^2~

 \mathrmch~ x
+ \mathrmsh~ x =
e^x, \mathrmch~ x
-\mathrmsh~ x =
e^-x

 \mathrmch ^2~x
-\mathrmsh ^2~x =
1

\mathrmch~ (a + b)
= \mathrmch~
a\mathrmch~ b
+ \mathrmsh~
a\mathrmsh~ b

\mathrmch~ (a - b)
= \mathrmch~
a\mathrmch~ b
-\mathrmsh~
a\mathrmsh~ b

\mathrmsh~ (a + b)
= \mathrmsh~
a\mathrmch~ b
+ \mathrmch~
a\mathrmsh~ b

\mathrmsh~ (a - b)
= \mathrmsh~
a\mathrmch~ b
-\mathrmch~
a\mathrmsh~ b

\mathrmch~ 2a =
2\mathrmch ^2~a -
1 = 1 + 2\mathrmsh~
^2a = \mathrmch~
^2a + \mathrmsh~
^2a

\mathrmsh~ 2a =
2\mathrmsh~
a\mathrmch~ a,
\mathrmth~ 2a =
2 \mathrmth~ a
\over
1+\mathrmth~
^2a

Si t = \mathrmth~ ( x
\over 2 ),

\mathrmch~ x = 1 +
t^2 \over 1 - t^2 ,
\mathrmsh~ x = 2t
\over 1 - t^2 ,
\mathrmth~ x = 2t
\over 1 + t^2

\paragraph{8.5.3 Fonctions hyperboliques réciproques}

Théorème~8.5.2 (i)
x\mapsto~\mathrmch~
x est un homéomorphisme croissant de {[}0,+\infty~{[} sur {[}1,+\infty~{[}~;
l'homéomorphisme réciproque est noté arg~
\mathrmch~ : {[}1,+\infty~{[}\rightarrow~
{[}0,+\infty~{[}~; arg~
\mathrmch~ est
C^\infty~ sur {]}1,+\infty~{[} et arg~
\mathrmch~ '(x) = 1
\over \sqrtx^2  -1 . (ii)
x\mapsto~\mathrmsh~
x est un C^\infty~ difféomorphisme croissant de {]} -\infty~,+\infty~{[} sur
{]} -\infty~,+\infty~{[}~; le difféomorphisme réciproque est noté
arg~
\mathrmsh~ :{]} -\infty~,+\infty~{[}\rightarrow~{]}
-\infty~,+\infty~{[} et on a arg~
\mathrmsh~ '(x) = 1
\over \sqrt1+x^2 . (iii)
x\mapsto~\mathrmth~
x est un C^\infty~ difféomorphisme croissant de {]} -\infty~,+\infty~{[} sur
{]} - 1,+1{[}~; le difféomorphisme réciproque est noté
arg~
\mathrmth~ :{]} - 1,1{[}\rightarrow~{]}
-\infty~,+\infty~{[} et on a arg~
\mathrmth~ '(x) = 1
\over 1-x^2 .

Pour t ≥ 1,\quad x = arg~
\mathrmch~ t
\Leftrightarrow t =\
\mathrmch x\text et x ≥ 0.

Pour t \in \mathbb{R}~,\quad x = arg~
\mathrmsh~ t
\Leftrightarrow t =\
\mathrmsh x.

Pour t \in{]} - 1,1{[}, \quad x =\
arg \mathrmth~ t
\Leftrightarrow t =\
\mathrmth x.

arg~
\mathrmch~ t
= log (t + \sqrtt~^2
 - 1),\quad arg~
\mathrmsh~ t
= log (t + \sqrtt~^2
 + 1),\quad arg~
\mathrmth~ t = 1
\over 2  log~ ( 1+t
\over 1-t )

{[}
{[}
{[}
{[}
