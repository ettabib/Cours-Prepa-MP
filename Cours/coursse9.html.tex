\textbf{Warning: 
requires JavaScript to process the mathematics on this page.\\ If your
browser supports JavaScript, be sure it is enabled.}

\begin{center}\rule{3in}{0.4pt}\end{center}

{[}
{[}
{[}{]}
{[}

\subsubsection{2.3 Rang}

\paragraph{2.3.1 Rang d'une famille de vecteurs}

Définition~2.3.1
\mathrmrg(xi)i\inI~
= dim~
\mathrmVect(xi~,i
\in I) =\
sup\\textbar{}J\textbar{}∣(xi)i\inJ\text
libre \

Démonstration L'égalité provient évidemment du théorème de la base
incomplète qui garantit que l'on peut extraire de la famille
(xi) une base du sous-espace
\mathrmVect(xi~).

Une application linéaire transformant une famille liée en une famille
liée, on a~:

Théorème~2.3.1 Soit u \in L(E,F) et (xi)i\inI une
famille de E. Alors
\mathrmrg(u(xi))i\inI~
\leq\mathrmrg(xi)i\inI~.

Recherche pratique~: voir le rang d'une matrice.

\paragraph{2.3.2 Rang d'une application linéaire}

Définition~2.3.2 Soit u \in L(E,F). Alors
\mathrmrg~u
= dim~
\mathrmIm~u \in \mathbb{N}~
\cup\ + \infty~\.

Remarque~2.3.1 Recherche pratique~: Si (ei)i\inI est
une base de E, (u(ei))i\inI est une famille
génératrice de \mathrmIm~u
et donc \mathrmrg~u
=\
\mathrmrg(u(ei))i\inI.

Théorème~2.3.2 Soit u \in L(E,F). (i) Si dim~ E
\textless{} +\infty~, alors u est de rang fini,
\mathrmrg~u
= dim E -\ dim~
\mathrmKer~u~; on a
\mathrmrg~u
= dim~ E si et seulement si u est in\\\\jmathmathmathmathectif (ii)
Si dim~ F \textless{} +\infty~, alors u est de rang
fini, \mathrmrg~u
\leq dim~ F~; on a
\mathrmrg~u
= dim~ F si et seulement si u est sur\\\\jmathmathmathmathectif

Démonstration Découle immédiatement des résultats sur la dimension.

Remarque~2.3.2 On a donc dans tous les cas
\mathrmrg~u
\leq min(\dim~
E,dim~ F)

Proposition~2.3.3 Soit u \in L(E,F),v \in L(F,G). Alors
\mathrmrg~v \cdot u
\leq\
min(\mathrmrgv,\\mathrmrg~u).
Si u est bi\\\\jmathmathmathmathectif,
\mathrmrg~v \cdot u
= \mathrmrg~v. Si v est
bi\\\\jmathmathmathmathectif, \mathrmrg~v \cdot u
= \mathrmrg~u.

Démonstration Il suffit de remarquer que
\mathrmIm~v \cdot u = v(u(E)).

Théorème~2.3.4 On suppose ici dim~ E
= dim~ F \textless{} +\infty~, u \in L(E,F). On a
équivalence de

\begin{itemize}
\itemsep1pt\parskip0pt\parsep0pt
\item
  (i) u in\\\\jmathmathmathmathective
\item
  (ii) u sur\\\\jmathmathmathmathective
\item
  (iii) u bi\\\\jmathmathmathmathective
\item
  (iv) \exists~v \in L(F,E), v \cdot u =
  \mathrmIdE
\item
  (v) \exists~v \in L(F,E), u \cdot v =
  \mathrmIdF
\end{itemize}

Démonstration L'équivalence entre (i), (ii) et (iii) est une compilation
des résultats précédents. De plus (iii) \rigtharrow~ (iv) et (v), (iv) \rigtharrow~ (i) et (v)
\rigtharrow~(ii), ce qui boucle la boucle.

{[}
{[}
{[}
{[}
