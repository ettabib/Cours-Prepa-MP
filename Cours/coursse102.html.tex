\textbf{Warning: 
requires JavaScript to process the mathematics on this page.\\ If your
browser supports JavaScript, be sure it is enabled.}

\begin{center}\rule{3in}{0.4pt}\end{center}

{[}
{[}
{[}{]}
{[}

\subsubsection{19.3 Equations de surfaces}

\paragraph{19.3.1 Surfaces cartésiennes et nappes paramétrées}

Nous avons vu précédemment que, au voisinage d'un point régulier, une
nappe paramétrée était équivalente à une nappe cartésienne, donc définie
par une équation du type z = f(x,y) dans un repère convenablement
choisi. Inversement toute nappe cartésienne est bien évidemment une
nappe paramétrée par x = u,y = v,z = f(u,v).

Pla\ccons nous maintenant du point de vue d'un
sous-ensemble de \mathbb{R}~^3 défini par une équation du type f(x,y,z)
= 0 où f est une fonction de classe C^k d'un ouvert U de
\mathbb{R}~^3 dans \mathbb{R}~. Soit donc \Sigma = \(x,y,z) \in
U∣f(x,y,z) = 0\. Supposons
qu'en un point (a,b,c) de \Sigma on ait ( \partial~f \over \partial~x
(a,b,c), \partial~f \over \partial~y (a,b,c), \partial~f
\over \partial~z (a,b,c))\neq~(0,0,0).
Quitte à permuter les noms des coordonnées, on peut supposer par exemple
 \partial~f \over \partial~z (a,b,c)\neq~0. Le
théorème des fonctions implicites nous garantit qu'il existe
U0 ouvert contenant (a,b), V 0 ouvert contenant c et
\phi : U0 \rightarrow~ V 0 de classe C^k telle que

\forall~(x,y) \in U0~,
\forall~z \in V 0~, f(x,y,z) = 0
\Leftrightarrow z = \phi(x,y)

Autrement dit \Sigma, au voisinage de (a,b,c) est l'image d'une nappe
cartésienne. Inversement, il est clair que l'image d'une nappe
cartésienne z = \phi(x,y) est définie par l'équation f(x,y,z) = 0 où
f(x,y,z) = z - \phi(x,y) (avec d'ailleurs  \partial~f \over \partial~z =
1). On dira que \Sigma est une surface cartésienne quand elle vérifie
\forall~(a,b,c) \in \Sigma, ( \partial~f \over \partial~x~
(a,b,c), \partial~f \over \partial~y (a,b,c), \partial~f
\over \partial~z (a,b,c))\neq~(0,0,0)
(la véritable dénomination étant en fait sous variété de dimension 2 de
\mathbb{R}~^3).

Ceci nous montre donc, qu'au moins localement les trois points de vue
(nappe cartésienne, nappe paramétrée et surfaces cartésiennes) sont
équivalents avec certaines hypothèses de régularité. On retiendra en
particulier sous ces trois formes les expressions du plan tangent et du
vecteur normal.

Nappes paramétrées (u,v)\mapsto~F(u,v) =
(\phi(u,v),\psi(u,v),\omega(u,v))

Le plan tangent en (u0,v0) est le plan
F(u0,v0) +\
\mathrmVect( \partial~F \over \partial~u
(u0,v0), \partial~F \over \partial~v
(u0,v0)) d'équation

\left
\textbar{}\matrix\,x -
\phi(u0,v0)& \partial~\phi \over \partial~u
(u0,v0)& \partial~\phi \over \partial~v
(u0,v0) \cr y -
\psi(u0,v0)& \partial~\psi \over \partial~u
(u0,v0)& \partial~\psi \over \partial~v
(u0,v0) \cr z -
\omega(u0,v0)& \partial~\omega \over \partial~u
(u0,v0)& \partial~\omega \over \partial~v
(u0,v0)\right \textbar{} = 0

avec comme vecteur normal  \partial~F \over \partial~u
(u0,v0) ∧ \partial~F \over \partial~v
(u0,v0) = \left
(\matrix\, \partial~\phi \over
\partial~u (u0,v0) \cr  \partial~\psi
\over \partial~u (u0,v0) \cr
 \partial~\omega \over \partial~u
(u0,v0)\right )
∧\left (\matrix\, \partial~\phi
\over \partial~v (u0,v0) \cr
 \partial~\psi \over \partial~v (u0,v0)
\cr  \partial~\omega \over \partial~v
(u0,v0)\right )

Nappes cartésiennes z = f(x,y)

En utilisant le paramétrage
(x,y)\mapsto~(x,y,f(x,y)) et les formules
précédentes, on obtient les formules suivantes.

Le plan tangent en (x0,y0) est le plan d'équation
(en utilisant les notations de Monge p = \partial~f \over \partial~x
(x0,y0), q = \partial~f \over \partial~y
(x0,y0))

\left
\textbar{}\matrix\,x -
x0&1&0 \cr y - y0&0&1
\cr z -
f(x0,y0)&p&q\right \textbar{} = 0

avec comme vecteur normal \left
(\matrix\,1 \cr 0
\cr p\right ) ∧\left
(\matrix\,0 \cr 1
\cr q\right ) = \left
(\matrix\,-p \cr -q
\cr 1 \right )

Surfaces cartésiennes f(x,y,z) = 0

A l'aide du théorème des fonctions implicites, on a obtenu les résultats
suivants.

Le plan tangent en (x0,y0,z0) est le plan
d'équation

(x - x0) \partial~f \over \partial~x
(x0,y0,z0) + (y - y0) \partial~f
\over \partial~y (x0,y0,z0) + (z
- z0) \partial~f \over \partial~z
(x0,y0,z0) = 0

avec le vecteur normal
\overrightarrowgradf(x0,y0,z0~)
= \left (\matrix\, \partial~f
\over \partial~x (x0,y0,z0)
\cr  \partial~f \over \partial~y
(x0,y0,z0) \cr  \partial~f
\over \partial~z
(x0,y0,z0)\right )

\paragraph{19.3.2 Cylindres}

Définition~19.3.1 Soit E un espace affine euclidien de dimension 3 et
\vecD une direction de droite. On dit qu'une partie \Sigma
de E est un cylindre de direction \vecD si, pour tout
m \in \Sigma, la droite m +\vec D est contenue dans \Sigma.

Définition~19.3.2 Les droites m +\vec D contenues
dans \Sigma sont appelées les génératrices du cylindre. Un sous-ensemble qui
rencontre toutes les génératrices est appelé un sous-ensemble directeur
du cylindre. Un sous-ensemble directeur plan est appelé une base du
cylindre.

On est parfois amené à rechercher un cylindre connaissant un
sous-ensemble directeur \Gamma et la direction du cylindre
\vecD = \mathbb{R}~\vecu. Nous supposerons
choisi un repère de E ce qui nous permet de supposer que E =
\mathbb{R}~^3. On posera donc \vecu = (\alpha~,\beta~,\gamma).

Premier cas \Gamma est l'image d'un arc paramétré
u\mapsto~(\phi(u),\psi(u),\omega(u)). On obtient immédiatement
une paramétrisation du cylindre par x = \phi(u) + \alpha~v,y = \psi(u) + \beta~v,z = \omega(u)
+ \gammav.

Deuxième cas \Gamma est donnée par deux équations f(x,y,z) = 0,g(x,y,z) = 0.
On écrit alors que

\begin{align*} m(x,y,z) \in \Sigma&
\Leftrightarrow & \exists~t \in \mathbb{R}~, m +
t\vecu \in \Gamma \%& \\ &
\Leftrightarrow & \exists~t \in \mathbb{R}~,
\left
\\matrix\,f(x + t\alpha~,y +
t\beta~,z + t\gamma) = 0 \cr g(x + t\alpha~,y + t\beta~,z + t\gamma) =
0\right .\%& \\
\end{align*}

et on élimine t entre ces équations.

Exemple~19.3.1 Cylindre de direction \vecu = (1,1,1)
de sous-ensemble directeur la parabole y^2 = 2px,z = 0. On
écrit

\begin{align*} m(x,y,z) \in \Sigma&
\Leftrightarrow & \exists~t \in \mathbb{R}~, m +
t\vecu \in \Gamma \%& \\ &
\Leftrightarrow & \exists~t \in \mathbb{R}~,
\left
\\matrix\,(y +
t)^2 = 2p(x + t) \cr z + t =
0\right .\%& \\ &
\Leftrightarrow & (y - z)^2 = 2p(x - z) \%&
\\ \end{align*}

et nous avons obtenu une équation du cylindre.

\paragraph{19.3.3 Cônes}

Définition~19.3.3 Soit E un espace affine euclidien de dimension 3 et S
un point de E. On dit qu'une partie \Sigma de E est un cône de sommet S si,
pour tout m \in \Sigma \diagdown\S\, la droite Sm est
contenue dans \Sigma.

Définition~19.3.4 Les droites Sm contenues dans \Sigma sont appelées les
génératrices du cône. Un sous-ensemble ne contenant pas le sommet qui
rencontre toutes les génératrices est appelé un sous ensemble directeur
du cône. Un sous-espace directeur plan est appelé une base du cône.

On est parfois amené à rechercher un cône connaissant un sous-ensemble
directeur \Gamma et le sommet S du cône. Nous supposerons choisi un repère de
E ce qui nous permet de supposer que E = \mathbb{R}~^3. On posera donc
S = (\alpha~,\beta~,\gamma).

Premier cas \Gamma est l'image d'un arc paramétré
u\mapsto~(\phi(u),\psi(u),\omega(u)). On obtient immédiatement
une paramétrisation du cône par x = v\phi(u) + (1 - v)\alpha~,y = v\psi(u) + (1 -
v)\beta~,z = v\omega(u) + (1 - v)\gamma.

Deuxième cas \Gamma est donnée par deux équations f(x,y,z) = 0,g(x,y,z) = 0.
On écrit alors que

\begin{align*} m(x,y,z) \in \Sigma
\diagdown\S& \Leftrightarrow &
\exists~t \in \mathbb{R}~, S +
t\overrightarrowSm \in \Gamma\%&
\\ \end{align*}

soit encore

\exists~t \in \mathbb{R}~, \left
\\matrix\,f(tx + (1 -
t)\alpha~,ty + (1 - t)\beta~,tz + (1 - t)\gamma) = 0 \cr g(tx + (1 -
t)\alpha~,ty + (1 - t)\beta~,tz + (1 - t)\gamma) = 0\right .

et on élimine t entre ces équations.

Exemple~19.3.2 Cône de sommet S = (0,0,1) de sous-ensemble directeur la
parabole y^2 = 2px,z = 0. On écrit

\begin{align*} m(x,y,z) \in \Sigma
\diagdown\S& \Leftrightarrow &
\exists~t \in \mathbb{R}~, S +
t\overrightarrowSm \in \Gamma \Leftrightarrow
\exists~t \in \mathbb{R}~, \left
\\matrix\,(ty)^2
= 2ptx \cr tz + (1 - t) = 0\right . \%&
\\ & \Leftrightarrow &
z\neq~1\text et
\left ( y \over 1 - z
\right )^2 = 2px \over 1 - z
\Leftrightarrow
z\neq~1\text et y^2
= 2px(1 - z)\%& \\
\end{align*}

et nous avons obtenu une équation du cône (privé de son sommet).

\paragraph{19.3.4 Surfaces de révolution}

Définition~19.3.5 Soit E un espace affine euclidien de dimension 3 et D
une droite de E. On dit qu'une partie \Sigma de E est une surface de
révolution d'axe D si, pour tout m \in \Sigma, le cercle Cm d'axe D
passant par m est contenu dans \Sigma.

Remarque~19.3.1 Les cercles Cm contenus dans \Sigma sont appelés
les parallèles de la surface de révolution. Un sous-ensemble qui
rencontre tous les parallèles est appelé un sous-ensemble directeur de
la surface de révolution. Un sous-ensemble directeur situé dans un plan
contenant l'axe D est appelé un méridien.

Remarque~19.3.2 Supposons que \Sigma soit défini par l'équation f(x,y,z) = 0
et supposons les axes choisis de telle sorte que D soit l'axe 0z. Posons
alors g(\rho,\theta,z) = f(\rhocos~
\theta,\rhosin~ \theta,z) en coordonnées cylindriques. Si g
ne dépend pas de \theta, on voit immédiatement que la surface est de
révolution, admettant pour méridiens les sous-ensembles g(\rho,z) = 0 dans
les plans \rhoOz. On retiendra en particulier que toute équation du type
F(x^2 + y^2,z) = 0 définit une surface de
révolution.

On est parfois amené à rechercher une surface de révolution connaissant
un sous-ensemble directeur \Gamma et l'axe de révolution. Nous supposerons
choisi un repère de E ce qui nous permet de supposer que E =
\mathbb{R}~^3. On posera alors D = (a,b,c) + \mathbb{R}~(\alpha~,\beta~,\gamma).

Premier cas \Gamma est l'image d'un arc paramétré
u\mapsto~(\phi(u),\psi(u),\omega(u)). On obtient immédiatement
une paramétrisation de la surface de révolution par (x,y,z) =
RD(v)(\phi(u),\psi(u),\omega(u)) où RD(\theta) désigne la rotation
d'axe D et d'angle \theta. Si le repère est bien choisi, on peut supposer que
D est l'axe 0z. Alors RD(\theta) est l'application
(x,y,z)\mapsto~(xcos~ \theta -
ysin \theta,x\sin~ \theta +
ycos~ \theta,z) si bien que l'on a la
paramétrisation de la nappe par

x = \phi(u)cos~ v -
\psi(u)sin~ v, y =
\phi(u)sin v + \psi(u)\cos~
v,z = \omega(u)

Deuxième cas \Gamma est donné par deux équations f(x,y,z) = 0,g(x,y,z) = 0.
Remarquons alors que Cm est l'intersection de la sphère de
centre A passant par m avec le plan orthogonal à
\vecu passant par m. Il admet donc pour équations (si
m a pour coordonnées (x0,y0,z0))

\left
\\matrix\,(x -
a)^2 + (y - b)^2 + (z - c)^2 =
(x0 - a)^2 + (y0 - b)^2 +
(z0 - c)^2 \cr \alpha~x + \beta~y + \gammaz =
\alpha~x0 + \beta~y0 + \gammaz0\right .

On écrit alors que

\begin{align*}
m(x0,y0,z0) \in \Sigma&
\Leftrightarrow & Cm \bigcap
\Gamma\neq~\varnothing~\%& \\
\end{align*}

soit encore \exists~x,y,z \in \mathbb{R}~,

\begin{align*} \left
\\matrix\,(x -
a)^2 + (y - b)^2 + (z - c)^2 =
(x0 - a)^2 + (y0 - b)^2 +
(z0 - c)^2 \cr \alpha~x + \beta~y + \gammaz =
\alpha~x0 + \beta~y0 + \gammaz0 \cr
f(x,y,z) = 0 \cr g(x,y,z) = 0\right .&
& \%& \\
\end{align*}

et on élimine x,y et z entre ces équations.

Dans le cas où D est l'axe Oz, on remplacera avantageusement la sphère
par un cylindre d'axe Oz et on obtiendra comme équation de Cm
\left
\\matrix\,x^2
+ y^2 = x0^2 + y0^2
\cr z = z0\right . et donc

\begin{align*}
m(x0,y0,z0) \in \Sigma&
\Leftrightarrow & Cm \bigcap
\Gamma\neq~\varnothing~ \%& \\ &
\Leftrightarrow & \exists~x,y,z \in \mathbb{R}~,
\left
\\matrix\,x^2
+ y^2 = x0^2 + y0^2
\cr z = z0 \cr f(x,y,z) = 0
\cr g(x,y,z) = 0\right .\%&
\\ \end{align*}

et on élimine x,y et z entre ces équations.

Exemple~19.3.3 Surface de révolution engendrée par la rotation de la
droite \Delta : x = 1, y = z autour de l'axe 0z. On a donc

\begin{align*}
m(x0,y0,z0) \in \Sigma&
\Leftrightarrow & Cm \bigcap
\Delta\neq~\varnothing~ \%& \\ &
\Leftrightarrow & \exists~x,y,z \in \mathbb{R}~,
\left
\\matrix\,x^2
+ y^2 = x0^2 + y0^2
\cr z = z0 \cr x = 1
\cr y = z\right .\%&
\\ & \Leftrightarrow &
x0^2 + y 0^2 = 1 + z
0^2 \%& \\
\end{align*}

si bien que la surface a pour équation cartésienne x^2 +
y^2 - z^2 = 1, comme équation cylindrique
\rho^2 - z^2 = 1 et que ses méridiennes sont des
hyperboles équilatères. Il s'agit là d'un hyperboloïde (à une nappe).

{[}
{[}
{[}
{[}
