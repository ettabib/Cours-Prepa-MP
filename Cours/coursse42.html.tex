\textbf{Warning: 
requires JavaScript to process the mathematics on this page.\\ If your
browser supports JavaScript, be sure it is enabled.}

\begin{center}\rule{3in}{0.4pt}\end{center}

{[}
{[}
{[}{]}
{[}

\subsubsection{7.8 Espaces de suites}

Définition~7.8.1 On dit qu'une suite (xn)n\in\mathbb{N}~ de
nombres réels ou complexes est sommable si la série
\\sum  xn~ est
absolument convergente.

Proposition~7.8.1 L'ensemble \ell^1(\mathbb{N}~) des suites sommables de
nombres complexes est un sous espace vectoriel de \mathbb{C}^\mathbb{N}~.
L'application u =
(un)n\in\mathbb{N}~\mapsto~\\textbar{}u\\textbar{}1
= \\sum ~
n=0^+\infty~\textbar{}un\textbar{} est une norme sur
cet espace vectoriel. L'application
u\mapsto~\\\sum
 n\in\mathbb{N}~un est linéaire de \ell^1(\mathbb{N}~) dans \mathbb{C}.

Démonstration Si (un) et (vn) sont deux suites
sommables et \alpha~,\beta~ \in \mathbb{C}, les suites (\textbar{}un\textbar{}) et
(\textbar{}vn\textbar{}) sont sommables~; il en est donc de
même de la suite (\textbar{}\alpha~\textbar{}\textbar{}un\textbar{}
+ \textbar{}\beta~\textbar{}\textbar{}vn\textbar{}) (résultat sur
les séries à réels positifs) et donc de la suite
(\textbar{}\alpha~un + \beta~vn\textbar{}) puisque
\textbar{}\alpha~un +
\beta~vn\textbar{}\leq\textbar{}\alpha~\textbar{}\textbar{}un\textbar{}
+ \textbar{}\beta~\textbar{}\textbar{}vn\textbar{}. Donc la suite
(\alpha~un + \beta~vn) est sommable. La suite nulle étant de
surcroît sommable, l'ensemble \ell^1(\mathbb{N}~) des suites sommables de
nombres complexes est un sous espace vectoriel de \mathbb{C}^\mathbb{N}~. La
vérification des propriétés d'une norme est élémentaire. On a alors

\begin{align*} \\sum
n=0^+\infty~(\alpha~u n + \beta~vn)& =&
limp\rightarrow~+\infty~~\\sum
n=0^p(\alpha~u n + \beta~vn) \%&
\\ & =&
\alpha~limp\rightarrow~+\infty~~\\sum
n=0^pu n +
\beta~limp\rightarrow~+\infty~\\sum
n=0^pv n\%& \\
& =& \alpha~\sum n=0^+\infty~u n~
+ \beta~\sum n=0^+\infty~v n~ \%&
\\ \end{align*}

d'où la linéarité de
u\mapsto~\\\sum
 n=0^+\infty~un.

Proposition~7.8.2 L'ensemble \ell^2(\mathbb{N}~) des suites de nombres
complexes dont les carrés forment une suite sommable est un sous-espace
vectoriel de \mathbb{C}^\mathbb{N}~. L'application (u,v) = \left
((un)n\in\mathbb{N}~,(vn)n\in\mathbb{N}~\right
)\mapsto~(u\mathrel∣v)
= \\sum ~
n=0^+\infty~\overlineunvn
est un produit scalaire hermitien sur cet espace~; en conséquence
l'application u =
(un)n\in\mathbb{N}~\mapsto~\\textbar{}u\\textbar{}2
= \left
(\\sum ~
n=0^+\infty~\textbar{}un\textbar{}^2\right
)^1\diagup2 est une norme sur cet espace vectoriel.

Démonstration Il est clair que si (un) est de carré sommable,
il en est de même de \alpha~(un) = (\alpha~un). Si
(un) et (vn) sont de carré sommable, l'inégalité
élémentaire \textbar{}un + vn\textbar{}^2
\leq 2\textbar{}un\textbar{}^2 +
2\textbar{}vn\textbar{}^2 montre que la suite
(un + vn) est de carré sommable. La suite nulle
étant de surcroît de carré sommable, les suites de carrés sommables
forment donc bien un sous-espace vectoriel de \mathbb{C}^\mathbb{N}~. Si
(un) et (vn) sont de carré sommable, l'inégalité
élémentaire
\textbar{}\overlineunvn\textbar{}\leq
1 \over 2 \textbar{}un\textbar{}^2
+ 1 \over 2
\textbar{}vn\textbar{}^2 montre que la suite
(\overlineunvn) est sommable. On
peut donc poser (u∣v)
= \\sum ~
n=0^+\infty~\overlineunvn.
Il est clair que
(u,v)\mapsto~(u\mathrel∣v) est
sesquilinéaire hermitienne. De plus, si u\neq~0,
(u∣u) \in \mathbb{R}~^+∗ ce qui montre que
cette forme sesquilinéaire est définie positive~; on a donc un produit
scalaire hermitien et la norme associée est
\\textbar{}u\\textbar{}2^2
= (u∣u).

Remarque~7.8.1 Le théorème ci dessus n'est plus valable pour des séries
convergentes~: posons an = bn = (-1)^n
\over \sqrtn+1 . On a
\textbar{}cn\textbar{} =\
\sum  k=0^n~ 1
\over \sqrt(k+1)(n-k+1) . Mais pour
k \in {[}0,n{]}, (k + 1)(n - k + 1) \leq ( n \over 2 +
1)^2 (facile). Donc \textbar{}cn\textbar{}≥ n+1
\over  n \over 2 +1 qui tend vers
2~; donc la suite (cn) ne tend pas vers 0 et la série
\\sum  cn~
diverge.

{[}
{[}
{[}
{[}
