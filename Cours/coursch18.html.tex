\textbf{Warning: 
requires JavaScript to process the mathematics on this page.\\ If your
browser supports JavaScript, be sure it is enabled.}

\begin{center}\rule{3in}{0.4pt}\end{center}

{[}
{[}
{[}{]}
{[}

\subsection{Chapitre~17\\Espaces affines}

Dans tout le chapitre K désignera un corps commutatif.

~17.1 {Généralités sur les espaces
affines} \\ ~~17.1.1 {Notion
d'espace affine} \\ ~~17.1.2

\\ ~~17.1.3 
\\ ~~17.1.4 {Parallélisme,
intersection} \\ ~~17.1.5
 \\
~~17.1.6 {Utilisation de repères
affines} \\ ~~17.1.7 {Formes
affines et sous-espaces affines} \\ ~17.2
 \\ ~~17.2.1
 \\
~~17.2.2 {Associativité des
barycentres} \\ ~~17.2.3
{Barycentres, sous-espaces
affines, applications affines} \\ ~~17.2.4
 \\
~17.3 
\\ ~~17.3.1 {Notion d'espace
affine euclidien} \\ ~~17.3.2
{Formule de Leibnitz et
applications} \\ ~~17.3.3
 \\ ~~17.3.4
{Forme réduite d'une isométrie
affine} \\ ~~17.3.5 {Distance à
un sous-espace affine} \\ ~~17.3.6
{Distance de deux sous-espaces
affines} \\ ~17.4 {Cercles,
sphères, triangle} \\ ~~17.4.1
 \\
~~17.4.2  \\
~~17.4.3 {Eléments de géométrie
du triangle}

{[}
{[}
{[}
{[}
