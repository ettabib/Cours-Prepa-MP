\textbf{Warning: 
requires JavaScript to process the mathematics on this page.\\ If your
browser supports JavaScript, be sure it is enabled.}

\begin{center}\rule{3in}{0.4pt}\end{center}

{[}
{[}
{[}{]}
{[}

\subsubsection{5.5 Compléments~: convexité dans les espaces vectoriels
normés}

\paragraph{5.5.1 Jauge d'un convexe}

Soit E un espace vectoriel normé réel et K un convexe borné qui contient
0 dans son intérieur. On définit alors une application \\\\jmathmathmathmathK de E
dans \mathbb{R}~^+ par

\\\\jmathmathmathmathK(x) = inf~ \\lambda~
\textgreater{} 0∣ x \over
\lambda~ \in K\

Cette définition a bien un sens, car si B(0,r) \subset~ K, on a  x
\over \lambda~ \in B(0,r) \in K dès que \lambda~ \textgreater{}
\\textbar{}x\\textbar{}
\over r .

Définition~5.5.1 La fonction \\\\jmathmathmathmathK est appelée la \\\\jmathmathmathmathauge du
convexe K.

Proposition~5.5.1 Soit E un espace vectoriel normé réel et K un convexe
borné qui contient 0 dans son intérieur. Alors l'application
\\\\jmathmathmathmathK vérifie

\begin{itemize}
\itemsep1pt\parskip0pt\parsep0pt
\item
  (i) \\\\jmathmathmathmathK(x) = 0 \Leftrightarrow x = 0
\item
  (ii) \\\\jmathmathmathmathK(\mux) = \mu\\\\jmathmathmathmathK(x) si \mu ≥ 0
\item
  (iii) \\\\jmathmathmathmathK(x + y) \leq \\\\jmathmathmathmathK(x) + \\\\jmathmathmathmathK(y)
\end{itemize}

Si de plus K = -K, alors \\\\jmathmathmathmathK est une norme.

Démonstration (i) Puisque K est borné, soit M ≥ 0 tel que
\forall~~y \in K,
\\textbar{}y\\textbar{} \leq M. Si
\\\\jmathmathmathmathK(x) = 0, il existe une suite \lambda~n tendant vers 0
telle que  x \over \lambda~n \in K, soit
\\textbar{}x\\textbar{} \leq M\lambda~n.
On a donc x = 0. La réciproque est évidente.

(ii) est évident puisque  x \over \lambda~ \in K
\Leftrightarrow \mux \over \mu\lambda~ \in K

(iii) Supposons que  x \over \lambda~ et  y
\over \mu appartiennent à K. Comme K est convexe, \lambda~ et \mu
positifs, on a aussi  1 \over \lambda~+\mu (\lambda~ x
\over \lambda~ + \mu y \over \mu ) \in K soit
encore  x+y \over \lambda~+\mu \in K. On a donc

\\lambda~ \textgreater{} 0∣ x
\over \lambda~ \in K\ + \\mu
\textgreater{} 0∣ y \over
\mu \in K\ \subset~\\nu \textgreater{}
0∣ x + y \over \nu \in
K\

En prenant les bornes inférieures on a donc \\\\jmathmathmathmathK(x + y) \leq
\\\\jmathmathmathmathK(x) + \\\\jmathmathmathmathK(y).

Si de plus, K = -K, on a \\\\jmathmathmathmathK(-x) = \\\\jmathmathmathmathK(x) et donc
\forall~\mu \in \mathbb{R}~, \\\\jmathmathmathmathK~(\mux) =
\textbar{}\mu\textbar{}\\\\jmathmathmathmathK(x) qui était la seule propriété des
normes qui manquait.

Remarque~5.5.1 On a évidemment, x \in K \rigtharrow~ \\\\jmathmathmathmathK(x) \leq 1 et
\\\\jmathmathmathmathK(x) \textless{} 1 \rigtharrow~ x \in K, autrement dit
B\\\\jmathmathmathmathK(0,1) \subset~ K \subset~ B'\\\\jmathmathmathmathK(0,1)~; si on
suppose de plus que K est fermé, on a facilement K =
B'\\\\jmathmathmathmathK(0,1)~; autrement dit un convexe, fermé, borné
et équilibré (K = -K) est une boule fermée pour une certaine norme~; la
réciproque étant évidente.

\paragraph{5.5.2 Pro\\\\jmathmathmathmathection sur un convexe fermé}

Théorème~5.5.2 Soit E un espace euclidien et K une partie non vide,
convexe fermée de E~; pour tout x de E, il existe un unique élément
pK(x) de K tel que d(x,pK(x)) = d(x,K). Pour y \in K,
on a

y = pK(x) \Leftrightarrow
\forall~~z \in K,\quad (x -
y∣z - y) \leq 0

Démonstration Nous allons donner une démonstration de ce résultat qui ne
fera pas appel à la dimension finie de E, mais uniquement au fait qu'il
est complet. Soit (yn) une suite de K qui vérifie
\\textbar{}x -
yn\\textbar{}^2 \leq
d(x,K)^2 + 1 \over n . L'égalité de la
médiane nous donne alors

\begin{align*}
\\textbar{}yp -
yq\\textbar{}^2&& \%&
\\ & =&
\\textbar{}(yp - x) - (yq -
x)\\textbar{}^2 \%&
\\ & =&
2\\textbar{}yp -
x\\textbar{}^2 +
2\\textbar{}y q -
x\\textbar{}^2 -\\textbar{}
(y p - x) + (yq -
x)\\textbar{}^2\%&
\\ & =&
2\\textbar{}yp -
x\\textbar{}^2 +
2\\textbar{}y q -
x\\textbar{}^2 - 4\\textbar{}
yp + yq \over 2 -
x)\\textbar{}^2 \%&
\\ \end{align*}

avec \\textbar{}x -
yp\\textbar{}^2 \leq
d(x,K)^2 + 1 \over p et
\\textbar{}x -
yq\\textbar{}^2 \leq
d(x,K)^2 + 1 \over q . Mais comme K est
convexe,  yp+yq \over 2 \in K et
donc \\textbar{} yp+yq
\over 2 - x)\\textbar{}^2 ≥
d(x,K)^2. On a donc \\textbar{}yp -
yq\\textbar{}^2 \leq 2( 1
\over p + 1 \over q ). La suite
(yn) est une suite de Cauchy dans E, donc elle converge. Soit
y sa limite dans E. Comme K est fermé, on a y \in K et on a évidemment en
passant à la limite à partir de d(x,K)^2
\leq\\textbar{} x -
yn\\textbar{}^2 \leq
d(x,K)^2 + 1 \over n , l'égalité d(x,K) =
d(x,y).

Soit y ainsi trouvé et soit z \in K. Pour tout t \in {[}0,1{]}, (1 - t)y +
tz \in K et donc \\textbar{}x - (1 - t)y -
tz\\textbar{}^2 ≥\\textbar{} x
- y\\textbar{}^2. En développant, on obtient
t^2\\textbar{}y -
z\\textbar{}^2 - 2t(x -
y∣z - y) ≥ 0. Pour t \in{]}0,1{]} on a donc
t\\textbar{}y - z\\textbar{}^2
- 2(x - y∣z - y) ≥ 0 et en faisant tendre t
vers 0, on obtient (x - y∣z - y) \leq 0.

Inversement supposons que \forall~~z \in
K,\quad (x - y∣z - y) \leq 0.
Alors

\begin{align*} \\textbar{}x -
z\\textbar{}^2& =&
\\textbar{}(x - y) - (z -
y)\\textbar{}^2 \%&
\\ & =& \\textbar{}x -
y\\textbar{}^2 +\\textbar{} z -
y\\textbar{}^2 - 2(x -
y∣z - y)\%& \\
& ≥& \\textbar{}x -
y\\textbar{}^2 \%&
\\ \end{align*}

avec égalité si et seulement si z = y. Ceci montre à la fois que d(x,y)
= d(x,K) et que y est unique.

Remarque~5.5.2 La condition (x - y∣z - y) \leq 0
correspond géométriquement à~: l'angle
(\overrightarrowyx,\overrightarrowyz)
est obtus.

\includegraphics{cours5x.png}

\paragraph{5.5.3 Hahn-Banach (version géométrique)}

Remarque~5.5.3 Il existe plusieurs théorèmes à la Hahn Banach. Certains
sont de type analytique et concernent des propriétés de prolongement de
formes linéaires ou de semi-normes d'un sous-espace vectoriel à l'espace
tout entier. D'autres sont de type géométrique et concernent des
propriétés de séparation d'un convexe et d'un point ou de deux convexes.
Nous avons choisi ici d'en présenter une version géométrique simple.

Théorème~5.5.3 Soit E un espace vectoriel normé de dimension finie, K un
convexe fermé non vide et x∉K. Alors il
existe un hyperplan affine qui sépare strictement x et K, c'est-à-dire
que x et K sont dans les deux demi-espaces ouverts définis par
l'hyperplan.

Démonstration Puisque toutes les normes sont équivalentes, on peut
supposer que E est muni d'une norme euclidienne. Soit alors y la
pro\\\\jmathmathmathmathection de x sur le convexe K. L'hyperplan médiateur du segment
{[}x,y{]} convient évidemment. ~~ \includegraphics{cours6x.png}

Remarque~5.5.4 Une autre fa\ccon de formuler le
théorème est de dire que si K est un convexe fermé et
x∉K, il existe une forme linéaire f sur E
telle que f(x) \textless{} inf~
y\inKf(y).

Corollaire~5.5.4 Soit E un espace vectoriel normé de dimension finie,
K1 un convexe compact non vide et K2 un convexe
fermé non vide tels que K1 \bigcap K2 = \varnothing~. Alors (i) il
existe un hyperplan H qui sépare strictement K1 et
K2 (ii) il existe une forme linéaire f telle que
supx\inK1~f(x)
\textless{} inf x\inK2~f(x)

Démonstration La fonction x\mapsto~d(x,K2)
est continue sur le compact K1, donc atteint sa borne
inférieure en x0. Il suffit alors d'appliquer la méthode
précédente à x0 et à K2.

\paragraph{5.5.4 L'enveloppe convexe~: Carathéodory et Krein Millman}

Définition~5.5.2 Soit E un \mathbb{R}~-espace vectoriel et A une partie de E.
L'ensemble des convexes contenant A admet un plus petit élément appelé
l'enveloppe convexe de A~: c'est encore l'ensemble des barycentres à
coefficients positifs de points de A.

Démonstration L'intersection de tous les convexes contenant A est encore
un convexe contenant A et c'est le plus petit. L'ensemble des
barycentres à coefficients positifs de points de A est un convexe (les
barycentres à coefficients positifs de barycentres à coefficients
positifs sont encore des barycentres à coefficients positifs) contenant
A donc il contient l'enveloppe convexe~; mais comme celle-ci est stable
par barycentrage à coefficients positifs, elle doit contenir tout
barycentre à coefficients positifs de points de A, d'où l'égalité.

Théorème~5.5.5 (Carathéodory). Soit n = dim~ E.
Alors l'enveloppe convexe de A est encore l'ensemble des barycentres à
coefficients positifs de n + 1 points de A.

Démonstration Il suffit évidemment de démontrer que si x est barycentre
à coefficients positifs de p ≥ n + 2 points de A, c'est encore un
barycentre à coefficients positifs de p - 1 points de A. Soit donc x
= \\sum ~
i=1^p\lambda~ixi avec \lambda~i ≥ 0 et
\\sum  \lambda~i~ = 1.
La famille (xi - xp)1\leqi\leqp-1 de E est une
famille de p - 1 ≥ n + 1 éléments dans E de dimension n, donc elle est
liée. On peut trouver
\alpha~1,\\ldots,\alpha~p-1~
non tous nuls tels que
\\sum ~
i=1^p-1\alpha~i(xi - xp) = 0.
Posons \alpha~p = -(\alpha~1 +
\\ldots~ +
\alpha~p-1). On a donc
\\sum ~
i=1^p\alpha~ixi = 0 avec
\\sum ~
i=1^p\alpha~i = 0. Soit t \in \mathbb{R}~^+. On a
alors x = \\sum ~
i=1^p(\lambda~i - t\alpha~i)xi avec
\\sum ~
i=1^p(\lambda~i - t\alpha~i) = 1. Il suffit alors
de choisir t de telle sorte que \forall~~i,
\lambda~i - t\alpha~i ≥ 0 avec pour un certain i0,
\lambda~i0 - t\alpha~i0 = 0 pour aboutir au
résultat souhaité. Or, si \alpha~i \leq 0, on a évidemment \lambda~i
- t\alpha~i ≥ 0. Il suffit donc de considérer les \alpha~i
\textgreater{} 0 et de prendre t =\
min\ \lambda~i \over
\alpha~i ∣\alpha~i \textgreater{}
0\ = \lambda~i0 \over
\alpha~i0 .

Corollaire~5.5.6 Soit E un \mathbb{R}~-espace vectoriel de dimension finie et A
une partie compacte de E. Alors l'enveloppe convexe de A est encore
compacte.

Démonstration Soit n = dim~ E,

C =
\(\lambda~1,\\ldots,\lambda~n+1~)
\in
\mathbb{R}~^n+1∣\forall~~i,
\lambda~ i ≥ 0\text et \\sum
\lambda~i = 1\

C est une partie compacte de \mathbb{R}~^n+1 (car fermée et bornée dans
un espace vectoriel normé~de dimension finie) et l'enveloppe convexe de
A est l'image de l'application continue \phi : C \times A^n+1 \rightarrow~ E
définie par
\phi(\lambda~1,\\ldots,\lambda~n+1,x1,\\\ldots,xn+1~)
= \\sum ~
i=1^n+1\lambda~ixi. Comme C \times
A^n+1 est compacte, cette image est compacte.

Remarque~5.5.5 Soit maintenant K un convexe. On peut essayer de trouver
une partie minimale de K qui engendre K, c'est-à-dire dont K soit
l'enveloppe convexe. Une telle partie doit évidemment contenir les
points de K qui ne sont pas barycentres d'autres points de K (autrement
que de fa\ccon triviale). Nous allons voir que pour
un convexe compact, ces points suffisent presque à engendrer K.

Définition~5.5.3 Soit K un convexe. Un point x de K est dit un point
extrémal de K si on a

\forall~~y,z \in K,\quad x \in {[}y,z{]} \rigtharrow~
x = y\text ou x = z

Un sous-ensemble S de K est dit extrémal si

\forall~~y,z \in K,\quad
{]}y,z{[}\bigcapS\neq~\varnothing~\rigtharrow~ y \in S\text et
z \in S

Lemme~5.5.7 Soit K un convexe compact, f une forme linéaire sur E, \mu
= supx\inK~f(x). Alors K' =
\x \in K∣f(x) =
\mu\ est un sous-ensemble compact extrémal de K.

Démonstration En effet, soit y,z \in K, x \in{]}y,z{[}\bigcapK'~; on a f(x) = \mu
avec x = ty + (1 - t)z et t \in{]}0,1{[}. Alors \mu = tf(y) + (1 - t)f(z)
avec t \textgreater{} 0, 1 - t \textgreater{} 0, f(y) \leq \mu, f(z) \leq \mu~;
ceci n'est possible que si f(y) = \mu et f(z) = \mu, soit y \in K' et z \in K'.

Théorème~5.5.8 (Krein-Millman). Soit E un \mathbb{R}~-espace vectoriel de
dimension finie et K un convexe compact de E. Alors K est l'adhérence de
l'enveloppe convexe de ses points extrémaux.

Démonstration Nous montrerons ce résultat par récurrence sur
dim~ E (le cas de la dimension 1 est laissé au
lecteur). Soit P l'ensemble des compacts extrémaux non vides de K.
Remarquons que tout intersection d'éléments de P est soit vide, soit
encore dans P. Soit S \inP. Montrons tout d'abord que S contient un point
extrémal. Si toute forme linéaire f est constante sur S, alors S est un
singleton réduit à un point extrémal. Sinon, soit f une forme linéaire
non constante sur S, \mu = supx\inS~f(x)
et S' = \x \in S∣f(x) =
\mu\. Alors S' est un sous ensemble convexe compact de
l'hyperplan H d'équation f(x) = \mu. En vectorialisant cet hyperplan, on
obtient par récurrence que S' admet un point extrémal x.

Montrons par l'absurde que x est un point extrémal de K. Si x \in{]}y,z{[}
avec y,z \in K, on a {]}y,z{[}\bigcapS\neq~\varnothing~, donc y \in S
et z \in S. Mais comme S' est un sous-ensemble extrémal de S et
{]}y,z{[}\bigcapS'\neq~\varnothing~, on a y,z \in S'~; ceci
contredit le fait que x soit un point extrémal de S'. On a donc montré
que toute partie compacte extrémale contenait un point extrémal.

Soit donc K0 l'adhérence de l'enveloppe convexe des points
extrémaux de K. On a K0 \subset~ K et puisque tout ensemble extrémal
contient un point extrémal, K0 rencontre tout ensemble
extrémal. Supposons que K0\neq~K et
soit x \in K \diagdown K0. D'après le théorème de Hahn Banach, il existe
une forme linéaire f telle que f(x)
\textgreater{}\
supy\inK0f(y). Soit \mu =\
supz\inKf(z) et S = \z \in
K∣f(z) = \mu\. S est non vide
(une fonction continue sur un compact atteint sa borne supérieure),
extrémal d'après le lemme précédent et S \bigcap K0 = \varnothing~ (car si y \in
K0, f(y) \textless{} f(x) \leq \mu). Donc S est un sous-ensemble
extrémal qui ne contient aucun point extrémal. C'est absurde. Donc K =
K0.

Exemple~5.5.1 Un polygone et plus généralement un polyèdre est enveloppe
convexe de ses sommets.

Remarque~5.5.6 En dimension finie, on peut affiner le résultat en
montrant qu'en fait K est l'enveloppe convexe de ses points extrémaux,
et pas seulement l'adhérence de l'enveloppe convexe. Ceci nécessite une
version plus fine de Hahn-Banach.

{[}
{[}
{[}
{[}
