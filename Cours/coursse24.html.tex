\textbf{Warning: 
requires JavaScript to process the mathematics on this page.\\ If your
browser supports JavaScript, be sure it is enabled.}

\begin{center}\rule{3in}{0.4pt}\end{center}

{[}
{[}
{[}{]}
{[}

\subsubsection{4.7 Espaces complets}

\paragraph{4.7.1 Suites de Cauchy}

Définition~4.7.1 Soit (E,d) un espace métrique. Une suite (xn)
de E est dite suite de Cauchy si elle vérifie

\forall~~\epsilon \textgreater{} 0,
\exists~N \in \mathbb{N}~,\quad p,q ≥ N \rigtharrow~
d(xp,xq) \textless{} \epsilon

Remarque~4.7.1 On peut sans nuire à la généralité remplacer par

\forall~~\epsilon \textgreater{} 0,
\exists~N \in \mathbb{N}~,\quad q \textgreater{} p
≥ N \rigtharrow~ d(xp,xq) \textless{} \epsilon

Remarque~4.7.2 Il s'agit là d'une notion métrique et non topologique. La
suite (n) est une suite de Cauchy dans \mathbb{R}~ muni de la distance de
\overline\mathbb{R}~, mais pas de Cauchy dans \mathbb{R}~ muni de la
distance usuelle. Par contre, pour deux distances équivalentes, les
suites de Cauchy sont les mêmes.

Théorème~4.7.1

\begin{itemize}
\itemsep1pt\parskip0pt\parsep0pt
\item
  (i) Toute suite convergente est une suite de Cauchy
\item
  (ii) Toute suite de Cauchy qui a une valeur d'adhérence est
  convergente.
\item
  (iii) Toute suite de Cauchy est bornée.
\item
  (iv) L'image par une application uniformément continue d'une suite de
  Cauchy est une suite de Cauchy.
\end{itemize}

Démonstration (i) Soit \ell = limxn~ et
\epsilon \textgreater{} 0. Il existe N \in \mathbb{N}~ tel que n ≥ N \rigtharrow~ d(xn,\ell)
\textless{} \epsilon\diagup2. Alors p,q ≥ N \rigtharrow~ d(xp,xq) \leq
d(xp,\ell),+d(\ell,xq) \textless{} \epsilon. Donc (xn)
est une suite de Cauchy.

(ii) Soit \ell une valeur d'adhérence de la suite de Cauchy (xn)
et \epsilon \textgreater{} 0. Il existe N \in \mathbb{N}~ tel que p,q ≥ N \rigtharrow~
d(xp,xq) \textless{} \epsilon\diagup2. De plus il existe
n0 ≥ N tel que d(xn0,\ell) \textless{} \epsilon\diagup2.
Pour n ≥ N, on a d(xn,\ell) \leq
d(xn,xn0) + d(xn0,\ell)
\textless{} \epsilon\diagup2 + \epsilon\diagup2 = \epsilon. Donc \ell est limite de (xn).

(iii) Il existe N \in \mathbb{N}~ tel que p,q ≥ N \rigtharrow~ d(xp,xq)
\textless{} 1. Alors
\xn∣n \in
\mathbb{N}~\
\subset~\x0,\\ldots,xN-1\~
\cup B'(xN,1) qui est un ensemble borné.

(iv) Soit \epsilon \textgreater{} 0. Il existe \eta \textgreater{} 0 tel que
d(x,x') \textless{} \eta \rigtharrow~ d(f(x),f(x')) \textless{} \epsilon. Pour ce \eta, il
existe N \in \mathbb{N}~ tel que p,q ≥ N \rigtharrow~ d(xp,xq) \textless{}
\eta. Alors p,q ≥ N \rigtharrow~ d(f(xp),f(xq)) \textless{} \epsilon.

Remarque~4.7.3 Une suite de Cauchy a donc soit aucune valeur d'adhérence
(si elle diverge), soit une valeur d'adhérence si elle converge.

L'image par une application continue d'une suite de Cauchy n'est pas en
général une suite de Cauchy~: prendre f :{]}0,+\infty~{[}\rightarrow~{]}0,+\infty~{[},
x\mapsto~1\diagupx et xn = 1\diagupn.

\paragraph{4.7.2 Espaces complets}

Définition~4.7.2 Un espace métrique (E,d) est dit complet si toute suite
de Cauchy de E converge dans E

Remarque~4.7.4 Il s'agit d'une notion métrique et non topologique~: bien
que la topologie soit la même, \mathbb{R}~ est complet pour la distance usuelle,
mais pas pour la distance de \overline\mathbb{R}~ (la suite (n)
est une suite de Cauchy non convergente)

Remarque~4.7.5 L'intérêt essentiel d'un espace complet est que dans un
tel espace, on peut assurer la convergence d'une suite sans exhiber au
préalable sa limite.

Théorème~4.7.2 Soit (E,d) un espace métrique et F une partie de E. (i)
Si (F,dF) est complet, alors F est fermé dans E (ii)
Inversement, si E est complet et F fermé dans E alors (F,dF)
est complet

Démonstration (i) Soit x \in E qui est limite d'une suite (xn)
de F. La suite (xn) est une suite de Cauchy~dans E, donc dans
F, donc admet une limite \ell dans F. mais l'unicité de la limite dans E
garantit que x = \ell \in F. Donc F est fermé dans E.

(ii) Soit (xn) une suite de Cauchy~dans F~; c'est aussi une
suite de Cauchy~dans E donc elle converge vers x \in E (car E est
complet)~; mais comme F est fermé et x est limite d'une suite d'éléments
de F, x appartient à F et il est évidemment limite dans F de la suite
(xn).

Proposition~4.7.3 Si (E1,d1) et
(E2,d2) sont deux espaces métriques complets, alors
l'espace métrique produit est complet.

Démonstration Soit (zn) une suite de Cauchy dans E =
E1 \times E2, zn = (xn,yn).
On a d(zp,zq) =\
max(d1(xp,xq),d2(yp,yq))
et donc d1(xp,xq) \leq
d(zp,zq). On en déduit que (xn) est une
suite de Cauchy~de E1 et donc converge vers x \in E1.
De même (yn) converge vers y dans E2 et alors
(zn) converge vers (x,y) \in E.

\paragraph{4.7.3 Propriétés des espaces complets}

Théorème~4.7.4 (théorème des fermés emboîtés) Soit (E,d) un espace
métrique complet et (Fn)n\in\mathbb{N}~ une suite de parties
fermées non vides de E vérifiant

\begin{itemize}
\itemsep1pt\parskip0pt\parsep0pt
\item
  (i) \forall~n, Fn+1 \subset~ Fn~
\item
  (ii) limn\rightarrow~+\infty~\delta(Fn~) = 0
\end{itemize}

Alors \⋂ ~
n\in\mathbb{N}~Fn est un singleton (et en particulier non vide).

Démonstration Choisissons xn \in Fn. Si q ≥ p, on a
xp,xq \in Fp et donc
d(xp,xq) \leq \delta(Fp) ce qui montre que la
suite (xn) est une suite de Cauchy. Par conséquent elle
converge dans E, soit a sa limite. On a a =\
limn\rightarrow~+\infty~xn+p et \forall~~n,
xn+p \in Fp. Comme Fp est fermé, a
appartient à Fp et donc a
\in\⋂ ~
p\in\mathbb{N}~Fp. Maintenant, si a,b
\in\⋂ ~
p\in\mathbb{N}~Fp, on a d(a,b) \leq \delta(Fp) pour tout p et
donc d(a,b) = 0, soit a = b.

Remarque~4.7.6 La condition
limn\rightarrow~+\infty~\delta(Fn~) = 0 est
essentielle pour démontrer que l'intersection est non vide comme le
montre l'exemple Fn = {[}n,+\infty~{[} dans \mathbb{R}~, où l'intersection est
vide.

Théorème~4.7.5 (Critère de Cauchy pour les fonctions) Soit E un espace
métrique, (F,d) un espace métrique complet, A \subset~ E, a
\in\overlineA, f une fonction de E vers F telle que A
\subset~ Def~ (f). Alors f admet une limite en a
suivant A si et seulement si~elle vérifie

\forall~~\epsilon \textgreater{} 0,
\exists~U \in V (a),\quad x,x' \in U \bigcap A \rigtharrow~
d(f(x),f(x')) \textless{} \epsilon

Démonstration La condition est nécessaire car si \ell
= limx\rightarrow~a,x\inA~f(x) et \epsilon \textgreater{}
0, il existe U \in V (a) tel que x \in U \bigcap A \rigtharrow~ d(f(x),\ell) \textless{} \epsilon\diagup2.
Alors, pour x,x' \in U \bigcap A on a d(f(x),f(x')) \leq d(f(x),\ell) + d(\ell,f(x'))
\textless{} \epsilon\diagup2 + \epsilon\diagup2 = \epsilon. Montrons maintenant qu'elle est suffisante.
Pour cela, soit (an) une suite de A convergeant vers a et \epsilon
\textgreater{} 0 et soit U \in V (a) tel que x,x' \in U \bigcap A \rigtharrow~ d(f(x),f(x'))
\textless{} \epsilon~; il existe N \in \mathbb{N}~ tel que n ≥ N \rigtharrow~ an \in U. Pour n
≥ N, on a an \in U \bigcap A et donc

p,q ≥ N \rigtharrow~ ap,aq \in U \bigcap A \rigtharrow~
d(f(ap),f(aq)) \textless{} \epsilon

La suite (f(an)) est donc une suite de Cauchy de F, donc elle
converge. On a donc montré que pour toute suite (an) de A de
limite a, la suite (f(an)) converge~; on en déduit que f a une
limite en a suivant A.

Théorème~4.7.6 (théorème du point fixe). Soit (E,d) un espace métrique
complet et f : E \rightarrow~ E une application contractante (lipschitzienne de
rapport strictement inférieur à 1). Alors f a un unique point fixe qui
est limite de toutes les suites (xn) définies par la
récurrence~: x0 \in E et xn+1 = f(xn).

Démonstration Ecrivons d(f(x),f(y)) \leq kd(x,y) avec k \textless{} 1. Pour
l'unicité, supposons que f(x) = x et f(y) = y. On a d(x,y) \leq kd(x,y)
avec k \textless{} 1 et d(x,y) ≥ 0. ce n'est possible que si d(x,y) = 0
et donc x = y. En ce qui concerne l'existence, soit x0 \in E et
la suite définie par la récurrence xn+1 = f(xn). Si
n ≥ 1, on a d(xn+1,xn) =
d(f(xn),f(xn-1)) \leq kd(xn,xn-1)
d'où en définitive d(xn+1,xn) \leq
k^nd(x0,x1). Mais alors, si q
\textgreater{} p on a

\begin{align*} d(xp,xq)& \leq&
d(xp,xp+1) +
\\ldots~ +
d(xq-1,xq) \%& \\ &
\leq& (k^p +
\\ldots~ +
k^q-1)d(x 0,x1) \leq k^p
d(x0,x1) \over 1 - k \%&
\\ \end{align*}

Comme k \textless{} 1, on a
limk^p~
d(x0,x1) \over 1-k = 0, et donc la
suite est une suite de Cauchy~; elle admet donc une limite x. On a x
= limxn+1~ =\
limf(xn) = f(limxn~) =
f(x) car f est continue. Donc x est point fixe de f.

{[}
{[}
{[}
{[}
