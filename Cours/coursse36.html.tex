\textbf{Warning: 
requires JavaScript to process the mathematics on this page.\\ If your
browser supports JavaScript, be sure it is enabled.}

\begin{center}\rule{3in}{0.4pt}\end{center}

{[}
{[}
{[}{]}
{[}

\subsubsection{7.2 Généralités sur les séries}

\paragraph{7.2.1 Notion de série}

Définition~7.2.1 Soit E un espace vectoriel normé~et (xn) une
suite de E. On appelle sommes partielles de la série
\\sum  xn~ les
Sn = \\sum ~
p=0^nxp (notée Sn(x) s'il y a risque
de confusion). On dit que la série converge si la suite des sommes
partielles converge dans E~; sa limite est alors appelée la somme de la
série et notée \\sum ~
n=0^+\infty~xn =\
limn\rightarrow~+\infty~\\\sum
 p=0^nxp. Une série non convergente est
dite divergente.

Remarque~7.2.1 Soit (an) une suite de E. Définissons une suite
(xn) par x0 = a0 et pour n ≥ 1,
xn = an - an-1. On a immédiatement
Sn(x) = an et donc la série
\\sum  xn~
converge si et seulement si~la suite (an) converge~; dans ce
cas on a d'ailleurs liman~
= \\sum ~
n=0^+\infty~xn. Ceci peut permettre dans certains cas
de ramener une étude de convergence de suite à une étude de convergence
de série.

Proposition~7.2.1 Soit E un espace vectoriel normé,
\\sum  xn~ et
\\sum  yn~ deux
séries d'éléments de E. On suppose qu'il existe N \in \mathbb{N}~ tel que n ≥ N \rigtharrow~
xn = yn (autrement dit les deux suites ne diffèrent
que par un nombre fini de termes). Alors les deux séries sont de même
nature (simultanément convergentes ou divergentes).

Démonstration Pour n ≥ N, on a Sn(x) = Sn(y) +
(SN(x) - SN(y)) donc l'une des suites Sn
converge si et seulement si~l'autre converge.

Remarque~7.2.2 En faisant tendre n vers + \infty~, on obtient S(x) = S(y) +
(SN(x) - SN(y)).

Définition~7.2.2 Soit E un espace vectoriel normé,
\\sum  xn~ une
série convergente et p \in \mathbb{N}~. Alors la série
\\sum ~
n≥p+1xn est convergente~; sa somme est notée
Rp (ou Rp(x)). On a par définition Sn +
Rn = \\sum ~
p=0^+\infty~xp et
limRn~ = 0.

Proposition~7.2.2 Soit E un espace vectoriel normé. Alors l'ensemble des
suites (xn) telles que la série
\\sum  xn~
convergent est un sous-espace vectoriel de E^\mathbb{N}~. L'application
(xn)n\in\mathbb{N}~\mapsto~\\\sum
 n=0^+\infty~xn est linéaire de ce sous-espace
vectoriel dans E.

Démonstration Il suffit de remarquer que si \alpha~ et \beta~ sont des scalaires,
Sn(\alpha~x + \beta~y) = \alpha~Sn(x) + \beta~Sn(y).

\paragraph{7.2.2 Terme général, critère de Cauchy}

Théorème~7.2.3 Si la série
\\sum  xn~
converge, alors la suite (xn) admet 0 pour limite.

Démonstration xn = Sn - Sn-1 et les deux
suites ont la même limite S =\
\sum  n=0^+\infty~xn~.

Théorème~7.2.4 (critère de Cauchy pour les séries). Soit E un espace
vectoriel normé~complet et
\\sum  xn~ une
série à termes de E. La série
\\sum  xn~
converge si et seulement si~elle vérifie

\forall~~\epsilon \textgreater{} 0,
\exists~N \in \mathbb{N}~, q ≥ p ≥ N
\rigtharrow~\\textbar{}\\sum
n=p^qx n\\textbar{}
\textless{} \epsilon

Démonstration C'est simplement le critère de Cauchy pour la suite
(Sn) des sommes partielles puisque
\\sum ~
n=p^qxn = Sq - Sp-1.

Exemple~7.2.1 La série harmonique
\\sum  n≥1~ 1
\over n diverge puisque  1 \over n+1
+ \\ldots~ + 1
\over 2n ≥ n \times 1 \over 2n = 1
\over 2 . La série ne vérifie donc pas le critère de
Cauchy (bien que lim~ 1 \over
n = 0), donc elle diverge.

{[}
{[}
{[}
{[}
