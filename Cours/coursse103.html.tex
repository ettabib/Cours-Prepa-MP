\textbf{Warning: 
requires JavaScript to process the mathematics on this page.\\ If your
browser supports JavaScript, be sure it is enabled.}

\begin{center}\rule{3in}{0.4pt}\end{center}

{[}
{[}
{[}{]}
{[}

\subsubsection{19.4 Quadriques}

\paragraph{19.4.1 Notion de quadrique}

Définition~19.4.1 Soit E un espace affine de dimension finie de
direction \vecE et F : E \rightarrow~ \mathbb{R}~. On dit que F est une
forme quadratique affine si elle vérifie les conditions équivalentes (i)
il existe a \in E, une forme quadratique \Phia sur
\vecE et une forme linéaire fa sur
\vecE telles que \forall~~x \in E,
F(x) = \Phia(\overrightarrowax) +
fa(\overrightarrowax) + F(a) (ii) pour tout
a \in E, il existe une forme quadratique \Phia sur
\vecE et une forme linéaire fa sur
\vecE telles que \forall~~x \in E,
F(x) = \Phia(\overrightarrowax) +
fa(\overrightarrowax) + F(a) (iii) pour
tout repère affine
(a,\overrightarrowe1,\\ldots,\overrightarrowen~),
il existe un polynôme P \in
\mathbb{R}~{[}X1,\\ldots,Xn~{]}
de degré inférieur ou égal à 2 tel que F(x) =
P(x1,\\ldots,xn~)
si
x1,\\ldots,xn~
sont les coordonnées de x dans ce repère. La forme quadratique
\Phia est en fait indépendante de a \in E~; on l'appelle la forme
quadratique principale de F.

Démonstration Il est clair que (ii) \rigtharrow~(i). Supposons (i) vérifié et soit
b \in E. On a alors par l'identité de polarisation

\begin{align*} F(x)& =&
\Phia(\overrightarrowab
+\overrightarrow bx) +
fa(\overrightarrowab
+\overrightarrow bx) + F(a) \%&
\\ & =&
\Phia(\overrightarrowbx) +
2\phia(\overrightarrowab,\overrightarrowbx)
+ fa(\overrightarrowbx) +
\Phia(\overrightarrowab) +
fa(\overrightarrowab) + F(a)\%&
\\ & =&
\Phib(\overrightarrowbx) +
fb(\overrightarrowbx) + F(b) \%&
\\ \end{align*}

en posant \Phib = \Phia et
fb(\overrightarrow\xi~) =
2\phia(\overrightarrowab,\overrightarrow\xi~)
+ fa(\overrightarrow\xi~). Ceci montre à la
fois que (i) \rigtharrow~(ii) et que \Phia ne dépend pas de a.

L'équivalence entre (i) et (iii) résulte immédiatement des isomorphismes
dé\\\\jmathmathmathmathà connus entre formes quadratiques et polynômes homogènes de degré 2,
formes linéaires et polynômes homogènes de degré 1, constantes et
polynômes homogènes de degré 0. La décomposition F(x) =
\Phia(\overrightarrowax) +
fa(\overrightarrowax) + F(a) correspond
exactement à la décomposition P = P2 + P1 +
P0 d'un polynôme de degré au plus 2 en un polynôme homogène de
degré 2, un polynôme homogène de degré 1 et une constante.

Remarque~19.4.1 Contrairement à la forme quadratique principale \Phi qui ne
dépend pas de a, la forme linéaire fa dépend de a. Supposons
que \Phi est non dégénérée~; on sait alors que l'on peut trouver un vecteur
\vecv \in\vec E tel que
\forall~\vec\xi~~
\in\vec E, fa(\vec\xi~) =
\phi(\vecv,\vec\xi~). Prenons alors b =
a - 1 \over 2 \vecv. On a alors
fb(\vec\xi~) =
2\phi(\overrightarrowab,\vec\xi~) +
fa(\vec\xi~) =
\phi(2\overrightarrowab +\vec
v,\vec\xi~) = \phi(0,\vec\xi~) = 0, si
bien que fb = 0.

Définition~19.4.2 On dit que a \in E est un centre de la forme quadratique
affine si fa = 0.

Remarque~19.4.2 On a donc montré que si \Phi est non dégénérée, F admet un
centre.

Définition~19.4.3 On dit qu'un sous-ensemble \Sigma de E est une quadrique
(ou une conique en dimension 2) s'il existe une forme quadratique affine
de forme quadratique principale non nulle telle que \Sigma =
\x \in E∣F(x) =
0\.

\paragraph{19.4.2 Réduction des quadriques}

Supposons que E est un espace affine euclidien. Soit \Sigma une quadrique
d'équation F(x) = 0 et soit \Phi la forme quadratique principale de F. On
sait qu'il existe une base orthonormée
(\overrightarrowe1,\\ldots,\overrightarrowen~)
de \vecE qui est orthogonale pour \Phi. La matrice de \Phi
dans cette base est alors
diag(\lambda~1,\\\ldots,\lambda~n~)
et quitte à permuter la base on peut supposer que
\lambda~1\neq~0,\\ldots,\lambda~r\mathrel\neq~0,\lambda~r+1~
= \\ldots~ =
\lambda~n = 0 pour un r \in {[}1,n{]} (car on a supposé
\Phi\neq~0). Dans tout repère
(a,\overrightarrowe1,\\ldots,\overrightarrowen~)
l'équation de \Sigma est donc de la forme \lambda~1x1^2 +
\\ldots~ +
\lambda~rxr^2 +\
\sum ~
i=1^n\alpha~ixi + k = 0 soit encore
\lambda~1(x1 + \alpha~1 \over
2\lambda~1 )^2 +
\\ldots~ +
\lambda~r(xr + \alpha~r \over
2\lambda~r )^2 +\
\sum ~
i=r+1^n\alpha~ixi + k' = 0 avec k' = k
-\\sum ~
i=1^r \alpha~i^2 \over
4\lambda~i^2 . En posant x1' = x1 +
\alpha~1 \over 2\lambda~1 ,xr' =
xr + \alpha~r \over 2\lambda~r ,
c'est-à-dire en faisant un changement d'origine du repère, on obtient un
nouveau repère
(a',\overrightarrowe1,\\ldots,\overrightarrowen~)
dans lequel l'équation devient \lambda~1x1^2 +
\\ldots~ +
\lambda~rxr^2 +\
\sum ~
i=r+1^n\alpha~ix'i + k' = 0.

S'il existe i ≥ r + 1 tel que \alpha~i\neq~0,
posons er+1' =
\\sum ~
i=r+1^n\alpha~ iei \over
\sqrt\\\sum
 i=r+1^n\alpha~i^2 . Alors
(e1,\\ldots,er,er+1~')
est une famille orthonormée, que nous pouvons compléter en une base
orthonormée
(e1,\\ldots,er,er+1',\\\ldots,en~').
Dans le repère
(a',e1,\\ldots,er,er+1',\\\ldots,en~')
les nouvelles coordonnées sont x'`1 =
x1,\\ldots,x'`r~
= xr,x'`r+1 =
\\sum ~
i=r+1^n\alpha~ ix'i \over
\sqrt\\\sum
 i=r+1^n\alpha~i^2 si bien que
l'équation devient dans ce repère \lambda~1x1^2 +
\\ldots~ +
\lambda~rxr^2 + \beta~x'`r+1 + k' = 0 avec
\beta~\neq~0. On écrit alors l'équation sous la forme
\lambda~1x1^2 +
\\ldots~ +
\lambda~rxr^2 + \beta~(x'`r+1 + k'
\over \beta~ ) = 0 et un nouveau changement d'origine ramène
à une équation \lambda~1y1^2 +
\\ldots~ +
\lambda~ryr^2 + \beta~yr+1 = 0.

Si par contre tous les \alpha~i sont nuls pour i ≥ r + 1 ou si r =
n, alors l'équation est dé\\\\jmathmathmathmathà réduite à la forme
\lambda~1x1^2 +
\\ldots~ +
\lambda~rxr^2 + k' = 0. On a donc démontré le
théorème suivant

Théorème~19.4.1 Soit \Sigma une quadrique. Alors il existe un repère
orthonormé
(a,\overrightarrowe1,\\ldots,\overrightarrowen~)
tel que l'équation de \Sigma dans ce repère soit de l'une des deux formes
suivantes

\begin{align*} \lambda~1x1^2 +
\\ldots + \lambda~~
rxr^2 + k& =& 0, r \leq n \%&
\\ \text(quadrique à
centre)& & \%& \\
\lambda~1x1^2 +
\\ldots + \lambda~~
rxr^2 + \beta~x r+1& =& 0, r \leq n - 1\%&
\\ \text(quadrique sans
centre)& & \%& \\
\end{align*}

avec
\lambda~1,\\ldots,\lambda~r~
non nuls. L'entier r est le rang de la forme quadratique principale et
\lambda~1,\\ldots,\lambda~r~
les valeurs propres non nulles (comptées avec leurs multiplicités) de la
matrice de la forme quadratique principale \Phi dans n'importe quelle base
orthonormée.

\paragraph{19.4.3 Classification des quadriques en dimension 2 et 3}

Dimension 2

Premier cas r = 2, \lambda~1\lambda~2 \textgreater{} 0~: on
obtient à partir de l'équation \lambda~1x^2 +
\lambda~2y^2 = -k que la conique est soit l'ensemble vide,
soit un point (si k = 0), soit une ellipse.

Deuxième cas r = 2, \lambda~1\lambda~2 \textless{} 0~: on obtient
à partir de l'équation \lambda~1x^2 +
\lambda~2y^2 = -k que la conique est soit la réunion de
deux droites sécantes (si k = 0), soit une hyperbole (si
k\neq~0).

Troisième cas r = 1 et conique sans centre~: on obtient à partir de
l'équation \lambda~1x^2 + \beta~y = 0 que la conique est une
parabole

Quatrième cas r = 1 et conique avec centre~: on obtient à partir de
l'équation \lambda~1x^2 + k = 0 que la conique est soit
l'ensemble vide, soit une droite soit la réunion de deux droites
parallèles.

Dimension 3

Premier cas r = 3, \lambda~1,\lambda~2 et \lambda~3 de même
signe (par exemple positifs)~; l'équation peut s'écrire sous la forme
\lambda~1x^2 + \lambda~2y^2 +
\lambda~3z^2 = k~; si k \textless{} 0, on obtient
l'ensemble vide~; si k = 0, la quadrique est réduite à un point~; si k
\textgreater{} 0, la quadrique se déduit par l'affinité
(x,y,z)\mapsto~(\sqrt\lambda~1x,\sqrt\lambda~2y,\sqrt\lambda~3z)
de la sphère x^2 + y^2 + z^2 = k, il
s'agit donc d'un ellipsoïde.

Deuxième cas r = 3, \lambda~1,\lambda~2 et \lambda~3 de signes
distincts. On peut par exemple supposer que \lambda~1 \textgreater{}
0,\lambda~2 \textgreater{} 0 et \lambda~3 \textless{} 0.
L'équation peut s'écrire sous la forme \lambda~1x2 +
\lambda~2y^2 + \lambda~3z^2 = k~; la
quadrique se déduit par l'affinité
(x,y,z)\mapsto~(\sqrt\lambda~1x,\sqrt\lambda~2y,\sqrt-\lambda~3z)
de la quadrique x^2 + y^2 - z^2 = k
autrement dit de la surface de révolution d'axe Oz dont une équation
cylindrique est \rho^2 - z^2 = k~; si k = 0, la
méridienne est la réunion de deux droites et la quadrique est un cône du
second degré~; si k\neq~0, la méridienne est une
hyperbole d'axe focal O\rho si k \textgreater{} 0, d'axe focal Oz si k
\textless{} 0~; dans le premier cas, la quadrique est un hyperboloïde à
une nappe obtenu par affinité à partir de la rotation d'une hyperbole
autour de son axe non focal, dans le second cas un hyperboloïde à deux
nappes, obtenu par affinité à partir de la rotation d'une hyperbole
autour de son axe focal

Troisième cas r = 2, quadrique à centre, \lambda~1\lambda~2
\textgreater{} 0. On peut écrire l'équation sous la forme
\lambda~1x^2 + \lambda~2y^2 = k~; il s'agit
soit de l'ensemble vide, soit d'une droite (si k = 0), soit d'un
cylindre d'axe Oz dont la base est une ellipse, c'est-à-dire d'un
cylindre elliptique.

Quatrième cas r = 2, quadrique à centre, \lambda~1\lambda~2
\textless{} 0. On peut écrire l'équation sous la forme
\lambda~1x^2 + \lambda~2y^2 = k~; il s'agit
soit de la réunion de deux plans sécants (si k = 0), soit d'un cylindre
d'axe Oz dont la base est une hyperbole, c'est-à-dire d'un cylindre
hyperbolique

Cinquième cas r = 2, quadrique sans centre, \lambda~1\lambda~2
\textgreater{} 0. On peut écrire l'équation sous la forme
\lambda~1x^2 + \lambda~2y^2 = \beta~z~; la
quadrique se déduit par l'affinité
(x,y,z)\mapsto~(\sqrt\lambda~1x,\sqrt\lambda~2y,\beta~z)
de la quadrique x^2 + y^2 = z autrement dit de la
surface de révolution d'axe Oz dont une équation cylindrique est
\rho^2 = z obtenue par rotation d'une parabole autour de son
axe~; il s'agit d'un paraboloïde elliptique.

Sixième cas r = 2, quadrique sans centre, \lambda~1\lambda~2
\textless{} 0. On peut écrire l'équation sous la forme
\lambda~1x^2 + \lambda~2y^2 = \beta~z~; la
quadrique se déduit par l'affinité
(x,y,z)\mapsto~(\sqrt\lambda~1x,\sqrt-\lambda~2y,\beta~z)
de la quadrique x^2 - y^2 = z~; il s'agit d'un
paraboloïde hyperbolique.

Septième cas r = 1, quadrique à centre. L'équation
\lambda~1x^2 + k = 0 définit soit l'ensemble vide, soit un
plan, soit la réunion de deux plans parallèles.

Huitième cas r = 1, quadrique sans centre. L'équation
\lambda~1x^2 = \beta~y définit un cylindre d'axe Oz dont la
base est une parabole. Il s'agit d'un cylindre parabolique.

\paragraph{19.4.4 Quadriques réglées, quadriques de révolution}

Parmi les neuf types de vraies quadriques, quatre sont des cylindres ou
des cônes qui sont évidemment des surfaces réglées. Il est clair qu'un
ellipsoïde qui est borné ne peut pas contenir de droites, dont ne peut
pas être réglé. Pour des raisons évidentes de non connexité, un
hyperboloïde à deux nappes ne peut pas contenir de droite (une telle
droite serait forcément horizontale car contenue dans un demi-espace
horizontal, or les sections horizontales de l'hyperboloïde sont des
cercles). Un paraboloïde elliptique étant situé dans un demi espace ne
peut évidemment pas contenir de droites (une telle droite serait
forcément horizontale car contenue dans un demi-espace horizontal, or
les sections horizontales du paraboloïde sont des cercles). Reste donc
le cas de l'hyperboloïde à une nappe et du paraboloïde hyperbolique.

En ce qui concerne l'hyperboloïde à une nappe, une équation réduite peut
s'écrire sous la forme  x^2 \over
a^2 + y^2 \over
b^2 - z^2 \over
c^2 = 1 soit encore

\left ( x \over a + z
\over c \right )\left (
x \over a - z \over c
\right ) = \left (1 + y
\over b \right )\left (1
- y \over b \right )

Cet hyperboloïde contient donc les deux familles de droites
(D\lambda~,\mu) et (\Delta\lambda~,\mu) définies pour
(\lambda~,\mu)\neq~(0,0) par

D\lambda~,\mu \left
\\matrix\,\lambda~\left
( x \over a + z \over c
\right ) = \mu\left (1 + y
\over b \right ) \cr
\cr \mu\left ( x \over a
- z \over c \right ) =
\lambda~\left (1 - y \over b
\right )\right .\quad
\text et \quad \Delta\lambda~,\mu
\left
\\matrix\,\lambda~\left
( x \over a + z \over c
\right ) = \mu\left (1 - y
\over b \right ) \cr
\cr \mu\left ( x \over a
- z \over c \right ) =
\lambda~\left (1 + y \over b
\right )\right .

Par tout point de l'hyperboloïde passe une et une seule droite de chaque
famille.

En ce qui concerne le paraboloïde hyperbolique, une équation réduite
peut s'écrire sous la forme  x^2 \over
a^2 - y^2 \over
b^2 = z soit encore

\left ( x \over a + y
\over b \right )\left (
x \over a - y \over b
\right ) = z

Cet hyperboloïde contient donc les deux familles de droites
(D\lambda~,\mu) et (\Delta\lambda~,\mu) définies pour
(\lambda~,\mu)\neq~(0,0) par

D\lambda~,\mu \left
\\matrix\,\lambda~\left
( x \over a + y \over b
\right ) = \muz \cr \cr
\mu\left ( x \over a - y
\over b \right ) =
\lambda~\right .\quad \text et
\quad \Delta\lambda~,\mu \left
\\matrix\,\lambda~\left
( x \over a + y \over b
\right ) = \mu \cr \cr
\mu\left ( x \over a - y
\over b \right ) =
\lambda~z\right .

Par tout point du paraboloïde hyperbolique passe une et une seule droite
de chaque famille.

En ce qui concerne la possibilité pour des quadriques d'être de
révolution, on constate immédiatement sur l'équation réduite que la
quadrique est de révolution si deux des \lambda~i sont égaux
(c'est-à-dire si la matrice de la forme quadratique principale \Phi dans
n'importe quelle base orthonormée admet une valeur propre double). Ceci
permet de compléter le tableau~:

\begin{center}\rule{3in}{0.4pt}\end{center}

\begin{center}\rule{3in}{0.4pt}\end{center}

\begin{center}\rule{3in}{0.4pt}\end{center}

\begin{center}\rule{3in}{0.4pt}\end{center}

Equation

Type

Réglé

De révolution

\begin{center}\rule{3in}{0.4pt}\end{center}

\begin{center}\rule{3in}{0.4pt}\end{center}

\begin{center}\rule{3in}{0.4pt}\end{center}

\begin{center}\rule{3in}{0.4pt}\end{center}

 x^2 \over a^2 +
y^2 \over b^2 + z^2
\over c^2 = 1

ellipsoïde

Non

si a = b ou b = c ou c = a

 x^2 \over a^2 +
y^2 \over b^2 - z^2
\over c^2 = 1

hyperboloïde à une nappe

Doublement

si a = b

 x^2 \over a^2 +
y^2 \over b^2 - z^2
\over c^2 = 0

cône du second degré

Oui

si a = b

 x^2 \over a^2 +
y^2 \over b^2 - z^2
\over c^2 = -1

hyperboloïde à deux nappes

Non

si a = b

 x^2 \over a^2 +
y^2 \over b^2 = 1

cylindre elliptique

Oui

si a = b

 x^2 \over a^2 -
y^2 \over b^2 = 1

cylindre hyperbolique

Oui

Non

z = x^2 \over a^2 +
y^2 \over b^2

paraboloïde elliptique

Non

si a = b

z = x^2 \over a^2 -
y^2 \over b^2

paraboloïde hyperbolique

Doublement

Non

2py = x^2

cylindre parabolique

Oui

Non

\begin{center}\rule{3in}{0.4pt}\end{center}

\begin{center}\rule{3in}{0.4pt}\end{center}

\begin{center}\rule{3in}{0.4pt}\end{center}

\begin{center}\rule{3in}{0.4pt}\end{center}

{[}
{[}
{[}
{[}
