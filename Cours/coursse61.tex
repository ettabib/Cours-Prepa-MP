\textbf{Warning: 
requires JavaScript to process the mathematics on this page.\\ If your
browser supports JavaScript, be sure it is enabled.}

\begin{center}\rule{3in}{0.4pt}\end{center}

{[}
{[}
{[}{]}
{[}

\subsubsection{10.2 Séries de fonctions}

\paragraph{10.2.1 Différents modes de convergence}

Remarque~10.2.1 Soit E un ensemble, F un espace vectoriel normé. Soit
(un)n\in\mathbb{N}~ une suite d'applications de E dans F. On
s'intéressera ici à la série
\\sum ~
n\in\mathbb{N}~un(x)~; c'est à la fois, pour chaque x \in E, une
série d'éléments de F et une suite de fonctions, la suite de ses sommes
partielles Sn =\
\sum  p=0^nup~. On peut
donc dé\\\\jmathmathmathmathà distinguer trois modes possibles de convergence de la série de
fonctions.

Définition~10.2.1 On dit que la série
\\sum ~
n\in\mathbb{N}~un d'applications de E dans F converge (i)
simplement sur E si pour chaque x \in E, la série
\\sum ~
n\in\mathbb{N}~un(x) converge (ii) absolument sur E si pour chaque
x \in E, la série \\sum ~
n\in\mathbb{N}~un(x) converge absolument (autrement dit la série
\\sum ~
n\in\mathbb{N}~\\textbar{}un(x)\\textbar{}
converge) (iii) uniformément sur E si la suite d'applications de E dans
F, x\mapsto~Sn(x)
= \\sum ~
p=0^nup(x) converge uniformément sur E

Remarque~10.2.2 Les résultats sur les séries à valeurs dans F montrent
que si F est complet, la convergence absolue implique la convergence
simple. De même, les résultats sur les suites de fonctions montrent que
la convergence uniforme implique la convergence simple. Bien entendu, le
critère de Cauchy uniforme peut s'appliquer à des séries de fonctions à
valeurs dans un espace vectoriel normé complet, et on obtient

Théorème~10.2.1 Soit E un ensemble et F un espace vectoriel normé
complet. Une série \\\sum
 n\in\mathbb{N}~un d'applications de E dans F est uniformément
convergente si et seulement si~elle vérifie le critère de Cauchy
uniforme

\forall~~\epsilon \textgreater{} 0,
\exists~N \in \mathbb{N}~, q ≥ p ≥ N
\rigtharrow~\forall~~x \in E,
\\textbar{}\\sum
n=p^qu n(x)\\textbar{}
\textless{} \epsilon

Démonstration C'est tout simplement le critère de Cauchy pour les suites
de fonctions en remarquant que
\\sum ~
n=p^qun = Sq - Sp-1.

Remarque~10.2.3 Nous allons introduire un quatrième mode de convergence
plus fort que les trois autres, la convergence normale~:

Définition~10.2.2 Soit
\\sum ~
n\in\mathbb{N}~un une série d'applications de E dans F. On dit
qu'elle converge normalement si elle vérifie les conditions équivalentes
(i) chaque un est une application bornée et la série (à termes
réels positifs) \\sum ~
n\in\mathbb{N}~\\textbar{}un\\textbar{}\infty~
est convergente (ii) il existe une série à termes réels positifs
\\sum ~
n\in\mathbb{N}~\alpha~n qui converge et qui vérifie
\forall~~x \in E,
\\textbar{}un(x)\\textbar{} \leq
\alpha~n.

Démonstration (i) \rigtharrow~(ii)~: prendre \alpha~n
=\\textbar{} un\\textbar{}\infty~.

(ii) \rigtharrow~(i)~: il suffit de remarquer que 0 \leq\\textbar{}
un\\textbar{}\infty~ \leq \alpha~n pour avoir la
convergence de \\sum ~
n\in\mathbb{N}~\\textbar{}un\\textbar{}\infty~.

Remarque~10.2.4 Montrer une convergence normale, c'est donc ma\\\\jmathmathmathmathorer
\\textbar{}un(x)\\textbar{} par
une série convergente indépendante de x. On constate que la convergence
normale n'est autre que la convergence absolue dans
(ℬ(E,F),\\textbar{}.\\textbar{}\infty~).

Théorème~10.2.2 Si F est complet, la convergence normale implique à la
fois la convergence absolue et la convergence uniforme.

Démonstration Pour tout x \in E, on a 0 \leq\\textbar{}
un(x)\\textbar{} \leq\\textbar{}
un\\textbar{}\infty~, et donc si la série
\\sum ~
n\in\mathbb{N}~\\textbar{}un\\textbar{}\infty~,
la série \\sum ~
n\in\mathbb{N}~\\textbar{}un(x)\\textbar{}
converge. Pour montrer la convergence uniforme, puisque F est complet,
il suffit de montrer que le critère de Cauchy uniforme est vérifié~;
mais on a, pour x \in E,

\\textbar{}\\sum
n=p^qu n(x)\\textbar{}
\leq\\sum
n=p^q\\textbar{}u
n(x)\\textbar{} \leq\\sum
n=p^q\\textbar{}u
n\\textbar{}\infty~

Comme la série \\sum ~
\\textbar{}un\\textbar{}\infty~
converge, pour \epsilon \textgreater{} 0, il existe N \in \mathbb{N}~ tel que q
\textgreater{} p ≥ N
\rigtharrow~\\sum ~
n=p^q\\textbar{}un\\textbar{}\infty~
\textless{} \epsilon. Alors

q ≥ p ≥ N \rigtharrow~\forall~~x \in E,
\\textbar{}\\sum
n=p^qu n(x)\\textbar{}
\textless{} \epsilon

Exemple~10.2.1 Soit \alpha~ \textgreater{} 0 et soit la série d'applications
de \mathbb{R}~^+ dans \mathbb{R}~,
\\sum  n≥1~ 1
\over n^\alpha~(1+nx) . On a 0 \leq 1
\over n^\alpha~(1+nx) \leq 1 \over
n^\alpha~ série indépendante de x. Cette dernière série converge
si \alpha~ \textgreater{} 1 et donc, si \alpha~ \textgreater{} 1, la série
\\sum  n≥1~ 1
\over n^\alpha~(1+nx) converge normalement sur
\mathbb{R}~^+. Si \alpha~ \leq 1, la série diverge au point 0. Elle ne peut pas
converger uniformément sur {]}0,+\infty~{[}, sinon elle vérifierait le critère
de Cauchy uniforme et pour \epsilon \textgreater{} 0, il existerait N \in \mathbb{N}~ tel
que q ≥ p ≥ N \rigtharrow~\forall~~x \in{]}0,+\infty~{[},
\\sum ~
n=p^q 1 \over n^\alpha~(1+nx)
\textless{} \epsilon~; mais alors, pour q et p fixés, en faisant tendre x vers
0 on obtiendrait \\\sum
 n=p^q 1 \over n^\alpha~ \leq \epsilon,
donc la série \\sum ~ 
1 \over n^\alpha~ convergerait par le critère de
Cauchy, ce qui est absurde. Par contre, l'équivalent  1
\over n^\alpha~(1+nx) ∼ 1 \over
xn^\alpha~+1 \textgreater{} 0 montre que si x \textgreater{} 0 la
série converge (car \alpha~ + 1 \textgreater{} 1). Donc la série converge
simplement sur {]}0,+\infty~{[} (et même absolument puisque c'est une série à
termes positifs). Si a \textgreater{} 0, on a, pour x \in {[}a,+\infty~{[}, 0
\leq 1 \over n^\alpha~(1+nx) \textless{} 1
\over n^\alpha~(1+na) ∼ 1 \over
an^\alpha~+1 \textgreater{} 0, ce qui montre que la série
converge normalement sur {[}a,+\infty~{[}.

Remarque~10.2.5 Le même argument utilisant le critère de Cauchy uniforme
permet de montrer que si
\\sum  un~ est
une série d'applications continues de E dans F (complet) qui converge
uniformément sur une partie A de E, alors elle converge encore
uniformément sur l'adhérence \overlineA de A~; la
plupart du temps, les convergences uniformes se produisent donc sur des
ensembles fermés et toute affirmation d'une convergence uniforme sur une
partie non fermée doit immédiatement susciter une inquiétude légitime
(même si parfois elle peut être infondée, une ou plusieurs des
un pouvant ne pas être continue).

\paragraph{10.2.2 Critères supplémentaires de convergence uniforme}

A part le critère de Cauchy uniforme, il y a peu de méthodes générales
permettant de montrer des convergences uniformes qui ne sont pas des
convergences normales. On retiendra cependant les deux cas suivants qui
sont importants.

Théorème~10.2.3~(convergence uniforme des séries alternées) Soit
(un)n\in\mathbb{N}~ une suite d'applications de l'ensemble E
dans \mathbb{R}~ vérifiant les hypothèses suivantes (i) pour chaque x \in E, la
suite (un(x))n\in\mathbb{N}~ est décroissante (ii) la suite
(un) converge uniformément vers la fonction nulle. Alors la
série \\sum ~
(-1)^nun converge uniformément sur E.

Démonstration Pour chaque x \in E, la série
\\sum ~
(-1)^nun(x) est convergente d'après le théorème sur
les séries alternées. De plus si l'on désigne par Sn sa somme
partielle d'indice n et par S sa somme, on sait (par le théorème sur les
séries alternées) que \textbar{}S(x) - Sn(x)\textbar{}\leq
un+1(x)~; la convergence uniforme de (un) vers la
fonction nulle implique donc la convergence uniforme de (Sn)
vers S.

Théorème~10.2.4~(critère d'Abel uniforme) Soit (an) une suite
d'applications de E dans \mathbb{R}~ et (un) une suite d'applications de
E dans l'espace vectoriel normé~complet F telles que (i)
\existsM ≥ 0, \\forall~~n \in \mathbb{N}~,
\forall~~x \in E,
\\textbar{}\\\sum
 p=0^nup(x)\\textbar{} \leq M
(ii) la suite (an) converge uniformément vers 0 en
décroissant. Alors la série
\\sum ~
anun converge uniformément

Démonstration On a, en posant Sn(x)
= \\sum ~
p=0^nup(x)

\begin{align*} \\sum
n=p^qa n(x)un(x)& =&
\sum n=p^qa~
n(x)(Sn(x) - Sn-1(x)) \%&
\\ & =& \\sum
n=p^qa n(x)Sn(x)
-\sum n=p^qa~
n(x)Sn-1(x) \%& \\ & =&
\sum n=p^qa~
n(x)Sn(x) -\\sum
n=p-1^q-1a n+1(x)Sn(x)\%&
\\ \text(changement
d'indices \$n - 1\mapsto~n\$)&& \%&
\\ & =& aq(x)Sq(x) -
ap(x)Sp-1(x) \%& \\
& \text & +\\sum
n=p^q-1(a n(x) -
an+1(x))Sn(x) \%& \\
\end{align*}

On a effectué ici une transformation d'Abel. Comme
\forall~n, \\forall~~x \in E,
\\textbar{}Sn(x)\\textbar{} \leq M
on a

\\textbar{}\\sum
n=p^qa
n(x)un(x)\\textbar{} \leq
M(\textbar{}aq(x)\textbar{} +
\textbar{}ap(x)\textbar{} + \\sum
n=p^q-1\textbar{}a n(x) -
an+1(x)\textbar{}) = 2Map(x)

en tenant compte de an(x) ≥ 0 et an(x) -
an+1(x) ≥ 0. Comme la suite (an) converge
uniformément vers 0, la série
\\sum ~
anun vérifie le critère de Cauchy uniforme, donc
elle converge uniformément.

\paragraph{10.2.3 Propriétés de la convergence uniforme}

Il suffit d'appliquer à la suite (Sn) d'applications de E dans
F les résultats sur les suites de fonctions pour obtenir les théorèmes
suivants

Théorème~10.2.5~(conservation de la continuité) Soit E un espace
métrique, F un espace vectoriel normé. Soit
\\sum ~
n\in\mathbb{N}~un une série d'applications de E dans F qui
converge simplement, de somme S : E \rightarrow~ F,
x\mapsto~S(x) =\
\sum  n=0^+\infty~un~(x).
Soit a \in E. On suppose que (i) chacune des un est continue au
point a (ii) il existe U voisinage de a telle que la série
\\sum  un~
converge uniformément sur U Alors S est continue au point a.

Démonstration Chacune des un étant continue en a, il en est de
même de Sn.

Corollaire~10.2.6 Soit E un espace métrique, F un espace vectoriel
normé. Soit \\sum ~
n\in\mathbb{N}~un une série d'applications continues de E dans F
qui converge uniformément. Alors la somme S de la série est continue.

Remarque~10.2.6 Il suffit évidemment que tout point ait un voisinage sur
lequel la série converge uniformément, ce que l'on appelle la
convergence uniforme locale.

Théorème~10.2.7~(interversion des limites) Soit E un espace métrique, F
un espace vectoriel normé complet. Soit
\\sum ~
n\in\mathbb{N}~un une série de fonctions de E dans F. Soit a \in E,
A \subset~ E tel que a \in\overlineA et
\forall~n \in \mathbb{N}~, A \subset~\ Def~
(un). On suppose que

\begin{itemize}
\itemsep1pt\parskip0pt\parsep0pt
\item
  (i) la série \\sum ~
  un converge uniformément sur A~; soit S sa somme
\item
  (ii) chacune des un a une limite \elln en a suivant A
\end{itemize}

Alors la série \\sum ~
\elln converge et x\mapsto~S(x) admet
\\sum ~
n=0^+\infty~\elln pour limite en a suivant A, autrement
dit

\sum n=0^+\infty~~\left
(lim x\rightarrow~a,x\inAun(x)\right ) =
limx\rightarrow~a,x\inA\left (\\sum
n=0^+\infty~u n(x)\right )

Démonstration Il suffit de remarquer que Sn
= \\sum ~
p=0^nup admet la limite
\\sum ~
p=0^n\ellp en a suivant A et d'appliquer le
théorème d'interversion des limites à la suite (Sn).

Remarque~10.2.7 Le résultat suivant s'applique en particulier dans le
cas où a = +\infty~ et A = \mathbb{N}~, c'est-à-dire au cas d'une suite double
(xn,p) d'éléments de E~: avec les hypothèses

\begin{itemize}
\itemsep1pt\parskip0pt\parsep0pt
\item
  (i) la série \\sum ~
  n\in\mathbb{N}~xn,p converge uniformément par rapport à p
\item
  (ii) limp\rightarrow~+\infty~xn,p~ =
  \elln
\end{itemize}

Alors la série \\sum ~
n\in\mathbb{N}~\elln converge et

\sum n=0^+\infty~~\left
(lim p\rightarrow~+\infty~xn,p\right ) =
limp\rightarrow~+\infty~\left (\\sum
n=0^+\infty~x n,p\right )

Exemple~10.2.2 Le résultat précédent utilise de manière essentielle la
convergence uniforme par rapport à p comme le montre l'exemple
xn,p = n \over n+p - n-1
\over n+p-1 = p \over (n+p)(n+p-1)
pour lequel on a

0 = \\sum
n=1^+\infty~\left (lim
p\rightarrow~+\infty~xn,p\right
)\neq~limp\rightarrow~+\infty~\left
(\sum n=1^+\infty~x~
n,p\right ) = 1

Théorème~10.2.8~(intégration) Soit
\\sum  un~ une
suite de fonctions réglées de {[}a,b{]} dans E (espace vectoriel normé
complet) qui converge uniformément sur {[}a,b{]}, de somme S : {[}a,b{]}
\rightarrow~ E. Alors S est réglée et la série
\\sum ~
n\in\mathbb{N}~\int ~
a^bun(t) dt converge, de somme
\int  a^b~S(t) dt, autrement dit
\int  a^b~\left
(\\sum ~
n=0^+\infty~un(t)\right ) dt
= \\sum ~
n=0^+\infty~\int ~
a^bun(t) dt (interversion du signe somme et du
signe intégrale).

Démonstration Il suffit d'appliquer le théorème correspondant sur les
suites de fonctions en remarquant que Sn est réglée et que
\int  a^bSn~(t) dt
= \\sum ~
p=0^n\int ~
a^bup(t) dt

Remarque~10.2.8 Comme pour les suites de fonctions, le fait que
l'intervalle soit borné est essentiel. Le résultat précédent ne s'étend
donc pas aux intégrales impropres sur des intervalles non bornés. Par
contre on a

Corollaire~10.2.9 Soit I un intervalle de \mathbb{R}~,
\\sum  un~ une
suite de fonctions réglées de I dans E (espace vectoriel normé complet)
qui converge uniformément sur I, de somme S : I \rightarrow~ E. Alors S est réglée.
Soit a \in I, Un(x) =\int ~
a^xun(t) dt et U(x)
=\int  a^x~S(t) dt. Alors la
série \\sum ~
n\in\mathbb{N}~Un converge uniformément sur tout segment inclus
dans I et elle admet U pour somme.

La convergence uniforme d'une série de fonctions dérivables n'implique
pas que la somme soit elle-même dérivable. C'est même de cette manière,
par limite uniforme, qu'ont été construits les premiers exemples de
fonctions continues n'admettant de dérivée en aucun point (voir
ci-dessous). Par contre on a

Théorème~10.2.10 Soit I un intervalle de \mathbb{R}~,
\\sum  un~ une
suite d'applications de I dans E qui converge simplement sur I, de somme
S : I \rightarrow~ E. On suppose que (i) chacune des un est de classe
\mathcal{C}^1 (ii) la série
\\sum  un~'
converge uniformément sur I Alors S est de classe \mathcal{C}^1 et
\forall~~x \in I, S'(x) =\
\sum  n=0^+\infty~un~'(x).

Démonstration Il suffit de remarquer que les Sn sont de classe
\mathcal{C}^1 et que Sn'(x) =\
\sum  p=0^nup~'(x). Il
ne reste plus qu'à appliquer le théorème correspondant sur les suites de
fonctions.

Remarque~10.2.9 Comme pour les suites de fonctions, il suffit, avec les
mêmes hypothèses, que la suite
\\sum  un~
converge en un point a pour qu'elle converge simplement sur I, cette
convergence étant d'ailleurs uniforme sur tout segment inclus dans I. On
retiendra donc, que pour montrer la dérivabilité d'une somme de série de
fonctions, il faut s'attacher à la convergence uniforme de la série des
dérivées, et non à celle de la série elle-même.

Exemple~10.2.3 Etude de la fonction \zeta de Riemann~: on pose, pour x
\textgreater{} 1, \zeta(x) =\
\sum  n=1^+\infty~~ 1
\over n^x . Pour a \textgreater{} 1 et x \in
{[}a,+\infty~{[}, on a  1 \over n^x \leq 1
\over n^a qui est une série convergente
indépendante de a. Donc la série converge normalement sur {[}a,+\infty~{[} et
la fonction \zeta est continue sur {[}a,+\infty~{[} quel que soit a \textgreater{}
1~; elle est donc continue sur {]}1,+\infty~{[}. La fonction x \rightarrow~ 1
\over n^x est de classe C^\infty~ et sa
dérivée p-ième est  (-1)^p(log~
n)^p \over n^x . Si a
\textgreater{} 1, la série de fonctions
\\sum  n≥1~
(-1)^p(log n)^p~
\over n^x converge normalement sur
{[}a,+\infty~{[} (car \left \textbar{}
(-1)^p(log n)^p~
\over n^x \right
\textbar{}\leq (log n)^p~
\over n^a qui est une série de Bertrand
convergente, indépendante de x) et donc \zeta est de classe C^p
sur {[}a,+\infty~{[} avec \zeta^(p)(x) =\
\sum  n=1^+\infty~~
(-1)^p(log n)^p~
\over n^x . Comme a est quelconque avec a
\textgreater{} 1, \zeta est de classe C^\infty~ sur {]}1,+\infty~{[} et on a
la formule ci-dessus.

Exemple~10.2.4 Nous allons donner un exemple de fonction continue sur un
intervalle, qui n'est dérivable en aucun point. Posons pour cela f(x)
= \\sum ~
n=0^+\infty~a^n cos~
(b^n\pi~x) avec 0 \textless{} a \textless{} 1 et b entier
multiple de 4. La ma\\\\jmathmathmathmathoration \left
\textbar{}a^ncos(b^n\pi~x)\right
\textbar{} \leq a^n montre que la série converge normalement sur
\mathbb{R}~ et que sa somme est donc une fonction continue sur \mathbb{R}~. On a 
f(x+h)-f(x) \over h =\
\sum  n=0^+\infty~a^n~
cos~
(b^n\pi~(x+h))-cos (b^n~\pi~x)
\over h . Prenons en particulier h = 1
\over b^p où p est un entier. On a alors,

\begin{align*} f(x + h) - f(x)
\over h & =& \\sum
n=0^+\infty~a^n \cos
(b^n\pi~x + b^nh\pi~) - \cos
(b^n\pi~x) \over h \%&
\\ & =& \\sum
n=0^pa^n \cos
(b^n\pi~x + b^nh\pi~) - \cos
(b^n\pi~x) \over h \%&
\\ \end{align*}

car b^nh = b^n-p est un entier pair pour n
\textgreater{} p. On peut donc écrire  f(x+h)-f(x)
\over h = Sp-1 -
2b^pa^p cos~
(b^p\pi~x) avec

\begin{align*} \textbar{}Sp-1\textbar{}&
=& \left \textbar{}\\sum
n=0^p-1a^n \cos
(b^n\pi~x + b^nh\pi~) - \cos
(b^n\pi~x) \over h \right
\textbar{}\%& \\ & =&
2\left \textbar{}\\sum
n=0^p-1a^n \sin
(b^n\pi~x + b^nh\pi~ \over 2
)\sin ( b^nh\pi~ \over 2 )
\over h \right \textbar{} \%&
\\ & \leq& 2\\sum
n=0^p-1a^n\left \textbar{}
\sin ( b^nh\pi~ \over 2 )
\over h \right \textbar{} \%&
\\ & \textless{}&
\pi~\\sum
n=0^p-1b^na^n \%&
\\ \end{align*}

en utilisant \textbar{}sin~ x\textbar{}
\textless{} x pour x \textgreater{} 0. Supposons a et b choisis de telle
sorte que ba - 1 \textgreater{} 2\pi~~; on a alors
\textbar{}Sp-1\textbar{} \textless{} \pi~
b^pa^p-1 \over ba-1 \textless{}
\pi~ b^pa^p \over ba-1 et donc
Sp-1 = \epsilonpb^pa^p avec
\textbar{}\epsilonp\textbar{} \textless{} 1 \over
2 . On a alors  f(x+h)-f(x) \over h =
a^pb^p(\epsilon p -
2cos (b^p~\pi~x)). En suivant la même
méthode on peutécrire  f(x+ h \over 2 )-f(x)
\over  h \over 2  =
a^pb^p(\etap -
2\sqrt2cos~
(b^p\pi~x + \pi~ \over 4 )) avec
\textbar{}\etap\textbar{} \textless{} 1 \over
2 . Mais les deux nombres b^p\pi~x et b^p\pi~x + \pi~
\over 4 différant de  \pi~ \over 4 ,
l'un des deux cosinus au moins est en valeur absolue supérieur à
sin  \pi~ \over 8~ (exercice
facile), et donc on a soit \left \textbar{} f(x+h)-f(x)
\over h \right \textbar{}≥
a^pb^p(2sin~  \pi~
\over 8 - 1 \over 2 ), soit
\left \textbar{} f(x+ h \over 2
)-f(x) \over  h \over 2 
\right \textbar{}≥
a^pb^p(2\sqrt2sin~
 \pi~ \over 8 - 1 \over 2 ). Comme
limp\rightarrow~+\infty~b^pa^p~
= +\infty~, on a donc
limsuph\rightarrow~0~\left
\textbar{} f(x+h)-f(x) \over h \right
\textbar{} = +\infty~, donc f n'est pas dérivable au point x.

\paragraph{10.2.4 Séries de fonctions intégrables sur un intervalle}

Remarque~10.2.10 Comme pour les suites de fonctions, les théorèmes du
type \\sum ~
\int  un =\\int ~
\\sum  un~
démontrés précédemment ont des hypothèses trop restrictives~: ils
nécessitent d'une part que l'intervalle soit borné et d'autre part que
la série de fonctions converge uniformément sur tout l'intervalle. La
théorie de Lebesgue étend également ces théorèmes à des situations plus
générales d'où nous extrairons un certain nombre de résultats utiles.

Théorème~10.2.11~(convergence monotone) Soit I un intervalle de \mathbb{R}~,
\\sum  un~ une
série de fonctions de I dans \mathbb{R}~ positives, continues par morceaux et
intégrables sur I~; on suppose que la série converge simplement sur I et
que sa somme S = \\sum ~
un est continue par morceaux. Alors la série
\\sum ~
n\in\mathbb{N}~\int  Iun~ converge
si et seulement si la fonction S est intégrable. Dans ces conditions on
a

\int  I~S =\\int
 I \\sum
n=0^+\infty~u n = \\sum
n=0^+\infty~\\\int
  Iun

Démonstration Il suffit d'appliquer le théorème de convergence monotone
pour les suites de fonctions à la suite Sn
= \\sum ~
p=0^nup. C'est une suite croissante de
fonctions positives, intégrables et continues par morceaux qui converge
simplement vers S. Donc la suite des intégrales
\int  ISn~
= \\sum ~
p=0^n\int  Iup~
converge si et seulement si S est intégrable et dans ce cas
\int  I~S =\
lim\int  ISn~~; donc la
série \\sum ~
n\in\mathbb{N}~\int  Iun~ converge
si et seulement si la fonction S est intégrable et dans ces conditions
on a

\int  I~S =\\int
 I \\sum
n=0^+\infty~u n = \\sum
n=0^+\infty~\\\int
  Iun

Théorème~10.2.12~(intégration terme à terme) Soit I un intervalle de \mathbb{R}~,
\\sum  un~ une
série de fonctions de I dans \mathbb{C} continues par morceaux et intégrables sur
I~; on suppose que la série converge simplement sur I et que sa somme S
= \\sum  un~
est continue par morceaux. Si la série
\\sum ~
n\in\mathbb{N}~\int ~
I\textbar{}un\textbar{} est convergente, alors S est
intégrable sur I, la série
\\sum ~
n\in\mathbb{N}~\int  Iun~ converge
et on a à la fois

\int  I~S =\\int
 I \\sum
n=0^+\infty~u n = \\sum
n=0^+\infty~\\\int
  Iun\text et
\\int  ~
I\textbar{}S\textbar{}\leq\\sum
n=0^+\infty~\\\int
  I\textbar{}un\textbar{}

Démonstration Soit Sn : I \rightarrow~ \mathbb{R}~^+,
t\mapsto~\\\sum
 k=0^nuk(t), Mn : I \rightarrow~
\mathbb{R}~^+,
t\mapsto~\\\sum
 k=0^n\textbar{}uk(t)\textbar{} et
hn : I \rightarrow~ \mathbb{R}~^+,
t\mapsto~min(\textbar{}S(t)\textbar{},Mn~(t)).
La formule classique min~(x,y) =
1\over 2(x + y -\textbar{}x - y\textbar{}) montre que
hn est continue par morceaux. Puisque chacune des fonctions
\textbar{}uk\textbar{} est intégrable sur I, il en est de même
de Mn et donc de hn qui est dominée par
Mn~; on a aussi

\int  Ihn~
\leq\int  IMn~ =
\sum k=0^n~
\\int  ~
I\textbar{}uk\textbar{}\leq\\sum
k=0^+\infty~\\\int
  I\textbar{}uk\textbar{}

Comme Mn+1(t) ≥ Mn(t), la suite
(hn)n\in\mathbb{N}~ est croissante. Fixons t \in I et \epsilon
\textgreater{} 0~; il existe N \in \mathbb{N}~ tel que n ≥ N \rigtharrow~\textbar{}S(t) -
Sn(t)\textbar{} \textless{} \epsilon~; on a alors, pour n ≥ N,
\textbar{}S(t)\textbar{}- \epsilon \leq\textbar{}Sn(t)\textbar{}\leq
Mn(t) et donc \textbar{}S(t)\textbar{}- \epsilon \leq hn(t)
=\
min(\textbar{}S(t)\textbar{},Mn(t))
\leq\textbar{}S(t)\textbar{}, ce qui montre que la suite
(hn)n\in\mathbb{N}~ converge simplement vers
\textbar{}S\textbar{}, qui est continue par morceaux. On peut donc
appliquer le théorème de convergence monotone à la suite
(hn)n\in\mathbb{N}~ et en déduire que \textbar{}S\textbar{} est
intégrable avec \int ~
I\textbar{}S\textbar{} =\
lim\int  Ihn~
\leq\\sum ~
k=0^+\infty~\int ~
I\textbar{}uk\textbar{}. On en conclut que S est
intégrable et que \left
\textbar{}\int  I~S\right
\textbar{}\leq\int ~
I\textbar{}S\textbar{}\leq\\\sum
 k=0^+\infty~\int ~
I\textbar{}uk\textbar{}.

On applique ce que l'on vient de démontrer à la série
\\sum ~
p≥n+1up et l'on obtient

\left \textbar{}\int  I~S
-\sum k=0^n~
\\int  ~
Iuk\right \textbar{} =
\left \textbar{}\int ~
I(S -\sum k=0^nu~
k)\right \textbar{} = \left
\textbar{}\int  I~
\sum k=n+1^+\infty~u~
k\right \textbar{}\leq\\sum
k=n+1^+\infty~\\\int
  I\textbar{}uk\textbar{}

qui tend vers 0 quand n tend vers + \infty~ (reste d'une série convergente).
On a donc \int  I~S
= \\sum ~
k=0^+\infty~\int ~
Iuk.

Remarque~10.2.11 Il est important de constater que l'hypothèse de
convergence de la série
\\sum ~
n\int ~
I\textbar{}un\textbar{} sert non seulement à garantir
l'intégrabilité de S et la convergence (absolue) de la série
\\sum ~
n\int  Iun~, mais est
également un argument essentiel de la démonstration de
\int  I~\
\sum  n=0^+\infty~un~
= \\sum ~
n=0^+\infty~\int ~
Iun, et donc de la validité du résultat. La série
\\sum  un~ peut
très bien converger avec une somme intégrable, la série
\\sum ~
\int  Iun~ convergeant (même
absolument) sans que l'on ait \int ~
I \\sum ~
n=0^+\infty~un =\
\sum ~
n=0^+\infty~\int ~
Iun

{[}
{[}
{[}
{[}
