\textbf{Warning: 
requires JavaScript to process the mathematics on this page.\\ If your
browser supports JavaScript, be sure it is enabled.}

\begin{center}\rule{3in}{0.4pt}\end{center}

{[}
{[}
{[}{]}
{[}

\subsection{Chapitre~20\\Intégrales curvilignes, intégrales multiples}

Avertissement~: ce chapitre essaye de donner une idée de la théorie des
intégrales curvilignes et des intégrales multiples. Il ne prétendra pas
au même degré de rigueur que le reste de l'ouvrage.

~20.1  \\
~~20.1.1 {Formes
différentielles sur un arc paramétré} \\ ~~20.1.2
{Intégrale d'une forme
différentielle sur un arc} \\ ~~20.1.3
{Formes différentielles exactes
et champs de gradients} \\ ~20.2
 \\
~~20.2.1 {Pavés et
subdivisions. Ensembles négligeables} \\ ~~20.2.2
{Intégrales multiples sur un
pavé de \mathbb{R}~^n} \\ ~~20.2.3
{Intégrales multiples sur une
partie de \mathbb{R}~^n} \\ ~~20.2.4
{Mesure d'un sous-ensemble
borné de \mathbb{R}~^n} \\ ~20.3
{Calcul des intégrales doubles et
triples} \\ ~~20.3.1 {Théorème
de Fubini sur une partie de \mathbb{R}~^2} \\ ~~20.3.2
{Théorème de Fubini sur une
partie de \mathbb{R}~^3} \\ ~~20.3.3
{Théorème de changement de
variables dans les intégrales multiples} \\ ~~20.3.4
 \\
~20.4 {Introduction aux
intégrales de surface}

{[}
{[}
{[}
{[}
