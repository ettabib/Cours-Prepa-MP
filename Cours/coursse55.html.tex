\textbf{Warning: 
requires JavaScript to process the mathematics on this page.\\ If your
browser supports JavaScript, be sure it is enabled.}

\begin{center}\rule{3in}{0.4pt}\end{center}

{[}
{[}
{[}{]}
{[}

\subsubsection{9.6 Intégration sur un intervalle quelconque~: fonctions
à valeurs complexes}

\paragraph{9.6.1 Fonctions à valeurs complexes intégrables}

Définition~9.6.1 Soit I un intervalle de \mathbb{R}~ et f : I \rightarrow~ \mathbb{C} continue par
morceaux. On dit que f est intégrable sur I si la fonction à valeurs
réelles positives \textbar{}f\textbar{} est intégrable sur I.

Théorème~9.6.1 Soit f : I \rightarrow~ \mathbb{C} continue par morceaux et intégrable. Alors
pour toute suite (Jn) croissante de segments contenus dans I
de réunion I, la suite (\int ~
Jnf)n\in\mathbb{N}~ est convergente. Sa limite est
indépendante de la suite (Jn) et notée
\int  I~f.

Démonstration Soit q \textgreater{} p. On a

\left \textbar{}\int ~
Jqf -\int ~
Jpf\right \textbar{} =
\left \textbar{}\int ~
Jq\diagdownJpf\right
\textbar{}\leq\int ~
Jq\diagdownJp\textbar{}f\textbar{}
=\int ~
Jq\textbar{}f\textbar{}-\int ~
Jp\textbar{}f\textbar{}

(avec un tout petit abus d'écriture en notant Jq \diagdown
Jp = {[}aq,ap{]} \cup
{[}bp,bq{]} la réunion de deux segments dis\\\\jmathmathmathmathoints ).
Comme \textbar{}f\textbar{} est intégrable, la suite
(\int ~
Jn\textbar{}f\textbar{}) converge, donc c'est une
suite de Cauchy, et par conséquent il en est de même de la suite
(\int  Jn~f) qui est donc
convergente. Si (Kn) est une autre suite de segments vérifiant
les mêmes propriétés, deux cas se présentent. Si
\forall~n, Jn \subset~ Kn~, alors

\left \textbar{}\int ~
Knf -\int ~
Jnf\right \textbar{} =
\left \textbar{}\int ~
Kn\diagdownJnf\right
\textbar{}\leq\int ~
Kn\diagdownJn\textbar{}f\textbar{}
=\int ~
Kn\textbar{}f\textbar{}-\int ~
Jn\textbar{}f\textbar{}

Mais les deux suites (\int ~
Jn\textbar{}f\textbar{}) et
(\int ~
Kn\textbar{}f\textbar{}) ont la même limite et donc
leur différence tend vers 0. Il en est donc de même de la différence des
suites (\int  Jn~f) et
(\int  Kn~f), qui, étant
convergentes, ont donc la même limite. Si Jn n'est pas
forcément inclus dans Kn il suffit d'écrire

lim\\int ~
Jnf =\
lim\int  Jn\cupKn~f
= lim\\int ~
Knf

Corollaire~9.6.2 Soit f : I \rightarrow~ \mathbb{C} continue par morceaux et intégrable.
Alors \left \textbar{}\int ~
If\right \textbar{}\leq\\int
 I\textbar{}f\textbar{}.

Démonstration Il suffit de passer à la limite à partir de l'inégalité
\left \textbar{}\int ~
Jnf\right \textbar{}
\leq\int ~
Jn\textbar{}f\textbar{}.

Théorème~9.6.3 Soit -\infty~ \textless{} a \textless{} b \leq +\infty~ et f :
{[}a,b{[}\rightarrow~ \mathbb{C} intégrable. Alors la fonction
x\mapsto~\int ~
a^xf(t) dt admet la limite \int ~
{[}a,b{[}f au point b.

Démonstration Soit a \textless{} x \textless{} y \textless{} b~; on a

\left \textbar{}\int ~
a^yf -\int ~
a^xf\right \textbar{} =
\left \textbar{}\int ~
x^yf\right
\textbar{}\leq\int ~
x^y\textbar{}f\textbar{} =\int ~
a^y\textbar{}f\textbar{}-\int ~
a^x\textbar{}f\textbar{}

Comme l'application
x\mapsto~\int ~
a^x\textbar{}f\textbar{} admet une limite au point b,
elle vérifie le critère de Cauchy~: pour tout \epsilon \textgreater{} 0, il
existe c \in {[}a,b{[} tel que c \textless{} x \textless{} y \textless{} b
\rigtharrow~\left \textbar{}\int ~
a^y\textbar{}f\textbar{}-\int ~
a^x\textbar{}f\textbar{}\right \textbar{}
\textless{} \epsilon~; alors l'inégalité ci dessus montre que
x\mapsto~\int ~
a^xf(t) dt vérifie également ce critère de Cauchy, donc
admet une limite au point b. Soit alors bn une suite
croissante de limite b. On a

limx\rightarrow~b\\int ~
a^xf(t) dt = lim~
n\rightarrow~+\infty~\int  a^bn ~f(t)
dt =\
limn\rightarrow~+\infty~\int ~
{[}a,bn{]}f =\int ~
{[}a,b{[}f

Proposition~9.6.4 Soit I un intervalle de \mathbb{R}~, f : I \rightarrow~ \mathbb{C} continue par
morceaux, intégrable sur I. Alors f est intégrable sur tout intervalle
I' inclus dans I.

Démonstration En effet l'intégrabilité de f équivaut à celle de
\textbar{}f\textbar{}.

Proposition~9.6.5 Soit f : I \rightarrow~ \mathbb{C} et \phi : I \rightarrow~ \mathbb{R}~^+ continues par
morceaux telles que 0 \leq\textbar{}f\textbar{}\leq \phi. Si \phi est intégrable sur
I il en est de même de f et \left
\textbar{}\int  I~f\right
\textbar{}\leq\int  I~\phi.

Démonstration Evident d'après les définitions.

Corollaire~9.6.6 Soit I un intervalle borné de \mathbb{R}~ et soit f : I \rightarrow~ \mathbb{C}
continue par morceaux et bornée. Alors f est intégrable sur I.

Démonstration Appliquer la proposition précédente avec \phi constante
ma\\\\jmathmathmathmathorant \textbar{}f\textbar{}.

Proposition~9.6.7 Soit I = {[}a,b{]} un segment de \mathbb{R}~, f : I \rightarrow~ \mathbb{C} continue
par morceaux. Alors f est intégrable sur I et
\int  I~f =\\int
 a^bf. De plus f est intégrable sur {]}a,b{[},
{[}a,b{[} et {]}a,b{]}, toutes ces intégrales étant égales.

Démonstration La fonction \textbar{}f\textbar{} est positive et continue
par morceaux, donc intégrable sur {[}a,b{]}. Donc f l'est également. On
sait alors que \textbar{}f\textbar{} est intégrable sur tout intervalle
inclus dans I et en particulier sur {]}a,b{[}, {[}a,b{[} et {]}a,b{]}~;
il en est donc de même pour f. De plus, si an = a + 1
\over n et bn = b - 1 \over
n , Jn = {[}an,bn{]} est une suite
croissante de segments dont la réunion est {]}a,b{[}, donc

\int  {]}a,b{[}~f
= lim\\int ~
an^bn f =\\int
 a^bf

par continuité de l'intégrale par rapport à ses bornes. On fait une
démonstration similaire pour {[}a,b{[} avec {[}a,bn{]} et
{]}a,b{]} avec {[}an,b{]}. Pour {[}a,b{]}, on prend
an = a et bn = b.

Théorème~9.6.8 Soit f,g : I \rightarrow~ \mathbb{C} continues par morceaux, soit \alpha~,\beta~ \in \mathbb{C}. Si
f et g sont intégrables sur I, il en est de même de \alpha~f + \beta~g et on a

\int  I~(\alpha~f + \beta~g) =
\alpha~\int  I~f +
\beta~\int  I~g

Autrement dit, l'ensemble des applications de I dans \mathbb{C} qui sont
intégrables sur I est un sous-espace vectoriel de l'espace vectoriel des
applications de I dans \mathbb{C} et l'application
f\mapsto~\int  I~f
est linéaire.

Démonstration L'intégrabilité est évidente à partir de l'inégalité
\textbar{}\alpha~f + \beta~g\textbar{}\leq\textbar{}\alpha~\textbar{}\textbar{}f\textbar{} +
\textbar{}\beta~\textbar{}\textbar{}g\textbar{} et du fait que
\textbar{}f\textbar{} et \textbar{}g\textbar{} étant intégrables, il en
est de même de \textbar{}\alpha~\textbar{}\textbar{}f\textbar{} +
\textbar{}\beta~\textbar{}\textbar{}g\textbar{}. Pour les égalités, il suffit
de prendre une suite (Jn) croissante de segments de réunion I
et de passer à la limite dans les formules

\int  Jn~(\alpha~f + \beta~g) =
\alpha~\int  Jn~f +
\beta~\int  Jn~g

Proposition~9.6.9 Soit I un intervalle de \mathbb{R}~, f : I \rightarrow~ \mathbb{R}~ continue par
morceaux. Soit a \in I^o. Alors f est intégrable sur I si et
seulement si elle est intégrable sur I\bigcap{]} -\infty~,a{]} et sur I \bigcap
{[}a,+\infty~{[}. Dans ce cas,

\int  I~f =\\int
 I\bigcap{]}-\infty~,a{]}f +\int ~
I\bigcap{[}a,+\infty~{[}f

Démonstration Le résultat similaire dé\\\\jmathmathmathmathà démontré pour
\textbar{}f\textbar{} démontre l'équivalence entre les diverses
intégrabilités. Soit alors Jn =
{[}an,bn{]} une suite croissante de segments de
réunion I. Pour n assez grand, on a an \leq a \leq bn car
a est dans l'intérieur de I. Mais ({[}an,a{]}) est une suite
croissante de segments de réunion I\bigcap{]} -\infty~,a{]} et
({[}a,bn{]}) est une suite croissante de segments de réunion I
\bigcap {[}a,+\infty~{[}. On peut donc passer à la limite dans la formule
\int  {[}an,bn{]}~f
=\int  {[}an,a{]}~f
+\int  {[}a,bn{]}~f, et on
obtient

\int  I~f =\\int
 I\bigcap{]}-\infty~,a{]}f +\int ~
I\bigcap{[}a,+\infty~{[}

\paragraph{9.6.2 Décomposition des fonctions à valeurs complexes}

Soit x \in \mathbb{R}~. On pose x^+ = max~(x,0)
et x^- = max~(-x,0). On a
x^+,x^-\in \mathbb{R}~^+, x = x^+ -
x^-, \textbar{}x\textbar{} = x^+ + x^-,
x^+ = 1 \over 2 (\textbar{}x\textbar{} +
x) et x^- = 1 \over 2
(\textbar{}x\textbar{}- x).

Remarque~9.6.1 Si f : I \rightarrow~ \mathbb{R}~, on peut ainsi lui associer des fonctions
f^+ et f^- à valeurs dans \mathbb{R}~^+. On a
f^+,f^-\in \mathbb{R}~^+, f = f^+ -
f^-, \textbar{}f\textbar{} = f^+ + f^-,
f^+ = 1 \over 2 (\textbar{}f\textbar{} +
f) et f^- = 1 \over 2
(\textbar{}f\textbar{}- f). Ces deux dernières formules montrent
clairement que si f est continue par morceaux, il en est de même de
f^+ et f^-.

Théorème~9.6.10 Soit f : I \rightarrow~ \mathbb{R}~ continue par morceaux. Alors f est
intégrable sur I si et seulement si les fonctions (à valeurs réelles
positives) f^+ et f^- le sont. Dans ce cas

\int  I~f =\\int
 If^+ -\int ~
If^-\text et
\int  I~\textbar{}f\textbar{}
=\int  If^+~
+\int  If^-~

Démonstration Si f est intégrable sur I, il en est de même pour
\textbar{}f\textbar{} et donc pour f^+ et f^-
puisque 0 \leq f^+ \leq\textbar{}f\textbar{} et 0 \leq
f^-\leq\textbar{}f\textbar{}. Inversement, si f^+ et
f^- sont intégrables, leur différence f l'est également. Les
formules proviennent de la linéarité de l'intégrale.

Théorème~9.6.11 Soit f : I \rightarrow~ \mathbb{C} continue par morceaux. Alors f est
intégrable sur I si et seulement si les fonctions (à valeurs réelles)
\mathrmRe~f et
\mathrmIm~f le sont. Dans ce
cas

\int  I~f =\\int
 I \mathrmRe~f +
i\int  I~\
\mathrmImf,\quad
\int  I\overlinef~ =
\overline\int  If~

Démonstration Si f est intégrable sur I, il en est de même pour
\textbar{}f\textbar{} et donc pour
\mathrmRe~f et
\mathrmIm~f puisque 0
\leq\textbar{}\mathrmRe~f\textbar{}\leq\textbar{}f\textbar{}
et 0
\leq\textbar{}\mathrmIm~f\textbar{}\leq\textbar{}f\textbar{}.
Inversement, si \mathrmRe~f
et \mathrmIm~f sont
intégrables, alors f =\
\mathrmRef +
i\mathrmIm~f l'est
également. Les formules proviennent de la linéarité de l'intégrale.

Remarque~9.6.2 La combinaison de ces deux théorèmes peut permettre de
ramener un problème sur des fonctions à valeurs complexes à des
problèmes sur des fonctions à valeurs réelles positives.

\paragraph{9.6.3 Convention et relation de Chasles}

Définition~9.6.2 Soit I un intervalle de \mathbb{R}~, f : I \rightarrow~ \mathbb{C} continue par
morceaux et intégrable. Soit a,b \in\overlineI. Alors
on posera

\int  a^b~f(t) dt =
\left \ \cases
\int  {]}a,b{[}~f &si a \textless{} b
\cr 0 &si a = b \cr
-\int  {]}b,a{[}~f&si b \textless{} a
 \right .

La définition a bien un sens puisque f est intégrable sur {]}a,b{[}\subset~ I
ou {]}b,a{[}\subset~ I suivant le cas.

Théorème~9.6.12 Soit I un intervalle de \mathbb{R}~, f : I \rightarrow~ \mathbb{C} continue par
morceaux et intégrable. Soit a,b,c \in\overlineI. Alors
on a

\int  a^c~f
=\int  a^b~f
+\int  b^c~f

Démonstration Etudier toutes les positions relatives de a,b et c.

\paragraph{9.6.4 Règles de comparaison}

Théorème~9.6.13 Soit f : {[}a,b{[}\rightarrow~ \mathbb{C} continue par morceaux et g :
{[}a,b{[}\rightarrow~ \mathbb{R}~^+ continue par morceaux, positive et intégrable.
On suppose qu'au voisinage de b on a f = O(g) (resp. f = o(g)). Alors f
est intégrable sur {[}a,b{[} et \int ~
{[}x,b{[}f(t) dt = O(\int ~
{[}x,b{[}g(t) dt) (resp. \int ~
{[}x,b{[}f(t) dt = o(\int ~
{[}x,b{[}g(t) dt))

Démonstration On a en effet \textbar{}f\textbar{} = O(g) (resp.
\textbar{}f\textbar{} = o(g)) et \left
\textbar{}\int ~
{[}x,b{[}f\right
\textbar{}\leq\int ~
{[}x,b{[}\textbar{}f\textbar{}. Il suffit donc d'appliquer le
théorème de comparaison à \textbar{}f\textbar{} et g.

Remarque~9.6.3 Il suffit pour appliquer le théorème précédent que la
condition de positivité de g soit vérifiée dans un voisinage de b.

\paragraph{9.6.5 Espaces de fonctions continues}

Théorème~9.6.14 Soit I un intervalle de \mathbb{R}~. L'ensemble des fonctions
continues et intégrables sur I à valeurs complexes est un sous-espace
vectoriel de l'espace C(I, \mathbb{C}). L'application
f\mapsto~\\textbar{}f\\textbar{}1
=\int  I~\textbar{}f\textbar{} est une
norme sur cet espace (appelée la norme de la convergence en moyenne).

Démonstration Vérification immédiate à partir des résultats précédents.

Théorème~9.6.15 Soit I un intervalle de \mathbb{R}~. L'ensemble des fonctions
continues à valeurs complexes dont le carré est intégrable sur I est un
sous-espace vectoriel de l'espace C(I, \mathbb{C}). L'application
(f,g)\mapsto~(f\mathrel∣g)
=\int  I\overlinef~g
est un produit scalaire hermitien sur cet espace. En particulier,
l'application
f\mapsto~\\textbar{}f\\textbar{}2
= (f∣f)^1\diagup2 est une norme sur cet
espace et on a l'inégalité de Cauchy-Schwarz
\textbar{}(f∣g)\textbar{}\leq\\textbar{}
f\\textbar{}2\\textbar{}g\\textbar{}2.

Démonstration Il est clair que si f est de carré intégrable, il en est
de même de \alpha~f pour \alpha~ \in \mathbb{C}. De plus l'inégalité élémentaire \textbar{}f +
g\textbar{}^2 \leq 2\textbar{}f\textbar{}^2 +
2\textbar{}g\textbar{}^2 montre que si f et g sont de carré
intégrables, il en est de même de f + g. Comme de surcroît il existe des
fonctions de carré intégrables (par exemple la fonction nulle),
celles-ci forment un sous-espace vectoriel de C(I, \mathbb{C}). L'inégalité
élémentaire \textbar{}\overlinefg\textbar{}\leq 1
\over 2 \textbar{}f\textbar{}^2 + 1
\over 2 \textbar{}g\textbar{}^2 montre que
si f et g sont de carré intégrables, \overlinegf est
intégrable ce qui permet de définir (f∣g)
=\int  I\overlinef~g.
L'application est visiblement sesquilinéaire hermitienne, on a
(f∣f) =\int ~
I\textbar{}f\textbar{}^2 ≥ 0 avec égalité si et
seulement si \textbar{}f\textbar{}^2 = 0, soit f = 0, puisque
f est continue. Les autres affirmations sont des conséquences des
résultats sur les produits scalaires hermitiens.

\paragraph{9.6.6 Notion d'intégrale impropre}

Définition~9.6.3 Soit -\infty~ \textless{} a \textless{} b \leq +\infty~ et f :
{[}a,b{[}\rightarrow~ E continue par morceaux. On dit que l'intégrale
\int  a^b~f(t) dt converge si
existe
limx\rightarrow~b,x\textless{}b~\\int
 a^xf(t) dt. Dans ce cas on pose
\int  a^b~f(t) dt
=\
limx\rightarrow~b,x\textless{}b\int ~
a^xf(t) dt.

On a une notion similaire avec -\infty~\leq a \textless{} b \textless{} +\infty~ et f
:{]}a,b{]} \rightarrow~ E continue par morceaux.

Remarque~9.6.4 Si l'intégrale ne converge pas, elle est dite divergente.
Si b \textless{} +\infty~ et si f est la restriction à {[}a,b{[} d'une
fonction réglée sur {[}a,b{]}, alors l'application
x\mapsto~\int ~
a^xf(t) dt est continue au point b~; l'intégrale impropre
est donc convergente et la valeur de l'intégrale impropre est donc la
valeur de l'intégrale, si bien qu'il n'y a pas d'ambiguïté dans la
notation \int  a^b~f(t) dt~; dans
ce cas nous parlerons d'une intégrale faussement impropre. Un exemple
typique est celui de \int  0^1~
sin t \over t~ dt qui est a
priori impropre en 0, mais qui est la restriction à {]}0,1{]} de la
fonction continue f(t) = \left \
\cases  sin~ t
\over t &si t\neq~0
\cr 1 &si t = 0 \cr 
\right ..

Proposition~9.6.16 Soit f : {[}a,b{[}\rightarrow~ E une fonction continue par
morceaux et c \in {[}a,b{[}. Alors l'intégrale \\int
 a^bf(t) dt converge si et seulement si~l'intégrale
\int  c^b~f(t) dt converge.

Démonstration On a \int  a^x~f(t)
dt =\int  a^c~f(t) dt
+\int  c^x~f(t) dt ce qui montre
que \int  a^x~f(t) dt a une
limite en b si et seulement si~\int ~
c^xf(t) dt en a une.

Remarque~9.6.5 Cette propriété montre que si f : {[}a,b{[}\rightarrow~ E est une
fonction continue par morceaux, la convergence de
\int  a^b~f(t) dt ne dépend que
de la restriction de f à un voisinage de b~; il s'agit donc d'une notion
locale en b.

Théorème~9.6.17 Si f est intégrable sur {[}a,b{[}, alors
\int  a^b~f(t) dt converge. Mais
la réciproque est fausse dans le cas général (mais vraie pour les
fonctions à valeurs dans \mathbb{R}~^+).

Démonstration On a vu que si f est intégrable sur {[}a,b{[}, alors
x\mapsto~\int ~
a^xf(t) dt admet la limite \int ~
If au point b. L'exemple suivant montre que la réciproque est
fausse.

Exemple~9.6.1 Etude de l'intégrale \int ~
1^+\infty~ sin~ t
\over t^\alpha~ dt pour \alpha~ \textgreater{} 0. On a
 sin t \over t^\alpha~~
= O( 1 \over t^\alpha~ ), donc si \alpha~
\textgreater{} 1 la fonction est intégrable.

Si 0 \textless{} \alpha~ \leq 1, on a après intégration par parties

\int  1^x~
sin t \over t^\alpha~~ dt
= cos 1 - \cos~ x
\over x^\alpha~ +\int ~
1^x cos~ t
\over t^\alpha~+1 dt

Mais limx\rightarrow~+\infty~~
cos x \over x^\alpha~~ =
0 et la fonction t\mapsto~
cos t \over t^\alpha~+1~
est intégrable puisque  cos~ t
\over t^\alpha~+1 = O( 1 \over
t^\alpha~+1 ). On en déduit que le terme de droite de l'égalité
ci dessus a une limite en + \infty~, et donc le terme de gauche aussi. En
conséquence, l'intégrale impropre \int ~
1^+\infty~ sin~ t
\over t^\alpha~ dt converge. Montrons que la
fonction n'est pas intégrable~; on a

\begin{align*} \int ~
1^x \textbar{}sin~ t\textbar{}
\over t^\alpha~ & ≥& \\int
 1^x sin ^2~t
\over t^\alpha~ dt \%&
\\ & =& 1 \over 2
\int  1^x~ 1
- cos~ (2t) \over
t^\alpha~ dt \%& \\ & =& 1
\over 2 \int ~
1^x 1 \over t^\alpha~ dt - 1
\over 2 \int ~
1^x cos~ (2t)
\over t^\alpha~ dt\%&
\\ \end{align*}

Mais l'intégrale \int  1^x~ 1
\over t^\alpha~ dt admet pour limite + \infty~ (car \alpha~ \leq
1), alors que l'intégrale \int ~
1^x cos~ (2t)
\over t^\alpha~ dt converge (même méthode
d'intégration par parties). On en déduit que
limx\rightarrow~+\infty~~\\int
 1^x sin ^2~t
\over t^\alpha~ dt = +\infty~ et donc aussi
limx\rightarrow~+\infty~~\\int
 1^x \textbar{} sin~
t\textbar{} \over t^\alpha~ dt = +\infty~.

{[}
{[}
{[}
{[}
