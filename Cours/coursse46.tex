%
\subsubsection{Fonctions réelles d'une variable réelle}
%
%\paragraph{Théorème de Rolle, formule des accroissements finis}
%
\begin{lem}
Soit $f : I \rightarrow \mathbb{R}$; si f admet en $c \in I^o$ un
extremum local et si f est dérivable au point c, alors $f'(c) = 0$.
\end{lem}
%
%Démonstration Supposons par exemple que f a en c un maximum local. Pour
%c - \eta \textless{} x \textless{} c, on a  f(x)-f(c)
%\over x-c ≥ 0 d'où en faisant tendre x vers c, f'(c) ≥
%0. Pour c \textless{} x \textless{} c + \eta, on a  f(x)-f(c)
%\over x-c \leq 0 d'où en faisant tendre x vers c, f'(c) \leq
%0. On a donc f'(c) = 0.
%
%Théorème~8.3.2 (Rolle). Soit f : {[}a,b{]} \rightarrow~ \mathbb{R}~, continue sur {[}a,b{]},
%dérivable sur {]}a,b{[} telle que f(a) = f(b). Alors il existe c
%\in{]}a,b{[} tel que f'(c) = 0.
%
%Démonstration Si f est constante sur {[}a,b{]}, n'importe quel c
%\in{]}a,b{[} convient. Sinon, par exemple, il existe x \in {[}a,b{]} tel que
%f(x) \textgreater{} f(a) = f(b). La fonction f est continue sur le
%compact {[}a,b{]} donc elle est bornée et atteint ses bornes. Soit c \in
%{[}a,b{]} tel que f(c) =\
%sup\f(t)∣t \in
%{[}a,b{]}\. On a f(c) ≥ f(x) \textgreater{} f(a) =
%f(b), donc c \in{]}a,b{[}. Mais alors, le lemme ci dessus garantit que
%f'(c) = 0.
%
%Corollaire~8.3.3 (formule des accroissements finis). Soit f : {[}a,b{]}
%\rightarrow~ \mathbb{R}~, continue sur {[}a,b{]}, dérivable sur {]}a,b{[}. Alors il existe c
%\in{]}a,b{[} tel que f(b) - f(a) = (b - a)f'(c).
%
%Démonstration On applique le théorème de Rolle à g(t) = f(t) -
%f(b)-f(a) \over b-a (t - a). On a g(b) = g(a) = f(a), g
%est, comme f, continue sur {[}a,b{]} et dérivable sur {]}a,b{[}. Donc il
%existe c \in{]}a,b{[} tel que g'(c) = 0~; mais g'(c) = f'(c) - f(b)-f(a)
%\over b-a d'où le résultat.
%
%\paragraph{8.3.2 Monotonie et dérivation}
%
%Théorème~8.3.4 Soit I un intervalle de \mathbb{R}~, f : I \rightarrow~ \mathbb{R}~ continue sur I et
%dérivable sur I^o. Alors (i) f est constante sur I si et
%seulement si~\forall~t \in I^o~, f'(t) = 0
%(ii) f est croissante sur I si et seulement
%si~\forall~t \in I^o~, f'(t) ≥ 0 (iii) f est
%décroissante sur I si et seulement si~\forall~~t \in
%I^o, f'(t) \leq 0
%
%Démonstration La définition de la dérivée f'(t)
%=\
%limx\rightarrow~t,x\neq~t f(x)-f(t)
%\over x-t montre clairement que les conditions sont
%nécessaires (prendre x \textgreater{} t et faire tendre x vers t).
%Inversement, si x,y \in I avec x \textless{} y, f est continue sur
%{[}x,y{]} \subset~ I et dérivable sur {]}x,y{[}\subset~ I^o et donc la
%formule des accroissements finis assure qu'il existe z \in{]}x,y{[}\subset~
%I^o tel que f(y) - f(x) = (y - x)f'(z), ce qui montre
%immédiatement que les conditions sont suffisantes.
%
%Corollaire~8.3.5 Soit I un intervalle de \mathbb{R}~, f : I \rightarrow~ \mathbb{R}~ continue sur I et
%dérivable sur I^o. Alors on a équivalence de (i) f est
%strictement croissante (ii) \forall~~t \in
%I^o, f'(t) ≥ 0 et \t \in
%I^o∣f'(t) = 0\
%est d'intérieur vide.
%
%Démonstration (i) \rigtharrow~(ii) Si f est strictement croissante, alors
%\forall~t \in I^o~, f'(t) ≥ 0~; supposons que
%\t \in I^o∣f'(t) =
%0\ n'est pas d'intérieur vide~; alors il contient un
%segment {[}a,b{]} avec a \textless{} b~; mais alors d'après le théorème
%précédent, f est constante sur {[}a,b{]} ce qui contredit la stricte
%monotonie de f.
%
%(ii) \rigtharrow~(i) On sait que si \forall~t \in I^o~,
%f'(t) ≥ 0, f est croissante~; supposons qu'elle n'est pas strictement
%croissante~; alors il existe a,b \in I tels que a \textless{} b et f(a) =
%f(b)~; en conséquence f est constante sur {]}a,b{[}\subset~ I^o et
%donc \forall~~t \in{]}a,b{[}, f'(t) = 0~; donc
%l'intervalle ouvert {]}a,b{[} est contenu dans l'intérieur de
%\t \in I^o∣f'(t) =
%0\, c'est absurde.
%
%\paragraph{8.3.3 Difféomorphismes}
%
%Théorème~8.3.6 Soit I et J deux intervalles de \mathbb{R}~ et f : I \rightarrow~ J un
%homéomorphisme. Soit a \in I un point où f est dérivable. Alors
%f^-1 est dérivable au point f(a) si et seulement
%si~f'(a)\neq~0. Dans ce cas,
%(f^-1)'(f(a)) = 1 \over f'(a) .
%
%Démonstration Posons g = f^-1. On a g \cdot f =
%\mathrmIdI. Si f est dérivable au point a et
%g dérivable au point f(a), le théorème de dérivation des fonctions
%composées assure que 1 = (\mathrmIdI)'(a) =
%(g \cdot f)'(a) = g'(f(a))f'(a), donc f'(a)\neq~0 et
%g'(f(a)) = 1 \over f'(a) . Inversement supposons que
%f'(a)\neq~0. On a alors
%limt\rightarrow~a,t\neq~a~
%t-a \over f(t)-f(a) = 1 \over f'(a)
%. Appliquons le théorème de composition des limites en posant t = g(u)
%(avec a = g(f(a))), en remarquant que u\neq~f(a)
%\rigtharrow~ g(u)\neq~a. On a donc, puisque g est continue
%au point f(a),
%
%limu\rightarrow~f(a),u\neq~f(a)~
%g(u) - g(f(a)) \over u - f(a) = 1
%\over f'(a)
%
%Donc g est dérivable au point f(a).
%
%Définition~8.3.1 Soit I et J deux intervalles de \mathbb{R}~~; on dit que f : I \rightarrow~
%J est un difféomorphisme de classe C^n (n ≥ 1) si f est
%bi\\\\jmathmathmathmathective et f et f^-1 sont de classe C^n.
%
%Théorème~8.3.7 Soit n ≥ 1, f : I \rightarrow~ \mathbb{R}~. On a équivalence de (i) f est un
%C^n difféomorphisme de I sur f(I) (ii) f est de classe
%C^n et f' ne s'annule pas.
%
%Démonstration (i) \rigtharrow~(ii) est clair d'après le théorème précédent.
%Inversement, supposons que f est de classe C^n et que f' ne
%s'annule pas. Alors f' garde un signe constant (elle est continue), et
%donc f est strictement monotone. Donc f définit un homéomorphisme de I
%sur J = f(I). Le théorème précédent assure que f^-1 est
%dérivable sur I et que (f^-1)' = 1 \over
%f'\cdotf^-1 ce qui garantit dé\\\\jmathmathmathmathà la continuité de
%(f^-1)'. Supposons alors que f^-1 est de classe
%C^k avec k \textless{} n. Comme f' est de classe
%C^k, f' \cdot f^-1 est de classe C^k~; il
%en est donc de même de  1 \over f'\cdotf^-1 ,
%donc de (f^-1)' et donc f^-1 est de classe
%C^k+1~; par récurrence, on en déduit que f^-1 est
%de classe C^n.
%
%\paragraph{8.3.4 Formule de Taylor Lagrange}
%
%Théorème~8.3.8 (Taylor-Lagrange). Soit f : {[}a,b{]} \rightarrow~ \mathbb{R}~ de classe
%C^n sur {[}a,b{]} et n + 1 fois dérivable sur {]}a,b{[}.
%Alors il existe c \in{]}a,b{[} tel que
%
%f(b) = f(a) + \sum k=1^n~
%f^(k)(a) \over k! (b - a)^k +
%f^(n+1)(c) \over (n + 1)! (b -
%a)^n+1
%
%Démonstration Posons \phi(t) = f(b) - f(t)
%-\\sum ~
%k=1^n f^(k)(t) \over k!
%(b - t)^k - \lambda~(b - t)^n+1 où \lambda~ est choisi de telle
%sorte que \phi(a) = 0 (c'est évidemment possible). Il est clair que \phi est
%continue sur {[}a,b{]}, dérivable sur {]}a,b{[} comme toutes les
%fonctions f^(k), 0 \leq k \leq n. De plus
%
%\begin{align*} \phi'(t)& =& -f'(t)
%-\sum k=1^n~
%f^(k+1)(t) \over k! (b - t)^k
%\%& \\ & \text &
%+\sum k=1^n~
%f^(k)(t) \over (k - 1)! (b -
%t)^k-1 + \lambda~(n + 1)(b - t)^n\%&
%\\ & =& -f'(t)
%-\sum l=2^n+1~
%f^(l)(t) \over (l - 1)! (b -
%t)^l-1 \%& \\ &
%\text & +\\sum
%k=1^n f^(k)(t) \over (k -
%1)! (b - t)^k-1 + \lambda~(n + 1)(b - t)^n\%&
%\\ & =& (b -
%t)^n\left ((n + 1)\lambda~ - f^(n+1)(t)
%\over n! \right ) \%&
%\\ \end{align*}
%
%(tous les autres termes se détruisent deux à deux). D'après le théorème
%de Rolle, il existe c \in{]}a,b{[} tel que \phi'(c) = 0, soit (b -
%c)^n\left ((n + 1)\lambda~ - f^(n+1)(c)
%\over n! \right ) = 0. Comme b -
%c\neq~0, on a \lambda~ = f^(n+1)(c)
%\over (n+1)! . En écrivant que \phi(a) = 0, on obtient
%alors la formule souhaitée.
%
%Remarque~8.3.1 Pour n = 0, on trouve comme cas particulier la formule
%des accroissements finis. La même formule est encore valable si on prend
%f : {[}b,a{]} \rightarrow~ \mathbb{R}~.
%
%\paragraph{8.3.5 Extensions du théorème des accroissements finis}
%
%Théorème~8.3.9 Soit f,g : {[}a,b{]} \rightarrow~ \mathbb{R}~ continues sur {[}a,b{]},
%dérivables sur {]}a,b{[}. Alors, il existe c \in{]}a,b{[} tel que
%\left
%\textbar{}\matrix\,f(b) - f(a)&f'(c)
%\cr g(b) - g(a)&g'(c)\right \textbar{}
%= 0.
%
%Démonstration Posons
%
%\phi(t) = \left
%\textbar{}\matrix\,f(b) - f(a)&f(t) -
%f(a) \cr g(b) - g(a)&g(t) - g(a)\right
%\textbar{}
%
%La fonction \phi est continue sur {[}a,b{]}, dérivable sur {]}a,b{[} avec
%\phi'(t) = \left
%\textbar{}\matrix\,f(b) - f(a)&f'(t)
%\cr g(b) - g(a)&g'(t)\right \textbar{}.
%Comme \phi(a) = \phi(b) = 0, le théorème de Rolle garantit l'existence d'un c
%\in{]}a,b{[} tel que \phi'(c) = 0.
%
%Corollaire~8.3.10 (règle de L'Hôpital). Soit f,g : I \rightarrow~ \mathbb{R}~ continues sur
%I, dérivables sur I \diagdown\a\. On suppose
%qu'il existe \eta \textgreater{} 0 tel que g' ne s'annule pas sur {]}a -
%\eta,a + \eta{[}\diagdown\a\ et que  f'
%\over g' a une limite \ell au point a. Alors  f(t)-f(a)
%\over g(t)-g(a) admet la même limite au point a.
%
%Démonstration Le théorème de Rolle garantit dé\\\\jmathmathmathmathà que g(t) - g(a) ne
%s'annule pas sur {]}a - \eta,a +
%\eta{[}\diagdown\a\. De plus le théorème
%précédent montre que pour t \in{]}a - \eta,a +
%\eta{[}\diagdown\a\, il existe ct
%\in{]}a,t{[} (ou {]}t,a{[}) tel que \left
%\textbar{}\matrix\,f(t) -
%f(a)&f'(ct) \cr g(t) -
%g(a)&g'(ct)\right \textbar{} = 0 soit 
%f(t)-f(a) \over g(t)-g(a) = f'(ct)
%\over g'(ct) . Quand t tend vers a, il en est
%de même de ct et le théorème de composition des limites donne
%
%limt\rightarrow~a,t\neq~a~
%f(t) - f(a) \over g(t) - g(a) = \ell
%
%\paragraph{8.3.6 Fonctions convexes de classe \mathcal{C}^1}
%
%Définition~8.3.2 Soit I un intervalle de \mathbb{R}~ et f : I \rightarrow~ \mathbb{R}~ une fonction de
%classe \mathcal{C}^1. On dit que f est convexe si f' est croissante.
%
%Remarque~8.3.2 Si f est de classe C^2, f est convexe si et
%seulement si~f'' est positive.
%
%Théorème~8.3.11 Soit I un intervalle de \mathbb{R}~ et f : I \rightarrow~ \mathbb{R}~ une fonction de
%classe \mathcal{C}^1 convexe. Alors (i) \forall~~a,b
%\in I, \forall~~t \in {[}0,1{]}, f(ta + (1 - t)b) \leq tf(a)
%+ (1 - t)f(b) (ii) \Gamma = \(x,y) \in
%\mathbb{R}~^2∣x \in I\text et
%y ≥ f(x)\ est une partie convexe de \mathbb{R}~^2
%(iii) \forall~~a,b \in I, f(b) ≥ f(a) + (b - a)f'(a)
%(iv) si a \in I, l'application I \diagdown\a\
%dans \mathbb{R}~, t\mapsto~pa(t) = f(t)-f(a)
%\over t-a est croissante (v)
%\forall~~a,b,c \in I, a \textless{} b \textless{} c \rigtharrow~
%f(b)-f(a) \over b-a \leq f(c)-f(a) \over
%c-a \leq f(c)-f(b) \over c-b
%
%Démonstration (i) On peut évidemment supposer a \textless{} b. D'après
%le théorème des accroissements finis, il existe c \in{]}a,b{[} tel que
%f(b) - f(a) = (b - a)f'(c). Posons c = t0a + (1 -
%t0)b. Soit \phi(t) = tf(a) + (1 - t)f(b) - f(ta + (1 - t)b) pour
%t \in {[}0,1{]}. Alors \phi est de classe \mathcal{C}^1 et \phi'(t) = f(a) -
%f(b) - (a - b)f'(ta + (1 - t)b) = (b - a)(f'(ta + (1 - t)b) -
%f'(t0a + (1 - t0)b). Comme f' est croissante et
%t\mapsto~ta + (1 - t)b est décroissante, la composée
%est décroissante et donc on a le tableau de variation
%
%\begin{center}\rule{3in}{0.4pt}\end{center}
%
%\begin{center}\rule{3in}{0.4pt}\end{center}
%
%\begin{center}\rule{3in}{0.4pt}\end{center}
%
%\begin{center}\rule{3in}{0.4pt}\end{center}
%
%\begin{center}\rule{3in}{0.4pt}\end{center}
%
%\begin{center}\rule{3in}{0.4pt}\end{center}
%
%t
%
%0
%
%t0
%
%1
%
%\begin{center}\rule{3in}{0.4pt}\end{center}
%
%\begin{center}\rule{3in}{0.4pt}\end{center}
%
%\begin{center}\rule{3in}{0.4pt}\end{center}
%
%\begin{center}\rule{3in}{0.4pt}\end{center}
%
%\begin{center}\rule{3in}{0.4pt}\end{center}
%
%\begin{center}\rule{3in}{0.4pt}\end{center}
%
%\phi'(t)
%
%+
%
%0
%
%-
%
%\begin{center}\rule{3in}{0.4pt}\end{center}
%
%\begin{center}\rule{3in}{0.4pt}\end{center}
%
%\begin{center}\rule{3in}{0.4pt}\end{center}
%
%\begin{center}\rule{3in}{0.4pt}\end{center}
%
%\begin{center}\rule{3in}{0.4pt}\end{center}
%
%\begin{center}\rule{3in}{0.4pt}\end{center}
%
%\phi(t)
%
%0
%
%\nearrow
%
%\searrow
%
%0
%
%\begin{center}\rule{3in}{0.4pt}\end{center}
%
%\begin{center}\rule{3in}{0.4pt}\end{center}
%
%\begin{center}\rule{3in}{0.4pt}\end{center}
%
%\begin{center}\rule{3in}{0.4pt}\end{center}
%
%\begin{center}\rule{3in}{0.4pt}\end{center}
%
%\begin{center}\rule{3in}{0.4pt}\end{center}
%
%ce qui montre que la fonction \phi est positive sur {[}0,1{]}.
%
%(ii) Soit (x1,y1) et (x2,y2)
%dans \Gamma et t \in {[}0,1{]}. On a
%
%ty1 + (1 - t)y2 ≥ tf(x1) + (1 -
%t)f(x2) ≥ f(tx1 + (1 - t)x2)
%
%donc t(x1,y1) + (1 - t)(x2,y2) \in
%\Gamma. Donc \Gamma est convexe.
%
%(iii) Posons \phi(t) = f(t) - f(a) - (t - a)f'(a). La fonction \phi est de
%classe \mathcal{C}^1 et \phi'(t) = f'(t) - f'(a). Comme f' est croissante,
%on a le tableau de variation
%
%\begin{center}\rule{3in}{0.4pt}\end{center}
%
%\begin{center}\rule{3in}{0.4pt}\end{center}
%
%\begin{center}\rule{3in}{0.4pt}\end{center}
%
%\begin{center}\rule{3in}{0.4pt}\end{center}
%
%\begin{center}\rule{3in}{0.4pt}\end{center}
%
%\begin{center}\rule{3in}{0.4pt}\end{center}
%
%t
%
%a
%
%\begin{center}\rule{3in}{0.4pt}\end{center}
%
%\begin{center}\rule{3in}{0.4pt}\end{center}
%
%\begin{center}\rule{3in}{0.4pt}\end{center}
%
%\begin{center}\rule{3in}{0.4pt}\end{center}
%
%\begin{center}\rule{3in}{0.4pt}\end{center}
%
%\begin{center}\rule{3in}{0.4pt}\end{center}
%
%\phi'(t)
%
%+
%
%0
%
%-
%
%\begin{center}\rule{3in}{0.4pt}\end{center}
%
%\begin{center}\rule{3in}{0.4pt}\end{center}
%
%\begin{center}\rule{3in}{0.4pt}\end{center}
%
%\begin{center}\rule{3in}{0.4pt}\end{center}
%
%\begin{center}\rule{3in}{0.4pt}\end{center}
%
%\begin{center}\rule{3in}{0.4pt}\end{center}
%
%\phi(t)
%
%\searrow
%
%0
%
%\nearrow
%
%\begin{center}\rule{3in}{0.4pt}\end{center}
%
%\begin{center}\rule{3in}{0.4pt}\end{center}
%
%\begin{center}\rule{3in}{0.4pt}\end{center}
%
%\begin{center}\rule{3in}{0.4pt}\end{center}
%
%\begin{center}\rule{3in}{0.4pt}\end{center}
%
%\begin{center}\rule{3in}{0.4pt}\end{center}
%
%ce qui montre que la fonction \phi est positive sur I.
%
%(iv) Posons pa(t) = f(t)-f(a) \over t-a si
%t\neq~a et pa(a) = f'(a). La fonction
%pa est continue sur I, dérivable sur I
%\diagdown\a\ et pa'(t) =
%f(a)-f(t)-(a-t)f'(t) \over (t-a)^2 ≥ 0
%d'après (iii). On en déduit que pa est croissante.
%
%(v) D'après (iv), on a pa(b) \leq pa(c) =
%pc(a) \leq pc(b) ce qui est le résultat souhaité.
%
%Théorème~8.3.12 Soit f : I \rightarrow~ \mathbb{R}~ de classe \mathcal{C}^1 convexe. Alors,
%pour tout
%(x1,\\ldots,xn~)
%\in I^n, pour toute famille
%(\alpha~1,\\ldots,\alpha~n~)
%\in (\mathbb{R}~^+)^n telle que \alpha~1 +
%\\ldots~ +
%\alpha~n = 1, on a
%
%f(\sum i=1^n\alpha~~
%ixi) \leq\\sum
%i=1^n\alpha~ if(xi)
%
%Démonstration Par récurrence sur n. Si n = 2, on a \alpha~2 = 1 -
%\alpha~1 et \alpha~1 \in {[}0,1{]}. L'inégalité se réduit à
%l'assertion (i) du théorème précédent. Supposons le résultat vrai pour n
%- 1 et montrons le pour n. Si \alpha~n = 0, on est immédiatement
%ramené au cas n - 1. On peut donc supposer
%\alpha~n\neq~0. Si \alpha~n = 1, alors
%tous les autres \alpha~i sont nuls et l'inégalité est triviale. On
%peut donc supposer \alpha~n \in{]}0,1{[}. On écrit alors
%\\sum ~
%i=1^n\alpha~ixi = \alpha~nxn
%+ (1 - \alpha~n)y avec y =
%\alpha~1x1+\\ldots+\alpha~n-1xn-1~
%\over
%\alpha~1+\\ldots+\alpha~n-1~
%= \beta~1x1 +
%\\ldots\beta~n-1xn-1~
%\in I. On a alors \beta~i ≥ 0 et
%\\sum ~
%i=1^n-1\beta~i = 1. On peut donc écrire (par
%l'hypothèse de récurrence) f(y)
%\leq\\sum ~
%i=1^n-1\beta~if(xi) soit
%
%\begin{align*} f(\\sum
%i=1^n\alpha~ ixi)& =&
%f(\alpha~nxn + (1 - \alpha~n)y) \leq
%\alpha~nf(xn) + (1 - \alpha~n)f(y) \%&
%\\ & \leq& \alpha~nf(xn) + (1
%- \alpha~n)\\sum
%i=1^n-1\beta~ if(xi) =
%\sum i=1^n\alpha~~
%if(xi)\%& \\
%\end{align*}
%
%puisque (1 - \alpha~n)\beta~i = \alpha~i.
%
\begin{cor}[inégalité de Holder]
	Soit $p,q \in \mathbb{R}^{+∗}$ tels
	que  $\frac{1}{p} + \frac{1}{q} = 1$.
Pour toute famille $a1,\ldots,a_n,b_1,\ldots,b_n$
de réels positifs, on a
\[
\sum_{i=1}^n a_i b_i \leq \left(\sum_{i=1}^{n} a_i^p\right)^{\frac{1}{p}} \left (\sum_{i=1}^{n} b_i^q \right)^{\frac{1}{q}}
\]
\end{cor}
%
%Démonstration Posons A = \left
%(\\sum ~
%i=1^nai^p\right
%)^1\diagupp, B = \left
%(\\sum ~
%i=1^nbi^q\right
%)^1\diagupq. La fonction exponentielle étant convexe sur \mathbb{R}~, on a
%\forall~s,t \in \mathbb{R}~, e~^ s \over
%p + t \over q  \leq 1 \over p
%e^s + 1 \over q e^t. Si
%ai et bi sont non nuls, en appliquant ceci à s =
%plog  ai \over A~
%et t = qlog  bi~
%\over B , on obtient  ai
%\over A  bi \over B \leq 1
%\over p  ai^p \over
%A^p + 1 \over q  bi^q
%\over B^q , inégalité qui reste vrai si
%aibi = 0~; en sommant de i = 1 \\\\jmathmathmathmathusque n on obtient
%
% 1 \over AB \\sum
%i=1^na ibi \leq 1
%\over pA^p  \\sum
%i=1^na i^p + 1 \over
%qB^q  \\sum
%i=1^nb i^q = 1 \over
%p + 1 \over q = 1
%
%soit \\sum ~
%i=1^naibi \leq AB ce qu'on voulait
%démontrer.
%
%Corollaire~8.3.14 (inégalité de Minkowski). Soit p ≥ 1. Pour toute
%famille
%a1,\\ldots,an,b1,\\\ldots,bn~
%de réels positifs, on a
%
% \left (\\sum
%i=1^n(a i +
%bi)^p\right )^1\diagupp
%\leq\left (\\sum
%i=1^na i^p\right
%)^1\diagupp + \left (\\sum
%i=1^nb i^p\right
%)^1\diagupp
%
%Démonstration C'est évident si p = 1~; si p \textgreater{} 1,
%définissons q par la condition  1 \over p + 1
%\over q = 1~; on écrit (ai +
%bi)^p = ai(ai +
%bi)^p-1 + bi(ai +
%bi)^p-1 et on applique deux fois l'inégalité de
%Hölder. On obtient alors
%
%\begin{align*} \\sum
%i=1^n(a i + bi)^p&
%\leq& \left (\\sum
%i=1^na i^p\right
%)^1\diagupp\left (\\sum
%i=1^n(a i +
%bi)^(p-1)q\right )^1\diagupq \%&
%\\ & \text &
%+\left (\\sum
%i=1^nb i^p\right
%)^1\diagupp\left (\\sum
%i=1^n(a i +
%bi)^(p-1)q\right )^1\diagupq\%&
%\\ \end{align*}
%
%Mais (p - 1)q = p et l'inégalité ci dessus s'écrit donc après mise en
%facteur
%
%\begin{align*} \left
%(\sum i=1^n(a i~ +
%bi)^p\right )^1\diagupp&& \%&
%\\ & \leq& \left
%(\left (\\sum
%i=1^na i^p\right
%)^1\diagupp + \left (\\sum
%i=1^nb i^p\right
%)^1\diagupp\right )\left
%(\sum i=1^n(a i~ +
%bi)^p\right )^1\diagupq\%&
%\\ \end{align*}
%
%Si \\sum ~
%i=1^n(ai + bi)^p = 0,
%l'inégalité cherchée est évidente~; sinon, on peut diviser les deux
%membres par \left
%(\\sum ~
%i=1^n(ai +
%bi)^p\right )^1\diagupq et on
%obtient (en tenant compte de 1 - 1 \over p = 1
%\over q )
%
% \left (\\sum
%i=1^n(a i +
%bi)^p\right )^1\diagupp
%\leq\left (\\sum
%i=1^na i^p\right
%)^1\diagupp + \left (\\sum
%i=1^nb i^p\right
%)^1\diagupp
%
%{[}
%{[}
%{[}
%{[}
