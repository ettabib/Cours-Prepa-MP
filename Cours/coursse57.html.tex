\textbf{Warning: 
requires JavaScript to process the mathematics on this page.\\ If your
browser supports JavaScript, be sure it is enabled.}

\begin{center}\rule{3in}{0.4pt}\end{center}

{[}
{[}
{[}{]}
{[}

\subsubsection{9.8 Généralités sur les intégrales impropres}

Remarque~9.8.1 Ce paragraphe et ceux qui suivent, qui correspondent aux
anciens programmes de classes préparatoires, font double emploi avec les
paragraphes sur les fonctions intégrables sur un intervalle. Ils ont été
maintenus ici par souci de compatibilité avec certains enseignements
universitaires.

Définition~9.8.1 On dit qu'une fonction est réglée sur un intervalle I
si elle est réglée sur tout segment inclus dans I.

\paragraph{9.8.1 Notion d'intégrale impropre}

Définition~9.8.2 Soit -\infty~ \textless{} a \textless{} b \leq +\infty~ et f :
{[}a,b{[}\rightarrow~ E réglée. On dit que l'intégrale \\int
 a^bf(t) dt converge si existe
limx\rightarrow~b,x\textless{}b~\\int
 a^xf(t) dt. Dans ce cas on pose
\int  a^b~f(t) dt
=\
limx\rightarrow~b,x\textless{}b\int ~
a^xf(t) dt. On a une notion similaire avec -\infty~\leq a
\textless{} b \textless{} +\infty~ et f :{]}a,b{]} \rightarrow~ E réglée.

Remarque~9.8.2 Si l'intégrale ne converge pas, elle est dite divergente.
Si b \textless{} +\infty~ et si f est la restriction à {[}a,b{[} d'une
fonction réglée sur {[}a,b{]}, alors l'application
x\mapsto~\int ~
a^xf(t) dt est continue au point b~; l'intégrale impropre
est donc convergente et la valeur de l'intégrale impropre est donc la
valeur de l'intégrale, si bien qu'il n'y a pas d'ambiguïté dans la
notation \int  a^b~f(t) dt~; dans
ce cas nous parlerons d'une intégrale faussement impropre. Un exemple
typique est celui de \int  0^1~
sin t \over t~ dt qui est a
priori impropre en 0, mais qui est la restriction à {]}0,1{]} de la
fonction continue f(t) = \left \
\cases  sin~ t
\over t &si t\neq~0
\cr 1 &si t = 0 \cr 
\right ..

Proposition~9.8.1 Soit f : {[}a,b{[}\rightarrow~ E une fonction réglée et c \in
{[}a,b{[}. Alors l'intégrale \int ~
a^bf(t) dt converge si et seulement si~l'intégrale
\int  c^b~f(t) dt converge.

Démonstration On a \int  a^x~f(t)
dt =\int  a^c~f(t) dt
+\int  c^x~f(t) dt ce qui montre
que \int  a^x~f(t) dt a une
limite en b si et seulement si~\int ~
c^xf(t) dt en a une.

Remarque~9.8.3 Cette propriété montre que si f : {[}a,b{[}\rightarrow~ E est une
fonction réglée, la convergence de \int ~
a^bf(t) dt ne dépend que de la restriction de f à un
voisinage de b~; il s'agit donc d'une notion locale en b.

\paragraph{9.8.2 Intégrales plusieurs fois impropres}

Définition~9.8.3 Soit -\infty~\leq a \textless{} b \leq +\infty~ et f :{]}a,b{[}\rightarrow~ E
réglée. On dit que l'intégrale \int ~
a^bf(t) dt converge si on a les conditions équivalentes
(i) il existe c \in{]}a,b{[} tel que les deux intégrales impropres
\int  a^c~f(t) dt et
\int  c^b~f(t) dt convergent (ii)
pour tout c \in{]}a,b{[}, les deux intégrales impropres
\int  a^c~f(t) dt et
\int  c^b~f(t) dt convergent
(iii) l'application
(x,y)\mapsto~\int ~
x^yf(t) dt admet une limite quand x tend vers a et y tend
vers b indépendamment l'un de l'autre. On pose alors
\int  a^b~f(t) dt
=\int  a^c~f(t) dt
+\int  c^b~f(t) dt
=\
limx\rightarrow~a,y\rightarrow~b\int ~
x^yf(t) dt.

Démonstration Découle immédiatement de la relation de Chasles.

Remarque~9.8.4 L'intégrale \int ~
-\infty~^+\infty~t dt diverge alors que
limx\rightarrow~+\infty~~\\int
 -x^xt dt = 0. Il est donc impératif dans (iii)
d'introduire deux variables x et y et de les faire varier
indépendamment.

Définition~9.8.4 Soit -\infty~\leq a0 \textless{} a1
\textless{} \\ldots~
\textless{} an \leq +\infty~ et f
:{]}a0,an{[}\diagdown\a1,\\ldots,an-1\~
\rightarrow~ E une fonction réglée. On dit que l'intégrale
\int ~
a0^anf(t) dt converge si chacune des
intégrales \int ~
ai-1^aif(t) dt converge. On pose
alors \int  a^b~f(t) dt
= \\sum ~
i=1^n\int ~
ai-1^aif(t) dt

Avec ces définitions, toutes les questions concernant des intégrales
plusieurs fois impropres se ramènent à des problèmes sur les intégrales
une fois impropres.

\paragraph{9.8.3 Opérations sur les intégrales impropres}

Théorème~9.8.2 L'ensemble des fonctions réglées de {[}a,b{[} dans E
telles que l'intégrale impropre \int ~
a^bf(t) dt converge est un sous espace vectoriel ~de
l'ensemble des fonctions réglées de {[}a,b{[} dans E. L'application
f\mapsto~\int ~
a^bf(t) dt est linéaire de cet espace vectoriel dans E.

Démonstration Il suffit d'écrire \int ~
a^x(\alpha~f(t) + \beta~g(t)) dt = \alpha~\int ~
a^xf(t) dt + \beta~\int ~
a^xg(t) dt et d'utiliser les théorèmes sur les limites.

Théorème~9.8.3 Soit u : E \rightarrow~ F une application linéaire continue, f :
{[}a,b{[}\rightarrow~ E réglée telle que \int ~
a^bf(t) dt converge. Alors \int ~
a^bu(f(t)) dt converge et \int ~
a^bu(f(t)) dt = u(\int ~
a^bf(t) dt).

Démonstration Il suffit d'écrire \int ~
a^xu(f(t)) dt = u(\int ~
a^xf(t) dt) et d'utiliser la continuité de u.

Corollaire~9.8.4 Soit E un espace vectoriel normé~de dimension finie,
(e1,\\ldots,en~)
une base de E, f : {[}a,b{[}\rightarrow~ E réglée. On écrit f(t) =
f1(t)e1 +
\\ldots~ +
fn(t)en. Alors \int ~
a^bf(t) dt converge si et seulement si~chacune des
intégrales \int ~
a^bfi(t) dt converge et alors

\int  a^b~f(t) dt =
(\int  a^bf 1~(t)
dt)e1 +
\\ldots~ +
(\int  a^bf n~(t)
dt)en

Démonstration Soit u : K^n \rightarrow~ E,
(x1,\\ldots,xn)\mapsto~x1e1~
+ \\ldots~ +
xnen. L'application linéaire u est un isomorphisme
d'espaces vectoriels et puisque les espaces sont de dimension finie, u
et u^-1 sont continues. Il suffit alors d'appliquer le
théorème précédent en remarquant que f = u \cdot
(f1,\\ldots,fn~)
et que
(f1,\\ldots,fn~)
= u^-1 \cdot f.

Changement de variables Soit \phi : {[}a,b{[}\rightarrow~ {[}\alpha~,\beta~{[} de classe
\mathcal{C}^1 telle que limu\rightarrow~b~\phi(u)
= \beta~. Soit f : {[}\alpha~,\beta~{[}\rightarrow~ E continue. Pour x \in {[}a,b{[}, on peut alors
écrire \int  a^x~f(\phi(u))\phi'(u) du
=\int  \phi(a)^\phi(x)~f(t) dt. On en
déduit par le théorème de composition des limites, que si l'intégrale
\int  \alpha~^\beta~~f(t) dt converge, alors
l'intégrale \int ~
a^bf(\phi(u))\phi'(u) du converge et que dans ce cas

\int  a^b~f(\phi(u))\phi'(u) du
=\int  \phi(a)^\beta~~f(t) dt

Inversement, si l'on suppose que \phi est un homéomorphisme de {[}a,b{[}
sur {[}\alpha~,\beta~{[}, on a, pour y \in {[}\alpha~,\beta~{[}, \int ~
\alpha~^yf(t) dt =\int ~
\phi^-1(\alpha~)^\phi^-1(y) f(\phi(u))\phi'(u) du et
alors la convergence de \int ~
a^bf(\phi(u))\phi'(u) du implique celle de
\int  \alpha~^\beta~~f(t) dt et l'égalité
ci-dessus. On retiendra en particulier

Théorème~9.8.5 Soit \phi : {[}a,b{[}\rightarrow~ {[}\alpha~,\beta~{[} un homéomorphisme. On
suppose que \phi est de classe \mathcal{C}^1 et que
limu\rightarrow~b~\phi(u) = \beta~. Soit f : {[}\alpha~,\beta~{[}\rightarrow~
E continue. Alors les deux intégrales impropres
\int  a^b~f(\phi(u))\phi'(u) du et
\int  \alpha~^\beta~~f(t) dt sont de même
nature (convergentes ou divergentes) et on a l'égalité

\int  a^b~f(\phi(u))\phi'(u) du
=\int  \phi(a)^\beta~~f(t) dt

Intégration par parties Soit f,g : {[}a,b{[}\rightarrow~ \mathbb{C} de classe
\mathcal{C}^1. Pour x \in {[}a,b{[}, on peut alors faire une intégration
par parties et écrire

\int  a^x~f(t)g'(t) dt =
\left {[}f(t)g(t)\right {]}
a^x -\int  a^x~f'(t)g(t)
dt

Si deux des trois termes qui dépendent de x admettent une limite en b,
alors le troisième aussi et on a alors

\int  a^b~f(t)g'(t) dt
= lim x\rightarrow~b~(f(x)g(x)) - f(a)g(a)
-\int  a^b~f'(t)g(t) dt

que l'on écrit encore

\int  a^b~f(t)g'(t) dt =
\left {[}f(t)g(t)\right {]}
a^b -\int  a^b~f'(t)g(t)
dt

Sinon, on conserve les intégrales partielles de a à x \\\\jmathmathmathmathusqu'à pouvoir
lever les indéterminations éventuelles.

Remarque~9.8.5 Le lecteur devra faire preuve d'une grande prudence~: une
intégration par parties peut facilement faire passer d'une intégrale
convergente à une intégrale divergente, en particulier avec des
fonctions comme le logarithme.

\paragraph{9.8.4 Intégrales et séries~: intégration par paquets}

Théorème~9.8.6 Soit f : {[}a,b{[}\rightarrow~ E réglée, (bn) une suite
strictement croissante de {[}a,b{[} de limite b. On pose pour n ≥ 1,
xn =\int ~
bn-1^bnf(t) dt. Alors

\begin{itemize}
\itemsep1pt\parskip0pt\parsep0pt
\item
  (i) si l'intégrale \int ~
  a^bf(t) dt converge, la série
  \\sum ~
  n≥1xn converge
\item
  (ii) la réciproque est exacte dans les deux cas suivants

  \begin{itemize}
  \itemsep1pt\parskip0pt\parsep0pt
  \item
    (a) la suite (bn - bn-1) est bornée et
    limt\rightarrow~b~f(t) = 0
  \item
    (b) E = \mathbb{R}~ et la fonction f est de signe constant sur chaque
    intervalle {[}bn-1,bn{]}.
  \end{itemize}
\end{itemize}

Démonstration (i) On a
\\sum ~
n=1^Nxn =\int ~
b0^bNf(t) dt = F(bN) avec
F(x) =\int  b0^x~f(t)
dt. Puisque l'intégrale converge, la fonction F a une limite au point
b~; le théorème de composition des limites assure alors l'existence de
limN\rightarrow~+\infty~F(bN~), donc la
convergence de la série~; on a d'ailleurs
\\sum ~
n=1^+\infty~xn =\int ~
b0^bf(t) dt.

(ii)(a) Soit x \textgreater{} b0 et soit p l'unique entier tel
que bp-1 \leq x \textless{} bp. On a alors

\sum n=1^px n~
-\\int  ~
b0^xf(t) dt =
\\int  ~
x^bp f(t) dt

Soit alors \epsilon \textgreater{} 0, K \textgreater{} 0 tel que
\forall~n \in \mathbb{N}~, bn - bn-1~ \leq K, c \in
{[}a,b{[} tel que t \in {[}c,b{[}\rigtharrow~\\textbar{}
f(t)\\textbar{} \textless{} \epsilon \over 2K
. Pour x \textgreater{} c, on a alors
\\textbar{}\\\sum
 n=1^pxn -\int ~
b0^xf(t) dt\\textbar{}
\leq\int ~
x^bp\\textbar{}f(t)\\textbar{}
dt \leq K \epsilon \over 2K = \epsilon \over 2 .
Soit S = \\sum ~
n=1^+\infty~xn et N \in \mathbb{N}~ tel que n \rigtharrow~ N
\rigtharrow~\\textbar{} S
-\\sum ~
k=1^nxk\\textbar{} \textless{}
\epsilon \over 2 . Pour x ≥ xN, on a p ≥ N et donc
pour x \textgreater{} max(c,bN~) on a

\begin{align*} \\textbar{}S
-\int  b0^x~f(t)
dt\\textbar{}& \leq& \\textbar{}S
-\sum k=1^px~
k\\textbar{} +\\textbar{}
\sum n=1^px n~
-\\int  ~
b0^xf(t) dt\\textbar{}\%&
\\ & \textless{}& \epsilon
\over 2 + \epsilon \over 2 = \epsilon \%&
\\ \end{align*}

ce qui montre la convergence de l'intégrale.

(ii)(b) La démonstration est similaire. Mais on écrit, en utilisant le
fait que f est de signe constant sur {[}bp-1,bp{]}

\begin{align*} \textbar{}\\sum
n=1^px n
-\\int  ~
b0^xf(t) dt\textbar{}& =&
\textbar{}\int  x^bp~
f(t) dt\textbar{} =\int ~
x^bp \textbar{}f(t)\textbar{} dt \%&
\\ & \leq& \int ~
bp-1^bp \textbar{}f(t)\textbar{} dt =
\textbar{}\int ~
bp-1^bp f(t) dt\textbar{}\%&
\\ & =& \textbar{}xp\textbar{}
\%& \\ \end{align*}

Puisque la série converge, limxn~ = 0
et donc, il existe M tel que n ≥ M \rigtharrow~\textbar{}xn\textbar{}
\textless{} \epsilon \over 2 . Soit N \in \mathbb{N}~ tel que n ≥ N
\rigtharrow~\textbar{}S -\\sum ~
k=1^nxk\textbar{} \textless{} \epsilon
\over 2 . Pour x ≥\
max(xN,xM), on a p ≥\
max(N,M) et donc

\textbar{}S -\int ~
b0^xf(t) dt\textbar{}\leq\textbar{}S
-\sum k=1^px~
k\textbar{} + \textbar{}\\sum
n=1^px n
-\\int  ~
b0^xf(t) dt\textbar{} \textless{} \epsilon
\over 2 + \epsilon \over 2 = \epsilon

ce qui montre la convergence de l'intégrale.

{[}
{[}
{[}
{[}
