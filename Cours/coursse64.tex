

\subsubsection{11.2 Somme d'une série entière}

\paragraph{11.2.1 Etude sur le disque ouvert de convergence (domaine
		complexe)}
~
%
		\begin{prop}[continuité de la somme] 
Soit
$\sum a_n z^n$ une série entière à coefficients dans E, de
rayon de convergence $R \textgreater{} 0$. Alors la fonction $S :
z\mapsto \sum_{n=0}^{+\infty} a_n z^n$ est continue sur le
disque $D(0,R) = \{ z \in K \∣ z \leq R \} $.
\end{prop}
%
%Démonstration On a vu en effet que la série convergeait normalement sur
%D'(0,r) pour r \leq R, donc S est continue sur un tel D'(0,r) et
%donc finalement sur D(0,R).
%
%Corollaire~11.2.2 (principe des zéros isolés). Soit
%\sum ~
%anz^n une série entière non nulle à coefficients
%dans E, de rayon de convergence R \textgreater{} 0. Alors, il existe \eta
%\textgreater{} 0 tel que la fonction S :
%z\mapsto~\\sum
% n=0^+\infty~anz^n ne s'annule pas sur
%D(0,\eta) \diagdown\0\.
%
%Démonstration Soit en effet p =\
				%min\k \in
				%\mathbb{N}~∣ak\mathrel\neq~0\.
				%On a alors S(z) =\ \sum
				% n=p^+\infty~anz^n =
				%z^p \sum ~
				%n=p^+\infty~anz^n-p =
				%z^p \sum ~
				%n=0^+\infty~an+pz^n. Mais la série entière
				%\sum ~
				%nan+pz^n a même rayon de convergence que la
				%série entière \sum ~
				%anz^n (facile) et sa somme définit donc une
				%fonction s continue sur D(0,R) avec s(0) =
				%ap\neq~0. Donc il existe \eta
				%\textgreater{} 0 tel que  z  \leq \eta \rigtharrow~
				%s(z)\neq~0. Mais alors, pour 0 \leq
				% z  \leq \eta, on a S(z) =
				%z^ps(z)\neq~0, ce que l'on voulait
				%démontrer.
				%
				%Corollaire~11.2.3 (principe d'identification). Soit
				%\sum ~
				%anz^n et et
				%\sum ~
				%bnz^n deux séries entières à coefficients dans E,
				%de rayons de convergence non nuls, de sommes S1 et
				%S2. Alors on a équivalence de (i) \forall~~n
				%\in \mathbb{N}~, an = bn (ii) il existe \eta \textgreater{} 0 tel
				%que \forall~z \in D(0,\eta), S1~(z) =
				%S2(z) (iii) il existe une suite (zn) de K formée
				%d'éléments distincts telle que limzn~
				%= 0 et \forall~n \in \mathbb{N}~, S1(zn~) =
%S2(zn)
	%
	%Démonstration Il suffit d'appliquer le principe des zéros isolés à la
	%série entière \sum ~
	%(an - bn)z^n dont la somme est
	%S1 - S2 dans le disque
	%D(0,min(R1,R2~)).
	%
	%Remarque~11.2.1 Le corollaire précédent qui garantit l'unicité du
	%développement en série entière d'une fonction est très souvent utilisé~;
	%il permet en particulier de travailler par identification. Il laisse
	%penser qu'il doit être possible de récupérer les valeurs des
	%coefficients an à partir de la somme S de la série. En fait,
	%dans une première approche, les techniques sont très différentes suivant
	%que le corps de base est \mathbb{C} ou \mathbb{R}~.
	%
	%Dans le cadre complexe, on a le théorème suivant qui relie les
	%coefficients du développement en série entière à la somme de la fonction
	%
	%Théorème~11.2.4 (formules de Cauchy). Soit E un \mathbb{C}-espace vectoriel normé
	%complet, \sum ~
	%anz^n une série entière à coefficients dans E, de
	%rayon de convergence R \textgreater{} 0, de somme S. Alors, pour tout r
	%\leq R, on a
	%
	%\forall~n \in \mathbb{N}~, an~ = 1
	%\over 2\pi~r^n \int ~
	%0^2\pi~S(re^i\theta)e^-in\theta d\theta
	%
	%Démonstration Puisque r \leq R, la série
	%\sum ~
	%\ an\ r^n
	%est convergente.
	%
	%On a S(re^i\theta)e^-in\theta =\
					%\sum ~
					%p=0^+\infty~apr^pe^i(p-n)\theta.
					%Mais l'inégalité
					%\ anr^pe^i(p-n)\theta\ 
					%\leq\ 
					%ap\ r^p montre que la série
					%converge normalement par rapport à \theta. On en déduit que
					%
					%\begin{align*} \int ~
					%0^2\pi~S(re^i\theta)e^-in\theta d\theta& =&
					%\int  0^2\pi~~
					%\sum p=0^+\infty~a~
					%pr^pe^i(p-n)\theta d\theta\%&
					%\\ & =& \sum
					%p=0^+\infty~a pr^p
					%\\int  ~
					%0^2\pi~e^i(p-n)\theta d\theta\%&
					%\\ & =& 2\pi~anr^n
					%\%& \\ \end{align*}
					%
					%car \int  0^2\pi~e^ik\theta~
					%d\theta = \left \ \cases 0
					%&si k\neq~0 \cr 2\pi~&si k = 0 
					%\right .. On obtient donc la formule ci dessus.
					%
					%Théorème~11.2.5 Soit \\sum
					% anzn une série entière de rayon de convergence
					%R et de somme S(z). Soit z0 \in \mathbb{C} tel que
					% z0  \leq R. Alors la fonction
					%S(z0 + u) est développable en série entière de u dans le
					%disque ouvert  u  \leq R
					%- z0 , ce que l'on traduit par~: la somme
					%d'une série entière est analytique dans son disque ouvert de
					%convergence.
					%
					%Démonstration Puisque  z0 +
					%u \leq z0  +  u 
					%\leq R, on peut écrire
					%
					%\begin{align*} S(z0 + u)& =&
					%\sum n=0^+\infty~a~
					%n(z0 + u)^n \%& \\
						%& =& \sum
						%n=0^+\infty~\left (\sum
								%m=0^nC n^ma
								%nz0^n-mu^m\right )\%&
						%\\ \end{align*}
						%
						%On considère alors la famille (xm,n)m,n\in\mathbb{N}~ définie
						%par
						%
						% xm,n = \left \
							 %\cases
							 %Cn^manz0^n-mu^m&si
							 %m \leq n \cr 0 &si m \textgreater{} n 
							 %\right .
							 %
							 %On a
							 %
							 %\sum m=0^+\infty~ x~
							 %m,n  = \sum
							 %m=0^n C n^ma
							 %nz0^n-mu^m  =  a
							 %n ( z0  +
									 % u )^n
							 %
							 %qui est une série convergente puisque la série
							 %\sum ~
							 %n an ( z0 
									 %+  u )^n converge (une série entière converge
										 %absolument dans son disque ouvert de convergence). Ceci montre que la
									 %famille (xm,n)m,n\in\mathbb{N}~ est sommable. On peut donc
									 %appliquer d'interversion des sommations et on a
									 %
									 %\begin{align*} S(z0 + u)& =&
									 %\sum n=0^+\infty~~\left
									 %(\sum m=0^+\infty~x~
											 %m,n\right ) = \sum
									 %m=0^+\infty~\left (\sum
											 %n=0^+\infty~x m,n\right )\%&
									 %\\ & =& \sum
									 %m=0^+\infty~u^m\left
									 %(\sum n=m^+\infty~C~
											 %n^ma nz0^n-m\right )
									 %
									 %\%& \\ \end{align*}
									 %
									 %ce qui montre le résultat.
									 %
									 %
									 %\paragraph{11.2.2 Etude sur le disque ouvert de convergence (domaine
											 %réel)}
									 %
									 %Avant de regarder le cas réel, nous allons démontrer le lemme suivant
									 %
									 %Lemme~11.2.6 Soit \sum ~
									 %anz^n une série entière à coefficients dans le
									 %K-espace vectoriel normé E, de rayon de convergence R. Soit F \in K(X) une
									 %fraction rationnelle non nulle et N \in \mathbb{N}~ tel que F n'ait pas de pôle
									 %entier supérieur à N. Alors la série entière
									 %\sum ~
									 %F(n)anz^n a encore pour rayon de convergence R.
									 %
									 %Démonstration Soit z \in K tel que  z  \leq R et
									 %soit r tel que  z  \leq r \leq R. La
									 %suite
									 %\ an\ r^n
									 %est donc bornée, par exemple ma\\\\jmathmathmathmathorée par M. On a alors, pour n ≥ N,
									 %\ F(n)anz^n\ 
									 %\leq M F(n) \left (
											 % z  \over r \right
											 %)^n ∼ \lambda~n^d\left (
												 % z  \over r \right
												 %)^n (où d est le degré de la fraction rationnelle, différence
													 %entre le degré de son numérateur et celui de son dénominateur, si bien
													 %que  F(t) ∼ \lambda~t^d au voisinage de + \infty~) qui
												 %tend vers 0 quand n tend vers + \infty~. On a donc R' ≥ R. Mais on a aussi
												 %an = 1 \over F (n)\left
												 %(F(n)an\right ) si bien que les suites
												 %(an) et (F(n)an) \\\\jmathmathmathmathouent ici un rôle parfaitement
												 %symétrique. On a donc aussi R ≥ R', soit R = R'.
												 %
												 %On va en déduire le théorème suivant
												 %
												 %Théorème~11.2.7 Soit E un \mathbb{R}~-espace vectoriel normé complet,
												 %\sum ~
												 %ant^n une série entière à coefficients dans E, de
												 %rayon de convergence R \textgreater{} 0, de somme S. Alors la fonction S
												 %est de classe C^\infty~ sur {]} - R,R{[} et
												 %\forall~p \in \mathbb{N}~, \\forall~~t \in{]} -
												 %R,R{[}
												 %
												 %\begin{align*} S^(p)(t)& =&
												 %\sum n=p^+\infty~~n(n -
														 %1)\ldots(n - p + 1)a~
												 %pt^n-p\%& \\ & =&
												 %\sum n=0^+\infty~~ (n + p)!
												 %\over n! an+pt^n \%&
												 %\\ \end{align*}
												 %
												 %Démonstration Les deux formules se déduisent l'une de l'autre par un
												 %changement d'indice (le changement de n - p en n). Il suffit donc de
												 %montrer la première. Mais le lemme précédent assure que la série entière
												 %\sum  n≥p~n(n
														 %- 1)\\ldots~(n - p +
															 %1)apt^n a même rayon de convergence R que la série
														 %de départ. Il en est donc de même de la série entière
														 %\sum  n≥p~n(n
																 %- 1)\\ldots~(n - p +
																	 %1)apt^n-p et cette série converge donc normalement
																 %sur {[}-r,r{]} pour r \leq R. Montrons donc le résultat par
																 %récurrence sur p. Pour p = 0, il n'y a rien à montrer. Supposons le
																 %résultat vrai pour p avec \forall~~t \in{]} - R,R{[},
																 %S^(p)(t) =\
																	    %\sum  n=p^+\infty~~n(n -
																			    %1)\\ldots~(n - p +
																				    %1)apt^n-p. La série dérivée
																			    %\sum ~
																			    %n≥p+1n(n -
																					    %1)\\ldots~(n - p +
																						    %1)(n - p)apt^n-p-1 converge normalement sur
																					    %{[}-r,+r{]} et le théorème de dérivation des séries de fonctions nous
																					    %garantit que S^(p) est de classe \mathcal{C}^1 (donc S de
																							    %classe C^p+1) sur {[}-r,r{]} avec
%\forall~t \in {[}-r,r{]}, S^(p)~(t)
	%= \sum ~
	%n=p^+\infty~n(n -
			%1)\\ldots~(n - p +
				%1)(n - p)apt^n-p-1~; mais comme r est quelconque (r
					%\leq R), S est de classe C^p+1 sur {]} - R,R{[} et la
				%formule ci-dessus y reste valable, ce qui achève la récurrence.
				%
				%Corollaire~11.2.8 Soit E un \mathbb{R}~-espace vectoriel normé complet,
				%\sum ~
				%ant^n une série entière à coefficients dans E, de
				%rayon de convergence R \textgreater{} 0, de somme S. Alors
	%
%\forall~n \in \mathbb{N}~, an = S^(n)~(0)
	%\over n!
	%
	%Démonstration Faire t = 0 dans la formule précédente.
	%
	%Remarque~11.2.2 Les coefficients an sont donc les mêmes que
	%ceux qui apparaissent dans un développement limité en 0 de la fonction
	%S.
	%
	%Le même argument de convergence normale sur {[}-r,r{]} \subset~{]} - R,R{[}
	%montrera le théorème suivant
	%
	%Théorème~11.2.9 Soit E un \mathbb{R}~-espace vectoriel normé complet,
	%\sum ~
	%ant^n une série entière à coefficients dans E, de
	%rayon de convergence R \textgreater{} 0, de somme S. Alors
	%
	%\forall~~t \in{]} - R,R{[}, \\int
	% 0^tS(u) du = \sum
	%n=0^+\infty~a n t^n+1
	%\over n + 1
	%
	%
	%\paragraph{11.2.3 Etude sur le cercle de convergence}
	%
	%On a vu qu'en un point du cercle de convergence, la série pouvait aussi
	%bien diverger que converger. Si la série converge, la question de la
	%continuité de la somme en ce point se pose immédiatement. En fait, on
	%peut montrer que sur \mathbb{C}, la somme peut très bien être discontinue en un
	%tel point, mais qu'il s'agit en fait d'une discontinuité tangentielle~:
	%il se peut que S(z) ne tende pas vers S(z0) quand z tend vers
	%z0 tangentiellement au cercle de convergence. Pour nous il
	%nous suffira de savoir que S(z) tend vers S(z0) quand z tend
	%vers z0 suivant un rayon, ce que garantit le théorème suivant
	%
	%Théorème~11.2.10 (Abel) Soit
	%\sum ~
	%anz^n une série entière à coefficients dans E, de
	%rayon de convergence R \textgreater{} 0 et S :
	%z\mapsto~\\sum
	% n=0^+\infty~anz^n continue sur le
	%disque D(0,R) = \z \in
	%K∣ z  \leq
	%R\. Soit z0 \in K tel que
	% z0  = R et la série
	%\sum ~
	%anz0^n converge. Alors
	%
	%\sum n=0^+\infty~a~
	%nz0^n = lim
%t\rightarrow~1^-S(tz0)
	%
	%Démonstration On considère la série de fonctions
	%\sum ~
	%anz0^nt^n, qui converge sur
	%{[}0,1{]}. Nous allons démontrer sa convergence uniforme~; ceci
	%garantira la continuité de sa somme au point 1, ce qui n'est autre
	%l'assertion à démontrer.
	%
	%Premier cas~: la série \\sum
	% anz0^n converge absolument. Alors on a
	%\forall~~t \in {[}0,1{]},
	%\ anz0^nt^n\ 
	%\leq\ 
	%anz0^n\ , série
	%convergente indépendante de t. Donc la série converge normalement.
	%
	%Deuxième cas~: le critère des séries alternées s'applique à la série
	%\sum ~
	%anz0^n, autrement dit
	%anz0^n = (-1)^nbn avec
	%(bn) qui tend vers 0 en décroissant. Alors
	%anz0^nt^n =
	%(-1)^nbnt^n, avec
	%t\mapsto~bnt^n qui tend
	%uniformément vers 0 en décroissant. Le critère de convergence uniforme
	%des séries alternées garantit la convergence uniforme de la série.
	%
	%Cas général~: nous allons montrer que la série de fonctions vérifie le
	%critère de Cauchy uniforme. Pour cela posons Rn
	%= \sum ~
	%k=n^+\infty~akz0^k. On a alors
	%
	%\begin{align*} \sum
	%n=p^qa nz0^nt^n&
	%=& \sum n=p^q(R n~ -
			%Rn+1)t^n = \sum
	%n=p^qR nt^n
	%-\sum n=p^qR~
	%n+1t^n\%& \\ & =&
	%\sum n=p^qR~
	%nt^n -\sum
	%n=p+1^q+1R nt^n-1 \%&
	%\\ & =& Rpt^p -
	%R q+1t^q -\sum
%n=p+1^qR n(t^n-1 - t^n)
	%\%& \\ \end{align*}
	%
	%On a limRn~ = 0 (reste d'une série
			%convergente). Soit \epsilon \textgreater{} 0~; il existe N \in \mathbb{N}~ (indépendant de
				%t) tel que n ≥ N \rigtharrow~\ 
			%Rn\  \leq \epsilon
			%\over 2 . Alors, en tenant compte de t^p ≥
			%0, t^q ≥ 0 et t^n-1 - t^n ≥ 0, on a
			%\forall~~t \in {[}0,1{]},
			%
			%\sum_{n=p}^{q}a_nz_0^n t^n\  \leq \epsilon
			%\over 2 (t^p + t^q +
					%\sum n=p+1^q(t^n-1~ -
						%t^n)) = 2t^p \epsilon \over 2 \leq \epsilon
			%
			%ce qui montre que la série vérifie le critère de Cauchy uniforme, donc
			%est uniformément convergente.
			%
			%Remarque~11.2.3 Une des premières utilités du théorème précédent est de
			%calculer la somme de certaines séries numériques du type
			%\sum ~
			%anz0^n~; il arrive en effet fréquemment que
			%la somme de la série entière
			%\sum ~
			%anz^n soit facile à calculer pour
			% z  \leq R (par exemple par dérivation ou par
					%résolution d'une certaine équation différentielle). Il suffit alors de
			%passer à la limite pour calculer la somme de la série.
			%
			%\begin{example}
			%
			%On cherche à calculer la somme de la série alternée
			%\sum ~
	%n=1^+\infty~ (-1)^n-1 \over n .
%Pour  t  \leq 1, on pose f(t)
	%= \sum ~
	%n=1^+\infty~ (-1)^n-1 \over n
	%t^n~; on sait que f est C^\infty~ sur {]} - 1,1{[} et
	%que f'(t) = \sum ~
	%n=1^+\infty~(-1)^n-1t^n-1 = 1
%\over 1+t . Comme f(0) = 0, on a f(t)
	%= log~ (1 + t). Le théorème précédent assure
	%que \sum ~
	%n=1^+\infty~ (-1)^n-1 \over n
	%=\
	  %limt\rightarrow~1^-log~ (1 + t)
	  %= log~ 2.
	  %\end{example}
	  %
	  %Suivant le même principe, le lecteur
	  %montrera que \sum ~
	  %n=0^+\infty~ (-1)^n \over 2n+1
	  %= \mathrmarctg~ 1 = \pi~
	  %\over 4 , en introduisant la série entière
	  %\sum ~
	  %n=0^+\infty~ (-1)^n \over 2n+1
	  %t^2n+1.
	  %
