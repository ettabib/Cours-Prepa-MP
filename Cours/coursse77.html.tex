\textbf{Warning: 
requires JavaScript to process the mathematics on this page.\\ If your
browser supports JavaScript, be sure it is enabled.}

\begin{center}\rule{3in}{0.4pt}\end{center}

{[}
{[}{]}
{[}

\subsubsection{14.1 Introduction~: transformée de Fourier sur les
groupes abéliens finis}

Ce paragraphe sert simplement d'introduction à la suite du chapitre.
Lors d'une première lecture il peut être sauté sans inconvénient.

\paragraph{14.1.1 Caractères des groupes abéliens finis}

Définition~14.1.1 Soit (G,.) un groupe abélien fini. On appelle
caractère de G tout morphisme de groupe \chi de G dans (\mathbb{C}^∗,.).
On note \hatG l'ensemble des caractères de G.

Remarque~14.1.1 On vérifie immédiatement que \hatG
est lui même muni d'une structure de groupes en posant (\chi\chi')(x) =
\chi(x)\chi'(x).

Proposition~14.1.1 Soit \chi \in\hat G. Alors
\forall~~x \in G, \textbar{}\chi(x)\textbar{} = 1.

Démonstration Puisque G est un groupe fini, tout élément est d'ordre
fini, et donc il existe n \in \mathbb{N}~ tel que x^n = e. On a donc 1 =
\chi(e) = \chi(x^n) = \chi(x)^n ce qui montre que \chi(x) est
une racine de l'unité donc de module 1.

Proposition~14.1.2 \hatG est un groupe abélien fini.

Démonstration On sait que \forall~~x \in G,
x^\textbar{}G\textbar{} = e. Le même raisonnement que ci
dessus monte que \chi(x) est une racine \textbar{}G\textbar{}-ième de
l'unité. Donc \hatG est un sous-ensemble de
l'ensemble des applications de G dans le groupe fini
\Gamma\textbar{}G\textbar{} des racines \textbar{}G\textbar{}-ièmes
de l'unité, donc il est fini.

Lemme~14.1.3 Soit G un groupe abélien fini et H un sous-groupe de G.
Soit \psi un caractère de H. Alors il existe un caractère \chi de G dont la
restriction à H est \psi.

Démonstration Pour des raisons de cardinal, il existe un sous-groupe K
maximal auquel \psi admet un prolongement \phi \in\hat K.
Nous allons montrer par l'absurde que K = G, ce qui démontrera le lemme.
Supposons donc que K\neq~G et soit x \in G \diagdown K.
L'ensemble des n \in ℤ tel que x^n \in K est un sous-groupe de ℤ,
donc de la forme dℤ pour un d \textgreater{} 0 (car
(x^\textbar{}G\textbar{} = e \in K) et
d\neq~1 (car x\mathrel∉K).
Soit \omega \in \mathbb{C} tel que \omega^d = \phi(x^d). Soit K' =
\x^nk∣n \in ℤ, k \in
K\ le sous-groupe de G engendré par K et x et
prolongeons \phi à K' en posant \phi'(x^nk) = \omega^n\phi(k).
Montrons tout d'abord que \phi' est bien définie. Si l'on a x^nk
= x^mk', on a x^n-m = k'k^-1 \in K et
donc d divise n - m. Posons n - m = dq si bien que x^dq =
k'k^-1~; on a alors \phi(k')\phi(k)^-1 =
\phi(x^dq) = \phi(x^d)^q =
(\omega^d)^q = \omega^dq = \omega^n-m, si
bien que \omega^n\phi(k) = \omega^m\phi(k') ce qui montre que \phi'
est bien définie. Il est élémentaire de vérifier que \phi' est encore un
morphisme de groupes et il est clair qu'il prolonge \phi et donc qu'il
prolonge \psi. Mais ceci contredit alors la maximalité de K. On a donc K =
G et donc \psi se prolonge à G tout entier.

Théorème~14.1.4 Soit x \in G, x\neq~e. Alors il
existe \chi \in\hat G tel que
\chi(x)\neq~1.

Démonstration Soit d l'ordre de x, \omega = exp~ (
2i\pi~ \over d ), H le sous-groupe engendré par x.
L'application x^n\mapsto~\omega^n
est bien définie (car si x^n = x^p, alors d divise
n - p et donc \omega^n = \omega^p) et c'est un caractère de
H (facile). Donc il existe \chi \in\hat G qui prolonge \psi.
On a en particulier \chi(x) = \omega\neq~1.

Définition~14.1.2 Si f est une application de G dans \mathbb{C}, on notera
\int  G~f =\
\sum  x\inG~f(x). En posant alors
(f∣g) = 1 \over
\textbar{}G\textbar{} \int ~
G\overlinefg, on munit l'espace E des
applications de G dans \mathbb{C} d'une structure d'espace hermitien.

Proposition~14.1.5

\begin{itemize}
\itemsep1pt\parskip0pt\parsep0pt
\item
  (i) Soit \chi \in\hat G. Alors
  \int  G~\chi = \left
  \\cases 0 &si
  \chi\neq~1 \cr
  \textbar{}G\textbar{}&si \chi = 1  \right ..
\item
  (ii) Soit x \in G. Alors
  \\sum ~
  \chi\in\hatG\chi(x) = \left
  \\cases 0 &si
  x\neq~e \cr
  \textbar{}\hatG\textbar{}&si x = e 
  \right .
\end{itemize}

Démonstration (i) Supposons que \chi\neq~1 et soit x
\in G tel que \chi(x)\neq~1. On a alors

\chi(x)\sum g\inG~\chi(g) =
\sum g\inG~\chi(xg) =
\sum g\inG~\chi(g)

puisque g\mapsto~xg est une bi\\\\jmathmathmathmathection de G sur lui
même. Comme \chi(x)\neq~1, on en déduit que
\\sum  g\inG~\chi(g)
= 0. Si par contre, \chi = 1, on a
\\sum  g\inG~\chi(g)
= \textbar{}G\textbar{}.

(ii) Si x\neq~e, soit \phi \in\hat
G tel que \phi(x)\neq~1. On a alors

\phi(x)\sum \chi\in\hatG~\chi(x)
= \\sum
\chi\in\hatG(\phi\chi)(x) =
\sum \chi\in\hatG~\chi(x)

puisque \chi\mapsto~\phi\chi est une bi\\\\jmathmathmathmathection de
\hatG sur lui même. Comme
\phi(x)\neq~1, on a
\\sum ~
\chi\in\hatG\chi(x) = 0. Si par contre, x = e, on a
pour tout \chi, \chi(x) = 1 et donc
\\sum ~
\chi\in\hatG\chi(x) =
\textbar{}\hatG\textbar{}.

Corollaire~14.1.6 \textbar{}G\textbar{} =
\textbar{}\hatG\textbar{}.

Démonstration Soit S =\
\sum ~
\chi\in\hatG\
\sum  x\inG~\chi(x). On a (en utilisant le
symbole de Kronecker \deltaa^b = \left
\ \cases 1&si a = b
\cr 0&si a\neq~b\\ 
\right . et en notant 1 le caractère constant égal à 1) S
= \\sum ~
\chi\in\hatG\textbar{}G\textbar{}\delta\chi^1
= \textbar{}G\textbar{}. Mais on a aussi S =\
\sum  x\inG~\
\sum  \chi\in\hatG~\chi(x)
= \\sum ~
x\inG\textbar{}\hatG\textbar{}\deltax^e
= \textbar{}\hatG\textbar{} d'où le résultat.

Corollaire~14.1.7 \hatG est une famille orthonormée
de E.

Démonstration On a (\chi∣\phi) = 1
\over \textbar{}G\textbar{} \\int
 G\overline\chi\phi = 0 si
\overline\chi\phi\neq~1, soit
\chi\neq~\phi (puisque \overline\chi(g)
= 1 \over \chi(g) comme nombre complexe de module 1). Si
par contre, \chi = \phi, on a \overline\chi\phi =
\textbar{}\chi\textbar{}^2 = 1 et donc
(\chi∣\phi) = 1.

\paragraph{14.1.2 Transformée de Fourier sur un groupe abélien fini}

Définition~14.1.3 Soit G un groupe abélien fini et soit f : G \rightarrow~ \mathbb{C}. On
définit la transformée de Fourier de f comme étant l'application
\hatf :\hat G \rightarrow~ \mathbb{C} définie par

\forall~\chi \in\hat G~,
\hatf(\chi) = (\chi∣f) =
1\over
\textbar{}G\textbar{}\int ~
Gf\overline\chi

Théorème~14.1.8 (cf Dirichlet). Soit f : G \rightarrow~ \mathbb{C}. Alors,

\forall~~x \in G, f(x) = \\sum
\chi\in\hatG\hatf(\chi)\chi(x)

Démonstration On a

\begin{align*} \\sum
\chi\in\hatG\hatf(\chi)\chi(x)&
=& 1 \over \textbar{}G\textbar{}
\sum \chi\in\hatG~
\\sum
y\inGf(y)\overline\chi(y)\chi(x)\%&
\\ & =& 1 \over
\textbar{}G\textbar{} \\sum
y\inGf(y)\\sum
\chi\in\hatG\chi(xy^-1) \%&
\\ \end{align*}

puisque \overline\chi(y) = 1 \over
\chi(y) = \chi(y^-1). Mais, d'après un résultat précédent
\\sum ~
\chi\in\hatG\chi(xy^-1) = 0 si
xy^-1\neq~e, soit
y\neq~x et
\\sum ~
\chi\in\hatG\chi(xy^-1) =
\textbar{}\hatG\textbar{} = \textbar{}G\textbar{} si
y = x. D'où ne persiste dans la somme que le terme pour y = x et donc
\\sum ~
\chi\in\hatG\hatf(\chi)\chi(x) =
f(x).

Corollaire~14.1.9 \hatG est une base orthonormée de
E.

Démonstration Le théorème précédent montre que c'est une famille
génératrice et on a vu précédemment que c'est une famille orthonormée,
donc libre.

Théorème~14.1.10 (cf Parseval-Plancherel). Soit f : G \rightarrow~ \mathbb{C}. Alors

\\textbar{}f\\textbar{}^2 =
(f∣f) = \\sum
\chi\in\hatG\textbar{}\hatf(\chi)\textbar{}^2

Démonstration On a vu précédemment que les \hatf(\chi)
sont les coordonnées de f dans la base orthonormée
\hatG de E et le carré de la norme de f dans E est la
somme des modules des carrés des coordonnées dans une base orthonormée.

{[}
{[}
