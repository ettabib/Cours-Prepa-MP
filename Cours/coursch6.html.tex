\textbf{Warning: 
requires JavaScript to process the mathematics on this page.\\ If your
browser supports JavaScript, be sure it is enabled.}

\begin{center}\rule{3in}{0.4pt}\end{center}

{[}
{[}
{[}{]}
{[}

\subsection{Chapitre~5\\Espaces vectoriels normés}

Tout au long du chapitre, K désigne le corps \mathbb{R}~ ou le corps \mathbb{C},
\textbar{}\lambda~\textbar{} désignant suivant le cas la valeur absolue ou le
module du scalaire \lambda~.

~5.1 {Notion d'espace vectoriel
normé} \\ ~~5.1.1 {Norme et
distance associée} \\ ~~5.1.2
 \\ ~~5.1.3
{Continuité des opérations
algébriques} \\ ~5.2 {Applications
linéaires continues} \\ ~~5.2.1
{Caractérisations et normes des
applications linéaires continues} \\ ~~5.2.2
{L'espace vectoriel normé des
applications linéaires continues de E dans F} \\ ~~5.2.3
 \\
~~5.2.4 {Caractérisation des
applications bilinéaires continues} \\ ~5.3
{Espaces vectoriels normés de
dimensions finies} \\ ~~5.3.1
 \\
~~5.3.2 {Propriétés topologiques
et métriques des espaces vectoriels normés de dimension finie} \\
~~5.3.3 {Continuité des
applications linéaires} \\ ~5.4
{Compléments: le théorème de Baire
et ses conséquences} \\ ~~5.4.1
 \\ ~~5.4.2
 \\ ~5.5
{Compléments: convexité dans les
espaces vectoriels normés} \\ ~~5.5.1
 \\ ~~5.5.2

\\ ~~5.5.3 {Hahn-Banach (version
géométrique)} \\ ~~5.5.4
{L'enveloppe convexe: Carathéodory
et Krein Millman}

{[}
{[}
{[}
{[}
