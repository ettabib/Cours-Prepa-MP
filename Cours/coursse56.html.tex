\textbf{Warning: 
requires JavaScript to process the mathematics on this page.\\ If your
browser supports JavaScript, be sure it is enabled.}

\begin{center}\rule{3in}{0.4pt}\end{center}

{[}
{[}
{[}{]}
{[}

\subsubsection{9.7 Développements asymptotiques et analyse numérique}

\paragraph{9.7.1 La formule d'Euler-Mac Laurin}

Proposition~9.7.1 Il existe une unique famille de polynômes
Bn(X) (polynômes de Bernoulli) dans \mathbb{R}~{[}X{]} vérifiant les
relations

\begin{itemize}
\itemsep1pt\parskip0pt\parsep0pt
\item
  (i) B0(X) = 1, B1(X) = X - 1
  \over 2 ,
\item
  (ii) Bn'(X) = nBn-1(X) pour n ≥ 1
\item
  (iii) B2n+1(0) = B2n+1(1) = 0 pour n ≥ 1
\end{itemize}

Démonstration La relation (ii) définit Bn à une constante près
et la relation (iii) fixe les deux constantes d'intégration qui se sont
introduites pour le passage de B2n-1 à B2n+1.

Théorème~9.7.2

\begin{itemize}
\itemsep1pt\parskip0pt\parsep0pt
\item
  (i) Bn est un polynôme normalisé de degré n.
\item
  (ii) On a Bn(1 - X) = (-1)^nBn(X) et en
  particulier B2n+1( 1 \over 2 ) = 0,
  B2n(1) = B2n(0) (noté b2n)
\item
  (iii) Bn(X + 1) - Bn(X) = nX^n-1
\end{itemize}

Démonstration (i) est évident par récurrence à partir de
Bn'(X) = nBn-1(X). Pour démontrer (ii), il suffit de
démontrer que, si l'on pose Cn(X) =
(-1)^nBn(1 - X), la suite (Cn) vérifie
les mêmes relations que la suite (Bn(X)), ce qui est immédiat.
On montre (iii) par récurrence sur n. La relation est vérifiée pour n =
1 et si elle est vérifiée pour n - 1, soit P(X) = Bn(X + 1) -
Bn(X) - nX^n-1. On a P'(X) = n(Bn-1(X +
1) - Bn-1(X) - (n - 1)X^n-2) = 0 par l'hypothèse de
récurrence. Mais d'autre part P(0) = Bn(1) - Bn(0) =
0 (par définition si n est impair, d'après l'assertion précédente si n
est pair), donc P est le polynôme nul.

Théorème~9.7.3 (formule d'Euler-Mac Laurin). Soit f : {[}0,1{]} \rightarrow~ E de
classe C^2n+1. Alors

\begin{align*} \int ~
0^1f(t) dt& =& 1 \over 2 (f(1) +
f(0)) \%& \\ & \text
& -\\sum
k=1^n(f^(2k-1)(1) -
f^(2k-1)(0)) b2k \over (2k)!
\%& \\ & \text & -
1 \over (2n + 1)! \int ~
0^1f^(2n+1)(t)B 2n+1(t) dt \%&
\\ \end{align*}

Démonstration Par récurrence sur n. Pour n = 0, on écrit

\begin{align*} \int ~
0^1f(t) dt& =& \int ~
0^1f(t)B 0(t) dt = \left
{[}f(t)B1(t)\right {]}0^1
-\int  0^1f'(t)B 1~(t)
dt\%& \\ & =& 1 \over
2 (f(1) + f(0)) -\int ~
0^1f'(t)B 1(t) dt \%&
\\ \end{align*}

Si la formule est vérifiée pour n, deux intégrations par parties donnent

\begin{align*} 1 \over (2n + 1)!
\int ~
0^1f^(2n+1)(t)B 2n+1(t)
dt\quad && \%& \\ &
=& 1 \over (2n + 1)! \left
{[}f^(2n+1)(t) B2n+2(t) \over 2n +
2 \right {]}0^1 \%&
\\ & \text & - 1
\over (2n + 2)! \int ~
0^1f^(2n+2)(t)B 2n+2(t) dt\%&
\\ & =& b2n+2
\over (2n + 2)! (f^(2n+1)(1) -
f^(2n+1)(0)) \%& \\ &
\text & -\Biggl ( 1
\over (2n + 2)! \left
{[}f^(2n+2)(t) B2n+3(t) \over 2n +
3 \right {]}0^1 \%&
\\ & \text & - 1
\over (2n + 3)! \int ~
0^1f^(2n+3)(t)B 2n+3(t)
dt\Biggr )\%& \\ &
=& b2n+2 \over (2n + 2)!
(f^(2n+1)(1) - f^(2n+1)(0)) \%&
\\ & \text & + 1
\over (2n + 3)! \int ~
0^1f^(2n+3)(t)B 2n+3(t) dt\%&
\\ \end{align*}

en tenant compte de B2n+2(0) = B2n+2(1) =
b2n+2 et de B2n+3(0) = B2n+3(1) = 0.

Remarque~9.7.1 On peut montrer que \forall~~x \in
{[}0,1{]}, \textbar{}Bn(x)\textbar{}\leq 4e^2\pi~ n!
\over (2\pi~)^n ce qui permet d'avoir une
estimation du reste. Si f est à valeurs réelles, on peut obtenir une
autre estimation du reste en montrant par récurrence que les polynômes
Bn ont les variations suivantes

̲ ̲ ̲ ̲ ̲ ̲ ̲ ̲ ̲ ̲

x

0

1\diagup2

1 ̲ ̲ ̲ ̲ ̲ ̲ ̲ ̲ ̲ ̲

B4n(x)

b4n \textless{} 0

\nearrow

0

\nearrow

\textgreater{} 0

\searrow

0

\searrow

b4n \textless{} 0 ̲ ̲ ̲ ̲ ̲ ̲ ̲ ̲ ̲ ̲

B4n+1(x)

0

\searrow

\nearrow

0

\nearrow

\searrow

0 ̲ ̲ ̲ ̲ ̲ ̲ ̲ ̲ ̲ ̲

B4n+2(x)

b4n+2 \textgreater{} 0

\searrow

0

\searrow

\textless{} 0

\nearrow

0

\nearrow

b4n+2 \textgreater{} 0 ̲ ̲ ̲ ̲ ̲ ̲ ̲ ̲ ̲ ̲

B4n+3(x)

0

\nearrow

\searrow

0

\searrow

\nearrow

0 ̲ ̲ ̲ ̲ ̲ ̲ ̲ ̲ ̲ ̲

\includegraphics{cours7x.png}

Ceci montre que les polynômes B2p - b2p sont de
signe constant sur {[}0,1{]}. On peut donc utiliser la première formule
de la moyenne, ce qui nous donne

\begin{align*} \int ~
0^1f^(2n+2)(t)B 2n+2(t)
dt\quad && \%& \\ & =&
b2n+2\int ~
0^1f^(2n+2)(t) dt +\\int
 0^1f^(2n+2)(t)(B 2n+2(t) -
b2n+2) dt\%& \\ & =&
b2n+2(f^(2n+1)(1) - f^(2n+1)(0)) \%&
\\ & \text &
+f^(2n+2)(\xi~)\int ~
0^1(B 2n+2(t) - b2n+2) dt \%&
\\ & =&
b2n+2(f^(2n+1)(1) - f^(2n+1)(0)) -
b 2n+2f^(2n+2)(\xi~) \%&
\\ \end{align*}

car \int  0^1B2n+2~(t)
dt = \left {[} B2n+3(t)
\over 2n+3 \right {]}0^1
= 0. On obtient, en reprenant la démonstration du lemme, la formule sous
la forme

\begin{align*} \int ~
0^1f(t) dt& =& 1 \over 2 (f(1) +
f(0)) \%& \\ & \text
& -\\sum
k=1^n+1(f^(2k-1)(1) -
f^(2k-1)(0)) b2k \over (2k)!
\%& \\ & \text & +
b2n+2 \over (2n + 1)! f^(2n+2)(\xi~)
\%& \\ \end{align*}

Exemple~9.7.1 Appliquons cette formule à f(t) = 1 \over
t+p . On va obtenir

\begin{align*} log~ (p + 1)
- log (p)& =& 1 \over 2~
( 1 \over p + 1 + 1 \over p ) \%&
\\ & \text &
+\sum k=1^n+1~( 1
\over (p + 1)^2k - 1 \over
p^2k ) b2k \over 2k \%&
\\ & \text & + (2n
+ 2)b2n+2 \over \xi~p^2n+3 \%&
\\ \end{align*}

avec \xi~p \in {[}p,p + 1{]} et donc \xi~p ∼ p. En sommant
de p = 1 \\\\jmathmathmathmathusque N - 1, on obtient

\begin{align*} log~ N& =&
\sum p=1^N~ 1
\over p - 1 \over 2 - 1
\over 2N + \\sum
k=1^n+1( 1 \over N^2k -
1) b2k \over 2k \%&
\\ & \text & +(2n +
2)b2n+2 \sum p=1^N-1~
1 \over \xi~p^2n+3 \%&
\\ \end{align*}

et en utilisant \\sum ~
p=N^+\infty~ 1 \over
\xi~p^2n+3 = O( 1 \over
N^2n+2 ) on obtient, après amalgame de tous les termes ne
dépendant pas de N en une constante \gamma,

\begin{align*} \\sum
p=1^N 1 \over p & =&
log N + \gamma + 1 \over 2N~
-\sum k=1^n b2k~
\over 2kN^2k + O( 1 \over
N^2n+2 )\%& \\
\end{align*}

\paragraph{9.7.2 Calcul approché d'intégrales}

Méthode des trapèzes

Soit f : {[}a,b{]} \rightarrow~ \mathbb{R}~ de classe C^2 et p \in \mathbb{N}~^∗.
Pour k \in {[}0,p{]} posons ak = a + k b-a
\over p . On approche la fonction f par la fonction \phi :
{[}a,b{]} \rightarrow~ E qui vérifie \forall~~k \in {[}0,p{]},
\phi(ak) = f(ak) et qui est linéaire sur chaque
intervalle {[}ak-1,ak{]}. On a immédiatement
\int ~
ak-1^ak\phi = (a k -
ak-1) f(ak)+f(ak-1) \over
2 (aire d'un trapèze). D'où, \int ~
a^b\phi = b-a \over n
\left ( f(a) \over 2
+ \\sum ~
k=1^p-1f(ak) + f(b) \over 2
\right ) = Tp avec les notations du paragraphe
précédent. On prendra donc comme valeur approchée de I
=\int  a^b~f,
\overlineI =\int ~
a^b\phi = Tp.

Ma\\\\jmathmathmathmathoration de l'erreur~: on cherche à ma\\\\jmathmathmathmathorer \textbar{}I
-\overlineI\textbar{} =
\textbar{}\int  a^b~(f -
\phi)\textbar{}. Posons g = f - \phi et calculons à l'aide d'une intégration
par parties l'intégrale suivante (en remarquant que la restriction de g
à {[}ak-1,ak{]} est de classe C^2 avec
g'`= f'')

\begin{align*} \int ~
ak-1^ak f'`(t)(t -
ak-1)(ak - t) dt&& \%&
\\ & =& \int ~
ak-1^ak g'`(t)(t -
ak-1)(ak - t) dt \%&
\\ & =& \left
{[}g'(t)(t - ak-1)(ak - t)\right
{]}ak-1^ak 
+\int  ak-1^ak~
g'(t)(2t - ak-1 - ak) dt\%&
\\ & =& \int ~
ak-1^ak g'(t)(2t - ak-1 -
ak) dt \%& \\ & =&
\left {[}g(t)(2t - ak-1 -
ak)\right
{]}ak-1^ak  -
2\int  ak-1^ak~
g(t) dt \%& \\ & =&
-2\int ~
ak-1^ak g(t) dt \%&
\\ \end{align*}

puisque g(ak-1) = g(ak) = 0. On a donc

\begin{align*} \left
\textbar{}\int ~
ak-1^ak g(t) dt\right
\textbar{}& =& 1 \over 2 \left
\textbar{}\int ~
ak-1^ak f''(t)(t -
ak-1)(ak - t) dt\right \textbar{}\%&
\\ & \leq& M2
\over 2 \int ~
ak-1^ak (t -
ak-1)(ak - t) dt \%&
\\ & =& M2
\over 12 (ak - ak-1)^3
= M2(b - a)^3 \over
12p^3 \%& \\
\end{align*}

En sommant de k = 1 à p, on obtient

\textbar{}I -\overlineI\textbar{}\leq M2(b
- a)^3 \over 12p^2

Application de la formule d'Euler-Mac Laurin

Soit alors f : {[}a,b{]} \rightarrow~ E de classe C^2n+1 et p \in
\mathbb{N}~^∗. Pour k \in {[}0,p{]} posons ak = a + k b-a
\over p ~; appliquons la formule précédente à
t\mapsto~f(ak-1 + t b-a
\over p ). On obtient alors (avec le changement de
variable x = ak-1 + t b-a \over p )

\begin{align*} \int ~
ak-1^ak f(x) dx& =& b - a
\over n \int ~
0^1f(a k-1 + t b - a \over p
) dt\%& \\ & =& b - a
\over 2p (f(ak-1) + f(ak)) \%&
\\ & -\\sum
k=1^n (b - a)^2k \over
p^2k (f^(2k-1)(a k) -
f^(2k-1)(a k-1)) b2k
\over (2k)! &\%& \\ &
\text & + (b - a)^2n+2
\over p^2n+2(2n + 1)! \rhon,k \%&
\\ \end{align*}

avec \rhon,k =\int ~
0^1f^(2n+1)(ak-1 + t b-a
\over p )B2n+1(t) dt. Posons M2n+1
=\
supt\in{[}a,b{]}\\textbar{}f^(2n+1)(t)\\textbar{}.
On a alors
\\textbar{}\rhon,k\\textbar{} \leq
M2n+1\int ~
0^1\textbar{}B2n+1(t)\textbar{} dt. Sommons
alors les égalités ci dessus, en posant

Tp = b - a \over n \left (
f(a) \over 2 + \\sum
k=1^p-1f(a k) + f(b) \over
2 \right )

on obtient,

\begin{align*} \int ~
a^bf(x) dx& =& T p
-\sum k=1^n~ (b -
a)^2k \over p^2k
(f^(2k-1)(b) - f^(2k-1)(a)) b2k
\over (2k)! \%& \\ &
\text & + (b - a)^2n+2
\over p^2n+2(2n + 1)! Sn,p \%&
\\ \end{align*}

avec Sn,p =\
\sum  k=1^p\rhon,k~ et
donc \\textbar{}Sn,p\\textbar{}
\leq\\sum ~
k=1^p\\textbar{}\rhon,k\\textbar{}
\leq pM2n+1\int ~
0^1\textbar{}B2n+1(t)\textbar{} dt. On obtient
donc

Théorème~9.7.4 Soit f : {[}a,b{]} \rightarrow~ E de classe C^2n+1 et p \in
\mathbb{N}~^∗. Pour k \in {[}0,p{]} posons ak = a + k b-a
\over p . Soit Tp = b-a
\over n \left ( f(a)
\over 2 +\
\sum  k=1^p-1f(ak~) +
f(b) \over 2 \right ) et M2n+1
=\
supt\in{[}a,b{]}\\textbar{}f^(2n+1)(t)\\textbar{}.
Alors

\begin{align*} \int ~
a^bf(x) dx& =& T p
-\sum k=1^n b2k~(b
- a)^2k \over p^2k(2k)!
(f^(2k-1)(b) - f^(2k-1)(a))\%&
\\ & \text & + (b
- a)^2n+2 \over p^2n+1(2n + 1)!
Rn,p \%& \\
\end{align*}

avec \\textbar{}Rn,p\\textbar{}
\leq M2n+1\int ~
0^1\textbar{}B2n+1(t)\textbar{} dt.

Remarque~9.7.2 Ce théorème nous donne un développement à un ordre
arbitraire de la différence entre l'intégrale et sa valeur approchée par
la méthode des trapèzes

Méthode de Simpson

La formule d'Euler-Mac Laurin, nous montre que si f est de classe
C^4, on a

I - Tp = \lambda~ \over p^2 + O( 1
\over p^4 )

On a donc également I - T2p = \lambda~ \over
4p^2 + O( 1 \over p^4 ) puis
4(I - T2p) - (I - Tp) = O( 1 \over
p^4 ) ou encore

I - 1 \over 3 (4T2p - Tp) = O(
1 \over p^4 )

Posons donc ak = a + k b-a \over 2p , on a

\begin{align*} Sp& =& (b - a)
\over 6p (Tp + 4T2p) \%&
\\ & =& (b - a) \over
6p (f(a) + 4f(a1) + 2f(a2) + 4f(a3) +
\\ldots~\%&
\\ & \text &
+2f(a2p-2) + 4f(a2p-1) + f(b)) \%&
\\ \end{align*}

On sait donc que l'on a une ma\\\\jmathmathmathmathoration du type

\textbar{}I - Sp\textbar{}\leq M \over
p^4

Remarque~9.7.3 On peut montrer qu'en fait \textbar{}I -
Sp\textbar{}\leq M4(b-a)^5
\over 2880p^4 avec M4
=\
supt\in{[}a,b{]}\\textbar{}f^(4)(t)\\textbar{},
ma\\\\jmathmathmathmathoration de peu d'intérêt dans la pratique vu la difficulté qu'il y a
habituellement à trouver un ma\\\\jmathmathmathmathorant de M4.

Méthode de Romberg

Elle consiste à généraliser la méthode qui nous a fait passer de la
méthode des trapèzes à la méthode de Simpson en utilisant le calcul de
Tp,T2p,T4p,\\ldots,T2^kp~
pour éliminer successivement les termes en  1 \over
p^2 , 1 \over p^4
,\\ldots~, 1
\over p^2k du développement asymptotique
donné par la formule d'Euler-Mac Laurin.

\paragraph{9.7.3 La méthode de Laplace}

C'est une méthode classique de recherche d'équivalents d'intégrales
dépendant d'un paramètre (ici n) consistant à remarquer qu'un intégrande
du type f(t)e^ng(t) va privilégier, pour n grand, les valeurs
de t pour lesquelles la fonction g atteint son maximum (car si x
\textless{} y, e^nx = o(e^ny)).

Proposition~9.7.5 Soit f :{]}a,b{[}\rightarrow~ \mathbb{R}~ continue intégrable sur
{]}a,b{[}. Soit g :{]}a,b{[}\rightarrow~ \mathbb{R}~ de classe C^2. On suppose que
g atteint son maximum en un point c \in{]}a,b{[} avec g''(c) \textless{}
0, f(c)\neq~0 et que \forall~~\eta
\textgreater{} 0,
sup\textbar{}t-c\textbar{}≥\eta~g(t)
\textless{} g(c). Alors, quand n tend vers + \infty~

\int  a^bf(t)e^ng(t)~
dt ∼ f(c)e^ng(c)\sqrt 2\pi~
\over n\textbar{}g''(c)\textbar{} 

Démonstration Quitte à changer f en - f, on peut supposer f(c)
\textgreater{} 0. Soit \alpha~ tel que 0 \textless{} \alpha~
\textless{}\
min(\textbar{}g''(c)\textbar{},f(c)) et soit \eta \textgreater{} 0 tel
que \textbar{}t - c\textbar{}\leq \eta \rigtharrow~\textbar{}f(t) - f(c)\textbar{}\leq \alpha~ et
\textbar{}g(t) - g(c) - (t-c)^2 \over 2
g''(c)\textbar{} \textless{} \alpha~ (t-c)^2 \over
2 (puisque g'(c) = 0). Sur {[}c - \eta,c + \eta{]}, on a f(c) - \alpha~
\textless{} f(t) \textless{} f(c) + \alpha~ et

g(c) + (t - c)^2 \over 2 (g'`(c) - \alpha~) \leq
g(t) \leq g(c) + (t - c)^2 \over 2 (g''(c) +
\alpha~)

On obtient donc

\begin{align*} (f(c) -
\alpha~)e^ng(c)\int ~
c-\eta^c+\eta exp~
\left (n (t - c)^2 \over 2
(g'`(c) - \alpha~)\right ) dt&& \%&
\\ & \leq& \int ~
c-\eta^c+\etaf(t)e^ng(t) dt \%&
\\ & \leq& (f(c) +
\alpha~)e^ng(c)\int ~
c-\eta^c+\eta exp~
\left (n (t - c)^2 \over 2
(g''(c) + \alpha~)\right ) dt\%&
\\ \end{align*}

Mais, si \lambda~ \textless{} 0, le changement de variable u =
\sqrt n\textbar{}\lambda~\textbar{} \over 2
 (t - c) donne

\begin{align*} \int ~
c-\eta^c+\eta exp~
\left (n\lambda~ (t - c)^2 \over 2
\right ) dt&& \%& \\ &
=& \sqrt 2 \over
n\textbar{}\lambda~\textbar{} \int  ~
-\sqrt n\textbar{}\lambda~\textbar{} \over
2  \eta^\sqrt n\textbar{}\lambda~\textbar{}
\over 2  \etae^-u^2  du \%&
\\ & ∼& \sqrt 2
\over n\textbar{}\lambda~\textbar{}
\int ~
-\infty~^+\infty~e^-u^2  du =
\sqrt 2\pi~ \over
n\textbar{}\lambda~\textbar{} \%& \\
\end{align*}

Posons In =
\sqrtne^-ng(c)\\int
 c-\eta^c+\etaf(t)e^ng(t) dt. On a donc
un \leq In \leq vn avec

\begin{align*} un& =& (f(c) -
\alpha~)\sqrtn\int ~
c-\eta^c+\eta exp~
\left (n (t - c)^2 \over 2
(g'`(c) - \alpha~)\right ) dt\%&
\\ & ∼& (f(c) -
\alpha~)\sqrt 2\pi~ \over \textbar{}g''(c) -
\alpha~\textbar{}  \%& \\
\end{align*}

et de même vn ∼ (f(c) + \alpha~)\sqrt 2\pi~
\over \textbar{}g''(c)+\alpha~\textbar{} . Donnons nous \epsilon
\textgreater{} 0 et soit \alpha~ tel que

\begin{itemize}
\itemsep1pt\parskip0pt\parsep0pt
\item
  (i) 0 \textless{} \alpha~ \textless{}\
  min(\textbar{}g''(c)\textbar{},f(c))
\item
  (ii) (f(c) - \alpha~)\sqrt 2\pi~ \over
  \textbar{}g'`(c)-\alpha~\textbar{}  \textgreater{}
  f(c)\sqrt 2\pi~ \over
  \textbar{}g''(c)\textbar{}  - \epsilon \over 2
\item
  (iii) (f(c) + \alpha~)\sqrt 2\pi~ \over
  \textbar{}g'`(c)+\alpha~\textbar{}  \textless{}
  f(c)\sqrt 2\pi~ \over
  \textbar{}g''(c)\textbar{}  + \epsilon \over 2
\end{itemize}

On prend le \eta correspondant comme ci-dessus. Alors comme
limun~ = (f(c) -
\alpha~)\sqrt 2\pi~ \over
\textbar{}g''(c)-\alpha~\textbar{}  et
limvn~ = (f(c) +
\alpha~)\sqrt 2\pi~ \over
\textbar{}g''(c)+\alpha~\textbar{} , il existe N \in \mathbb{N}~ tel que

\begin{align*} n ≥ N \rigtharrow~ f(c)\sqrt
2\pi~ \over \textbar{}g'`(c)\textbar{} - \epsilon
\over 2 \textless{} un \leq vn
\textless{} f(c)\sqrt 2\pi~ \over
\textbar{}g''(c)\textbar{}  + \epsilon \over 2 & &
\%& \\ \end{align*}

Pour n ≥ N on a donc f(c)\sqrt 2\pi~
\over \textbar{}g'`(c)\textbar{}  - \epsilon
\over 2 \textless{} In \textless{}
f(c)\sqrt 2\pi~ \over
\textbar{}g''(c)\textbar{}  + \epsilon \over 2 .

Soit M =\
sup\textbar{}t-c\textbar{}≥\etag(t) \textless{} g(c). On a alors

\left
\textbar{}\sqrtne^-ng(c)\\int
 \textbar{}t-c\textbar{}≥\etaf(t)e^ng(t)
dt\right
\textbar{}\leq\sqrtne^-n(g(c)-M)\\int
 a^b\textbar{}f(t)\textbar{} dt

qui tend vers 0 quand n tend vers + \infty~. Soit donc N' \in \mathbb{N}~ tel que n \leq N'
\rigtharrow~\sqrtne^-n(g(c)-M)\\int
 a^b\textbar{}f(t)\textbar{} dt \textless{} \epsilon
\over 2 . Alors, pour n ≥\
max(N,N'), on a

\begin{align*} f(c)\sqrt 2\pi~
\over \textbar{}g'`(c)\textbar{} - \epsilon& \textless{}&
\sqrtne^-ng(c)\\int
 a^bf(t)e^ng(t) dt\%&
\\ & \textless{}&
f(c)\sqrt 2\pi~ \over
\textbar{}g''(c)\textbar{}  + \epsilon \%&
\\ \end{align*}

ce qui montre le résultat.

Remarque~9.7.4 Pour appliquer la méthode précédente, il suffit en fait
qu'il existe un n0 \in \mathbb{N}~ tel que \int ~
a^b\textbar{}f(t)\textbar{}e^n0g(t)
dt converge~: on peut alors écrire, en posant f1(t) =
f(t)e^n0g(t)

\begin{align*} \int ~
a^bf(t)e^ng(t) dt& =&
\int  a^bf~
1(t)e^(n-n0)g(t) dt \%&
\\ & ∼&
f1(c)e^(n-n0)g(c)\sqrt
2\pi~ \over (n - n0)\textbar{}g'`(c)\textbar{}
\%& \\ & ∼&
f(c)e^ng(c)\sqrt 2\pi~ \over
n\textbar{}g''(c)\textbar{}  \%& \\
\end{align*}

Exemple~9.7.2 Ecrivons

n! = \Gamma(n + 1) =\int ~
0^+\infty~t^ne^-t dt =
n^n+1\int ~
0^+\infty~(ue^-u)^n du

avec le changement de variable t = nu. Pour trouver un équivalent de
\int ~
0^+\infty~(ue^-u)^n du, on peut appliquer
la méthode de Laplace, en tenant compte de la remarque ci-dessus avec
n0 = 1. On prend donc f(u) = 1, g(u) =\
log (ue^-u) = -u + log~ u qui
atteint son maximum au point 1 avec g''(1) = -1, g(1) = -1. D'où

\int ~
0^+\infty~(ue^-u)^n du ∼
e^-n\sqrt 2\pi~ \over n 

et donc n! ∼\sqrt2\pi~n n^n
\over e^n .

{[}
{[}
{[}
{[}
